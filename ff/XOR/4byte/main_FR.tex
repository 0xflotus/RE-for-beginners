\clearpage
\subsection{Chiffrement le plus simple possible avec un XOR de 4-octets}

Si un pattern plus long était utilisé, comme un pattern de 4 octets, ça serait
facile à repérer.

Par exemple, voici le début du fichier kernel32.dll (version 32-bit de Windows Server
2008):

\begin{figure}[H]
\centering
\myincludegraphics{ff/XOR/4byte/original1.png}
\caption{Fichier original}
\end{figure}

\clearpage

Ici, il est \q{chiffré} avec une clef de 4-octet:

\begin{figure}[H]
\centering
\myincludegraphics{ff/XOR/4byte/encrypted1.png}
\caption{Fichier \q{chiffré}}
\end{figure}

Il est facile de repérer les 4 symboles récurrents.

En effet, l'entête d'un fichier PE comporte de longues zones de zéro, ce qui explique
que la clef devient visible.

\clearpage

Voici le début d'un entête PE au format hexadécimal:

\begin{figure}[H]
\centering
\myincludegraphics{ff/XOR/4byte/original2.png}
\caption{Entête PE}
\end{figure}

\clearpage

Le voici \q{chiffré}:

\begin{figure}[H]
\centering
\myincludegraphics{ff/XOR/4byte/encrypted2.png}
\caption{Entête PE \q{chiffré}}
\end{figure}

Il est facile de repérer que la clef est la séquence de 4 octets suivant: \TT{8C 61 D2 63}.

Avec cette information, c'est facile de déchiffrer le fichier entier.

Il est important de garder à l'esprit ces propriétés importantes des fichiers PE:
1) l'entête PE comporte de nombreuses zones remplies de zéro;
2) toutes les sections PE sont complétées avec des zéros jusqu'à une limite de page
(4096 octets),
donc il y a d'habitude de longues zones à zéro après chaque section.

Quelques autres formats de fichier contiennent de longues zones de zéro.

C'est typique des fichiers utilisés par les scientifiques et les ingénieurs logiciels.

Pour ceux qui veulent inspecter ces fichiers eux-même, ils sont téléchargeables ici:
\url{http://go.yurichev.com/17352}.

\subsubsection{\Exercise}

\begin{itemize}
	\item \url{http://challenges.re/50}
\end{itemize}

