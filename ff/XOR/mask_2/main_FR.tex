% TODO translate
\subsection{Chiffrement simple utilisant un masque XOR, cas II}
\label{XOR_mask_2}

J'ai un autre fichier chiffré, qui est clairement chiffré avec quelque chose de simple,
comme un XOR:

\begin{figure}[H]
\centering
\myincludegraphics{ff/XOR/mask_2/cipher.png}
\caption{Fichier chiffré dans Midnight Commander}
\end{figure}

Le fichier chiffré peut être téléchargé
\href{https://github.com/DennisYurichev/RE-for-beginners/blob/master/ff/XOR/mask_2/files/cipher.txt}{ici}.

L'utilitaire Linux \IT{ent} indique environ $\textasciitilde{}7.5$ bits par octet,
et ceci est un haut niveau d'entropie (\myref{entropy}), proche de celui de fichiers
compressés ou chiffrés correctement.
Mais encore, nous distinguons clairement quelques patterns, il y a quelques blocs
avec une taille de 17 octets, et nous pouvons voir des sortes d'échelles, se décalant
d'un octet à chaque ligne de 16 octets.

On sait aussi que le texte clair est en anglais.

Maintenant, supposons que ce morceau de texte est chiffré par un simple XOR avec une
clef de 17 octets.

J'ai essayé de repérer des blocs de 17 octets se répétant avec Mathematica, comme
je l'ai fait dans l'exemple précédant (\myref{XOR_mask_1}):

\begin{lstlisting}[caption=Mathematica,style=custommath]
In[]:=input = BinaryReadList["/home/dennis/tmp/cipher.txt"];

In[]:=blocks = Partition[input, 17];

In[]:=Sort[Tally[blocks], #1[[2]] > #2[[2]] &]

Out[]:={{{248,128,88,63,58,175,159,154,232,226,161,50,97,127,3,217,80},1},
{{226,207,67,60,42,226,219,150,246,163,166,56,97,101,18,144,82},1},
{{228,128,79,49,59,250,137,154,165,236,169,118,53,122,31,217,65},1},
{{252,217,1,39,39,238,143,223,241,235,170,91,75,119,2,152,82},1},
{{244,204,88,112,59,234,151,147,165,238,170,118,49,126,27,144,95},1},
{{241,196,78,112,54,224,142,223,242,236,186,58,37,50,17,144,95},1},
{{176,201,71,112,56,230,143,151,234,246,187,118,44,125,8,156,17},1},
...
{{255,206,82,112,56,231,158,145,165,235,170,118,54,115,9,217,68},1},
{{249,206,71,34,42,254,142,154,235,247,239,57,34,113,27,138,88},1},
{{157,170,84,32,32,225,219,139,237,236,188,51,97,124,21,141,17},1},
{{248,197,1,61,32,253,149,150,235,228,188,122,97,97,27,143,84},1},
{{252,217,1,38,42,253,130,223,233,226,187,51,97,123,20,217,69},1},
{{245,211,13,112,56,231,148,223,242,226,188,118,52,97,15,152,93},1},
{{221,210,15,112,28,231,158,141,233,236,172,61,97,90,21,149,92},1}}
\end{lstlisting}

Pas de chance, chaque bloc de 17 octets est unique dans le fichier, et n'apparaît
donc qu'une fois.
Peut-être n'y a-t-il pas de zone de 17 octets à zéro, ou de zone contenant seulement
des espaces.
C'est possible toutefois: de telles séries d'espace peuvent être absentes dans des
textes composés rigoureusement.

La première idée est d'essayer toutes les clefs de 17 octets possible et trouver
celles qui donnent un résultat lisible après déchiffrement.
La force brute n'est pas une option, car il y a $256^{17}$ clefs possible ($\textasciitilde{}10^{40}$),
c'est beaucoup trop.
Mais il y a une bonne nouvelle: qui a dit que nous devons tester la clef de 17 octets
en entier, pourquoi ne pas teste chaque octet séparémment?
C'est possible en effet.

Maintenant, l'algorithme est:

\begin{itemize}
\item essayer chacun des 25 octets pour le premier octet de la clef;
\item déchiffrer le 1er octet de chaque bloc de 17 octets du fichier;
\item est-ce que tous les octets déchiffrés sont imprimable? garder un oeil dessus;
\item faire de même pour chacun des 17 octets de la clef.
\end{itemize}

J'ai écrit le script Python suivant pour essayer cette idée:

\begin{lstlisting}[caption=Python script,style=custompy]
each_Nth_byte=[""]*KEY_LEN

content=read_file(sys.argv[1])
# split input by 17-byte chunks:
all_chunks=chunks(content, KEY_LEN)
for c in all_chunks:
    for i in range(KEY_LEN):
        each_Nth_byte[i]=each_Nth_byte[i] + c[i]

# try each byte of key
for N in range(KEY_LEN):
    print "N=", N
    possible_keys=[]
    for i in range(256):
        tmp_key=chr(i)*len(each_Nth_byte[N])
        tmp=xor_strings(tmp_key,each_Nth_byte[N])
        # are all characters in tmp[] are printable?
        if is_string_printable(tmp)==False:
	    continue
	possible_keys.append(i)
    print possible_keys, "len=", len(possible_keys)
\end{lstlisting}

(La version complète du code source est
 \href{https://github.com/DennisYurichev/RE-for-beginners/blob/master/ff/XOR/mask_2/files/decrypt2.py}{ici}.)

Voici sa sortie:

\begin{lstlisting}
N= 0
[144, 145, 151] len= 3
N= 1
[160, 161] len= 2
N= 2
[32, 33, 38] len= 3
N= 3
[80, 81, 87] len= 3
N= 4
[78, 79] len= 2
N= 5
[142, 143] len= 2
N= 6
[250, 251] len= 2
N= 7
[254, 255] len= 2
N= 8
[130, 132, 133] len= 3
N= 9
[130, 131] len= 2
N= 10
[206, 207] len= 2
N= 11
[81, 86, 87] len= 3
N= 12
[64, 65] len= 2
N= 13
[18, 19] len= 2
N= 14
[122, 123] len= 2
N= 15
[248, 249] len= 2
N= 16
[48, 49] len= 2
\end{lstlisting}

Donc, il y a 2 ou 3 octets possible pour chaque octet de l clef de 17 octets.
C'est mieux que 256 octets pour chaque octet, mais encore beaucoup trop.
Il y a environ 1 million de clefs possible:

\begin{lstlisting}[caption=Mathematica,style=custommath]
In[]:= 3*2*3*3*2*2*2*2*3*2*2*3*2*2*2*2*2
Out[]= 995328
\end{lstlisting}

Il est possible de les vérifier toutes, mais alors nous devons vérifier visuellement
si le texte déchiffré à l'air d'un texte en anglais.

Prenons en compte le fait que nous avons à faire avec 1) un langage naturel 2) de l'anglais.
Les langages naturels ont quelques caractéristiques statistiques importantes.
Tout d'abord, le ponctuation et la longueur des mots.
Quelle est la longueur moyenne des mots en anglais?
Comptons les espaces dans quelques textes bien connus en anglais avec Mathematica.

Voici le fichier texte de \href{http://www.gutenberg.org/cache/epub/100/pg100.txt}{\q{The Complete Works of William Shakespeare}}
provenant de la bibliothèque Gutenberg.

\begin{lstlisting}[caption=Mathematica,style=custommath]
In[]:= input = BinaryReadList["/home/dennis/tmp/pg100.txt"];

In[]:= Tally[input]
Out[]= {{239, 1}, {187, 1}, {191, 1}, {84, 39878}, {104, 
  218875}, {101, 406157}, {32, 1285884}, {80, 12038}, {114, 
  209907}, {111, 282560}, {106, 2788}, {99, 67194}, {116, 
  291243}, {71, 11261}, {117, 115225}, {110, 216805}, {98, 
  46768}, {103, 57328}, {69, 42703}, {66, 15450}, {107, 29345}, {102, 
  69103}, {67, 21526}, {109, 95890}, {112, 46849}, {108, 146532}, {87,
   16508}, {115, 215605}, {105, 199130}, {97, 245509}, {83, 
  34082}, {44, 83315}, {121, 85549}, {13, 124787}, {10, 124787}, {119,
   73155}, {100, 134216}, {118, 34077}, {46, 78216}, {89, 9128}, {45, 
  8150}, {76, 23919}, {42, 73}, {79, 33268}, {82, 29040}, {73, 
  55893}, {72, 18486}, {68, 15726}, {58, 1843}, {65, 44560}, {49, 
  982}, {50, 373}, {48, 325}, {91, 2076}, {35, 3}, {93, 2068}, {74, 
  2071}, {57, 966}, {52, 107}, {70, 11770}, {85, 14169}, {78, 
  27393}, {75, 6206}, {77, 15887}, {120, 4681}, {33, 8840}, {60, 
  468}, {86, 3587}, {51, 343}, {88, 608}, {40, 643}, {41, 644}, {62, 
  440}, {39, 31077}, {34, 488}, {59, 17199}, {126, 1}, {95, 71}, {113,
   2414}, {81, 1179}, {63, 10476}, {47, 48}, {55, 45}, {54, 73}, {64, 
  3}, {53, 94}, {56, 47}, {122, 1098}, {90, 532}, {124, 33}, {38, 
  21}, {96, 1}, {125, 2}, {37, 1}, {36, 2}}

In[]:= Length[input]/1285884 // N
Out[]= 4.34712
\end{lstlisting}

Il y a 1285884 espaces dans l'ensemble du fichier, et la fréquence de l'occurrence
des espaces est de 1 par $\textasciitilde{}4.3$ caractères.

Maintenant voici \href{http://www.gutenberg.org/ebooks/11}{Alice's Adventures in Wonderland, par Lewis Carroll}
de la même bibliothèque:

\begin{lstlisting}[caption=Mathematica,style=custommath]
In[]:= input = BinaryReadList["/home/dennis/tmp/pg11.txt"];

In[]:= Tally[input]
Out[]= {{239, 1}, {187, 1}, {191, 1}, {80, 172}, {114, 6398}, {111, 
  9243}, {106, 222}, {101, 15082}, {99, 2815}, {116, 11629}, {32, 
  27964}, {71, 193}, {117, 3867}, {110, 7869}, {98, 1621}, {103, 
  2750}, {39, 2885}, {115, 6980}, {65, 721}, {108, 5053}, {105, 
  7802}, {100, 5227}, {118, 911}, {87, 256}, {97, 9081}, {44, 
  2566}, {121, 2442}, {76, 158}, {119, 2696}, {67, 185}, {13, 
  3735}, {10, 3735}, {84, 571}, {104, 7580}, {66, 125}, {107, 
  1202}, {102, 2248}, {109, 2245}, {46, 1206}, {89, 142}, {112, 
  1796}, {45, 744}, {58, 255}, {68, 242}, {74, 13}, {50, 12}, {53, 
  13}, {48, 22}, {56, 10}, {91, 4}, {69, 313}, {35, 1}, {49, 68}, {93,
   4}, {82, 212}, {77, 222}, {57, 11}, {52, 10}, {42, 88}, {83, 
  288}, {79, 234}, {70, 134}, {72, 309}, {73, 831}, {85, 111}, {78, 
  182}, {75, 88}, {86, 52}, {51, 13}, {63, 202}, {40, 76}, {41, 
  76}, {59, 194}, {33, 451}, {113, 135}, {120, 170}, {90, 1}, {122, 
  79}, {34, 135}, {95, 4}, {81, 85}, {88, 6}, {47, 24}, {55, 6}, {54, 
  7}, {37, 1}, {64, 2}, {36, 2}}

In[]:= Length[input]/27964 // N
Out[]= 5.99049
\end{lstlisting}

Le résultat est différent, sans soute à cause d'un formatage des textes différents
(indentation ou remplissage).

Ok, donc supposons que la fréquence moyenne de l'espace en anglais est de 1 espace
tous les 4..7 caractères.

Maintenant, encore une bonne nouvelle: nous pouvons mesurer la fréquence des espaces
au fur et à mesure du déchiffrement de notre fichier.
Maintenant je compte les espaces dans chaque \IT{slice} et jette les clefs de 1 octets
qui produise un résultat avec un nombre d'espaces trop petit (ou trop grand, mais
c'est presque impossible avec une si petite clef):

\begin{lstlisting}[caption=Python script,style=custompy]
each_Nth_byte=[""]*KEY_LEN

content=read_file(sys.argv[1])
# split input by 17-byte chunks:
all_chunks=chunks(content, KEY_LEN)
for c in all_chunks:
    for i in range(KEY_LEN):
        each_Nth_byte[i]=each_Nth_byte[i] + c[i]

# try each byte of key
for N in range(KEY_LEN):
    print "N=", N
    possible_keys=[]
    for i in range(256):
        tmp_key=chr(i)*len(each_Nth_byte[N])
        tmp=xor_strings(tmp_key,each_Nth_byte[N])
        # are all characters in tmp[] are printable?
        if is_string_printable(tmp)==False:
	    continue
        # count spaces in decrypted buffer:
	spaces=tmp.count(' ')
	if spaces==0:
            continue
	spaces_ratio=len(tmp)/spaces
	if spaces_ratio<4:
	    continue
	if spaces_ratio>7:
	    continue
	possible_keys.append(i)
    print possible_keys, "len=", len(possible_keys)
\end{lstlisting}

(La version complète du code source se trouve
\href{https://github.com/DennisYurichev/RE-for-beginners/blob/master/ff/XOR/mask_2/files/decrypt3.py}{ici}.)

Ceci nous donne un seul octet possible pour chaque octet de la clef:

\begin{lstlisting}
N= 0
[144] len= 1
N= 1
[160] len= 1
N= 2
[33] len= 1
N= 3
[80] len= 1
N= 4
[79] len= 1
N= 5
[143] len= 1
N= 6
[251] len= 1
N= 7
[255] len= 1
N= 8
[133] len= 1
N= 9
[131] len= 1
N= 10
[207] len= 1
N= 11
[86] len= 1
N= 12
[65] len= 1
N= 13
[18] len= 1
N= 14
[122] len= 1
N= 15
[249] len= 1
N= 16
[49] len= 1
\end{lstlisting}

Vérifions cette clef dans Mathematica:

\begin{lstlisting}[caption=Mathematica,style=custommath]
In[]:= input = BinaryReadList["/home/dennis/tmp/cipher.txt"];

In[]:= blocks = Partition[input, 17];

In[]:= key = {144, 160, 33, 80, 79, 143, 251, 255, 133, 131, 207, 86, 65, 18, 122, 249, 49};

In[]:= EncryptBlock[blk_] := BitXor[key, blk]

In[]:= encrypted = Map[EncryptBlock[#] &, blocks];

In[]:= BinaryWrite["/home/dennis/tmp/plain2.txt", Flatten[encrypted]]

In[]:= Close["/home/dennis/tmp/plain2.txt"]
\end{lstlisting}

Et le texte brut est:

\begin{lstlisting}
Mr. Sherlock Holmes, who was usually very late in the mornings, save
upon those not infrequent occasions when he was up all night, was seated
at the breakfast table. I stood upon the hearth-rug and picked up the
stick which our visitor had left behind him the night before. It was a
fine, thick piece of wood, bulbous-headed, of the sort which is known as
a "Penang lawyer." Just under the head was a broad silver band nearly
an inch across. "To James Mortimer, M.R.C.S., from his friends of the
C.C.H.," was engraved upon it, with the date "1884." It was just such a
stick as the old-fashioned family practitioner used to carry--dignified,
solid, and reassuring.

"Well, Watson, what do you make of it?"

Holmes was sitting with his back to me, and I had given him no sign of
my occupation.

...
\end{lstlisting}

(La version complète de ce texte se trouve
\href{https://github.com/DennisYurichev/RE-for-beginners/blob/master/ff/XOR/mask_2/files/plain.txt}{ici}.)

Le texte semble correct.
Oui, j'ai créé cet exemple de toutes pièces et j'ai choisi un texte très connu de
Conan Doyle, mais c'est très proche de ce que j'ai eu à faire il y a quelques temps.

\subsubsection{Autres idées à envisager}

Si nous échouions avec le comptage des espaces, il y a d'autres idées à essayer:

\begin{itemize}

\item Prenons en considération le fait que les lettres minuscules sont plus fréquentes
que celles en majuscule.

\item Analyse des fréquences.

\item Il y a aussi une bonne technique pour détecter le langage d'un texte: les trigrammes.
Chaque langage possède des triplets de lettres fréquences, qui peuvent être \q{the}
et \q{tha} en anglais.
En lire plus à ce sujet:
\href{http://odur.let.rug.nl/~vannoord/TextCat/textcat.pdf}{N-Gram-Based Text Categorization},
\url{http://code.activestate.com/recipes/326576/}.
Fait suffisamment intéressant, la détection des trigrammes peut être utilisée lorsque
vous décryptez un texte chiffré progressivement, comme dans cet exemple (yous devez
juste tester les 3 caractères décryptez adjacents).

Pour les systèmes non-latin encodés en UTF-8, les choses peuvent être plus simples.
Par exemple, les textes en russe encodés en UTF-8 ont chaque octet intercalé avec
des octets 0xD0/0xD1.
C'est parce que les caractères cyrilliques sont situés dans le 4ème bloc de la table
Unicode.
D'autres systèmes d'écriture on leurs propres blocs.

\end{itemize}

