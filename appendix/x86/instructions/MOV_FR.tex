\myindex{x86!\Instructions!MOV}
\item[MOV] charger une valeur.
Le nom de cette instruction est inapproprié, ce qui entraîne des confusions (la donnée
n'est pas déplacée, mais copiée), dans d'autres architectures la même instruction
est en général appelée \q{LOAD} et/ou \q{STORE} ou quelque chose comme ça.

Une chose importante: si vous mettez la partie 16-bit basse d'un registre 32-bit
en mode 32-bit, les 16-bit haut restent comme ils étaient.
Mais si vous modifiez la partie 32-bit basse d'un registre en mode 64-bit, les 32-bits
haut du registre seront mis à zéro.

Peut-être que ça a été fait pour simplifier le portage du code sur x86-64.

