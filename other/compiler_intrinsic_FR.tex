\mysection{Fonctions intrinsèques du compilateur}
\myindex{Compiler intrinsic}
\label{sec:compiler_intrinsic}

\myindex{x86!\Instructions!ROL}
\myindex{x86!\Instructions!ROR}

Les fonctions intrinsèques sont spécifiques à chaque compilateur. Ce ne sont pas des fonctions que
vous pouvez retrouver dans une bibliothèque.
Le compilateur génère une séquence spécifique de code machine lorsqu'il rencontre la fonction
intrinsèque. Le plus souvent, il s'agit d'une pseudo fonction qui correspond à une instruction d'un
\ac{CPU} particulier.\\
\\
Par exemple, il n'existe pas d'opérateur de décalage cyclique dans les langages \CCpp. La plupart
des \ac{CPU}s supportent cependant des instructions de ce type.
Pour faciliter la vie des programmeurs, le compilateur MSVC propose de telles pseudo fonctions
\IT{\_rotl()} and \IT{\_rotr()}\FNMSDNROTxURL{}
qui sont directement traduites par le compilateur vers les instructions x86 ROL/ROR. \\
\\
Les fonctions intrinsèques qui permettent de générer des instructions SSE en sont un autre exemple.

La liste complète des fonctions intrinsèques proposées par le compilateur MSVC figurent dans le
\href{http://go.yurichev.com/17254}{MSDN}.

