\mysection{Compiler Anomalien}
\label{anomaly:Intel}
\myindex{\CompilerAnomaly}

\subsection{\oracle 11.2 und Intel C++ 10.1}

\myindex{Intel C++}
\myindex{\oracle}
\myindex{x86!\Instructions!JZ}

Der Intel C++ 10.1-Compiler, der für \oracle 11.2 für Linux 86 genutzt wurde, kann
zwei \JZ in einer Reihe ausgeben. Es gibt keine Referenz zum zweiten \JZ. Das zweite
ist also ohne Bedeutung.

\begin{lstlisting}[caption=kdli.o from libserver11.a,style=customasmx86]
.text:08114CF1                   loc_8114CF1: ; CODE XREF: __PGOSF539_kdlimemSer+89A
.text:08114CF1                                ; __PGOSF539_kdlimemSer+3994
.text:08114CF1 8B 45 08              mov     eax, [ebp+arg_0]
.text:08114CF4 0F B6 50 14           movzx   edx, byte ptr [eax+14h]
.text:08114CF8 F6 C2 01              test    dl, 1
.text:08114CFB 0F 85 17 08 00 00     jnz     loc_8115518
.text:08114D01 85 C9                 test    ecx, ecx
.text:08114D03 0F 84 8A 00 00 00     jz      loc_8114D93
.text:08114D09 0F 84 09 08 00 00     jz      loc_8115518
.text:08114D0F 8B 53 08              mov     edx, [ebx+8]
.text:08114D12 89 55 FC              mov     [ebp+var_4], edx
.text:08114D15 31 C0                 xor     eax, eax
.text:08114D17 89 45 F4              mov     [ebp+var_C], eax
.text:08114D1A 50                    push    eax
.text:08114D1B 52                    push    edx
.text:08114D1C E8 03 54 00 00        call    len2nbytes
.text:08114D21 83 C4 08              add     esp, 8
\end{lstlisting}

\begin{lstlisting}[caption=from the same code,style=customasmx86]
.text:0811A2A5                   loc_811A2A5: ; CODE XREF: kdliSerLengths+11C
.text:0811A2A5                                ; kdliSerLengths+1C1
.text:0811A2A5 8B 7D 08              mov     edi, [ebp+arg_0]
.text:0811A2A8 8B 7F 10              mov     edi, [edi+10h]
.text:0811A2AB 0F B6 57 14           movzx   edx, byte ptr [edi+14h]
.text:0811A2AF F6 C2 01              test    dl, 1
.text:0811A2B2 75 3E                 jnz     short loc_811A2F2
.text:0811A2B4 83 E0 01              and     eax, 1
.text:0811A2B7 74 1F                 jz      short loc_811A2D8
.text:0811A2B9 74 37                 jz      short loc_811A2F2
.text:0811A2BB 6A 00                 push    0
.text:0811A2BD FF 71 08              push    dword ptr [ecx+8]
.text:0811A2C0 E8 5F FE FF FF        call    len2nbytes
\end{lstlisting}

Dies ist vermutlich ein Fehler im Codegenerator der während der Tests nicht
gefunden wurde. Der resultierende Code funktioniert trotzdem.

\input{other/anomaly2_DE}

\subsection{Zusammenfassung}

Andere Compiler-Anomalien in diesem Buch:
\myref{anomaly:LLVM}, \myref{loops_iterators_loop_anomaly}, \myref{Keil_anomaly},
\myref{MSVC2013_anomaly},
\myref{MSVC_double_JMP_anomaly},
\myref{MSVC2012_anomaly}.

Diese Beispiele werden in diesem Buch gezeigt, um zu verdeutlichen, das solche Fehler
in den Compilern möglich sind und es gelegentlich keinen Sinn ergibt sich den Kopf
darüber zu zerbrechen warum der Compiler diesen \q{seltsamen} Code erzeugte.
