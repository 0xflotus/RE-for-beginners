% TODO resync with EN version
\chapter{Bücher / Lesenswerte Blogs}

\mysection{Bücher und andere Materialien}

\subsection{Reverse Engineering}

\begin{itemize}
\item Eldad Eilam, \IT{Reversing: Secrets of Reverse Engineering}, (2005)

\item Bruce Dang, Alexandre Gazet, Elias Bachaalany, Sebastien Josse, \IT{Practical Reverse Engineering: x86, x64, ARM, Windows Kernel, Reversing Tools, and Obfuscation}, (2014)

\item Michael Sikorski, Andrew Honig, \IT{Practical Malware Analysis: The Hands-On Guide to Dissecting Malicious Software}, (2012)

\item Chris Eagle, \IT{IDA Pro Book}, (2011)

\item Reginald Wong, \IT{Mastering Reverse Engineering: Re-engineer your ethical hacking skills}, (2018)

\end{itemize}


Ebenfalls das Buch von Kris Kaspersky.

\subsection{Windows}

\input{Win_reading}

\subsection{\CCpp}

\input{CCppBooks}

\subsection{x86 / x86-64}

\label{x86_manuals}
\begin{itemize}
\item Intel Handbücher\footnote{\AlsoAvailableAs \url{http://www.intel.com/content/www/us/en/processors/architectures-software-developer-manuals.html}}

\item AMD Handbücher\footnote{\AlsoAvailableAs \url{http://developer.amd.com/resources/developer-guides-manuals/}}

\item \AgnerFog{}\footnote{\AlsoAvailableAs \url{http://agner.org/optimize/microarchitecture.pdf}}

\item \AgnerFogCC{}\footnote{\AlsoAvailableAs \url{http://www.agner.org/optimize/calling_conventions.pdf}}

\item \IntelOptimization

\item \AMDOptimization
\end{itemize}

Etwas veraltet aber immer noch interessant zu lesen:

\MAbrash\footnote{\AlsoAvailableAs \url{https://github.com/jagregory/abrash-black-book}}
(Er ist bekannt für seine Arbeiten auf dem Gebiet der Low-Level Optimierung in Projekten wie Windows NT 3.1 und id Quake).

\subsection{ARM}

\begin{itemize}
\item ARM Handbücher\footnote{\AlsoAvailableAs \url{http://infocenter.arm.com/help/index.jsp?topic=/com.arm.doc.subset.architecture.reference/index.html}}

\item \ARMSevenRef

\item \ARMSixFourRefURL

\item \ARMCookBook\footnote{\AlsoAvailableAs \url{http://go.yurichev.com/17273}}
\end{itemize}

\subsection{Assembly language}

Richard Blum --- Professional Assembly Language.

\subsection{Java}

\JavaBook.

\subsection{UNIX}

\TAOUP

% subsection:
\input{crypto_reading}

\mysection{Anderes}

\HenryWarren.

% TODO! shouldn't be here!
Es gibt zwei exzellente \ac{RE}-relevante Subreddits auf reddit.com:
\href{http://go.yurichev.com/17027}{reddit.com/r/ReverseEngineering/} und
\href{http://go.yurichev.com/17028}{reddit.com/r/remath}
(über die Themen die sich mit \ac{RE} und Mathematik überschneiden).

Es gibt auch einen \ac{RE}relevanten Teil auf der Stack Exchange-Website:

\par \href{http://go.yurichev.com/17029}{reverseengineering.stackexchange.com}.

Im IRC gibt es einen \#\#re Channel auf
FreeNode\footnote{\href{http://go.yurichev.com/17030}{freenode.net}}.
