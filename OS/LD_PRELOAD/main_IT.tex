\subsection{\IT{LD\_PRELOAD} hack in Linux}

\myindex{LD\_PRELOAD}
\label{ld_preload}

Questo ci permette di caricare le nostre librerie dinamiche prima delle altre, anche quelle del sistema, come libc.so.6.

Questo a sua volta ci permette di \q{sostituire} la funzione che abbiamo scritto con quella nelle librerie del sistema.
Ad esempio, è facile intercettare tutte le chiamate da  
time(), read(), write(), etc. \\
\\
\myindex{uptime}
Proviamo ad ingannare l'utility \IT{uptime}.
Come sappiamo, essa ci dice da quanto tempo il computer sta lavorando.
\myindex{strace}
Con l'aiuto di strace(\myref{strace}), è possibile osservare che l'utility prende questa informazione dal file \TT{/proc/uptime}:

\begin{lstlisting}
$ strace uptime 
...
open("/proc/uptime", O_RDONLY)          = 3
lseek(3, 0, SEEK_SET)                   = 0
read(3, "416166.86 414629.38\n", 2047)  = 20
...
\end{lstlisting}

Questo non è un file presente su disco ma uno virtuale, il suo contenuto è generato al volo nel Linux kernel.
Contiene due numeri:

\begin{lstlisting}
$ cat /proc/uptime
416690.91 415152.03
\end{lstlisting}

Da Wikipedia possiamo imparare che
\footnote{\href{http://go.yurichev.com/17043}{wikipedia}}:

\begin{framed}
\begin{quotation}
Il primo numero è il numero totale di secondi che il sistema è acceso.
Il secondo numero è la quantità di tempo che la macchina è rimasta in attesa (idle), in secondi.
\end{quotation}
\end{framed}

\myindex{\CStandardLibrary!open()}
\myindex{\CStandardLibrary!read()}
\myindex{\CStandardLibrary!close()}

Proviamo a scrivere la nostra libreria dinamica con le funzioni open(), read(), e close().

Come prima cosa, la funzione open() confronterà il nome del file da aprire con con quello che ci serve,
se l'esito è positivo, scriverà il descrittore del file aperto.

In secondo luogo, la funzione read(), se chiamata per tale descrittore del file, sostituirà l'output, altrimenti chiamerà 
la funzione read() originale dalla libreria libc.so.6.
E quindi la funzione close(), chiuderà il file che abbiamo utilizzato.

\myindex{dlopen()}
\myindex{dlsym()}

Useremo le funzioni dlopen() e dlsym() per determinare l'indirizzo della funzione originale in libc.so.6.
Dobbiamo usare queste funzioni per passare il controllo alla \q{vera} funzione.

\myindex{\CStandardLibrary!strcmp()}

D'altra parte, se intercettiamo la chiamata a strcmp() e monitoriamo ogni confronto tra stringhe
nel programma, dovremmo implementare una nostra versione di strcmp(), senza usare la funzione originale.
\footnote{Ad esempio, in questo articolo trovi quanto facilmente strcmp() riesce ad intercettare le chiamate
\footnote{\href{http://go.yurichev.com/17143}{yurichev.com}}
written by Yong Huang}

\lstinputlisting[style=customc]{OS/LD_PRELOAD/fool_uptime.c}
( \href{https://github.com/DennisYurichev/RE-for-beginners/blob/master/OS/LD_PRELOAD/fool_uptime.c}{Source code at GitHub} )
% FIXME go.yurichev.com...

Compiliamolo con librerie dinamiche comuni:

\begin{lstlisting}
gcc -fpic -shared -Wall -o fool_uptime.so fool_uptime.c -ldl
\end{lstlisting}

Avviamo \IT{uptime} caricando prima le nostre librerie:

\begin{lstlisting}
LD_PRELOAD=`pwd`/fool_uptime.so uptime
\end{lstlisting}

Osserviamo che:

\begin{lstlisting}
 01:23:02 up 24855 days,  3:14,  3 users,  load average: 0.00, 0.01, 0.05
\end{lstlisting}

Se la variabile d'ambiente \IT{LD\_PRELOAD} punta sempre al nome del file ed al percorso della nostra libreria, 
deve essere per forza avviato per tutti i programmi che andremo ad avviare. \\
\\
Altri esempi:

\begin{itemize}

\item
Semplice intercettazione della funzione strcmp() (Yong Huang) 
\url{http://go.yurichev.com/17143}

\item
Kevin Pulo---Fun with LD\_PRELOAD. Molti esempio ed idee.
\href{http://go.yurichev.com/17145}{yurichev.com}

\item
Funzioni che intercettano file, per la compressione/decompressione di file al volo (zlibc). \url{http://go.yurichev.com/17146}

\end{itemize}
