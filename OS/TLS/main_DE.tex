\mysection{lokaler Thread-Speicher}
\label{TLS}
\myindex{TLS}

TLS ist ein Datenbereich der für jeden einzelnen Thread spezifisch ist.
Jeder Thread kann hier alles speichern was er möchte.
Eine bekanntes Beispiel ist die globale C-Standardvariable \IT{errno}.

Mehrere Threads können Funktionen simultan aufrufen, die einen Fehlercode in \IT{errno} zurückgeben.
Eine globale Variable würde hier nicht korrekt für Multithread-Programme funktionieren,
aus diesem Grund muss \IT{errno} im \ac{TLS} gesichert.

\myindex{\Cpp!C++11}
Im C++11-Standard wurde ein neues Schlüsselwort \IT{thread\_local} eingeführt,
um anzuzeigen, dass jeder Thread eine eigene Kopie dieser Variable, die initialisiert werde kann
und sich auf dem \ac{TLS} befindet\footnote{\myindex{C11} C11 hat ebenfalls einen (optionalen) Thread-Support}:

\begin{lstlisting}[caption=C++11]
#include <iostream>
#include <thread>

thread_local int tmp=3;

int main()
{
	std::cout << tmp << std::endl;
};
\end{lstlisting}

Kompiliert mit MinGW GCC 4.8.1 jedoch nicht MSVC 2012.

Wenn es um PE-Dateien geht, also die ausführbaren Dateien, wird die \IT{tmp}-Variable
in der Sektion mit dem Namen \ac{TLS} gesichert.

\subsection{Nochmals Linearer Kongruenzgenerator}
\label{LCG_TLS}

Der Pseudozufallszahlen-Generator der in \myref{LCG_simple} bereits erwähnt wurde hat einen Nachteil:
Er ist nicht Thread-sicher, weil er eine interne Zustandsvariable hat, die von verschiedenen Threads
gleichzeitig gelesen und verändert werden kann.

% subsections
\input{OS/TLS/LCG_win32_DE}
\input{OS/TLS/LCG_linux_DE}
