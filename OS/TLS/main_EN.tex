\mysection{Thread Local Storage}
\label{TLS}
\myindex{TLS}

TLS is a data area, specific to each thread. Every thread can store what it needs there.
One well-known example is the C standard global variable \IT{errno}. 

Multiple threads may simultaneously call functions
which return an error code in \IT{errno}, so a global variable will not work correctly here for multi-threaded programs,
so \IT{errno} must be stored in the \ac{TLS}. \\
\\
\myindex{\Cpp!C++11}
In the C++11 standard, a new \IT{thread\_local} modifier was added, showing that each thread has its own version of the variable,
it can be initialized, and it is located in the \ac{TLS}
\footnote{\myindex{C11} C11 also has thread support, optional though}:

\begin{lstlisting}[caption=C++11,style=customc]
#include <iostream>
#include <thread>

thread_local int tmp=3;

int main()
{
	std::cout << tmp << std::endl;
};
\end{lstlisting}

Compiled in MinGW GCC 4.8.1, but not in MSVC 2012.

If we talk about PE files, in the resulting executable file, the \IT{tmp} 
variable is to be allocated in the section devoted to the \ac{TLS}.

\subsection{Linear congruential generator revisited}
\label{LCG_TLS}

The pseudorandom number generator we considered earlier \myref{LCG_simple} has a flaw:
it's not thread-safe, because it has an internal state variable which can be read and/or modified in different threads simultaneously.

% subsections
\input{OS/TLS/LCG_win32_EN}
\input{OS/TLS/LCG_linux_EN}

