\mysection{Methoden zur Argumentenübergabe (Aufrufkonventionen)}
\label{sec:callingconventions}

\subsection{cdecl}
\myindex{cdecl}
\label{cdecl}

Hierbei handelt es sich um die am weitesten verbreitete Methode um in \CCpp-Sprachen
Argumente an Funktionen zu übergeben.

Der \gls{caller} muss den Wert des \gls{stack pointer} (\ESP) auf den ursprünglichen Stand
bringen, nachdem \gls{callee}-Funktion beendet wurde.

\begin{lstlisting}[caption=cdecl]
push Argument3
push Argument2
push Argument1
call Funktion
add esp, 12 ; gibt ESP zurueck
\end{lstlisting}

\subsection{stdcall}
\label{sec:stdcall}
\myindex{stdcall}

\newcommand{\SIZEOFINT}{Die Größe einer Variablen vom Datentyp \Tint ist 4 in x86-Systemen und 8 in x64-Systemen}

Dies ist fast gleich zu der \IT{cdecl}-Aufrufkonvention, mit Ausnahme, dass die
\gls{callee} den Wert von \ESP auf den ursprünglichen Wert setzen muss. Dies geschieht
durch die \TT{RET x} anstatt \RET, wobei gilt \TT{x = Nummer des Arguments * sizeof(int)\footnote{\SIZEOFINT}}.

Der \gls{caller} passt den \gls{stack pointer} nicht an, es sind also keine \TT{add esp, x}-Anweisungen vorhanden.

\begin{lstlisting}[caption=stdcall]
push Argument3
push Argument2
push Argument1
call Funktion

Funktion:
... tue etwas ...
ret 12
\end{lstlisting}

Diese Methode ist in Win32-Standard-Bibliotheken allgegenwärtig, fehlt jedoch in Win64 (siehe unten).

\par Beispielsweise kann die Funktion von \myref{src:passing_arguments_ex} genommen werden und durch
Hinzufügen des \TT{\_\_stdcall}-Modifizierers leicht verändert werden:

\begin{lstlisting}
int __stdcall f2 (int a, int b, int c)
{
	return a*b+c;
};
\end{lstlisting}

Das Kompilat ist fast das gleiche wie bei \myref{src:passing_arguments_ex_MSVC_cdecl},
jedoch wird \TT{RET 12} anstatt \TT{RET} genutzt. Der \ac{SP} wird im \gls{caller} nicht
aktualisiert.

Als Konsequenz daraus kann die Anzahl der Funktionsargumente einfach von der \TT{RETN n}-Anweisung
abgeleitet werden, indem $n$ durch 4 geteilt wird

%TODO Compilation of german version fails with style=customasm
%\lstinputlisting[caption=MSVC 2010,style=customasm]{OS/calling_conventions/stdcall_ex.asm}
\lstinputlisting[caption=MSVC 2010]{OS/calling_conventions/stdcall_ex.asm}

\subsubsection{Funktionen mit einer variablen Anzahl von Argumenten}

\printf-ähnliche Funktionen sind vielleicht die einzigen Funktionen in \CCpp mit einer Variablen
Anzahl von Argumenten, aber mit ihnen kann ein wichtiger Unterschied zwischen \IT{cdecl} und
\IT{stdcall} veranschaulicht werden.
Beginnen wir mit der Vorstellung, dass der Compiler die Anzahl der Argumente für jeden Aufruf
von \printf kennt.

Die aufgerufene und bereits kompilierte \printf-Funktion befindet sich in der Datei MSVCRT.DLL
(wenn über Windows geredet wird) und hat aber keinerlei Informationen darüber wie viele Argumente
übergeben wurden; dies kann jedoch über den Formatstring herausgefunden werden.

%TODO: The following sentence is much too long.
Wenn \printf eine \IT{stdcall}-Funktion wäre und den \gls{stack pointer} durch Zählen der Zahl
der Argumente im Formatstring auf den ursprünglichen Wert setzen würde, kann eine gefährliche
Situation entstehen, da ein Schreibfehler des Programmierers zu einem plötzliche Programmabsturz
führen könnte.
Aus diesem Grund ist für diese Art von Funktionen \IT{stdcall} ungeeignet und \IT{cdecl} ist
zu bevorzugen.

\subsection{fastcall}
\label{fastcall}
\myindex{fastcall}

Dies ist der allgemeine Name für eine Methode, in der einige Argumente mittels Registern und
der Rest über den Stack übergeben werden. Für ältere CPUs ist \IT{fastcall} schneller als
\IT{cdecl} und \IT{stdcall} wegen der geringeren Stack-Nutzung.
Auf neueren \ac{CPU}s wird dieser Ansatz vermutlich keine signifikante Geschwindigkeitserhöhung
nach sich ziehen.

\IT{fastcall} ist nicht standardisiert, so das verschiedene Compiler eine unterschiedliche
Umsetzung machen können.
Dies kann zu Problemen führen, wenn zwei DLLs genutzt werden, von denen eine die andere nutzt
und durch das Nutzen verschiedener Compiler unterschiedliche \IT{fastcall}-Aufrufkonventionen
genutzt werden.

Sowohl MSVC als auch GCC übergeben das erste und zweite Argument über \ECX und \EDX und den Rest
der Arguments mittels des Stacks.

Der \gls{stack pointer} muss vom \gls{callee} auf den ursprünglichen Wert gesetzt werden,
wie in \IT{stdcall} auch.

\begin{lstlisting}[caption=fastcall]
push Argument3
mov edx, Argument2
mov ecx, Argument1
call Funktion

Funktion:
.. tue etwas ..
ret 4
\end{lstlisting}

Beispielsweise kann die Funktion von \myref{src:passing_arguments_ex} genommen werden und durch
Hinzufügen des \TT{\_\_fastcall}-Modifizierers leicht verändert werden:

\begin{lstlisting}
int __fastcall f3 (int a, int b, int c)
{
	return a*b+c;
};
\end{lstlisting}

Nachfolgend das Kompilat:

%TODO Compilation of german version fails with style=customasm
%\lstinputlisting[caption=\Optimizing MSVC 2010 /Ob0,style=customasm]{OS/calling_conventions/fastcall_ex.asm}
\lstinputlisting[caption=\Optimizing MSVC 2010 /Ob0]{OS/calling_conventions/fastcall_ex.asm}

Es ist erkennbar, dass der \gls{callee} den \ac{SP} mit der \TT{RETN}-Anweisung
und einem Operanden auf den ursprünglichen Wert setzt.

Dies bedeutet, dass die Anzahl der Argumente ebenfalls einfach abgeleitet werden kann.

\subsubsection{GCC regparm}

\newcommand{\URLREGPARMM}{\url{http://go.yurichev.com/17040}}

Dies ist in gewisser Weise die Weiterentwicklung von \IT{fastcall}\footnote{\URLREGPARMM}.
Mit der \TT{-mregparm}-Option ist es möglich festzulegen, wieviele Argumente per Register
übergeben werden (maximal 3).
Aus diesem Grund werden die Register \EAX, \EDX und \ECX genutzt.

Natürlich werden, wenn die Anzahl der Argumente kleiner als drei ist, nicht alle drei
Register genutzt.

Der \gls{caller} setzt den \gls{stack pointer} auf den initialen Zustand.

Als Beispiel siehe (\myref{regparm}).

\subsubsection{Watcom/OpenWatcom}
\myindex{OpenWatcom}

Hier erfolgt der Aufruf mit der \q{Register-Aufruf-Konvention}.
Die ersten vier Argumente werden in den Registern \EAX, \EDX, \EBX und \ECX übergeben,
der Rest auf dem Stack.

Diese Funktionen haben einen Unterstrich an den Funktionsnamen angehängt, um sie von
den anderen Aufrufkonventionen unterscheiden zu können.

\subsection{thiscall}
\myindex{thiscall}

Hier wird der \ITthis-Zeiger des Objekts an die Methode in \Cpp übergeben.

In MSVC wird \ITthis üblicherweise im \ECX-Register übergeben.

In GCC wird der \ITthis-Zeiger im ersten Argument der Methode übergeben.
Es ist sehr offensichtlich dass intern alle Methoden ein zusätzliches Argument haben.

Als Beispiel siehe (\myref{thiscall}).

\subsection{x86-64}
\myindex{x86-64}

\subsubsection{Windows x64}
\label{sec:callingconventions_win64}

Die Art Argumente zu Übergeben ähnelt in Win64 in gewisser Weise \TT{fastcall}.
Die ersten vier Argumente werden in den Registern \RCX, \RDX, \Reg{8} und \Reg{9}
übergeben und der Rest auf dem Stack.
Der \gls{caller} muss Platz für 32 Byte, also 4 64-Bit-Werte bereitstellen, so
dass der \gls{callee} dort die ersten vier Argumente speichern kann.
Kurze Funktionen können die Werte der Argumente direkt aus den Registern lesen,
während längere Funktionen diese für späteren Gebrauch zwischenspeichern sollten.

Der \gls{caller} muss den \gls{stack pointer} auf den vorherigen Zustand zurücksetzen.

Diese Aufrufkonvention wird auch in den Windows x86-64-System-DLLs genutzt
(anstatt \IT{stdcall} in Win32).

Beispiel:

\lstinputlisting[style=customc]{OS/calling_conventions/x64.c}

%TODO Compilation of german version fails with style=customasm
%\lstinputlisting[caption=MSVC 2012 /0b,style=customasm]{OS/calling_conventions/x64_MSVC_Ob.asm}
\lstinputlisting[caption=MSVC 2012 /0b]{OS/calling_conventions/x64_MSVC_Ob.asm}

\myindex{Scratch space}

Es ist hier klar erkennbar, wie sieben Argumente übergeben werden: vier in den
Registern und die drei restlichen auf dem Stack.

Der Code des f1()-Funktionsprolog sichert die Argumente in dem \q{Scratch Space},
einer Stelle auf dem Stack, die genau für diese Zwecke existiert.

Dies ist so realisiert, weil der Compiler nicht sicher sein kann, dass genug Register
ohne diese vier genutzt werden können und sie sonst durch die Argumente verändert
werden können, bis die Funktion beendet wird.

Die Bereitstellung des \q{Scratch Space} auf dem Stack ist Aufgabe der aufrufenden Funktion.

%TODO Compilation of german version fails with style=customasm
%\lstinputlisting[caption=\Optimizing MSVC 2012 /0b,style=customasm]{OS/calling_conventions/x64_MSVC_Ox_Ob.asm}
\lstinputlisting[caption=\Optimizing MSVC 2012 /0b]{OS/calling_conventions/x64_MSVC_Ox_Ob.asm}

Wenn das Beispiel ohne Optimierung compiliert wird, ist das Ergebnis fast das gleiche,
lediglich der \q{Scratch Space} ist unnötig und wird nicht genutzt.

\myindex{x86!\Instructions!LEA}
\label{using_MOV_and_pack_of_LEA_to_load_values}

Beachtenswert ist auch, wie MSVC 2012 das Laden von einfachen Werten in Register
durch das Nutzen von \LEA (\myref{sec:LEA}) optimiert.
\INS{MOV} wäre hier ein Byte länger (5 anstatt 4).

Ein weiteres Beispiel so eines Sachverhalts ist: \myref{TaskMgr_LEA}.

\myparagraph{Windows x64: Übergeben von \ITthis (\CCpp)}

Der \ITthis-Zeiger wird in \RCX übergeben, das erste Argument der Methode ist \RDX, usw.
Ein Beispiel ist hier zu sehen: \myref{simple_CPP_MSVC_x64}.

\subsubsection{Linux x64}

Die Art wie Linux für x86-64 Argumente übergibt ist fast die Gleiche wie in Windows,
jedoch werden sechs anstatt vier Register genutzt (\RDI, \RSI, \RDX, \RCX, \Reg{8}, \Reg{9})
und es gibt keinen \q{Scratch Space}, auch wenn der \gls{callee} die Registerwerte auf
dem Stack speichern kann wenn er dies will oder muss.

%TODO Compilation of german version fails with style=customasm
%\lstinputlisting[caption=\Optimizing GCC 4.7.3,style=customasm]{OS/calling_conventions/x64_linux_O3.s}
\lstinputlisting[caption=\Optimizing GCC 4.7.3]{OS/calling_conventions/x64_linux_O3.s}

\myindex{AMD}
Zur Beachtung: die Werte werden hier in die 32-Bit-Teile der Register (z.B.EAX) geschrieben,
aber nicht in die kompletten 64-Bit-Register (RAX).
Dies wird gemacht, weil jeder Schreibzugang auf die niederwertigen 32-Bit-Teile eines
Registers automatisch den höherwertigen Teil zurücksetzt.

Vermutlich wurde dies bei AMD so eingeführt um die Portierung des Codes zu x86-64 zu vereinfachen.

\subsection{Rückgabewerte von \Tfloat- und \Tdouble-Typen}
\myindex{float}
\myindex{double}

In allen Konventionen außer in Win64, werden di Werte vom Typ \Tfloat oder \Tdouble
in dem FPU-Register \ST{0} zurückgegeben.

In Win64 werden die Werte vom Typ \Tfloat oder \Tdouble in den niederwertigen 32 oder
64 Bit des \XMM{0}-Registers zurückgegeben.

\subsection{Verändern von Argumenten}

Manchmal fragen \CCpp{}-Programmierer (obwohl nicht auf diese \ac{PL} beschränkt),
was passieren kann wenn die Funktionsargumente verändert werden.

Die Antwort ist einfach: die Argumente sind auf dem Stack gespeichert und hier
werden auch die Veränderungen vorgenommen.

Die aufzurufende Funktion wird diese nicht nach dem Verlassen des \gls{callee}
nutzen (der Autor dieser Linien hat so einen Fall in der Praxis noch nie gesehen).

\lstinputlisting[style=customc]{OS/calling_conventions/change_arguments.c}

%TODO Compilation of german version fails with style=customasm
%\lstinputlisting[caption=MSVC 2012,style=customasm]{OS/calling_conventions/change_arguments.asm}
\lstinputlisting[caption=MSVC 2012]{OS/calling_conventions/change_arguments.asm}

% TODO (OllyDbg) пример как в стеке меняется $a$

Also: ja, die Argumente können einfach modifiziert werden.
Natürlich, wenn diese keine \IT{Referenz} in  \Cpp{} (\myref{cpp_references}) ist,
und nicht die Daten verändert werden auf die der Zeiger zeigt, wird der Effekt nicht
außerhalb der aktuellen Funktion sichtbar sein.

Theoretisch kann der \gls{caller} die modifizierten Argumente auf irgendeine Weise
nutzen nachdem der \gls{callee} beendet wurde.
Vielleicht wenn dies direkt in Assembler programmiert ist.

Beispielsweise wird Code wie folgt von gewöhnlichen \CCpp-Compilern erzeugt:

\begin{lstlisting}
	push	456	; wird gleich b sein
	push	123	; wird gleich a sein
	call	f	; f() modifiziert das erste Argument
	add	esp, 2*4
\end{lstlisting}

Der Code kann wie folgt neu geschrieben werden:

\begin{lstlisting}
	push	456	; wird gleich b sein
	push	123	; wird gleich a sein
	call	f	; f() modifiziert das erste Argument
	pop	eax
	add	esp, 4
	; EAX=Erstes Argument von f() modifiziert in f()
\end{lstlisting}

Es ist scher vorzustellen warum jemand dies tun sollte, aber in der Praxis ist es möglich.
Nichtsdestotrotz bietet der \CCpp-Standard keinen Möglichkeit dies zu tun.

% sections
\subsection{Gesetzte Bits zählen}
Hier ist ein einfaches Beispiel einer Funktion, die die Anzahl der gesetzten
Bits in einem Eingabewert zählt.

Diese Operation wird auch \q{population count}\footnote{moderne x86 CPUs
(die SSE4 unterstützen) haben zu diesem Zweck sogar einen eigenen POPCNT Befehl}
genannt.

\lstinputlisting[style=customc]{patterns/14_bitfields/4_popcnt/shifts.c}
In dieser Schleife wird der Wert von $i$ schrittweise von 0 bis 31 erhöht,
sodass der Ausdruck $1 \ll i$ von 1 bis \TT{0x80000000} zählt.
In natürlicher Sprache würden wir diese Operation als \IT{verschiebe 1 um n
Bits nach links} beschreiben.
Mit anderen Worten: Der Ausdruck $1 \ll i$ erzeugt alle möglichen Bitpositionen
in einer 32-Bit-Zahl.
Das freie Bit auf der rechten Seite wird jeweils gelöscht.

\label{2n_numbers_table}
Hier ist eine Tabelle mit allen Werten von $1 \ll i$ 
für $i=0 \ldots 31$:

\small
\begin{center}
\begin{tabular}{ | l | l | l | l | }
\hline
\HeaderColor \CCpp Ausdruck & 
\HeaderColor Zweierpotenz & 
\HeaderColor Dezimalzahl & 
\HeaderColor Hexadezimalzahl \\
\hline
$1 \ll 0$ & $2^{0}$ & 1 & 1 \\
\hline
$1 \ll 1$ & $2^{1}$ & 2 & 2 \\
\hline
$1 \ll 2$ & $2^{2}$ & 4 & 4 \\
\hline
$1 \ll 3$ & $2^{3}$ & 8 & 8 \\
\hline
$1 \ll 4$ & $2^{4}$ & 16 & 0x10 \\
\hline
$1 \ll 5$ & $2^{5}$ & 32 & 0x20 \\
\hline
$1 \ll 6$ & $2^{6}$ & 64 & 0x40 \\
\hline
$1 \ll 7$ & $2^{7}$ & 128 & 0x80 \\
\hline
$1 \ll 8$ & $2^{8}$ & 256 & 0x100 \\
\hline
$1 \ll 9$ & $2^{9}$ & 512 & 0x200 \\
\hline
$1 \ll 10$ & $2^{10}$ & 1024 & 0x400 \\
\hline
$1 \ll 11$ & $2^{11}$ & 2048 & 0x800 \\
\hline
$1 \ll 12$ & $2^{12}$ & 4096 & 0x1000 \\
\hline
$1 \ll 13$ & $2^{13}$ & 8192 & 0x2000 \\
\hline
$1 \ll 14$ & $2^{14}$ & 16384 & 0x4000 \\
\hline
$1 \ll 15$ & $2^{15}$ & 32768 & 0x8000 \\
\hline
$1 \ll 16$ & $2^{16}$ & 65536 & 0x10000 \\
\hline
$1 \ll 17$ & $2^{17}$ & 131072 & 0x20000 \\
\hline
$1 \ll 18$ & $2^{18}$ & 262144 & 0x40000 \\
\hline
$1 \ll 19$ & $2^{19}$ & 524288 & 0x80000 \\
\hline
$1 \ll 20$ & $2^{20}$ & 1048576 & 0x100000 \\
\hline
$1 \ll 21$ & $2^{21}$ & 2097152 & 0x200000 \\
\hline
$1 \ll 22$ & $2^{22}$ & 4194304 & 0x400000 \\
\hline
$1 \ll 23$ & $2^{23}$ & 8388608 & 0x800000 \\
\hline
$1 \ll 24$ & $2^{24}$ & 16777216 & 0x1000000 \\
\hline
$1 \ll 25$ & $2^{25}$ & 33554432 & 0x2000000 \\
\hline
$1 \ll 26$ & $2^{26}$ & 67108864 & 0x4000000 \\
\hline
$1 \ll 27$ & $2^{27}$ & 134217728 & 0x8000000 \\
\hline
$1 \ll 28$ & $2^{28}$ & 268435456 & 0x10000000 \\
\hline
$1 \ll 29$ & $2^{29}$ & 536870912 & 0x20000000 \\
\hline
$1 \ll 30$ & $2^{30}$ & 1073741824 & 0x40000000 \\
\hline
$1 \ll 31$ & $2^{31}$ & 2147483648 & 0x80000000 \\
\hline
\end{tabular}
\end{center}
\normalsize
Diese Konstanten (Bitmasken) tauchen im Code oft auf und ein Reverse Engineer
muss in der Lage sein, sie schnell und sicher zu erkennen.

% TBT
Es dazu jedoch nicht notwendig, die Dezimalzahlen (Zweierpotenzen) größer
65535 auswendig zu kennen. Die hexadezimalen Zahlen sind leicht zu merken.

Die Konstanten werden häufig verwendet um Flags einzelnen Bits zuzuordnen. 
Hier ist zum Beispiel ein Auszug aus \TT{ssl\_private.h} aus dem Quellcode von
Apache 2.4.6:

\begin{lstlisting}[style=customc]
/**
 * Define the SSL options
 */
#define SSL_OPT_NONE           (0)
#define SSL_OPT_RELSET         (1<<0)
#define SSL_OPT_STDENVVARS     (1<<1)
#define SSL_OPT_EXPORTCERTDATA (1<<3)
#define SSL_OPT_FAKEBASICAUTH  (1<<4)
#define SSL_OPT_STRICTREQUIRE  (1<<5)
#define SSL_OPT_OPTRENEGOTIATE (1<<6)
#define SSL_OPT_LEGACYDNFORMAT (1<<7)
\end{lstlisting}

Zurück zu unserem Beispiel.

Das Makro \TT{IS\_SET} prüft auf Anwesenheit von Bits in $a$.
\myindex{x86!\Instructions!AND}

Das Makro \TT{IS\_SET} entspricht dabei dem logischen (\IT{AND})
und gibt 0 zurück, wenn das entsprechende Bit nicht gesetzt ist, oder die
Bitmaske, wenn das Bit gesetzt ist.
Der Operator \IT{if()} wird in \CCpp ausgeführt, wenn der boolesche Ausdruck
nicht null ist (er könnte sogar 123456 sein), weshalb es meistens richtig
funktioniert.


% subsections
\subsubsection{x86}

\myparagraph{MSVC}

Kompilieren wir das Beispiel:

\lstinputlisting[caption=MSVC 2008,style=customasmx86]{patterns/13_arrays/1_simple/simple_msvc.asm}

\myindex{x86!\Instructions!SHL}
Soweit nichts Außergewöhnliches, nur zwei Schleifen: die erste füllt mit Werten auf und die zweite gibt Werte aus.
% TBT
Der Befehl \TT{shl ecx, 1} wird für die Multiplikation mit 2 in \ECX verwendet; mehr dazu unten~\myref{SHR}.

Auf dem Stack werden 80 Bytes für das Array reserviert: 20 Elemente von je 4 Byte.

\clearpage
Untersuchen wir dieses Beispiel in \olly.
\myindex{\olly}

Wir erkennen wie das Array befüllt wird:

jedes Element ist ein 32-Bit-Wort vom Typ \Tint und der Wert ist der Index multipliziert mit 2:

\begin{figure}[H]
\centering
\myincludegraphics{patterns/13_arrays/1_simple/olly.png}
\caption{\olly: nach dem Füllen des Arrays}
\label{fig:array_simple_olly}
\end{figure}
Da sich dieses Array auf dem Stack befindet, finden wir dort alle seine 20 Elemente.

\myparagraph{GCC}

Hier ist was GCC 4.4.1 erzeugt:

\lstinputlisting[caption=GCC 4.4.1,style=customasmx86]{patterns/13_arrays/1_simple/simple_gcc.asm}
Die Variable $a$ ist übrigens vom Typ \IT{int*} (Pointer auf \Tint{})--man kann einen Pointer auf ein Array an eine
andere Funktion übergeben, aber es ist richtiger zu sagen, dass der Pointer auf das erste Element des Arrays übergeben
wird. (Die Adressen der übrigen Elemente werden in bekannter Weise berechnet.)

Wenn man diesen Pointer mittels \IT{a[idx]} indiziert, wird \IT{idx} zum Pointer addiert und das dort abgelegte Element
(auf das der berechnete Pointer zeigt) wird zurückgegeben.

Ein interessantes Beispiel: ein String wie \IT{\q{string}} ist ein Array von Chars und hat den Typ \IT{const
char[]}.

Auch auf diesen Pointer kann ein Index angewendet werden.

Das ist der Grund warum es es möglich ist, Dinge wie \TT{\q{string}[i]} zu schreiben--es handelt sich dabei um einen
korrekten \CCpp Ausdruck!


\input{patterns/14_bitfields/4_popcnt/x64_DE}
\subsubsection{ARM}

\myparagraph{\OptimizingKeilVI (\ThumbMode)}

\lstinputlisting[style=customasmARM]{patterns/04_scanf/1_simple/ARM_IDA.lst}

\myindex{\CLanguageElements!\Pointers}
Damit \scanf Elemente einlesen kann, benötigt die Funktion einen Paramter--einen Pointer vom Typ \Tint.
\Tint hat die Größe 32 Bit, wir benötigen also 4 Byte, um den Wert im Speicher abzulegen, und passt daher genau in ein 32-Bit-Register.
\myindex{IDA!var\_?}
Auf dem Stack wird Platz für die lokalen Variable \GTT{x} reserviert und IDA bezeichnet diese Variable mit \IT{var\_8}. 
Eigentlich ist aber an dieser Stelle gar nicht notwendig, Platz auf dem Stack zu reservieren, da \ac{SP} (\gls{stack pointer} 
bereits auf die Adresse zeigt und auch direkt verwendet werden kann.

Der Wert von \ac{SP} wird also in das \Reg{1} Register kopiert und zusammen mit dem Formatierungsstring an \scanf übergeben.

% TBT here
%\INS{PUSH/POP} instructions behaves differently in ARM than in x86 (it's the other way around).
%They are synonyms to \INS{STM/STMDB/LDM/LDMIA} instructions.
%And \INS{PUSH} instruction first writes a value into the stack, \IT{and then} subtracts \ac{SP} by 4.
%\INS{POP} first adds 4 to \ac{SP}, \IT{and then} reads a value from the stack.
%Hence, after \INS{PUSH}, \ac{SP} points to an unused space in stack.
%It is used by \scanf, and by \printf after.

%\INS{LDMIA} means \IT{Load Multiple Registers Increment address After each transfer}.
%\INS{STMDB} means \IT{Store Multiple Registers Decrement address Before each transfer}.

\myindex{ARM!\Instructions!LDR}
Später wird mithilfe des \INS{LDR} Befehls dieser Wert vom Stack in das \Reg{1} Register verschoben um an \printf übergeben werden zu können.

\myparagraph{ARM64}

\lstinputlisting[caption=\NonOptimizing GCC 4.9.1 ARM64,numbers=left,style=customasmARM]{patterns/04_scanf/1_simple/ARM64_GCC491_O0_DE.s}

Im Stack Frame werden 32 Byte reserviert, was deutlich mehr als benötigt ist. Vielleicht handelt es sich um eine Frage des Aligning (dt. Angleichens) von Speicheradressen.
Der interessanteste Teil ist, im Stack Frame einen Platz für die Variable $x$ zu finden (Zeile 22).
Warum 28? Irgendwie hat der Compiler entschieden die Variable am Ende des Stack Frames anstatt an dessen Beginn abzulegen.
Die Adresse wird an \scanf übergeben; diese Funktion speichert den Userinput an der genannten Adresse im Speicher.
Es handelt sich hier um einen 32-Bit-Wert vom Typ \Tint. 
Der Wert wird in Zeile 27 abgeholt und dann an \printf übergeben.



\subsubsection{MIPS}
% FIXME better start at non-optimizing version?
Die Funktion verwendet eine Menge S-Register, die gesichert werden müssen. Das ist der Grund dafür, dass die Werte im
Funktionsprolog gespeichert und im Funktionsepilog wiederhergestellt werden.

\lstinputlisting[caption=\Optimizing GCC 4.4.5
(IDA),style=customasmMIPS]{patterns/13_arrays/1_simple/MIPS_O3_IDA_DE.lst}
Interessant: es gibt zwei Schleifen und die erste benötigt $i$ nicht; sie benötigt nur $i\cdot 2$ (erhöht um 2 bei
jedem Iterationsschritt) und die Adresse im Speicher (erhöht um 4 bei jedem Iterationsschritt).

Wir sehen hier also zwei Variablen: eine (in \$V0), die jedes Mal um 2 erhöht wird, und eine andere (in\$V1), die um 4
erhöht wird.

Die zweite Schleife ist der Ort, an dem \printf aufgerufen wird und dem Benutzer den Wert von $i$ zurückliefert, es gibt
also eine Variable die in \$S0 inkrementiert wird und eine Speicheradresse in \$S1, die jedes Mal um 4 erhöht wird.

% TBT
Das erinnert uns an die Optimierung von Schleifen, die wir früher betrachtet haben: \myref{loop_iterators}.

Das Ziel der Optimierung ist es, die Multiplikationen loszuwerden.


