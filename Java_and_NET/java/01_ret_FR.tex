% TODO proof-reading
\subsection{Renvoyer une valeur}

La fonction Java la plus simple est probablement celle qui renvoie une valeur.

Il faut garder en tête qu'il n'y a pas de fonction \q{libre} en Java au sens commun,
ce sont des \q{méthodes}.

Chaque méthode est liée à une classe, donc il n'est pas possible de définir
une méthode en dehors d'une classe.

Mais nous allons quand même les appeler des \q{fonctions}, par simplicité.

\begin{lstlisting}[style=customjava]
public class ret
{
	public static int main(String[] args) 
	{
		return 0;
	}
}
\end{lstlisting}

Compilons le :

\begin{lstlisting}
javac ret.java
\end{lstlisting}

\dots et décompilons le en utilisant l'outil Java standard :

\begin{lstlisting}
javap -c -verbose ret.class
\end{lstlisting}

Nous obtenons :

\begin{lstlisting}[caption=JDK 1.7 (excerpt)]
  public static int main(java.lang.String[]);
    flags: ACC_PUBLIC, ACC_STATIC
    Code:
      stack=1, locals=1, args_size=1
         0: iconst_0      
         1: ireturn       
\end{lstlisting}

Les développeurs Java ont décidé que comme 0 est l'une des constantes les plus utilisées en programmation,
alors il existe une instruction à part sur un short d'un octet, \GTT{iconst\_0} qui pousse 0.

\footnote{Comme en MIPS où un registre séparé existe pour la constante zéro : \myref{MIPS_zero_register}.}.
Il y a aussi \GTT{iconst\_1} (qui pousse 1), \GTT{iconst\_2}, etc., jusqu'à \GTT{iconst\_5}.

Il existe également l'instruction \GTT{iconst\_m1} qui pousse -1.

La pile est utilisée en JVM pour passer des données à une fonction appelée et également pour renvoyer des valeurs.
Donc \GTT{iconst\_0} pousse 0 sur la pile.
\GTT{ireturn} renvoie une valeur entière (\IT{i} dans le nom signifie \IT{integer}) depuis le \ac{TOS}.

Réécrivons légèrement notre exemple, pour qu'il renvoie 1234 maintenant :

\begin{lstlisting}[style=customjava]
public class ret
{
	public static int main(String[] args)
	{
		return 1234;
	}
}
\end{lstlisting}

\dots nous avons :

\begin{lstlisting}[caption=JDK 1.7 (excerpt)]
  public static int main(java.lang.String[]);
    flags: ACC_PUBLIC, ACC_STATIC
    Code:
      stack=1, locals=1, args_size=1
         0: sipush        1234
         3: ireturn       
\end{lstlisting}

\GTT{sipush} (\IT{short integer}) pousse 1234 sur la pile.
\IT{short} dans le nom implique qu'une valeur de 16-bit va être poussée. 
Le nombre 1234 tiens bien, en effet, dans une valeur de 16-bit.

Qu'en est-il des valeurs plus grandes ?

\begin{lstlisting}[style=customjava]
public class ret
{
	public static int main(String[] args) 
	{
		return 12345678;
	}
}
\end{lstlisting}

\begin{lstlisting}[caption=Constant pool]
...
   #2 = Integer            12345678
...
\end{lstlisting}

\begin{lstlisting}
  public static int main(java.lang.String[]);
    flags: ACC_PUBLIC, ACC_STATIC
    Code:
      stack=1, locals=1, args_size=1
         0: ldc           #2                  // int 12345678
         2: ireturn       
\end{lstlisting}

Ce n'est pas possible d'encoder un nombre 32-bit dans une instruction opcode JVM, 
les développeurs n'ont pas laissé une telle possibilité.

Donc le nombre 32-bit 12345678 est enregistré dans ce qu'on appelle le \q{constant pool} qui est, disons, 
la bibliothèque des constantes les plus utilisées (incluant les strings, objects, etc.).

Cette façon de passer des constantes n'est pas propre à JVM.

MIPS, ARM et les autres CPUs RISC ne peuvent pas non plus encoder un nombre 32-bit 
dans un opcode de 32-bit, donc le code CPU RISC (incluant MIPS et ARM) doit construire la valeur 
en plusieurs, ou le garder dans le segment des données :
\myref{ARM_big_constants}, \myref{MIPS_big_constants}.

Le code MIPS a aussi traditionnellement un pool de constantes, nommé \q{literal pool}, les segments
sont nommés \q{.lit4} (pour des nombres flottants constants de simples précisions sur 32-bit) et \q{.lit8}
(pour des nombres flottants constants de double précision sur 64-bit).

Essayons quelques autres types de données !

Boolean:

\begin{lstlisting}[style=customjava]
public class ret
{
	public static boolean main(String[] args) 
	{
		return true;
	}
}
\end{lstlisting}

\begin{lstlisting}
  public static boolean main(java.lang.String[]);
    flags: ACC_PUBLIC, ACC_STATIC
    Code:
      stack=1, locals=1, args_size=1
         0: iconst_1      
         1: ireturn       
\end{lstlisting}

Ce bytecode JVM n'est pas différent de celui qui retourne l'entier 1.

Les emplacements de données 32-bits dans la pile sont aussi utilisés ici pour des valeurs booléennes, comme en \CCpp.

Mais on ne peut pas utiliser la valeur booléenne retournée comme un entier ou vice versa -- l'information du type est enregistrée dans le fichier de classe et et vérifiée à l'exécution.

C'est la même histoire qu'un \IT{short} 16-bit :

\begin{lstlisting}[style=customjava]
public class ret
{
	public static short main(String[] args) 
	{
		return 1234;
	}
}
\end{lstlisting}

\begin{lstlisting}
  public static short main(java.lang.String[]);
    flags: ACC_PUBLIC, ACC_STATIC
    Code:
      stack=1, locals=1, args_size=1
         0: sipush        1234
         3: ireturn       
\end{lstlisting}

\dots et \IT{char}!

\begin{lstlisting}[style=customjava]
public class ret
{
	public static char main(String[] args) 
	{
		return 'A';
	}
}
\end{lstlisting}

\begin{lstlisting}
  public static char main(java.lang.String[]);
    flags: ACC_PUBLIC, ACC_STATIC
    Code:
      stack=1, locals=1, args_size=1
         0: bipush        65
         2: ireturn       
\end{lstlisting}

\INS{bipush} signifie \q{push byte}.
Inutile de préciser qu'un \IT{char} en Java est un caractère UTF-16 16-bit, 
ce qui équivaut à un \IT{short}, mais le code ASCII du caractère \q{A} est 65, et c'est possible
d'utiliser cette instruction pour pousser un octet dans la pile.

Essayons aussi un \IT{byte}:

\begin{lstlisting}[style=customjava]
public class retc
{
	public static byte main(String[] args) 
	{
		return 123;
	}
}
\end{lstlisting}

\begin{lstlisting}
  public static byte main(java.lang.String[]);
    flags: ACC_PUBLIC, ACC_STATIC
    Code:
      stack=1, locals=1, args_size=1
         0: bipush        123
         2: ireturn       
\end{lstlisting}

On peut se demander, pourquoi s'embêter avec avec un type de donnée \IT{short} de 16-bit qui fonctionne en interne
comme un entier 32-bit ?

Pourquoi utiliser un type de donnée \IT{char} si c'est pareil qu'un type de donnée \IT{short} ?

La réponse est simple : pour le contrôle du type de données et pour la lisibilité du code source.

Un \IT{char} peut être essentiellement le même qu'un \IT{short}, mais nous saisissons rapidement que c'est un substitut pour 
un caractère 16-bit, et non pour une autre valeur entière.

Quand on utilise \IT{short}, nous montrons à tout le monde que la plage de la variable est limitée à 16 bits.

C'est une très bonne idée d'utiliser le type \IT{boolean} où c'est nécessaire, 
plutôt que le \IT{int} de style C.

Il y a aussi un type de donnée entier sur 64-bits en Java :

\begin{lstlisting}[style=customjava]
public class ret3
{
	public static long main(String[] args)
	{
		return 1234567890123456789L;
	}
}
\end{lstlisting}

\begin{lstlisting}[caption=Constant pool]
...
   #2 = Long               1234567890123456789l
...
\end{lstlisting}

\begin{lstlisting}
  public static long main(java.lang.String[]);
    flags: ACC_PUBLIC, ACC_STATIC
    Code:
      stack=2, locals=1, args_size=1
         0: ldc2_w        #2                  // long 1234567890123456789l
         3: lreturn       
\end{lstlisting}

Le nombre 64-bit est aussi stocké dans le pool de constantes, \INS{ldc2\_w} le charge et \INS{lreturn} 
(\IT{long return}) le retourne.

L'instruction \INS{ldc2\_w} est aussi utilisée pour charger des nombres flottants double précision 
(qui occupent aussi 64 bits) depuis le pool de constantes :

\begin{lstlisting}[style=customjava]
public class ret
{
	public static double main(String[] args)
	{
		return 123.456d;
	}
}
\end{lstlisting}

\begin{lstlisting}[caption=Constant pool]
...
   #2 = Double             123.456d
...
\end{lstlisting}

\begin{lstlisting}
  public static double main(java.lang.String[]);
    flags: ACC_PUBLIC, ACC_STATIC
    Code:
      stack=2, locals=1, args_size=1
         0: ldc2_w        #2                  // double 123.456d
         3: dreturn       
\end{lstlisting}

\INS{dreturn} signifie \q{return double}.

Et enfin, un nombre flottant simple précision :

\begin{lstlisting}[style=customjava]
public class ret
{
	public static float main(String[] args)
	{
		return 123.456f;
	}
}
\end{lstlisting}

\begin{lstlisting}[caption=Constant pool]
...
   #2 = Float              123.456f
...
\end{lstlisting}

\begin{lstlisting}
  public static float main(java.lang.String[]);
    flags: ACC_PUBLIC, ACC_STATIC
    Code:
      stack=1, locals=1, args_size=1
         0: ldc           #2                  // float 123.456f
         2: freturn       
\end{lstlisting}

L'instruction \INS{ldc} utilisée ici est la même que celle pour charger des nombres entiers de 32-bit
depuis le pool de constantes.

\INS{freturn} signifie \q{return float}.

Maintenant, qu'en est-il de la fonction qui ne retourne rien ?

\begin{lstlisting}[style=customjava]
public class ret
{
	public static void main(String[] args) 
	{
		return;
	}
}
\end{lstlisting}

\begin{lstlisting}
  public static void main(java.lang.String[]);
    flags: ACC_PUBLIC, ACC_STATIC
    Code:
      stack=0, locals=1, args_size=1
         0: return        
\end{lstlisting}

Cela signifie que l'instruction \INS{return} est utilisée pour retourner le contrôle sans renvoyer
une vraie valeur.

En sachant cela, il est très facile de déduire le type renvoyé par des fonctions (ou des méthodes) depuis la dernière instruction.

