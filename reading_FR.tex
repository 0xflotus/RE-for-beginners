\chapter{Livres/blogs qui valent le détour}

\section{Livres et autres matériels}

\subsection{Rétro-ingénierie}

\begin{itemize}
\item Eldad Eilam, \IT{Reversing: Secrets of Reverse Engineering}, (2005)

\item Bruce Dang, Alexandre Gazet, Elias Bachaalany, Sebastien Josse, \IT{Practical Reverse Engineering: x86, x64, ARM, Windows Kernel, Reversing Tools, and Obfuscation}, (2014)

\item Michael Sikorski, Andrew Honig, \IT{Practical Malware Analysis: The Hands-On Guide to Dissecting Malicious Software}, (2012)

\item Chris Eagle, \IT{IDA Pro Book}, (2011)

\item Reginald Wong, \IT{Mastering Reverse Engineering: Re-engineer your ethical hacking skills}, (2018)

\end{itemize}


Également, les livres de Kris Kaspersky.

\subsection{Windows}

\input{Win_reading}

\subsection{\CCpp}

\input{CCppBooks}

\subsection{Architecture x86 / x86-64}

\label{x86_manuals}
\begin{itemize}
\item Manuels Intel\footnote{\AlsoAvailableAs \url{http://www.intel.com/content/www/us/en/processors/architectures-software-developer-manuals.html}}

\item Manuels AMD\footnote{\AlsoAvailableAs \url{http://developer.amd.com/resources/developer-guides-manuals/}}

\item \AgnerFog{}\footnote{\AlsoAvailableAs \url{http://agner.org/optimize/microarchitecture.pdf}}

\item \AgnerFogCC{}\footnote{\AlsoAvailableAs \url{http://www.agner.org/optimize/calling_conventions.pdf}}

\item \IntelOptimization

\item \AMDOptimization
\end{itemize}

Quelque peu vieux, mais toujours intéressant à lire :

\MAbrash\footnote{\AlsoAvailableAs \url{https://github.com/jagregory/abrash-black-book}}
(il est connu pour son travail sur l'optimisation bas niveau pour des projets tels que Windows NT 3.1 et id Quake).

\subsection{ARM}

\begin{itemize}
\item Manuels ARM\footnote{\AlsoAvailableAs \url{http://infocenter.arm.com/help/index.jsp?topic=/com.arm.doc.subset.architecture.reference/index.html}}

\item \ARMSevenRef

\item \ARMSixFourRefURL

\item \ARMCookBook\footnote{\AlsoAvailableAs \url{http://go.yurichev.com/17273}}
\end{itemize}

\subsection{Langage d'assemblage}

Richard Blum --- Professional Assembly Language.

\subsection{Java}

\JavaBook.

\subsection{UNIX}

\TAOUP

\subsection{Programmation en général}

\begin{itemize}

\item \RobPikePractice

\item \HenryWarren
Certaines personnes disent ques les trucs et astuces de ce livre ne sont plus pertinents
aujourd'hui, car ils n'étaient valables que pour les \ac{CPU}s \ac{RISC}, où les instructions
de branchement sont coûteuses.
Néanmoins, ils peuvent énormément aider à comprendre l'algèbre booléenne et toutes les
mathématiques associées.

\item{(Pour les passionnés avec des connaissances en informatique et mathématiques) Donald E. Knuth, \IT{The Art of Computer Programming}}.

\end{itemize}

%subsection:
\input{crypto_reading}


