\chapter{Books/blogs worth reading}

\section{Books and other materials}

\subsection{Reverse Engineering}

\begin{itemize}
\item Eldad Eilam, \IT{Reversing: Secrets of Reverse Engineering}, (2005)

\item Bruce Dang, Alexandre Gazet, Elias Bachaalany, Sebastien Josse, \IT{Practical Reverse Engineering: x86, x64, ARM, Windows Kernel, Reversing Tools, and Obfuscation}, (2014)

\item Michael Sikorski, Andrew Honig, \IT{Practical Malware Analysis: The Hands-On Guide to Dissecting Malicious Software}, (2012)

\item Chris Eagle, \IT{IDA Pro Book}, (2011)

\item Reginald Wong, \IT{Mastering Reverse Engineering: Re-engineer your ethical hacking skills}, (2018)

\end{itemize}


Also, Kris Kaspersky's books.

\subsection{Windows}

\input{Win_reading}

\subsection{\CCpp}

\input{CCppBooks}

\subsection{x86 / x86-64}

\label{x86_manuals}
\begin{itemize}
\item Intel manuals\footnote{\AlsoAvailableAs \url{http://www.intel.com/content/www/us/en/processors/architectures-software-developer-manuals.html}}

\item AMD manuals\footnote{\AlsoAvailableAs \url{http://developer.amd.com/resources/developer-guides-manuals/}}

\item \AgnerFog{}\footnote{\AlsoAvailableAs \url{http://agner.org/optimize/microarchitecture.pdf}}

\item \AgnerFogCC{}\footnote{\AlsoAvailableAs \url{http://www.agner.org/optimize/calling_conventions.pdf}}

\item \IntelOptimization

\item \AMDOptimization
\end{itemize}

Somewhat outdated, but still interesting to read:

\MAbrash\footnote{\AlsoAvailableAs \url{https://github.com/jagregory/abrash-black-book}}
(he is known for his work on low-level optimization for such projects as Windows NT 3.1 and id Quake).

\subsection{ARM}

\begin{itemize}
\item ARM manuals\footnote{\AlsoAvailableAs \url{http://infocenter.arm.com/help/index.jsp?topic=/com.arm.doc.subset.architecture.reference/index.html}}

\item \ARMSevenRef

\item \ARMSixFourRefURL

\item \ARMCookBook\footnote{\AlsoAvailableAs \url{http://go.yurichev.com/17273}}
\end{itemize}

\subsection{Assembly language}

Richard Blum --- Professional Assembly Language.

\subsection{Java}

\JavaBook.

\subsection{UNIX}

\TAOUP

\subsection{Programming in general}

\begin{itemize}

\item \RobPikePractice

\item \HenryWarren.
Some people say tricks and hacks from the book are not relevant today because they were good only for \ac{RISC} \ac{CPU}s,
where branching instructions are expensive.
Nevertheless, these can help immensely to understand boolean algebra and what all the mathematics near it.

\item (For hard-core geeks with computer science and mathematical background) Donald E. Knuth, \IT{The Art of Computer Programming}.

\end{itemize}

% subsection:
\input{crypto_reading}

