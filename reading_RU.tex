% TODO sync with English version
\chapter{Что стоит почитать}

\mysection{Книги и прочие материалы}

\subsection{Reverse Engineering}

Также, книги Криса Касперски.

Дмитрий Скляров --- ``Искусство защиты и взлома информации''.

\begin{itemize}
\item Eldad Eilam, \IT{Reversing: Secrets of Reverse Engineering}, (2005)

\item Bruce Dang, Alexandre Gazet, Elias Bachaalany, Sebastien Josse, \IT{Practical Reverse Engineering: x86, x64, ARM, Windows Kernel, Reversing Tools, and Obfuscation}, (2014)

\item Michael Sikorski, Andrew Honig, \IT{Practical Malware Analysis: The Hands-On Guide to Dissecting Malicious Software}, (2012)

\item Chris Eagle, \IT{IDA Pro Book}, (2011)

\item Reginald Wong, \IT{Mastering Reverse Engineering: Re-engineer your ethical hacking skills}, (2018)

\end{itemize}


\subsection{Windows}

\input{Win_reading}

\subsection{\CCpp}

\input{CCppBooks}

\subsection{x86 / x86-64}

\label{x86_manuals}
\begin{itemize}
\item Документация от Intel\footnote{\AlsoAvailableAs \url{http://www.intel.com/content/www/us/en/processors/architectures-software-developer-manuals.html}}

\item Документация от AMD\footnote{\AlsoAvailableAs \url{http://developer.amd.com/resources/developer-guides-manuals/}}

\item \AgnerFog{}\footnote{\AlsoAvailableAs \url{http://agner.org/optimize/microarchitecture.pdf}}

\item \AgnerFogCC{}\footnote{\AlsoAvailableAs \url{http://www.agner.org/optimize/calling_conventions.pdf}}

\item \IntelOptimization

\item \AMDOptimization
\end{itemize}

Немного устарело, но всё равно интересно почитать:

\MAbrash\footnote{\AlsoAvailableAs \url{https://github.com/jagregory/abrash-black-book}}
(он известен своей работой над низкоуровневой оптимизацией в таких проектах как Windows NT 3.1 и id Quake).

\subsection{ARM}

\begin{itemize}
\item Документация от ARM\footnote{\AlsoAvailableAs \url{http://infocenter.arm.com/help/index.jsp?topic=/com.arm.doc.subset.architecture.reference/index.html}}

\item \ARMSevenRef

\item \ARMSixFourRefURL

\item \ARMCookBook\footnote{\AlsoAvailableAs \url{http://go.yurichev.com/17273}}
\end{itemize}

\subsection{Язык ассемблера}

Richard Blum --- Professional Assembly Language.

\subsection{Java}

\JavaBook.

\subsection{UNIX}

\TAOUP

\subsection{Программирование}

\begin{itemize}

\item \RobPikePractice

\item Александр Шень\footnote{\url{http://imperium.lenin.ru/~verbit/Shen.dir/shen-progra.html}}

\item \HenryWarren.
Некоторые люди говорят, что трюки и хаки из этой книги уже не нужны, потому что годились только для \ac{RISC}-процессоров,
где инструкции перехода слишком дорогие.
Тем не менее, всё это здорово помогает лучше понять булеву алгебру и всю математику рядом.

\item (Для хард-корных гиков от информатики и математики) Дональд Кнут, \IT{Искусство программирования}.

\end{itemize}

% subsection:
\input{crypto_reading}
