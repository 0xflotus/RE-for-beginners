\subsubsection{\WorkingWithFloatAsWithStructSubSubSectionName}
\label{sec:floatasstruct}

FPUについてのセクション(\myref{sec:FPU})ですでに述べたように、
\Tfloat と \Tdouble の両方は、\IT{符号}、
\IT{仮数部}(または\IT{小数部})、\IT{指数部}で構成されます。 
しかし、これらのフィールドを直接編集することはでるでしょうか? \Tfloat で試してみましょう。

\input{float_IEEE754.tex}

\lstinputlisting[style=customc]{patterns/15_structs/6_bitfields/float/float_JPN.c}

\TT{float\_as\_struct}構造体は、 \Tfloat と同じ量のメモリ、すなわち4バイトまたは32ビットを占有します。

ここで、入力値に負の符号を設定し、指数に2を加えることによって、
整数を\TT{$2^2$}、すなわち4で乗算します。

最適化をオンにしないでMSVC 2008でコンパイルしましょう:

\lstinputlisting[caption=\NonOptimizing MSVC 2008,style=customasmx86]{patterns/15_structs/6_bitfields/float/float_msvc_JPN.asm}

少し冗長です。
\Ox フラグを付けてコンパイルした場合、\TT{memcpy()}呼び出しはなく、
\TT{f}変数が直接使用されます。 
しかし、最適化されていないバージョンを見れば分かりやすくなります。

GCC 4.4.1を \Othree つきで実行したらどうなりますか?

\lstinputlisting[caption=\Optimizing GCC 4.4.1,style=customasmx86]{patterns/15_structs/6_bitfields/float/float_gcc_O3_JPN.asm}

\ttf 関数はほぼ理解できます。 しかし、興味深いのは、GCCが構造体フィールドを持つこのような問題にかかわらず、
コンパイル時に\TT{f(1.234)}の結果を計算でき、
コンパイル時に事前計算されて \printf にこの引数を用意したことです。
