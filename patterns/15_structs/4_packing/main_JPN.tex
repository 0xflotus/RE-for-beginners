\subsection{\StructurePackingSectionName}
\label{structure_packing}

1つ重要なことは、構造内のフィールドのパッキングです\footnote{参照: \URLWPDA}。

簡単な例を考えてみましょう:

\lstinputlisting[style=customc]{patterns/15_structs/4_packing/packing.c}

見てきたように、2つの \Tchar フィールド(それぞれ1バイト)と2つの \Tint(それぞれ4バイト)があります。

% subsections:
\subsubsection{x86}

このようにコンパイルされます。

\lstinputlisting[caption=MSVC 2012 /GS- /Ob0,label=src:struct_packing_4,numbers=left,style=customasmx86]{patterns/15_structs/4_packing/packing_JPN.asm}

構造全体を渡しますが、実際には、構造体は
一時的な領域にコピーされて、(スタック内の領域は10行目に割り当てられ、
次に4つのフィールドはすべて1つずつ、12行目から19行目にコピーされます)
そのポインタ(アドレス)が渡されます。

\ttf{} 関数が構造体を変更するかどうかわからないため、
構造体がコピーされます。 
それが変更された場合、 \main の構造体はそのままでいなければなりません。

私たちは \CCpp ポインタを使うことができました。結果のコードはほぼ同じですが、
コピーは行いません。

次に見るように、各フィールドのアドレスは4バイトの境界に揃えられています。 
だからこそ、各 \Tchar が( \Tint のように)4バイトを占めるのです。なぜでしょうか? 
CPUが整列したアドレスでメモリにアクセスし、メモリからデータをキャッシュする方が簡単であるためです。

しかし、あまり経済的ではありません。

オプション(\TT{/Zp1})(nバイト境界で構造体をパックする \IT{/Zp[n]})で
コンパイルしてみましょう。

\lstinputlisting[caption=MSVC 2012 /GS- /Zp1,label=src:struct_packing_1,numbers=left,style=customasmx86]{patterns/15_structs/4_packing/packing_msvc_Zp1_JPN.asm}

Now the structure takes only 10 bytes and each \Tchar value takes 1 byte. What does it give to us?
Size economy. And as drawback~---the CPU accessing these fields slower than it could.

構造体は10バイトしかなく、各 \Tchar 値は1バイト必要です。それは私たちに何を与えるのですか?
サイズ経済。そして欠点として、CPUはこれらのフィールドにアクセスするのが遅くなります。

\label{short_struct_copying_using_MOV}

構造体も \main にコピーされます。フィールド単位ではなく、3つの \MOV ペアを使用して直接10バイトをコピーします。
なぜ4ではないのでしょうか?

コンパイラは、3つの \MOV ペアを使用して10バイトをコピーする方が、2つの32ビットワードと
4つの \MOV ペアを使用して2バイトをコピーするよりも優れていると判断しました。

ちなみに、\TT{memcpy()}関数を呼び出す代わりに \MOV を使用するようなコピーの実装は、
\TT{memcpy()}の呼び出しよりも速いため、広く使用されています。
\myref{copying_short_blocks}

簡単に推測できるように、構造体が多くのソースファイルとオブジェクトファイルで使用されている場合、
構造体パッキングについてはすべて同じ規則でコンパイルする必要があります。

\newcommand{\FNURLMSDNZP}{\footnote{\href{http://go.yurichev.com/17067}
{MSDN: Working with Packing Structures}}}
\newcommand{\FNURLGCCPC}{\footnote{\href{http://go.yurichev.com/17068}
{Structure-Packing Pragmas}}}

各構造体フィールドの配置方法を設定するMSVC \TT{/Zp}オプションの他に、
\TT{\#pragma pack}コンパイラオプションもあります。このオプションはソースコード内で直接定義できます。 
MSVC\FNURLMSDNZP と GCC\FNURLGCCPC{} の両方で利用できます。

16ビットのフィールドで構成される\TT{SYSTEMTIME}構造体に戻りましょう。
私たちのコンパイラは、1バイト境界でパックすることをどうやって知っていますか?

\TT{WinNT.h}ファイルはこれを持っています:

\begin{lstlisting}[caption=WinNT.h,style=customc]
#include "pshpack1.h"
\end{lstlisting}

そしてこれを。

\begin{lstlisting}[caption=WinNT.h,style=customc]
#include "pshpack4.h"                   // 4バイトパッキングがデフォルト
\end{lstlisting}

PshPack1.h ファイルはこのようになっています。

\begin{lstlisting}[caption=PshPack1.h,style=customc]
#if ! (defined(lint) || defined(RC_INVOKED))
#if ( _MSC_VER >= 800 && !defined(_M_I86)) || defined(_PUSHPOP_SUPPORTED)
#pragma warning(disable:4103)
#if !(defined( MIDL_PASS )) || defined( __midl )
#pragma pack(push,1)
#else
#pragma pack(1)
#endif
#else
#pragma pack(1)
#endif
#endif /* ! (defined(lint) || defined(RC_INVOKED)) */
\end{lstlisting}

コンパイラは \TT{\#pragma pack} の後で定義される構造体をパックする方法を知らせます。

\clearpage
\myparagraph{\olly フィールドはデフォルトでパックされる}
\myindex{\olly}

\olly で我々の例を(フィールドがデフォルト(4バイト)で整列される)試してみましょう。

\begin{figure}[H]
\centering
\myincludegraphics{patterns/15_structs/4_packing/olly_packing_4.png}
\caption{\olly: \printf が実行される前}
\label{fig:packing_olly_4}
\end{figure}

データウィンドウに4つフィールドが見えます。

しかし、ランダム値(0x30, 0x37, 0x01)はどこから来だのでしょう、最初の(a)と3番目のフィールド(c)の次でしょうか。

私たちの \myref{src:struct_packing_4} のリストを見ると、最初のフィールドと3番目のフィールドが
\Tchar だと分かります。したがって、それぞれ1と3(6行目と8行目)が書かれています。

32ビットワードの残りの3バイトはメモリ内で変更されていません! 
したがって、ランダムなごみが残っています。
\myindex{x86!\Instructions!MOVSX}

このゴミは \printf の出力には何の影響も与えません。その値は、 \MOVSX 命令を使用して準備されるため、
ワードではなくバイトを使用します。
\lstref{src:struct_packing_4} (34行目と38行目)

ちなみに、 \Tchar はMSVCとGCCでデフォルトでは符号ありなので、
\MOVSX (符号拡張)命令がここで使用されます。 
\TT{符号なしのchar}データ型または\TT{uint8\_t}がここで使用された場合、
代わりに \MOVZX 命令が使用されます。

\clearpage
\myparagraph{\olly 1バイト境界でアラインメントされるフィールド}
\myindex{\olly}

ここでははっきりしています。4フィールドは10バイトを占め、値は並べて保存されます。

\begin{figure}[H]
\centering
\myincludegraphics{patterns/15_structs/4_packing/olly_packing_1.png}
\caption{\olly: \printf が実行される前}
\label{fig:packing_olly_1}
\end{figure}


\subsubsection{ARM}

\myparagraph{\OptimizingKeilVI (\ThumbMode)}

\lstinputlisting[caption=\OptimizingKeilVI (\ThumbMode),style=customasmARM]{patterns/15_structs/4_packing/packing_Keil_thumb.asm}

思い出されるように、ここでは1つのポインタの代わりに構造体が渡されます。
ARMの最初の4つの関数引数はレジスタを介して渡されるので、
構造体のフィールドは\TT{R0-R3}を介して渡されます。

\myindex{ARM!\Instructions!LDRB}
\myindex{x86!\Instructions!MOVSX}
\TT{LDRB}はメモリから1バイトをロードし、その符号を考慮して32ビットに拡張します。 
これはx86の \MOVSX に似ています。 
ここでは、構造体からフィールド $a$ および $c$ をロードするために使用されます。

\myindex{Function epilogue}

私たちが簡単に見つけたもう1つのことは、関数エピローグの代わりに、別の関数のエピローグにジャンプすることです! 
確かに、これはまったく異なる機能であり、私たちとは何ら関係がありませんでしたが、まったく同じ
エピローグを持っています
(おそらく、ローカル変数を5つ含んでいるからです
($5*4=0x14$))。

また近くに位置しています(アドレスを見てください)。

実際、私たちが必要としているように動作すれば、
どのエピローグが実行されるかは問題ではありません。

どうやら、Keilは別の関数の一部を再利用して節約するように決めているようです。

エピローグはジャンプに4バイト取ります。

\myparagraph{ARM + \OptimizingXcodeIV (\ThumbTwoMode)}

\lstinputlisting[caption=\OptimizingXcodeIV (\ThumbTwoMode),style=customasmARM]{patterns/15_structs/4_packing/packing_Xcode_thumb.asm}

\myindex{ARM!\Instructions!SXTB}
\myindex{x86!\Instructions!MOVSX}
\TT{SXTB} (\IT{Signed Extend Byte}) はx86の \MOVSX に似ています。
残りの部分はすべて同じです。

\subsubsection{MIPS}
\label{MIPS_structure_big_endian}

\lstinputlisting[caption=\Optimizing GCC 4.4.5 (IDA),numbers=left,style=customasmMIPS]{patterns/15_structs/4_packing/MIPS_O3_IDA_JPN.lst}

構造体フィールドはレジスタ \$A0..\$A3 に入ってから \printf のために \$A1..\$A3 に再整理され、
4番目のフィールド(\$A3 から)は\INS{SW}を使ってローカルスタックを経由して渡されます。

しかし、2つのSRA(\q{Shift Word Right Arithmetic})命令があり、これは \Tchar フィールドを準備します。 
なぜでしょうか?

MIPSはデフォルトではビッグエンディアンアーキテクチャです\myref{sec:endianness}。私たちが動かすDebian Linuxもビッグエンディアンです。

したがって、バイト変数が32ビット構造のスロットに格納されるとき、それらは高位31~24ビットを占有します。

また、\Tchar 変数を32ビット値に拡張する必要がある場合は、それを24ビット右にシフトする必要があります。

\Tchar は符号付き型なので、ここでは論理シフトの代わりに算術シフトが使用されます。


\subsubsection{もう一言}

関数の引数として構造体を渡すのは(構造体へのポインタを渡すのではなく)構造体のフィールドを
1つ1つ渡すのと同じです。

構造体のフィールドがデフォルトでパックされる場合、f()関数は以下のように書き換える可能です。

\begin{lstlisting}[style=customc]
void f(char a, int b, char c, int d)
{
    printf ("a=%d; b=%d; c=%d; d=%d\n", a, b, c, d);
};
\end{lstlisting}

そして同じコードになります。
