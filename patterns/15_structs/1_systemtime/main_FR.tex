\subsection{MSVC: exemple SYSTEMTIME}
\label{sec:SYSTEMTIME}

\newcommand{\FNSYSTEMTIME}{\footnote{\href{http://go.yurichev.com/17260}{MSDN: SYSTEMTIME structure}}}

Considérons la structure win32 SYSTEMTIME\FNSYSTEMTIME{} qui décrit un instant dans le temps. Voici
comment elle est définie:

\begin{lstlisting}[caption=WinBase.h,style=customc]
typedef struct _SYSTEMTIME {
  WORD wYear;
  WORD wMonth;
  WORD wDayOfWeek;
  WORD wDay;
  WORD wHour;
  WORD wMinute;
  WORD wSecond;
  WORD wMilliseconds;
} SYSTEMTIME, *PSYSTEMTIME;
\end{lstlisting}

Écrivons une fonction C pour récupérer l'instant qu'il est:

\lstinputlisting[style=customc]{patterns/15_structs/1_systemtime/systemtime.c}

Le résultat de la compilation avec MSVC 2010 donne:

\lstinputlisting[caption=MSVC 2010 /GS-,style=customasmx86]{patterns/15_structs/1_systemtime/systemtime.asm}

16 octets sont réservés sur la pile pour cette structure, ce qui correspond exactement à
\TT{sizeof(WORD)*8}. La structure comprend effectivement 8 variables d'un WORD chacun.

\newcommand{\FNMSDNGST}{\footnote{\href{http://go.yurichev.com/17261}{MSDN: GetSystemTime function}}}

Faites attention au fait que le premier membre de la structure est le champ \TT{wYear}.
On peut donc considérer que la fonction \TT{GetSystemTime()}\FNSYSTEMTIME reçoit comme argument
un pointeur sur la structure SYSTEMTIME, ou bien qu'elle reçoit un pointeur sur le champ \TT{wYear}.
Et en fait c'est exactement la même chose!
\TT{GetSystemTime()} écrit l'année courante dans à l'adresse du WORD qu'il a reçu, avance de 2
octets, écrit le mois courant et ainsi de suite.

\clearpage
\myparagraph{\olly et les champs alignés par défaut}
\myindex{\olly}

Examinons dans \olly notre exemple lorsque les champs sont alignés par défaut sur des frontières de 4 octets:

\begin{figure}[H]
\centering
\myincludegraphics{patterns/15_structs/4_packing/olly_packing_4.png}
\caption{\olly: Before \printf execution}
\label{fig:packing_olly_4}
\end{figure}

Nous voyons nos quatre champs dans la fenêtre de données.

Mais d'où viennent ces octets aléatoires (0x30, 0x37, 0x01) situé à côté des premier (a) et troisième (c)
champs ?

Si nous revenons à notre listing \myref{src:struct_packing_4}, nous constatons que ces deux champs sont de
type \Tchar. Seul un octet est écrit pour chacun d'eux: 1 et 3 respectivement (lignes 6 et 8).

Les trois autres octets des deux mots de 32 bits ne sont pas modifiés en mémoire! Des débris aléatoires des 
précédentes opérations demeurent donc là.

\myindex{x86!\Instructions!MOVSX}

Ces débris n'influencent nullement le résultat de la fonction \printf parce que les valeurs qui lui sont 
passées sont préparés avec l'instruction \MOVSX qui opère sur des octets et non pas sur des mots: 
\lstref{src:struct_packing_4} (lignes 34 et 38).

L'instruction \MOVSX (extension de signe) est utilisée ici car le type \Tchar est par défaut une valeur 
signée pour MSVC et GCC. Si l'un des types \TT{unsigned char} ou \TT{uint8\_t} était utilisé ici, ce serait 
l'instruction \MOVZX que le compilateur aurait choisi.

\clearpage
\myparagraph{\olly et les champs alignés sur des frontières de 1 octet}
\myindex{\olly}

Les choses sont beaucoup plus simples ici. Les 4 champs occupent 10 octets et les valeurs sont stockées 
côte-à-côte.

\begin{figure}[H]
\centering
\myincludegraphics{patterns/15_structs/4_packing/olly_packing_1.png}
\caption{\olly: Avant appel de la fonction \printf}
\label{fig:packing_olly_1}
\end{figure}


\subsubsection{Remplacer la structure par un tableau}

Le fait que les champs d'une structure ne sont que des variables situées côte-à-côte peut être
aisément démontré de la manière suivante.
Tout en conservant à l'esprit la description de la structure \TT{SYSTEMTIME}, il est possible de
réécrire cet exemple simple de la manière suivante:

\lstinputlisting[style=customc]{patterns/15_structs/1_systemtime/systemtime2.c}

Le compilateur ronchonne certes un peu:

\begin{lstlisting}
systemtime2.c(7) : warning C4133: 'function' : incompatible types - from 'WORD [8]' to 'LPSYSTEMTIME'
\end{lstlisting}

Mais, il consent quand même à produire le code suivant:

\lstinputlisting[caption=\NonOptimizing MSVC 2010,style=customasmx86]{patterns/15_structs/1_systemtime/systemtime2.asm}

Qui fonctionne à l'identique du précédent!

Il est extrêmement intéressant de constater que le code assembleur produit est impossible à
distinguer de celui produit par la compilation précédente.

Et ainsi celui qui observe ce code assembleur est incapable de décider avec certitude si une
structure ou un tableau était déclaré dans le code source en C.

Cela étant, aucun esprit sain ne s'amuserait à déclarer un tableau ici. Car il faut aussi compter
avec la possibilité que la structure soit modifiée par les développeurs, que les champs soient
triés dans un autre ordre ...

Nous n'étudierons pas cet exemple avec \olly, car les résultats seraient identiques à ceux que nous
avons observé en utilisant la structure.

