\subsection{Charger une constante dans un registre}
\label{ARM_big_constants}

\subsubsection{ARM 32-bit}
\label{ARM_big_constants_loading}

Comme nous le savons déjà, toutes les instructions ont une taille de 4 octets en
mode ARM et de 2 octets en mode Thumb.

Mais comment peut-on charger une valeur 32-bit dans un registre, s'il n'est pas possible
de l'encoder dans une instruction?

Essayons:

\begin{lstlisting}[style=customc]
unsigned int f()
{
	return 0x12345678;
};
\end{lstlisting}

\begin{lstlisting}[caption=GCC 4.6.3 -O3 \ARMMode,style=customasmARM]
f:
        ldr     r0, .L2
        bx      lr
.L2:
        .word   305419896 ; 0x12345678
\end{lstlisting}

Donc, la valeur \TT{0x12345678} est simplement stockée à part en mémoire et chargée
si besoin.

Mais il est possible de se débarrasser de l'accès supplémentaire en mémoire.

\begin{lstlisting}[caption=GCC 4.6.3 -O3 -march{=}armv7-a (\ARMMode),style=customasmARM]
movw    r0, #22136      ; 0x5678
movt    r0, #4660       ; 0x1234
bx      lr
\end{lstlisting}

Nous voyons que la valeur est chargée dans le registre par parties, la partie basse
en premier (en utilisant \INS{MOVW}), puis la partie haute (en utilisant \INS{MOVT}).

Ceci implique que 2 instructions sont nécessaires en mode ARM pour charger une valeur
32-bit dans un registre.

Ce n'est pas un problème, car en fait il n'y pas beaucoup de constantes dans du code
réel (excepté pour 0 et 1).

Est-ce que ça signifie que la version à deux instructions est plus lente que celle
à une instruction?

C'est discutable. Le plus souvent, les processeurs ARM modernes sont capable de détecter
de telle séquences et les exécutent rapidement.

D'un autre côté, \IDA est capable de détecter ce genre de patterns dans le code et
désassemble cette fonction comme:

\begin{lstlisting}[style=customasmARM]
MOV    R0, 0x12345678
BX     LR
\end{lstlisting}

\subsubsection{ARM64}

\begin{lstlisting}[style=customc]
uint64_t f()
{
	return 0x12345678ABCDEF01;
};
\end{lstlisting}

\begin{lstlisting}[caption=GCC 4.9.1 -O3,style=customasmARM]
mov	x0, 61185   ; 0xef01
movk	x0, 0xabcd, lsl 16
movk	x0, 0x5678, lsl 32
movk	x0, 0x1234, lsl 48
ret
\end{lstlisting}

\myindex{ARM!\Instructions!MOVK}
\TT{MOVK} signifie \q{MOV Keep} (déplacer garder), i.e., elle écrit une valeur
16-bit dans le registre, sans affecter le reste des bits.
\myindex{ARM!Optional operators!LSL}
Le suffixe \TT{LSL} signifie décaler la valeur à gauche de 16, 32 et 48 bits à chaque
étape. Le décalage est fait avant le chargement.

Ceci implique que 4 instructions sont nécessaires pour charger une valeur de 64-bit
dans un registre.

\myparagraph{Charger un nombre à virgule flottante dans un registre}

Il est possible de stocker un nombre à virgule flottante dans un D-registre en utilisant
une seule instruction.

Par exemple:

\begin{lstlisting}[style=customc]
double a()
{
	return 1.5;
};
\end{lstlisting}

\begin{lstlisting}[caption=GCC 4.9.1 -O3 + objdump,style=customasmARM]
0000000000000000 <a>:
   0:   1e6f1000        fmov    d0, #1.500000000000000000e+000
   4:   d65f03c0        ret
\end{lstlisting}

Le nombre $1.5$ a en effet été encodé dans une instruction 32-bit.
Mais comment?
\myindex{ARM!\Instructions!FMOV}

En ARM64, il y a 8 bits dans l'instruction \TT{FMOV} pour encoder certains nombres
à virgule flottante.

L'algorithme est appelé \TT{VFPExpandImm()} en \ARMSixFourRefURL.
\myindex{minifloat}
Ceci est aussi appelé \IT{minifloat}\footnote{\href{http://go.yurichev.com/17139}{Wikipédia}}
(mini flottant).

Nous pouvons essayer différentes valeurs. Le compilateur est capable d'encoder $30.0$
et $31.0$, mais il ne peut pas encoder $32.0$, car 8 octets doivent être alloués
pour ce nombre au format IEEE 754:

\begin{lstlisting}[style=customc]
double a()
{
	return 32;
};
\end{lstlisting}

\begin{lstlisting}[caption=GCC 4.9.1 -O3,style=customasmARM]
a:
	ldr	d0, .LC0
	ret
.LC0:
	.word	0
	.word	1077936128
\end{lstlisting}
