\subsection{メモリブロックコピールーチン}
\label{loop_memcpy}

実世界のメモリコピールーチンは、反復ごとに4バイトまたは8バイトをコピーし、\ac{SIMD}、
ベクトル化などを使用します。

しかし、話を簡単にするために、この例は可能な限り簡単にしています。

\lstinputlisting[style=customc]{memcpy.c}

\subsubsection{簡単な実装}

\lstinputlisting[caption=GCC 4.9 x64 optimized for size (-Os),style=customasmx86]{patterns/09_loops/memcpy/memcpy_GCC49_x64_Os_JPN.s}

\lstinputlisting[caption=GCC 4.9 ARM64 optimized for size (-Os),style=customasmARM]{patterns/09_loops/memcpy/memcpy_GCC49_ARM64_Os_JPN.s}

\lstinputlisting[caption=\OptimizingKeilVI (\ThumbMode),style=customasmARM]{patterns/09_loops/memcpy/memcpy_Keil_Thumb_O3_JPN.s}

\subsubsection{ARM in ARM mode}

ARMモードのKeilは、条件付き接尾辞を最大限に活用しています。

\lstinputlisting[caption=\OptimizingKeilVI (\ARMMode),style=customasmARM]{patterns/09_loops/memcpy/memcpy_Keil_ARM_O3_JPN.s}

そのため、2ではなく1つの分岐命令しか存在しません。

\subsubsection{MIPS}

\lstinputlisting[caption=GCC 4.4.5 optimized for size (-Os) (IDA),style=customasmMIPS]{patterns/09_loops/memcpy/memcpy_MIPS_Os_IDA_JPN.lst}

\myindex{MIPS!\Instructions!LBU}
\myindex{MIPS!\Instructions!SB}

ここでは、2つの新しい命令があります:\INS{LBU}(\q{Load Byte Unsigned})と\INS{SB}(\q{Store Byte})

ARMと同様に、MIPSレジスタはすべて32ビット幅であり、x86などの1バイト幅のレジスタ部分はありません。

したがって、1バイトを処理する場合は、32ビットのレジスタ全体を割り当てる必要があります。

\INS{LBU}は1バイトをロードし、他のすべてのビット(\q{Unsigned})をクリアします。
\myindex{MIPS!\Instructions!LB}

一方、\INS{LB}(\q{Load Byte})命令は、ロードされたバイトを32ビット値に符号拡張します。

\INS{SB}は、レジスタの最下位8ビットから1バイトをメモリに書き込むだけです。

\subsubsection{ベクトル化}

\Optimizing GCCはこの例でもっと多くのことを行うことができます: \myref{vec_memcpy}.
