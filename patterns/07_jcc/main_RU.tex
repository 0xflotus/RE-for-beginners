\mysection{Условные переходы}
\label{sec:Jcc}
\myindex{\CLanguageElements!if}

% sections
\subsection{\RU{Простой пример}\EN{Simple example}\DE{einfaches Beispiel}
\FR{Exemple simple}\ITA{Esempio semplice}\JPN{シンプルな例}
}

\lstinputlisting[style=customc]{patterns/07_jcc/simple/ex.c}

% subsections
\EN{\subsection{x86}

\subsubsection{MSVC}

Here is what we get after compilation (MSVC 2010 Express):

\lstinputlisting[label=src:passing_arguments_ex_MSVC_cdecl,caption=MSVC 2010 Express,style=customasmx86]{patterns/05_passing_arguments/msvc_EN.asm}

\myindex{x86!\Registers!EBP}

What we see is that the \main function pushes 3 numbers onto the stack and calls \TT{f(int,int,int).} 

Argument access inside \ttf is organized with the help of macros like:\\
\TT{\_a\$ = 8}, 
in the same way as local variables, but with positive offsets (addressed with \IT{plus}).
So, we are addressing the \IT{outer} side of the \gls{stack frame} by adding the \TT{\_a\$} macro to the value in the \EBP register.

\myindex{x86!\Instructions!IMUL}
\myindex{x86!\Instructions!ADD}

Then the value of $a$ is stored into \EAX. After \IMUL instruction execution, the value in \EAX is 
a \gls{product} of the value in \EAX and the content of \TT{\_b}.

After that, \ADD adds the value in \TT{\_c} to \EAX.

The value in \EAX does not need to be moved: it is already where it must be.
On returning to \gls{caller}, it takes the \EAX value and uses it as an argument to \printf.

\input{patterns/05_passing_arguments/olly_EN}

\subsubsection{GCC}

Let's compile the same in GCC 4.4.1 and see the results in \IDA:

\lstinputlisting[caption=GCC 4.4.1,style=customasmx86]{patterns/05_passing_arguments/gcc_EN.asm}

The result is almost the same with some minor differences discussed earlier.

The \gls{stack pointer} is not set back after the two function calls(f and printf), 
because the penultimate \TT{LEAVE} (\myref{x86_ins:LEAVE}) 
instruction takes care of this at the end.
}
\RU{\subsection{x86}

\subsubsection{MSVC}

Рассмотрим пример, скомпилированный в (MSVC 2010 Express):

\lstinputlisting[label=src:passing_arguments_ex_MSVC_cdecl,caption=MSVC 2010 Express,style=customasmx86]{patterns/05_passing_arguments/msvc_RU.asm}

\myindex{x86!\Registers!EBP}
Итак, здесь видно: в функции \main заталкиваются три числа в стек и вызывается функция \TT{f(int,int,int)}.
 
Внутри \ttf доступ к аргументам, также как и к локальным переменным, происходит через макросы: 
\TT{\_a\$ = 8}, но разница в том, что эти смещения со знаком \IT{плюс}, 
таким образом если прибавить макрос \TT{\_a\$} к указателю на \EBP, то адресуется \IT{внешняя} 
часть \glslink{stack frame}{фрейма} стека относительно \EBP.

\myindex{x86!\Instructions!IMUL}
\myindex{x86!\Instructions!ADD}
Далее всё более-менее просто: значение $a$ помещается в \EAX. 
Далее \EAX умножается при помощи инструкции \IMUL на то, что лежит в \TT{\_b}, 
и в \EAX остается \glslink{product}{произведение} этих двух значений.

Далее к регистру \EAX прибавляется то, что лежит в \TT{\_c}.

Значение из \EAX никуда не нужно перекладывать, оно уже лежит где надо. 
Возвращаем управление вызывающей функции~--- она возьмет значение из \EAX и отправит его в \printf.

\input{patterns/05_passing_arguments/olly_RU}

\subsubsection{GCC}

Скомпилируем то же в GCC 4.4.1 и посмотрим результат в \IDA:

\lstinputlisting[caption=GCC 4.4.1,style=customasmx86]{patterns/05_passing_arguments/gcc_RU.asm}

Практически то же самое, если не считать мелких отличий описанных ранее.

После вызова обоих функций \glslink{stack pointer}{указатель стека} не возвращается назад, 
потому что предпоследняя инструкция \TT{LEAVE} (\myref{x86_ins:LEAVE}) делает это за один раз, в конце исполнения.

}
\DE{\subsubsection{x86}

\myparagraph{MSVC}

Kompilieren wir das Beispiel:

\lstinputlisting[caption=MSVC 2008,style=customasmx86]{patterns/13_arrays/1_simple/simple_msvc.asm}

\myindex{x86!\Instructions!SHL}
Soweit nichts Außergewöhnliches, nur zwei Schleifen: die erste füllt mit Werten auf und die zweite gibt Werte aus.
% TBT
Der Befehl \TT{shl ecx, 1} wird für die Multiplikation mit 2 in \ECX verwendet; mehr dazu unten~\myref{SHR}.

Auf dem Stack werden 80 Bytes für das Array reserviert: 20 Elemente von je 4 Byte.

\clearpage
Untersuchen wir dieses Beispiel in \olly.
\myindex{\olly}

Wir erkennen wie das Array befüllt wird:

jedes Element ist ein 32-Bit-Wort vom Typ \Tint und der Wert ist der Index multipliziert mit 2:

\begin{figure}[H]
\centering
\myincludegraphics{patterns/13_arrays/1_simple/olly.png}
\caption{\olly: nach dem Füllen des Arrays}
\label{fig:array_simple_olly}
\end{figure}
Da sich dieses Array auf dem Stack befindet, finden wir dort alle seine 20 Elemente.

\myparagraph{GCC}

Hier ist was GCC 4.4.1 erzeugt:

\lstinputlisting[caption=GCC 4.4.1,style=customasmx86]{patterns/13_arrays/1_simple/simple_gcc.asm}
Die Variable $a$ ist übrigens vom Typ \IT{int*} (Pointer auf \Tint{})--man kann einen Pointer auf ein Array an eine
andere Funktion übergeben, aber es ist richtiger zu sagen, dass der Pointer auf das erste Element des Arrays übergeben
wird. (Die Adressen der übrigen Elemente werden in bekannter Weise berechnet.)

Wenn man diesen Pointer mittels \IT{a[idx]} indiziert, wird \IT{idx} zum Pointer addiert und das dort abgelegte Element
(auf das der berechnete Pointer zeigt) wird zurückgegeben.

Ein interessantes Beispiel: ein String wie \IT{\q{string}} ist ein Array von Chars und hat den Typ \IT{const
char[]}.

Auch auf diesen Pointer kann ein Index angewendet werden.

Das ist der Grund warum es es möglich ist, Dinge wie \TT{\q{string}[i]} zu schreiben--es handelt sich dabei um einen
korrekten \CCpp Ausdruck!

}
\FR{\subsubsection{x86}

\myindex{x86!\Instructions!LOOP}

Il y a une instruction \LOOP spéciale en x86 qui teste le contenu du registre \ECX
et si il est différent de 0, le \glslink{decrement}{décrémente} et continue l'exécution
au label de l'opérande \LOOP.
Probablement que cette instruction n'est pas très pratique, et il n'y a aucun compilateur
moderne qui la génère automatiquement.
Donc, si vous la rencontrez dans du code, il est probable qu'il s'agisse de code
assembleur écrit manuellement.

\par

En \CCpp les boucles sont en général construites avec une déclaration \TT{for()},
\TT{while()} ou \TT{do/while()}.

Commençons avec \TT{for()}.
\myindex{\CLanguageElements!for}

Cette déclaration définit l'initialisation de la boucle (met le compteur à sa valeur
initiale), la condition de boucle (est-ce que le compteur est plus grand qu'une limite?),
qu'est-ce qui est fait à chaque itération (\glslink{increment}{incrémenter}/\glslink{decrement}{décrémenter})
et bien sûr le corps de la boucle.

\lstinputlisting[style=customc]{patterns/09_loops/simple/loops_1_FR.c}

Le code généré consiste également en quatre parties.

Commençons avec un exemple simple:

\lstinputlisting[label=loops_src,style=customc]{patterns/09_loops/simple/loops_2.c}

Résultat (MSVC 2010):

\lstinputlisting[caption=MSVC 2010,style=customasmx86]{patterns/09_loops/simple/1_MSVC_FR.asm}

Comme nous le voyons, rien de spécial.

GCC 4.4.1 génère presque le même code, avec une différence subtile:

\lstinputlisting[caption=GCC 4.4.1,style=customasmx86]{patterns/09_loops/simple/1_GCC_FR.asm}

Maintenant, regardons ce que nous obtenons avec l'optimisation  (\TT{\Ox}):

\lstinputlisting[caption=\Optimizing MSVC,style=customasmx86]{patterns/09_loops/simple/1_MSVC_Ox.asm}

Ce qui se passe alors, c'est que l'espace pour la variable $i$ n'est plus alloué
sur la pile locale, mais utilise un registre individuel pour cela, \ESI. Ceci est
possible pour ce genre de petites fonctions, où il n'y a pas beaucoup de variables
locales.

Il est très important que la fonction \ttf ne modifie pas la valeur de \ESI.
Notre compilateur en est sûr ici.
Et si le compilateur décide d'utiliser le registre \ESI aussi dans la fonction \ttf,
sa valeur devra être sauvegardée lors du prologue de la fonction et restaurée lors
de son épilogue, presque comme dans notre listing: notez les \TT{PUSH ESI/POP ESI}
au début et à la fin de la fonction.

Essayons GCC 4.4.1 avec l'optimisation la plus performante (option \Othree):

\lstinputlisting[caption=GCC 4.4.1 \Optimizing,style=customasmx86]{patterns/09_loops/simple/1_GCC_O3.asm}

\myindex{Loop unwinding}

Hé, GCC a juste déroulé notre boucle.

Le \glslink{loop unwinding}{déroulement de boucle} est un avantage lorsqu'il n'y a pas
beaucoup d'itérations et que nous pouvons économiser du temps d'exécution en supprimant
les instructions de gestion de la boucle.
D'un autre côté, le code est étonnement plus gros.

Dérouler des grandes boucles n'est pas recommandé de nos jours, car les grosses fonctions
ont une plus grande empreinte sur le cache%
%
\footnote{Un très bon article à ce sujet: \DrepperMemory.
D'autres recommandations sur l'expansion des boucles d'Intel sont ici:
\InSqBrackets{\IntelOptimization 3.4.1.7}.}.

Ok, augmentons la valeur maximale de la variable $i$ à 100 et essayons à nouveau.
GCC donne:

\lstinputlisting[caption=GCC,style=customasmx86]{patterns/09_loops/simple/2_GCC_FR.asm}

C'est assez similaire à ce que MSVC 2010 génère avec l'optimisation (\Ox), avec l'exception
que le registre \EBX est utilisé pour la variable $i$.

GCC est sûr que ce registre ne sera pas modifié à l'intérieur de la fonction \ttf,
et si il l'était, il serait sauvé dans le prologue de la fonction et restauré dans
l'épilogue, tout comme dans la fonction \main.

\clearpage
\subsubsection{x86: \olly}
\myindex{\olly}

Compilons notre exemple dans MSVC 2010 avec les options \Ox et \Obzero, puis chargeons
le dans \olly.

Il semble qu'\olly soit capable de détecter des boucles simples et les affiche entre
parenthèses, par commodité.

\begin{figure}[H]
\centering
\myincludegraphics{patterns/09_loops/simple/olly1.png}
\caption{\olly: début de \main}
\label{fig:loops_olly_1}
\end{figure}

En traçant (F8~--- \stepover) nous voyons \ESI 
s'\glslink{increment}{incrémenter}.
Ici, par exemple, $ESI=i=6$:

\begin{figure}[H]
\centering
\myincludegraphics{patterns/09_loops/simple/olly2.png}
\caption{\olly: le corps de la boucle vient de s'exécuter avec $i=6$}
\label{fig:loops_olly_2}
\end{figure}

9 est la dernière valeur de la boucle.
C'est pourquoi \JL ne s'exécute pas après l'\glslink{increment}{incrémentation},
et que le fonction se termine.

\begin{figure}[H]
\centering
\myincludegraphics{patterns/09_loops/simple/olly3.png}
\caption{\olly: $ESI=10$, fin de la boucle}
\label{fig:loops_olly_3}
\end{figure}

\subsubsection{x86: tracer}
\myindex{tracer}

Comme nous venons de le voir, il n'est pas très commode de tracer manuellement dans
le débogueur.
C'est pourquoi nous allons essayer \tracer.

Nous ouvrons dans \IDA l'exemple compilé, trouvons l'adresse de l'instruction \INS{PUSH ESI}
(qui passe le seul argument à \ttf), qui est \TT{0x401026} dans ce cas et nous lançons
le \tracer:

\begin{lstlisting}
tracer.exe -l:loops_2.exe bpx=loops_2.exe!0x00401026
\end{lstlisting}

\TT{BPX} met juste un point d'arrêt à l'adresse et \tracer va alors afficher l'état
des registres.

Voici ce que l'on voit dans \TT{tracer.log}:

\lstinputlisting{patterns/09_loops/simple/tracer.log}

Nous voyons comment la valeur du registre \ESI change de 2 à 9.

Encore plus que ça, \tracer peut collecter les valeurs des registres pour toutes
les adresses dans la fonction. C'est appelé \IT{trace} ici. Chaque instruction est
tracée, toutes les valeurs intéressantes des registres sont enregistrées.

Ensuite, un script \IDA\ .idc est généré, qui ajoute des commentaires.
Donc, dans \IDA, nous avons appris que l'adresse de la fonction \main est \TT{0x00401020}
et nous lançons:

\begin{lstlisting}
tracer.exe -l:loops_2.exe bpf=loops_2.exe!0x00401020,trace:cc
\end{lstlisting}

\TT{BPF} signifie mettre un point d'arrêt sur la fonction.

Comme résultat, nous obtenons les scripts \TT{loops\_2.exe.idc} et \TT{loops\_2.exe\_clear.idc}.

\clearpage
Nous chargeons \TT{loops\_2.exe.idc} dans \IDA et voyons:

\begin{figure}[H]
\centering
\myincludegraphics{patterns/09_loops/simple/IDA_tracer_cc.png}
\caption{\IDA avec le script .idc chargé}
\label{fig:loops_IDA_tracer}
\end{figure}

Nous voyons que \ESI varie de 2 à 9 au début du corps de boucle, mais de 3 à
0xA (10) après l'incrément.
Nous voyons aussi que \main se termine avec 0 dans \EAX.

\tracer génère également \TT{loops\_2.exe.txt}, qui contient des informations sur
le nombre de fois qu'une instruction a été exécutée et les valeurs du registre:

\lstinputlisting[caption=loops\_2.exe.txt]{patterns/09_loops/simple/loops_2.exe.txt}
\myindex{\GrepUsage}
Nous pouvons utiliser grep ici.

}
\JPN{\subsubsection{x86}

\myindex{x86!\Instructions!LOOP}

x86命令セットには \ECX というレジスタが値をチェックする特別な \LOOP 命令になります。そして
0でなければ、\gls{decrement} \ECX し、 \LOOP オペランドのラベルに制御フローを渡します。
おそらくこの命令はあまり便利ではなく、自動的にそれを発行する最新のコンパイラはありません。
したがって、コードのどこかでこの命令を見ると、これは手作業で書かれたアセンブリコードである可能性が高いです。

\par

\CCpp のループでは通常\TT{for()}、\TT{while()} または \TT{do/while()}文を使用して構成されます。

\TT{for()}を使ってみましょう。
\myindex{\CLanguageElements!for}

この文はループの初期化を定義し(ループカウンタを初期値にセット)、
ループ条件(がリミットより大きいか?)が各イテレーション(\gls{increment}/\gls{decrement})で実行され、
ループボディも当然実行されます。

\lstinputlisting[style=customc]{patterns/09_loops/simple/loops_1_JPN.c}

生成されたコードは4つの部分で構成されています。

簡単な例から始めましょう:

\lstinputlisting[label=loops_src,style=customc]{patterns/09_loops/simple/loops_2.c}

結果(MSVC 2010):

\lstinputlisting[caption=MSVC 2010,style=customasmx86]{patterns/09_loops/simple/1_MSVC_JPN.asm}

我々が見るように、特別なものはありません。

GCC 4.4.1はほぼ同じコードを出力しますが、微妙な違いが1つあります:

\lstinputlisting[caption=GCC 4.4.1,style=customasmx86]{patterns/09_loops/simple/1_GCC_JPN.asm}

最適化を有効にして(\TT{\Ox})取得した内容を見てみましょう。

\lstinputlisting[caption=\Optimizing MSVC,style=customasmx86]{patterns/09_loops/simple/1_MSVC_Ox.asm}

ここで起こるのは、 $i$ 変数のスペースがローカルスタックにはもう割り当てられず、
\ESI のための個別のレジスタを使用するということです。 
これは、ローカル変数があまりないような小さな関数で可能です。

とても重要なことは、 \ttf 関数が \ESI の値を変更してはならないことです。 
私たちのコンパイラは確かにそうしています。
コンパイラが \ttf で \ESI レジスタを使用することを決定した場合、
その値は関数のプロローグに保存され、関数のエピローグで復元されなければなりません。
リストのようになります。関数の開始と終了での\TT{PUSH ESI/POP ESI}に注意してください。

最適化を最大にしてGCC 4.4.1を試してみましょう( \Othree オプション):

\lstinputlisting[caption=\Optimizing GCC 4.4.1,style=customasmx86]{patterns/09_loops/simple/1_GCC_O3.asm}

\myindex{Loop unwinding}

えっ、GCCは単に私たちのループを巻き戻してしまいました。

\Gls{loop unwinding} は反復回数が多くなく、
すべてのループサポート命令を削除することで実行時間を短縮できる場合に利点があります。 
逆の場合では、明らかにコードが大きくなります。

大規模な関数は大量のキャッシュフットプリント%
%
\footnote{非常によい記事: \DrepperMemory 
インテルのループアンローリングに関するその他の推奨事項はこちら:
\InSqBrackets{\IntelOptimization 3.4.1.7}}
を必要とする可能性があるため、大きなアンロールループは現代では推奨されません。

OK、 $i$ 変数の最大値を100に増やして、もう一度試してみましょう。 GCCの結果は以下の通り:

\lstinputlisting[caption=GCC,style=customasmx86]{patterns/09_loops/simple/2_GCC_JPN.asm}

これは、EBXレジスタが $i$ 変数に割り当てられていることを除いて、
最適化ありのMSVC 2010(\Ox)と非常によく似ています。

GCCはこのレジスタが \ttf 関数の内部で変更されないことをわかっています。
もしそうであれば、 \main 関数のように関数プロローグに保存され、エピローグで復元されます。

\clearpage
\subsubsection{x86: \olly}
\myindex{\olly}

\Ox と \Obzero オプションを使用してMSVC 2010のサンプルをコンパイルし、 
\olly にロードしてみましょう。

\olly は単純なループを検出し、便宜上、角カッコで表示してくれます。

\begin{figure}[H]
\centering
\myincludegraphics{patterns/09_loops/simple/olly1.png}
\caption{\olly: \main 開始}
\label{fig:loops_olly_1}
\end{figure}

By tracing (F8~--- \stepover) we see \ESI 
\glslink{increment}{incrementing}.
Here, for instance, $ESI=i=6$:

トレースすることにより(F8。ステップオーバ)、ESIが増加することがわかります。 ここで、例えば、ESI = i = 6:

\begin{figure}[H]
\centering
\myincludegraphics{patterns/09_loops/simple/olly2.png}
\caption{\olly: ループボディが $i=6$ で実行}
\label{fig:loops_olly_2}
\end{figure}

9は最後のループ値です。
そのため、 \JL は\gls{increment}後に実行されず、関数は終了します。

\begin{figure}[H]
\centering
\myincludegraphics{patterns/09_loops/simple/olly3.png}
\caption{\olly: $ESI=10$、ループ終了}
\label{fig:loops_olly_3}
\end{figure}

\subsubsection{x86: tracer}
\myindex{tracer}

見てきたように、デバッガで手動でトレースするのはあまり便利ではありません。 
それがトレーサを試みる理由です。

コンパイルされたサンプルを \IDA で開き、命令\INS{PUSH ESI}( \ttf へ1つ引数を渡す)
のアドレスを見つけます。この場合は \TT{0x401026} で、トレーサを実行してみます。

\begin{lstlisting}
tracer.exe -l:loops_2.exe bpx=loops_2.exe!0x00401026
\end{lstlisting}

\TT{BPX} はアドレスにブレークポイントを設定するだけで、 \tracer はレジスタの状態を出力します。

\TT{tracer.log}では、このようになります。

\lstinputlisting{patterns/09_loops/simple/tracer.log}

\ESI レジスタの値が2から9に変化する様子を見ています。

\tracer はそれ以上にも、関数内のすべてのアドレスのレジスタ値を収集できます。 
これを\IT{trace}といいます。 
すべての命令がトレースされ、興味深いレジスタ値がすべて記録されます。

次に、コメントを追加する \IDA .idcスクリプトが生成されます。 
したがって、 \IDA では、 \main 関数のアドレスは\TT{0x00401020}であり、次のように実行されます。

\begin{lstlisting}
tracer.exe -l:loops_2.exe bpf=loops_2.exe!0x00401020,trace:cc
\end{lstlisting}

\TT{BPF}は、関数にブレークポイントを設定します。

その結果、\TT{loops\_2.exe.idc}および\TT{loops\_2.exe\_clear.idc}スクリプトが取得されます。

\clearpage
\TT{loops\_2.exe.idc}を \IDA にロードすると次のようになります。

\begin{figure}[H]
\centering
\myincludegraphics{patterns/09_loops/simple/IDA_tracer_cc.png}
\caption{.idc-scriptを \IDA でロードした}
\label{fig:loops_IDA_tracer}
\end{figure}

\ESI はループ本体の開始時には2から9、
インクリメント後は3から0xA(10)になります。
\main が \EAX 0で終了していることもわかります。

\tracer はまた、各命令が何回実行されたかに
関する情報とレジスタ値を含む\TT{loops\_2.exe.txt}も生成します。

\lstinputlisting[caption=loops\_2.exe.txt]{patterns/09_loops/simple/loops_2.exe.txt}
\myindex{\GrepUsage}
ここではgrepを使うことができます。
}


\mysection{\oracle}
\label{oracle}

% sections
\EN{\input{examples/oracle/1_version_EN}}\RU{\input{examples/oracle/1_version_RU}}
\EN{\input{examples/oracle/2_ksmlru_EN}}\RU{\input{examples/oracle/2_ksmlru_RU}}
\EN{\input{examples/oracle/3_timer_EN}}\RU{\input{examples/oracle/3_timer_RU}}


\subsubsection{MIPS}

\ifdefined\RUSSIAN
\lstinputlisting[caption=\NonOptimizing GCC 4.4.5 (IDA),style=customasmMIPS]{patterns/09_loops/simple/MIPS_O0_IDA_RU.lst}

\myindex{MIPS!\Pseudoinstructions!B}
Новая для нас инструкция это \INS{B}. Вернее, это псевдоинструкция (\INS{BEQ}).
\fi

\ifdefined\ENGLISH
\lstinputlisting[caption=\NonOptimizing GCC 4.4.5 (IDA),style=customasmMIPS]{patterns/09_loops/simple/MIPS_O0_IDA_EN.lst}

\myindex{MIPS!\Pseudoinstructions!B}
The instruction that's new to us is \TT{B}. It is actually the pseudo instruction (\INS{BEQ}).
\fi

\ifdefined\FRENCH
\lstinputlisting[caption=GCC 4.4.5 \NonOptimizing (IDA),style=customasmMIPS]{patterns/09_loops/simple/MIPS_O0_IDA_FR.lst}

\myindex{MIPS!\Pseudoinstructions!B}
L'instruction qui est nouvelle pour nous est \TT{B}. C'est la pseudo instruction (\INS{BEQ}).
\fi

\ifdefined\JAPANESE
\lstinputlisting[caption=GCC 4.4.5 \NonOptimizing (IDA),style=customasmMIPS]{patterns/09_loops/simple/MIPS_O0_IDA_JPN.lst}

\myindex{MIPS!\Pseudoinstructions!B}
新しい命令は\TT{B}です。 実際には擬似命令(\INS{BEQ})です。
\fi



\mysection{\oracle}
\label{oracle}

% sections
\EN{\input{examples/oracle/1_version_EN}}\RU{\input{examples/oracle/1_version_RU}}
\EN{\input{examples/oracle/2_ksmlru_EN}}\RU{\input{examples/oracle/2_ksmlru_RU}}
\EN{\input{examples/oracle/3_timer_EN}}\RU{\input{examples/oracle/3_timer_RU}}


\mysection{\oracle}
\label{oracle}

% sections
\EN{\input{examples/oracle/1_version_EN}}\RU{\input{examples/oracle/1_version_RU}}
\EN{\input{examples/oracle/2_ksmlru_EN}}\RU{\input{examples/oracle/2_ksmlru_RU}}
\EN{\input{examples/oracle/3_timer_EN}}\RU{\input{examples/oracle/3_timer_RU}}


\mysection{\oracle}
\label{oracle}

% sections
\EN{\input{examples/oracle/1_version_EN}}\RU{\input{examples/oracle/1_version_RU}}
\EN{\input{examples/oracle/2_ksmlru_EN}}\RU{\input{examples/oracle/2_ksmlru_RU}}
\EN{\input{examples/oracle/3_timer_EN}}\RU{\input{examples/oracle/3_timer_RU}}



\subsection{\Conclusion{}}

\subsubsection{x86}

Примерный скелет условных переходов:

\begin{lstlisting}[caption=x86,style=customasmx86]
CMP register, register/value
Jcc true ; §cc=код условия§
false:
... код, исполняющийся, если сравнение ложно ...
JMP exit 
true:
... код, исполняющийся, если сравнение истинно ...
exit:
\end{lstlisting}

\subsubsection{ARM}

\begin{lstlisting}[caption=ARM,style=customasmARM]
CMP register, register/value
Bcc true ; §cc=код условия§
false:
... код, исполняющийся, если сравнение ложно ...
JMP exit 
true:
... код, исполняющийся, если сравнение истинно ...
exit:
\end{lstlisting}

\subsubsection{MIPS}

\begin{lstlisting}[caption=Проверка на ноль,style=customasmMIPS]
BEQZ REG, label
...
\end{lstlisting}

\begin{lstlisting}[caption=Меньше ли нуля? (используя псевдоинструкцию),style=customasmMIPS]
BLTZ REG, label
...
\end{lstlisting}

\begin{lstlisting}[caption=Проверка на равенство,style=customasmMIPS]
BEQ REG1, REG2, label
...
\end{lstlisting}

\begin{lstlisting}[caption=Проверка на неравенство,style=customasmMIPS]
BNE REG1, REG2, label
...
\end{lstlisting}

\begin{lstlisting}[caption=Проверка на меньше (знаковое),style=customasmMIPS]
SLT REG1, REG2, REG3
BEQ REG1, label
...
\end{lstlisting}

\begin{lstlisting}[caption=Проверка на меньше (беззнаковое),style=customasmMIPS]
SLTU REG1, REG2, REG3
BEQ REG1, label
...
\end{lstlisting}

\subsubsection{Без инструкций перехода}

\myindex{ARM!\Instructions!MOVcc}
\myindex{x86!\Instructions!CMOVcc}
\myindex{ARM!\Instructions!CSEL}

Если тело условного выражения очень короткое, может быть
использована инструкция условного копирования: \INS{MOVcc} в ARM (в режиме ARM), \INS{CSEL} в ARM64, \INS{CMOVcc} в x86.

\myparagraph{ARM}

В режиме ARM можно использовать условные суффиксы для некоторых инструкций:

\begin{lstlisting}[caption=ARM (\ARMMode),style=customasmARM]
CMP register, register/value
instr1_cc ; инструкция, которая будет исполнена, если условие истинно
instr2_cc ; еще инструкция, которая будет исполнена, если условие истинно
... и т.д....
\end{lstlisting}

Нет никаких ограничений на количество инструкций с условными суффиксами до тех пор,
пока флаги CPU не были модифицированы одной из таких инструкций.

% FIXME: list of such instructions or \myref{} to it

\myindex{ARM!\Instructions!IT}
В режиме Thumb есть инструкция \INS{IT}, позволяющая дополнить следующие 4 инструкции суффиксами, задающими
условие.

Читайте больше об этом: \myref{ARM_Thumb_IT}.

\begin{lstlisting}[caption=ARM (\ThumbMode),style=customasmARM]
CMP register, register/value
ITEEE EQ ; выставить такие суффиксы: if-then-else-else-else
instr1   ; инструкция будет исполнена, если истинно
instr2   ; инструкция будет исполнена, если ложно
instr3   ; инструкция будет исполнена, если ложно
instr4   ; инструкция будет исполнена, если ложно
\end{lstlisting}

\subsection{\Exercise}

(ARM64) Попробуйте переписать код в \lstref{cond_ARM64} 
убрав все инструкции условного перехода, и используйте инструкцию \INS{CSEL}.

