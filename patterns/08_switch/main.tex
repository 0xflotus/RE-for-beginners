\mysection{\SwitchCaseDefaultSectionName}
\myindex{\CLanguageElements!switch}

% sections
\subsection{\RU{Если вариантов мало}\EN{Small number of cases}\DE{Kleine Anzahl von Fällen}\FR{Petit nombre de cas}\ITA{Pochi casi}\JPN{小さな数のcase}}

\lstinputlisting[style=customc]{patterns/08_switch/1_few/few.c}

\EN{\subsection{x86}

\subsubsection{MSVC}

Here is what we get after compilation (MSVC 2010 Express):

\lstinputlisting[label=src:passing_arguments_ex_MSVC_cdecl,caption=MSVC 2010 Express,style=customasmx86]{patterns/05_passing_arguments/msvc_EN.asm}

\myindex{x86!\Registers!EBP}

What we see is that the \main function pushes 3 numbers onto the stack and calls \TT{f(int,int,int).} 

Argument access inside \ttf is organized with the help of macros like:\\
\TT{\_a\$ = 8}, 
in the same way as local variables, but with positive offsets (addressed with \IT{plus}).
So, we are addressing the \IT{outer} side of the \gls{stack frame} by adding the \TT{\_a\$} macro to the value in the \EBP register.

\myindex{x86!\Instructions!IMUL}
\myindex{x86!\Instructions!ADD}

Then the value of $a$ is stored into \EAX. After \IMUL instruction execution, the value in \EAX is 
a \gls{product} of the value in \EAX and the content of \TT{\_b}.

After that, \ADD adds the value in \TT{\_c} to \EAX.

The value in \EAX does not need to be moved: it is already where it must be.
On returning to \gls{caller}, it takes the \EAX value and uses it as an argument to \printf.

\clearpage
\myparagraph{\Optimizing MSVC + \olly}
\myindex{\olly}

We can try this (optimized) example in \olly.  Here is the first iteration:

\begin{figure}[H]
\centering
\myincludegraphics{patterns/10_strings/1_strlen/olly1.png}
\caption{\olly: first iteration start}
\label{fig:strlen_olly_1}
\end{figure}

We see that \olly found a loop and, for convenience, \IT{wrapped} its instructions in brackets.
By clicking the right button on \EAX, we can choose 
\q{Follow in Dump} and the memory window scrolls to the right place.
Here we can see the string \q{hello!} in memory.
There is at least
one zero byte after it and then random garbage.

If \olly sees a register with a valid address in it, that points to some string, 
it is shown as a string.

\clearpage
Let's press F8 (\stepover) a few times, to get to the start of the body of the loop:

\begin{figure}[H]
\centering
\myincludegraphics{patterns/10_strings/1_strlen/olly2.png}
\caption{\olly: second iteration start}
\label{fig:strlen_olly_2}
\end{figure}

We see that \EAX contains the address of the second character in the string.

\clearpage

We have to press F8 enough number of times in order to escape from the loop:

\begin{figure}[H]
\centering
\myincludegraphics{patterns/10_strings/1_strlen/olly3.png}
\caption{\olly: pointers difference to be calculated now}
\label{fig:strlen_olly_3}
\end{figure}

We see that \EAX now contains the address of zero byte that's right after the string plus 1 (because INC EAX was executed regardless of whether
we exit from the loop or not).
Meanwhile, \EDX hasn't changed,
so it still pointing to the start of the string.

The difference between these two addresses is being calculated now.

\clearpage
The \SUB instruction just got executed:

\begin{figure}[H]
\centering
\myincludegraphics{patterns/10_strings/1_strlen/olly4.png}
\caption{\olly: \EAX to be decremented now}
\label{fig:strlen_olly_4}
\end{figure}

The difference of pointers is in the \EAX register now---7.
Indeed, the length of the \q{hello!} string is 6, 
but with the zero byte included---7.
But \TT{strlen()} must return the number of non-zero characters in the string.
So the decrement executes and then the function returns.


\subsubsection{GCC}

Let's compile the same in GCC 4.4.1 and see the results in \IDA:

\lstinputlisting[caption=GCC 4.4.1,style=customasmx86]{patterns/05_passing_arguments/gcc_EN.asm}

The result is almost the same with some minor differences discussed earlier.

The \gls{stack pointer} is not set back after the two function calls(f and printf), 
because the penultimate \TT{LEAVE} (\myref{x86_ins:LEAVE}) 
instruction takes care of this at the end.
}
\RU{\subsection{x86}

\subsubsection{MSVC}

Рассмотрим пример, скомпилированный в (MSVC 2010 Express):

\lstinputlisting[label=src:passing_arguments_ex_MSVC_cdecl,caption=MSVC 2010 Express,style=customasmx86]{patterns/05_passing_arguments/msvc_RU.asm}

\myindex{x86!\Registers!EBP}
Итак, здесь видно: в функции \main заталкиваются три числа в стек и вызывается функция \TT{f(int,int,int)}.
 
Внутри \ttf доступ к аргументам, также как и к локальным переменным, происходит через макросы: 
\TT{\_a\$ = 8}, но разница в том, что эти смещения со знаком \IT{плюс}, 
таким образом если прибавить макрос \TT{\_a\$} к указателю на \EBP, то адресуется \IT{внешняя} 
часть \glslink{stack frame}{фрейма} стека относительно \EBP.

\myindex{x86!\Instructions!IMUL}
\myindex{x86!\Instructions!ADD}
Далее всё более-менее просто: значение $a$ помещается в \EAX. 
Далее \EAX умножается при помощи инструкции \IMUL на то, что лежит в \TT{\_b}, 
и в \EAX остается \glslink{product}{произведение} этих двух значений.

Далее к регистру \EAX прибавляется то, что лежит в \TT{\_c}.

Значение из \EAX никуда не нужно перекладывать, оно уже лежит где надо. 
Возвращаем управление вызывающей функции~--- она возьмет значение из \EAX и отправит его в \printf.

\clearpage
\myparagraph{\Optimizing MSVC + \olly}
\myindex{\olly}

Можем попробовать этот (соптимизированный) пример в \olly.  Вот самая первая итерация:

\begin{figure}[H]
\centering
\myincludegraphics{patterns/10_strings/1_strlen/olly1.png}
\caption{\olly: начало первой итерации}
\label{fig:strlen_olly_1}
\end{figure}

Видно, что \olly обнаружил цикл и, для удобства, \IT{свернул} инструкции тела цикла в скобке.

Нажав правой кнопкой на \EAX, можно выбрать \q{Follow in Dump} 
и позиция в окне памяти будет как раз там, где надо.

Здесь мы видим в памяти строку \q{hello!}.
После неё имеется как минимум 1 нулевой байт, затем случайный мусор.
Если \olly видит, что в регистре содержится адрес какой-то строки, он показывает эту строку.

\clearpage
Нажмем F8 (\stepover) столько раз, чтобы текущий адрес снова был в начале тела цикла:

\begin{figure}[H]
\centering
\myincludegraphics{patterns/10_strings/1_strlen/olly2.png}
\caption{\olly: начало второй итерации}
\label{fig:strlen_olly_2}
\end{figure}

Видно, что \EAX уже содержит адрес второго символа в строке.

\clearpage
Будем нажимать F8 достаточное количество раз, чтобы выйти из цикла:

\begin{figure}[H]
\centering
\myincludegraphics{patterns/10_strings/1_strlen/olly3.png}
\caption{\olly: сейчас будет вычисление разницы указателей}
\label{fig:strlen_olly_3}
\end{figure}

Увидим, что \EAX теперь содержит адрес нулевого байта, следующего сразу за строкой плюс 1 (потому что INC EAX исполнился вне зависимости
от того, выходим мы из цикла, или нет).

А \EDX так и не менялся~--- он всё ещё указывает на начало строки.
Здесь сейчас будет вычисляться разница между этими двумя адресами.

\clearpage
Инструкция \SUB исполнилась:

\begin{figure}[H]
\centering
\myincludegraphics{patterns/10_strings/1_strlen/olly4.png}
\caption{\olly: сейчас будет декремент \EAX}
\label{fig:strlen_olly_4}
\end{figure}

Разница указателей сейчас в регистре \EAX~--- 7.

Действительно, длина строки \q{hello!}~--- 6, 
но вместе с нулевым байтом --- 7.
Но \TT{strlen()} должна возвращать количество ненулевых символов в строке.
Так что сейчас будет исполняться декремент и выход из функции.



\subsubsection{GCC}

Скомпилируем то же в GCC 4.4.1 и посмотрим результат в \IDA:

\lstinputlisting[caption=GCC 4.4.1,style=customasmx86]{patterns/05_passing_arguments/gcc_RU.asm}

Практически то же самое, если не считать мелких отличий описанных ранее.

После вызова обоих функций \glslink{stack pointer}{указатель стека} не возвращается назад, 
потому что предпоследняя инструкция \TT{LEAVE} (\myref{x86_ins:LEAVE}) делает это за один раз, в конце исполнения.

}
\DE{\subsubsection{x86}

\myparagraph{MSVC}

Kompilieren wir das Beispiel:

\lstinputlisting[caption=MSVC 2008,style=customasmx86]{patterns/13_arrays/1_simple/simple_msvc.asm}

\myindex{x86!\Instructions!SHL}
Soweit nichts Außergewöhnliches, nur zwei Schleifen: die erste füllt mit Werten auf und die zweite gibt Werte aus.
% TBT
Der Befehl \TT{shl ecx, 1} wird für die Multiplikation mit 2 in \ECX verwendet; mehr dazu unten~\myref{SHR}.

Auf dem Stack werden 80 Bytes für das Array reserviert: 20 Elemente von je 4 Byte.

\clearpage
Untersuchen wir dieses Beispiel in \olly.
\myindex{\olly}

Wir erkennen wie das Array befüllt wird:

jedes Element ist ein 32-Bit-Wort vom Typ \Tint und der Wert ist der Index multipliziert mit 2:

\begin{figure}[H]
\centering
\myincludegraphics{patterns/13_arrays/1_simple/olly.png}
\caption{\olly: nach dem Füllen des Arrays}
\label{fig:array_simple_olly}
\end{figure}
Da sich dieses Array auf dem Stack befindet, finden wir dort alle seine 20 Elemente.

\myparagraph{GCC}

Hier ist was GCC 4.4.1 erzeugt:

\lstinputlisting[caption=GCC 4.4.1,style=customasmx86]{patterns/13_arrays/1_simple/simple_gcc.asm}
Die Variable $a$ ist übrigens vom Typ \IT{int*} (Pointer auf \Tint{})--man kann einen Pointer auf ein Array an eine
andere Funktion übergeben, aber es ist richtiger zu sagen, dass der Pointer auf das erste Element des Arrays übergeben
wird. (Die Adressen der übrigen Elemente werden in bekannter Weise berechnet.)

Wenn man diesen Pointer mittels \IT{a[idx]} indiziert, wird \IT{idx} zum Pointer addiert und das dort abgelegte Element
(auf das der berechnete Pointer zeigt) wird zurückgegeben.

Ein interessantes Beispiel: ein String wie \IT{\q{string}} ist ein Array von Chars und hat den Typ \IT{const
char[]}.

Auch auf diesen Pointer kann ein Index angewendet werden.

Das ist der Grund warum es es möglich ist, Dinge wie \TT{\q{string}[i]} zu schreiben--es handelt sich dabei um einen
korrekten \CCpp Ausdruck!

}
\FR{\subsubsection{x86}

Regardons ce que nous obtenons avec MSVC 2010:

\lstinputlisting[caption=MSVC 2010,style=customasmx86]{patterns/12_FPU/2_passing_floats/MSVC_FR.asm}

\myindex{x86!\Instructions!FLD}
\myindex{x86!\Instructions!FSTP}

\FLD et \FSTP déplacent des variables entre le segment de données et la pile du FPU.
\GTT{pow()}\footnote{une fonction C standard, qui élève un nombre à la puissance
donnée (puissance)} prend deux valeurs depuis la pile et renvoie son résultat dans
le registre \ST{0}.
\printf prend 8 octets de la pile locale et les interprète comme des variables de
type \Tdouble.

À propos, une paire d'instructions \MOV pourrait être utilisée ici pour déplacer
les valeurs depuis la mémoire vers la pile, car les valeurs en mémoire sont stockées
au format IEEE 754, et pow() les prend aussi dans ce format, donc aucune conversion
n'est nécessaire.
C'est fait ainsi dans l'exemple suivant, pour ARM: \myref{FPU_passing_floats_ARM}.

}
\ITA{\input{patterns/08_switch/1_few/x86_ITA}}
\JPN{\subsubsection{x86}

\myparagraph{\NonOptimizing MSVC}

結果 (MSVC 2010):

\lstinputlisting[caption=MSVC 2010,style=customasmx86]{patterns/08_switch/1_few/few_msvc.asm}

実際、switch()でいくつかのcaseを持つ私たちの関数は、この構造に似ています。

\lstinputlisting[label=switch_few_ifelse,style=customc]{patterns/08_switch/1_few/few_analogue.c}

\myindex{\CLanguageElements!switch}
\myindex{\CLanguageElements!if}

いくつかのcaseでswitch()を使用する場合、ソースコード内の実際のswitch()か、
単にif文の組であるかどうかを確認することは不可能です。
\myindex{\SyntacticSugar}

これはswitch()が多段にネストされたif文との糖衣構文のようなものであることを意味します。

コンパイラが入力変数 $a$ を一時的なローカル変数\TT{tv64}に移動することを除いて、
生成されたコードには特に新しいことはありません。
\footnote{スタック内のローカル変数には接頭辞\TT{tv}が付きます。MSVCが内部変数として使用するために命名しています。}

これをGCC 4.4.1でコンパイルすると、最大限の最適化(\Othree option)を有効にしても
ほぼ同じ結果になります。

\myparagraph{\Optimizing MSVC}

% TODO separate various kinds of \TT
% idea: enclose command lines in a specific environment, like \cmdline{} 
% assembly instructions in \asm{} (now both \TT and \q{} are used),
% variables in,  like \var{}
% messages (string constants) in something else, like \strconst
% to separate them all. Now they all use \TT, which is not best
% \INS{} for all instructions including operands? --DY
では、MSVC(\Ox)の最適化を有効にしましょう:\TT{cl 1.c /Fa1.asm /Ox}

\label{JMP_instead_of_RET}
\lstinputlisting[caption=MSVC,style=customasmx86]{patterns/08_switch/1_few/few_msvc_Ox.asm}

ここで、汚いハックを見ることができます。

\myindex{x86!\Instructions!JZ}
\myindex{x86!\Instructions!JE}
\myindex{x86!\Instructions!SUB}

最初に、 $a$ の値を \EAX に置き、0を引きます。 EAXの値が0かどうかを確認するために行われますが、
そうであれば、 \ZF フラグがセットされます(例えば、0からの減算は0)
最初の条件ジャンプ \JE (\IT{Jump if Equal} またはあ同義語 \JZ~---\IT{Jump if Zero})は実行され、
制御フローは\TT{\$LN4@f}ラベルに渡されます。ここでは、 \TT{'zero'}メッセージが出力されます。
最初のジャンプが実行されない場合は、入力値から1が減算され、結果が0の場合、対応するジャンプが実行されます。

また、ジャンプが全く実行されない場合、制御フローは文字列引数\TT{'something unknown'}を \printf に渡します。

\label{jump_to_last_printf}
\myindex{\Stack}

次に、文字列ポインタが $a$ 変数に置かれ、 \printf が \CALL ではなく \JMP を介して呼び出されます。 
簡単に説明するとこうなります:
\gls{caller} は値をスタックにプッシュし、 \CALL 経由で関数を呼び出します。
\CALL 自体は戻りアドレス(\ac{RA})をスタックにプッシュし、関数アドレスへの無条件ジャンプを行います。
スタックポインタを移動させる命令が含まれていないため、任意の実行時点での関数は、次のスタックレイアウトを持ちます。

\begin{itemize}
\item\ESP---points to \ac{RA}
\item\TT{ESP+4}---points to the $a$ variable 
\end{itemize}

反対に、\printf をここで呼び出さなければならないときは、文字列を指し示す必要がある最初の\printf 引数を除いて、
全く同じスタックレイアウトが必要です。それが私たちのコードがすることです。

ファンクションの最初の引数を文字列のアドレスに置き換え、
関数 \ttf を直接呼び出しずに直接 \printf を呼び出すかのように、 \printf にジャンプします。
\printf は文字列を \gls{stdout} に出力し、 \RET 命令を実行します。スタックから\ac{RA}を取り出し、
制御フローは \ttf ではなく \ttf 関数の終りをバイパスして、 \ttf の \gls{caller} です。

\myindex{\CStandardLibrary!longjmp()}
\newcommand{\URLSJ}{\href{http://go.yurichev.com/17121}{wikipedia}}

% TODO \myref{}
\printf はすべての場合に  \ttf 関数の終わりで右に呼ばれるので、これはすべて可能です。
ある意味では、\TT{longjmp()}\footnote{\URLSJ}関数に似ています。
そしてもちろん、それはスピードのためにすべて行われます。

ARMコンパイラと同様のケースは、\q{\PrintfSeveralArgumentsSectionName}セクションに記載されています。
こちら:~(\myref{ARM_B_to_printf})

\clearpage
\mysubparagraph{\olly}

この例は扱いにくいので、 \olly でトレースしてみましょう。

\olly はそのようなswitch()構文を検出することができ、有用なコメントを追加することができます。
\EAX の値は最初は2で、それは関数への入力値です:

\begin{figure}[H]
\centering
\myincludegraphics{patterns/08_switch/1_few/olly1.png}
\caption{\olly: \EAX 
は最初の(そして唯一の)関数への引数を含んでいます}
\label{fig:switch_few_olly1}
\end{figure}

\clearpage
0は \EAX から2を引いた値です。
もちろん、 \EAX にはまだ2が入っています。
しかし、 \ZF フラグは0になり、結果の値がゼロでないことを示します。

\begin{figure}[H]
\centering
\myincludegraphics{patterns/08_switch/1_few/olly2.png}
\caption{\olly: \SUB の実行}
\label{fig:switch_few_olly2}
\end{figure}

\clearpage
\DEC が実行され、 \EAX には1が入ります。
しかし1はゼロではないので、 \ZF フラグはまだ0です:

\begin{figure}[H]
\centering
\myincludegraphics{patterns/08_switch/1_few/olly3.png}
\caption{\olly: 最初の \DEC 実行}
\label{fig:switch_few_olly3}
\end{figure}

\clearpage
次の \DEC が実行されます。
\EAX は最終的に0になり、結果がゼロであるため \ZF フラグが設定されます。

\begin{figure}[H]
\centering
\myincludegraphics{patterns/08_switch/1_few/olly4.png}
\caption{\olly: 2回目の \DEC 実行}
\label{fig:switch_few_olly4}
\end{figure}

\olly は、このジャンプが今行われることを示しています。

\clearpage
\q{two}という文字列へのポインタが今スタックに書き込まれます:

\begin{figure}[H]
\centering
\myincludegraphics{patterns/08_switch/1_few/olly5.png}
\caption{\olly: 
文字列へのポインタは、最初の引数の場所に書き込まれる}
\label{fig:switch_few_olly5}
\end{figure}

% TODO: homogenize numbers
% now they are inconsistent: sometimes plain text, sometimes in math mode
% some kind of \expr{} both for numbers and expressions? --DY
注意:関数の現在の引数は2であり、2はスタックに\TT{0x001EF850}のアドレスにあります。

\clearpage
\MOV はアドレス\TT{0x001EF850}の文字列にポインタを書き込みます(スタックウィンドウを参照)。
その後、ジャンプが発生します。
これはMSVCR100.DLLの \printf 関数の最初の命令です(この例は/MDスイッチでコンパイルされています)。

\begin{figure}[H]
\centering
\myincludegraphics{patterns/08_switch/1_few/olly6.png}
\caption{\olly: MSVCR100.DLLでの \printf の最初の命令}
\label{fig:switch_few_olly6}
\end{figure}

今や \printf は\TT{0x00FF3010}の文字列を唯一の引数として扱い、文字列を出力します。

\clearpage
これが \printf の最後の命令です。

\begin{figure}[H]
\centering
\myincludegraphics{patterns/08_switch/1_few/olly7.png}
\caption{\olly: MSVCR100.DLLの \printf の最後の命令}
\label{fig:switch_few_olly7}
\end{figure}

文字列 \q{two} はコンソールウィンドウに表示されます。

\clearpage
F7またはF8を押して(\stepover) リターンすると\dots \ttf ではなく、 \main にいきます。

\begin{figure}[H]
\centering
\myincludegraphics{patterns/08_switch/1_few/olly8.png}
\caption{\olly: \main へのリターン}
\label{fig:switch_few_olly8}
\end{figure}

はい、 \printf の中心から \main に直接ジャンプしました。
なぜならスタックの\ac{RA}は \ttf ではなく、 \main の場所を指しているからです。
\CALL \TT{0x00FF1000}は \ttf を呼び出した実際の命令です。


}

\EN{\subsubsection{ARM}

\myparagraph{\OptimizingKeilVI (\ThumbMode)}

\lstinputlisting[style=customasmARM]{patterns/04_scanf/1_simple/ARM_IDA.lst}

\myindex{\CLanguageElements!\Pointers}

In order for \scanf to be able to read item it needs a parameter---pointer to an \Tint.
\Tint is 32-bit, so we need 4 bytes to store it somewhere in memory, and it fits exactly in a 32-bit register.
\myindex{IDA!var\_?}
A place for the local variable \GTT{x} is allocated in the stack and \IDA
has named it \IT{var\_8}. It is not necessary, however, to allocate a such since \ac{SP} (\gls{stack pointer}) is already pointing to that space and it can be used directly.

So, \ac{SP}'s value is copied to the \Reg{1} register and, together with the format-string, passed to \scanf.

\INS{PUSH/POP} instructions behaves differently in ARM than in x86 (it's the other way around).
They are synonyms to \INS{STM/STMDB/LDM/LDMIA} instructions.
And \INS{PUSH} instruction first writes a value into the stack, \IT{and then} subtracts \ac{SP} by 4.
\INS{POP} first adds 4 to \ac{SP}, \IT{and then} reads a value from the stack.
Hence, after \INS{PUSH}, \ac{SP} points to an unused space in stack.
It is used by \scanf, and by \printf after.

\INS{LDMIA} means \IT{Load Multiple Registers Increment address After each transfer}.
\INS{STMDB} means \IT{Store Multiple Registers Decrement address Before each transfer}.

\myindex{ARM!\Instructions!LDR}
Later, with the help of the \INS{LDR} instruction, this value is moved from the stack to the \Reg{1} register in order to be passed to \printf.

\myparagraph{ARM64}

\lstinputlisting[caption=\NonOptimizing GCC 4.9.1 ARM64,numbers=left,style=customasmARM]{patterns/04_scanf/1_simple/ARM64_GCC491_O0_EN.s}

There is 32 bytes are allocated for stack frame, which is bigger than it needed. Perhaps some memory aligning issue?
The most interesting part is finding space for the $x$ variable in the stack frame (line 22).
Why 28? Somehow, compiler decided to place this variable at the end of stack frame instead of beginning.
The address is passed to \scanf, which just stores the user input value in the memory at that address.
This is 32-bit value of type \Tint.
The value is fetched at line 27 and then passed to \printf.

}
\RU{\subsubsection{ARM}

\myparagraph{\OptimizingKeilVI (\ThumbMode)}

\lstinputlisting[style=customasmARM]{patterns/04_scanf/1_simple/ARM_IDA.lst}

\myindex{\CLanguageElements!\Pointers}
Чтобы \scanf мог вернуть значение, ему нужно передать указатель на переменную типа \Tint.
\Tint~--- 32-битное значение, для его хранения нужно только 4 байта, и оно помещается в 32-битный регистр.

\myindex{IDA!var\_?}
Место для локальной переменной \GTT{x} выделяется в стеке, \IDA наименовала её \IT{var\_8}. 
Впрочем, место для неё выделять не обязательно, т.к. \glslink{stack pointer}{указатель стека} \ac{SP} уже указывает на место, 
свободное для использования.
Так что значение указателя \ac{SP} копируется в регистр \Reg{1}, и вместе с format-строкой, 
передается в \scanf.

Инструкции \INS{PUSH/POP} в ARM работают иначе, чем в x86 (тут всё наоборот).
Это синонимы инструкций \INS{STM/STMDB/LDM/LDMIA}.
И инструкция \INS{PUSH} в начале записывает в стек значение, \IT{затем} вычитает 4 из \ac{SP}.
\INS{POP} в начале прибавляет 4 к \ac{SP}, \IT{затем} читает значение из стека.
Так что после \INS{PUSH}, \ac{SP} указывает на неиспользуемое место в стеке.
Его и использует \scanf, а затем и \printf.

\INS{LDMIA} означает \IT{Load Multiple Registers Increment address After each transfer}.
\INS{STMDB} означает \IT{Store Multiple Registers Decrement address Before each transfer}.

\myindex{ARM!\Instructions!LDR}
Позже, при помощи инструкции \INS{LDR}, это значение перемещается из стека в регистр \Reg{1}, чтобы быть переданным в \printf.

\myparagraph{ARM64}

\lstinputlisting[caption=\NonOptimizing GCC 4.9.1 ARM64,numbers=left,style=customasmARM]{patterns/04_scanf/1_simple/ARM64_GCC491_O0_RU.s}

Под стековый фрейм выделяется 32 байта, что больше чем нужно. Может быть, это связано с выравниваем по границе памяти?
Самая интересная часть~--- это поиск места под переменную $x$ в стековом фрейме (строка 22).
Почему 28? Почему-то, компилятор решил расположить эту переменную в конце стекового фрейма, а не в начале.
Адрес потом передается в \scanf, которая просто сохраняет значение, введенное пользователем, в памяти по этому адресу.
Это 32-битное значение типа \Tint.
Значение загружается в строке 27 и затем передается в \printf.

}
\DE{\subsubsection{ARM}

\myparagraph{\OptimizingKeilVI (\ThumbMode)}

\lstinputlisting[style=customasmARM]{patterns/04_scanf/1_simple/ARM_IDA.lst}

\myindex{\CLanguageElements!\Pointers}
Damit \scanf Elemente einlesen kann, benötigt die Funktion einen Paramter--einen Pointer vom Typ \Tint.
\Tint hat die Größe 32 Bit, wir benötigen also 4 Byte, um den Wert im Speicher abzulegen, und passt daher genau in ein 32-Bit-Register.
\myindex{IDA!var\_?}
Auf dem Stack wird Platz für die lokalen Variable \GTT{x} reserviert und IDA bezeichnet diese Variable mit \IT{var\_8}. 
Eigentlich ist aber an dieser Stelle gar nicht notwendig, Platz auf dem Stack zu reservieren, da \ac{SP} (\gls{stack pointer} 
bereits auf die Adresse zeigt und auch direkt verwendet werden kann.

Der Wert von \ac{SP} wird also in das \Reg{1} Register kopiert und zusammen mit dem Formatierungsstring an \scanf übergeben.

% TBT here
%\INS{PUSH/POP} instructions behaves differently in ARM than in x86 (it's the other way around).
%They are synonyms to \INS{STM/STMDB/LDM/LDMIA} instructions.
%And \INS{PUSH} instruction first writes a value into the stack, \IT{and then} subtracts \ac{SP} by 4.
%\INS{POP} first adds 4 to \ac{SP}, \IT{and then} reads a value from the stack.
%Hence, after \INS{PUSH}, \ac{SP} points to an unused space in stack.
%It is used by \scanf, and by \printf after.

%\INS{LDMIA} means \IT{Load Multiple Registers Increment address After each transfer}.
%\INS{STMDB} means \IT{Store Multiple Registers Decrement address Before each transfer}.

\myindex{ARM!\Instructions!LDR}
Später wird mithilfe des \INS{LDR} Befehls dieser Wert vom Stack in das \Reg{1} Register verschoben um an \printf übergeben werden zu können.

\myparagraph{ARM64}

\lstinputlisting[caption=\NonOptimizing GCC 4.9.1 ARM64,numbers=left,style=customasmARM]{patterns/04_scanf/1_simple/ARM64_GCC491_O0_DE.s}

Im Stack Frame werden 32 Byte reserviert, was deutlich mehr als benötigt ist. Vielleicht handelt es sich um eine Frage des Aligning (dt. Angleichens) von Speicheradressen.
Der interessanteste Teil ist, im Stack Frame einen Platz für die Variable $x$ zu finden (Zeile 22).
Warum 28? Irgendwie hat der Compiler entschieden die Variable am Ende des Stack Frames anstatt an dessen Beginn abzulegen.
Die Adresse wird an \scanf übergeben; diese Funktion speichert den Userinput an der genannten Adresse im Speicher.
Es handelt sich hier um einen 32-Bit-Wert vom Typ \Tint. 
Der Wert wird in Zeile 27 abgeholt und dann an \printf übergeben.


}
\FR{\subsubsection{ARM + \NonOptimizingXcodeIV (\ThumbTwoMode)}
\label{FPU_passing_floats_ARM}

\lstinputlisting[style=customasmARM]{patterns/12_FPU/2_passing_floats/Xcode_thumb_O0.asm}

Comme nous l'avons déjà mentionné, les pointeurs sur des nombres flottants 64-bit
sont passés dans une paire de R-registres.

Ce code est un peu redondant (probablement car l'optimisation est désactivée),
puisqu'il est possible de charger les valeurs directement dans les R-registres sans
toucher les D-registres.

Donc, comme nous le voyons, la fonction \GTT{\_pow} reçoit son premier argument dans
\Reg{0} et \Reg{1}, et le second dans \Reg{2} et \Reg{3}.
La fonction laisse son résultat dans \Reg{0} et \Reg{1}.
Le résultat de \GTT{\_pow} est déplacé dans \GTT{D16}, puis dans la paire \Reg{1}
et \Reg{2}, d'où \printf prend le nombre résultant.

\subsubsection{ARM + \NonOptimizingKeilVI (\ARMMode)}

\lstinputlisting[style=customasmARM]{patterns/12_FPU/2_passing_floats/Keil_ARM_O0.asm}

Les D-registres ne sont pas utilisés ici, juste des paires de R-registres.

\subsubsection{ARM64 + GCC (Linaro) 4.9 \Optimizing}

\lstinputlisting[caption=GCC (Linaro) 4.9 \Optimizing,style=customasmARM]{patterns/12_FPU/2_passing_floats/ARM64_FR.s}

Les constantes sont chargées dans \RegD{0} et \RegD{1}: \TT{pow()} les prend d'ici.
Le résultat sera dans \RegD{0} après l'exécution de \TT{pow()}.
Il est passé à  \printf sans aucune modification ni déplacement, car \printf
prend ces arguments de \glslink{integral type}{type intégral} et pointeurs depuis
des X-registres, et les arguments en virgule flottante depuis des D-registres.

}
\ITA{\subsubsection{ARM}

\myparagraph{\OptimizingKeilVI (\ThumbMode)}

\lstinputlisting[style=customasmARM]{patterns/04_scanf/1_simple/ARM_IDA.lst}

\myindex{\CLanguageElements!\Pointers}

Affinche' \scanf possa leggere l'input, necessita di un parametro ---puntatore ad un \Tint.
\Tint e' 32-bit, quindi servono 4 byte per memorizzarlo da qualche parte in memoria, e entra perfettamente in un registro a 32-bit.
\myindex{IDA!var\_?}
Uno spazio per la variabile locale \GTT{x} e' allocato nello stack e \IDA
lo ha chiamato \IT{var\_8}. Non e' comunque necessario allocarlo in questo modo poiche' \ac{SP} (\gls{stack pointer}) punta gia' a quella posizione e puo' essere usato direttamente.

Successivamente il valore di \ac{SP} e' copiato nel registro \Reg{1} e sono passati, insieme alla format-string, a \scanf.

% TBT here
%\INS{PUSH/POP} instructions behaves differently in ARM than in x86 (it's the other way around).
%They are synonyms to \INS{STM/STMDB/LDM/LDMIA} instructions.
%And \INS{PUSH} instruction first writes a value into the stack, \IT{and then} subtracts \ac{SP} by 4.
%\INS{POP} first adds 4 to \ac{SP}, \IT{and then} reads a value from the stack.
%Hence, after \INS{PUSH}, \ac{SP} points to an unused space in stack.
%It is used by \scanf, and by \printf after.

%\INS{LDMIA} means \IT{Load Multiple Registers Increment address After each transfer}.
%\INS{STMDB} means \IT{Store Multiple Registers Decrement address Before each transfer}.

\myindex{ARM!\Instructions!LDR}
Questo valore, con l'aiuto dell'istruzione \INS{LDR} , viene poi spostato dallo stakc al registro \Reg{1} per essere passato a \printf.

\myparagraph{ARM64}

\lstinputlisting[caption=\NonOptimizing GCC 4.9.1 ARM64,numbers=left,style=customasmARM]{patterns/04_scanf/1_simple/ARM64_GCC491_O0_EN.s}

Ci sono 32 byte allocati per lo stack frame, che e' piu' grande del necessario. Forse a causa di meccanismi di allineamento della memoria?
La parte piu' interessante e' quella in cui trova spazio per la variabile $x$ nello stack frame (riga 22).
Perche' 28? Il compilatore ha in qualche modo deciso di piazzare questa variabile alla fine dello stack frame anziche' all'inizio.
L'indirizzo e' passato a \scanf, che memorizzera' il valore immesso dall'utente nella memoria a quell'indirizzo.
Si tratta di un valore a 32-bit di tipo \Tint.
Il valore e' recuperato successivamente a riga 27 e passato a \printf.

}
\JPN{\subsubsection{ARM: \OptimizingKeilVI (\ARMMode)}
\myindex{\CLanguageElements!switch}

\lstinputlisting[style=customasmARM]{patterns/08_switch/1_few/few_ARM_ARM_O3.asm}

繰り返しますが、このコードを調べることで、元のソースコードのswitch()か
単なるif()文の集合かどうかはわかりません。

\myindex{ARM!\Instructions!ADRcc}

とにかく、ここでは、 $R0=0$ の場合にのみトリガされる \ADREQ(\IT{Equal})のような述語命令を再度参照し、
文字列 \IT{<<zero\textbackslash{}n>>} を \Reg{0} にコピーします。
\myindex{ARM!\Instructions!BEQ}
$R0=0$ の場合、次の命令\ac{BEQ}は制御フローを\TT{loc\_170}にリダイレクトします。

巧妙な読者は、\Reg{0}レジスタに既に値を埋め込んでいるので、
\ac{BEQ}が正しくトリガされるかどうかを尋ねるかもしれません。

はい、\ac{BEQ}は\CMP 命令で設定されたフラグをチェックし、
\ADREQ はフラグをまったく変更しません。

命令の残りの部分は既に慣れ親しんでいます。
最後に \printf を1回呼び出すだけですが、ここではこのトリック~(\myref{ARM_B_to_printf})を既に調べています。
最後に \printf{} には3つのパスがあります。

\myindex{ARM!\Instructions!ADRcc}
\myindex{ARM!\Instructions!CMP}
$a=2$ かどうかを確認するには、最後の命令\TT{CMP R0, \#2}が必要です。

それが真でない場合、\ADRNE は $a$ がすでに0に等しいとチェックされているので、
\IT{<<something unknown \textbackslash{}n>>}文字列へのポインタを\Reg{0}にロードします 
または1であり、この時点で $a$ 変数がこれらの数値と等しくないことがわかります。
$R0=2$ の場合、文字列 \IT{<<two\textbackslash{}n>>} へのポインタは \ADREQ によって\Reg{0}にロードされます。

\subsubsection{ARM: \OptimizingKeilVI (\ThumbMode)}

\lstinputlisting[style=customasmARM]{patterns/08_switch/1_few/few_ARM_thumb_O3.asm}

% FIXME а каким можно? к каким нельзя? \myref{} ->

既に言及したように、Thumbモードのほとんどの命令に条件付き述語を追加することはできないため、
ここのThumbコードは、わかりやすいx86 \ac{CISC}スタイルコードと多少似ています。

\subsubsection{ARM64: \NonOptimizing GCC (Linaro) 4.9}

\lstinputlisting[style=customasmARM]{patterns/08_switch/1_few/ARM64_GCC_O0_JPN.lst}

入力値のタイプは \Tint なので、 \RegX{0} レジスタ全体ではなくレジスタ \RegW{0} が使用されます。

文字列ポインタは\INS{ADRP}/\INS{ADD}命令ペアを使用して\q{\HelloWorldSectionName}
の例~\myref{pointers_ADRP_and_ADD}と同じように \puts に渡されます。

\subsubsection{ARM64: \Optimizing GCC (Linaro) 4.9}

\lstinputlisting[style=customasmARM]{patterns/08_switch/1_few/ARM64_GCC_O3_JPN.lst}

より最適化されたコード。
\Reg{0}がゼロの場合、\TT{CBZ}(\IT{Compare and Branch on Zero})命令はジャンプします。
また、\myref{JMP_instead_of_RET}の前に説明したように、 \puts を呼び出す代わりに直接ジャンプすることもできます。
}

\EN{\subsubsection{MIPS}
% FIXME better start at non-optimizing version?

The function uses a lot of S- registers which must be preserved, so that's why its 
values are saved in the function prologue and restored in the epilogue.

\lstinputlisting[caption=\Optimizing GCC 4.4.5 (IDA),style=customasmMIPS]{patterns/13_arrays/1_simple/MIPS_O3_IDA_EN.lst}

Something interesting: there are two loops and the first one doesn't need $i$, it needs only 
$i*2$ (increased by 2 at each iteration) and also the address in memory (increased by 4 at each iteration).

So here we see two variables, one (in \$V0) increasing by 2 each time, and another (in \$V1) --- by 4.

The second loop is where \printf is called and it reports the value of $i$ to the user, 
so there is a variable
which is increased by 1 each time (in \$S0) and also a memory address (in \$S1) increased by 4 each time.

That reminds us of loop optimizations: \myref{loop_iterators}.

Their goal is to get rid of multiplications.

}
\RU{\subsubsection{MIPS}
% FIXME better start at non-optimizing version?
Функция использует много S-регистров, которые должны быть сохранены. Вот почему их значения сохраняются
в прологе функции и восстанавливаются в эпилоге.

\lstinputlisting[caption=\Optimizing GCC 4.4.5 (IDA),style=customasmMIPS]{patterns/13_arrays/1_simple/MIPS_O3_IDA_RU.lst}

Интересная вещь: здесь два цикла и в первом не нужна переменная $i$, а нужна только переменная
$i*2$ (скачущая через 2 на каждой итерации) и ещё адрес в памяти (скачущий через 4 на каждой итерации).

Так что мы видим здесь две переменных: одна (в \$V0) увеличивается на 2 каждый раз, и вторая (в \$V1) --- на 4.

Второй цикл содержит вызов \printf. Он должен показывать значение $i$ пользователю,
поэтому здесь есть переменная, увеличивающаяся на 1 каждый раз (в \$S0), а также адрес в памяти (в \$S1) 
увеличивающийся на 4 каждый раз.

Это напоминает нам оптимизацию циклов: \myref{loop_iterators}.
Цель оптимизации в том, чтобы избавиться от операций умножения.

}
\DE{\subsubsection{MIPS}
% FIXME better start at non-optimizing version?
Die Funktion verwendet eine Menge S-Register, die gesichert werden müssen. Das ist der Grund dafür, dass die Werte im
Funktionsprolog gespeichert und im Funktionsepilog wiederhergestellt werden.

\lstinputlisting[caption=\Optimizing GCC 4.4.5
(IDA),style=customasmMIPS]{patterns/13_arrays/1_simple/MIPS_O3_IDA_DE.lst}
Interessant: es gibt zwei Schleifen und die erste benötigt $i$ nicht; sie benötigt nur $i\cdot 2$ (erhöht um 2 bei
jedem Iterationsschritt) und die Adresse im Speicher (erhöht um 4 bei jedem Iterationsschritt).

Wir sehen hier also zwei Variablen: eine (in \$V0), die jedes Mal um 2 erhöht wird, und eine andere (in\$V1), die um 4
erhöht wird.

Die zweite Schleife ist der Ort, an dem \printf aufgerufen wird und dem Benutzer den Wert von $i$ zurückliefert, es gibt
also eine Variable die in \$S0 inkrementiert wird und eine Speicheradresse in \$S1, die jedes Mal um 4 erhöht wird.

% TBT
Das erinnert uns an die Optimierung von Schleifen, die wir früher betrachtet haben: \myref{loop_iterators}.

Das Ziel der Optimierung ist es, die Multiplikationen loszuwerden.
}
\FR{\subsubsection{MIPS}

MIPS peut supporter plusieurs coprocesseurs (jusqu'à 4), le zérotième\footnote{Barbarisme
pour rappeler que les indices commencent à zéro.} est un coprocesseur contrôleur
spécial, et celui d'indice 1 est le FPU.

Comme en ARM, le coprocesseur MIPS n'est pas une machine à pile, il comprend 32 registres
32-bit (\$F0-\$F31):
\myref{MIPS_FPU_registers}.

Lorsque l'on doit travailler avec des valeurs \Tdouble 64-bit, une paire de F-registres
32-bit est utilisée.

\lstinputlisting[caption=GCC 4.4.5 \Optimizing (IDA),style=customasmMIPS]{patterns/12_FPU/1_simple/MIPS_O3_IDA_FR.lst}

Les nouvelles instructions ici sont:

\myindex{MIPS!\Instructions!LWC1}
\myindex{MIPS!\Instructions!DIV.D}
\myindex{MIPS!\Instructions!MUL.D}
\myindex{MIPS!\Instructions!ADD.D}
\begin{itemize}

\item \INS{LWC1} charge un mot de 32-bit dans un registre du premier coprocesseur
(d'où le \q{1} dans le nom de l'instruction).
\myindex{MIPS!\Pseudoinstructions!L.D}

Une paire d'instructions \INS{LWC1} peut être combinée en une pseudo instruction \INS{L.D}.

\item \INS{DIV.D}, \INS{MUL.D}, \INS{ADD.D} effectuent respectivement la division,
la multiplication, et l'addition (\q{.D} est le suffixe standard pour la double précision,
\q{.S} pour la simple précision)

\end{itemize}

\myindex{MIPS!\Instructions!LUI}
\myindex{\CompilerAnomaly}
\label{MIPS_FPU_LUI}

Il y a une anomalie bizarre du compilateur: l'instruction \INS{LUI} que nous avons
marqué avec un point d'interrogation.
Il m'est difficile de comprendre pourquoi charger une partie de la constante de type
64-bit \Tdouble dans le registre \$V0. Cette instruction n'a pas d'effet.
% TODO did you try checking out compiler source code?
Si quelqu'un en sait plus sur ceci, s'il vous plaît, envoyez moi un email\footnote{\EMAIL}.

}
\ITA{\input{patterns/08_switch/1_few/MIPS_ITA}}
\JPN{\subsubsection{MIPS}

\lstinputlisting[caption=\Optimizing GCC 4.4.5 (IDA),style=customasmMIPS]{patterns/08_switch/1_few/MIPS_O3_IDA_JPN.lst}

\myindex{MIPS!\Instructions!JR}

関数は常に \puts を呼び出すことで終了するので、\puts(\INS{JR}:\q{Jump Register})へのジャンプは\q{jump and link}ではなく、ここにあります。
私たちは以前これについて話しました:\myref{JMP_instead_of_RET}

\myindex{MIPS!Load delay slot}
\INS{LW}命令の後に\INS{NOP}命令も表示されることがよくあります。
これは\q{load delay slot}:MIPSの別の\IT{delay slot}です。
\myindex{MIPS!\Instructions!LW}

\INS{LW}がメモリから値をロードする間に、\INS{LW}の次の命令が実行されることがあります。

ただし、次の命令は\INS{LW}の結果を使用してはなりません。

現代のMIPS CPUは、次の命令が\INS{LW}の結果を使用するのを待つ機能を持っているので、これは幾分時代遅れですが、
GCCは古いMIPS CPU用にNOPを追加します。
一般に、無視することができます。
}

\subsubsection{\Conclusion{}}

\EN{A \IT{switch()} with few cases is indistinguishable from an \IT{if/else} construction, for example:}
\RU{Оператор \IT{switch()} с малым количеством вариантов трудно отличим от применения конструкции \IT{if/else}:}
\DEph{}
\FR{Un \IT{switch()} avec peu de cas est indistinguable d'une construction avec \IT{if/else}, par exemple:}
\ITA{Uno \IT{switch()} con un piccolo numero di casi è indistinguibile da un costrutto \IT{if/else} , per esempio:}
\JPN{ほとんどの場合switch()はif / else構造と区別できません:}
\lstref{switch_few_ifelse}.



\EN{\mysection{Returning Values}
\label{ret_val_func}

Another simple function is the one that simply returns a constant value:

\lstinputlisting[caption=\EN{\CCpp Code},style=customc]{patterns/011_ret/1.c}

Let's compile it.

\subsection{x86}

Here's what both the GCC and MSVC compilers produce (with optimization) on the x86 platform:

\lstinputlisting[caption=\Optimizing GCC/MSVC (\assemblyOutput),style=customasmx86]{patterns/011_ret/1.s}

\myindex{x86!\Instructions!RET}
There are just two instructions: the first places the value 123 into the \EAX register,
which is used by convention for storing the return
value, and the second one is \RET, which returns execution to the \gls{caller}.

The caller will take the result from the \EAX register.

\subsection{ARM}

There are a few differences on the ARM platform:

\lstinputlisting[caption=\OptimizingKeilVI (\ARMMode) ASM Output,style=customasmARM]{patterns/011_ret/1_Keil_ARM_O3.s}

ARM uses the register \Reg{0} for returning the results of functions, so 123 is copied into \Reg{0}.

\myindex{ARM!\Instructions!MOV}
\myindex{x86!\Instructions!MOV}
It is worth noting that \MOV is a misleading name for the instruction in both the x86 and ARM \ac{ISA}s.

The data is not in fact \IT{moved}, but \IT{copied}.

\subsection{MIPS}

\label{MIPS_leaf_function_ex1}

The GCC assembly output below lists registers by number:

\lstinputlisting[caption=\Optimizing GCC 4.4.5 (\assemblyOutput),style=customasmMIPS]{patterns/011_ret/MIPS.s}

\dots while \IDA does it by their pseudo names:

\lstinputlisting[caption=\Optimizing GCC 4.4.5 (IDA),style=customasmMIPS]{patterns/011_ret/MIPS_IDA.lst}

The \$2 (or \$V0) register is used to store the function's return value.
\myindex{MIPS!\Pseudoinstructions!LI}
\INS{LI} stands for ``Load Immediate'' and is the MIPS equivalent to \MOV.

\myindex{MIPS!\Instructions!J}
The other instruction is the jump instruction (J or JR) which returns the execution flow to the \gls{caller}.

\myindex{MIPS!Branch delay slot}
You might be wondering why the positions of the load instruction (LI) and the jump instruction (J or JR) are swapped. This is due to a \ac{RISC} feature called ``branch delay slot''.

The reason this happens is a quirk in the architecture of some RISC \ac{ISA}s and isn't important for our
purposes---we must simply keep in mind that in MIPS, the instruction following a jump or branch instruction
is executed \IT{before} the jump/branch instruction itself.

As a consequence, branch instructions always swap places with the instruction executed immediately beforehand.

In practice, functions which merely return 1 (\IT{true}) or 0 (\IT{false}) are very frequent.

The smallest ever of the standard UNIX utilities, \IT{/bin/true} and \IT{/bin/false} return 0 and 1 respectively, as an exit code.
(Zero as an exit code usually means success, non-zero means error.)
}
\RU{\mysection{Оптимизации циклов}

% subsections:
\subsection{Странная оптимизация циклов}

Это самая простая (из всех возможных) реализация memcpy():

\begin{lstlisting}[style=customc]
void memcpy (unsigned char* dst, unsigned char* src, size_t cnt)
{
	size_t i;
	for (i=0; i<cnt; i++)
		dst[i]=src[i];
};
\end{lstlisting}

Как минимум MSVC 6.0 из конца 90-х вплоть до MSVC 2013 может выдавать вот такой странный код (этот листинг создан MSVC 2013
x86):

\lstinputlisting[style=customasmx86]{advanced/500_loop_optimizations/1_1_RU.lst}

Это всё странно, потому что как люди работают с двумя указателями? Они сохраняют два адреса в двух регистрах или двух
ячейках памяти.
Компилятор MSVC в данном случае сохраняет два указателя как один указатель (\IT{скользящий dst} в \EAX)
и разницу между указателями \IT{src} и \IT{dst} (она остается неизменной во время исполнения цикла, в \ESI).
\myindex{\CLanguageElements!ptrdiff\_t}
(Кстати, это тот редкий случай, когда можно использовать тип ptrdiff\_t.)
Когда нужно загрузить байт из \IT{src}, он загружается на \IT{diff + скользящий dst} и сохраняет байт просто на
\IT{скользящем dst}.

Должно быть это какой-то трюк для оптимизации. Но я переписал эту ф-цию так:

\lstinputlisting[style=customasmx86]{advanced/500_loop_optimizations/1_2.lst}

\dots и она работает также быстро как и \IT{соптимизированная} версия на моем Intel Xeon E31220 @ 3.10GHz.
Может быть, эта оптимизация предназначалась для более старых x86-процессоров 90-х, т.к., этот трюк использует
как минимум древний MS VC 6.0?

Есть идеи?

\myindex{Hex-Rays}
Hex-Rays 2.2 не распознает такие шаблонные фрагменты кода (будем надеятся, это временно?):

\begin{lstlisting}[style=customc]
void __cdecl f1(char *dst, char *src, size_t size)
{
  size_t counter; // edx@1
  char *sliding_dst; // eax@2
  char tmp; // cl@3

  counter = size;
  if ( size )
  {
    sliding_dst = dst;
    do
    {
      tmp = (sliding_dst++)[src - dst];         // разница (src-dst) вычисляется один раз, перед телом цикла
      *(sliding_dst - 1) = tmp;
      --counter;
    }
    while ( counter );
  }
}
\end{lstlisting}

Тем не менее, этот трюк часто используется в MSVC (и не только в самодельных ф-циях \IT{memcpy()}, но также и во многих
циклах, работающих с двумя или более массивами), так что для реверс-инжиниров стоит помнить об этом.

% <!-- As of why writting occurred after <b>dst</b> incrementing? -->


\subsection{Возврат строки}

Классическая ошибка из \RobPikePractice{}:

\begin{lstlisting}[style=customc]
#include <stdio.h>

char* amsg(int n, char* s)
{
        char buf[100];

        sprintf (buf, "error %d: %s\n", n, s) ;

        return buf;
};

int main()
{
        printf ("%s\n", amsg (1234, "something wrong!"));
};
\end{lstlisting}

Она упадет.
В начале, попытаемся понять, почему.

Это состояние стека перед возвратом из amsg():

% FIXME! TikZ or whatever
\begin{lstlisting}
§(низкие адреса)§

§[amsg(): 100 байт]§
§[RA]                               <- текущий SP§
§[два аргумента amsg]§
§[что-то еще]§
§[локальные переменные main()]§

§(высокие адреса)§
\end{lstlisting}

Когда управление возвращается из amsg() в \main, пока всё хорошо.
Но когда \printf вызывается из \main, который, в свою очередь, использует стек для своих нужд, затирая 100-байтный буфер.
В лучшем случае, будет выведен случайный мусор.

Трудно поверить, но я знаю, как это исправить:

\begin{lstlisting}[style=customc]
#include <stdio.h>

char* amsg(int n, char* s)
{
        char buf[100];

        sprintf (buf, "error %d: %s\n", n, s) ;

        return buf;
};

char* interim (int n, char* s)
{
        char large_buf[8000];
        // используем локальный массив.
        // а иначе компилятор выбросит его при оптимизации, как неиспользуемый.
        large_buf[0]=0;
        return amsg (n, s);
};

int main()
{
        printf ("%s\n", interim (1234, "something wrong!"));
};
\end{lstlisting}

Это заработает если скомпилировано в MSVC 2013 без оптимизаций и с опцией \TT{/GS-}\footnote{Выключить защиту от переполнения буфера}.
MSVC предупредит: ``warning C4172: returning address of local variable or temporary'', но код запустится и сообщение выведется.
Посмотрим состояние стека в момент, когда amsg() возвращает управление в interim():

\begin{lstlisting}
§(низкие адреса)§

§[amsg(): 100 байт]§
§[RA]                                      <- текущий SP§
§[два аргумента amsg()]§
§[вледения interim(), включая 8000 байт]§
§[еще что-то]§
§[локальные переменные main()]§

§(высокие адреса)§
\end{lstlisting}

Теперь состояние стека на момент, когда interim() возвращает управление в \main{}:

\begin{lstlisting}
§(низкие адреса)§

§[amsg(): 100 байт]§
§[RA]§
§[два аргумента amsg()]§
§[вледения interim(), включая 8000 байт]§
§[еще что-то]                              <- текущий SP§
§[локальные переменные main()]§

§(высокие адреса)§
\end{lstlisting}

Так что когда \main вызывает \printf, он использует стек в месте, где выделен буфер в interim(),
и не затирает 100 байт с сообщение об ошибке внутри, потому что 8000 байт (или может быть меньше) это достаточно для всего,
что делает \printf и другие нисходящие ф-ции!

Это также может сработать, если между ними много ф-ций, например:
\main $\rightarrow$ f1() $\rightarrow$ f2() $\rightarrow$ f3() ... $\rightarrow$ amsg(),
и тогда результат amsg() используется в \main.
Дистанция между \ac{SP} в \main и адресом буфера \TT{buf[]} должна быть достаточно длинной.

Вот почему такие ошибки опасны: иногда ваш код работает (и бага прячется незамеченной). иногда нет.
\label{heisenbug}
\myindex{Хейзенбаги}
Такие баги в шутку называют хейзенбаги или шрёдинбаги\footnote{\url{https://en.wikipedia.org/wiki/Heisenbug}}.



}
\DE{\subsection{Gesetzte Bits zählen}
Hier ist ein einfaches Beispiel einer Funktion, die die Anzahl der gesetzten
Bits in einem Eingabewert zählt.

Diese Operation wird auch \q{population count}\footnote{moderne x86 CPUs
(die SSE4 unterstützen) haben zu diesem Zweck sogar einen eigenen POPCNT Befehl}
genannt.

\lstinputlisting[style=customc]{patterns/14_bitfields/4_popcnt/shifts.c}
In dieser Schleife wird der Wert von $i$ schrittweise von 0 bis 31 erhöht,
sodass der Ausdruck $1 \ll i$ von 1 bis \TT{0x80000000} zählt.
In natürlicher Sprache würden wir diese Operation als \IT{verschiebe 1 um n
Bits nach links} beschreiben.
Mit anderen Worten: Der Ausdruck $1 \ll i$ erzeugt alle möglichen Bitpositionen
in einer 32-Bit-Zahl.
Das freie Bit auf der rechten Seite wird jeweils gelöscht.

\label{2n_numbers_table}
Hier ist eine Tabelle mit allen Werten von $1 \ll i$ 
für $i=0 \ldots 31$:

\small
\begin{center}
\begin{tabular}{ | l | l | l | l | }
\hline
\HeaderColor \CCpp Ausdruck & 
\HeaderColor Zweierpotenz & 
\HeaderColor Dezimalzahl & 
\HeaderColor Hexadezimalzahl \\
\hline
$1 \ll 0$ & $2^{0}$ & 1 & 1 \\
\hline
$1 \ll 1$ & $2^{1}$ & 2 & 2 \\
\hline
$1 \ll 2$ & $2^{2}$ & 4 & 4 \\
\hline
$1 \ll 3$ & $2^{3}$ & 8 & 8 \\
\hline
$1 \ll 4$ & $2^{4}$ & 16 & 0x10 \\
\hline
$1 \ll 5$ & $2^{5}$ & 32 & 0x20 \\
\hline
$1 \ll 6$ & $2^{6}$ & 64 & 0x40 \\
\hline
$1 \ll 7$ & $2^{7}$ & 128 & 0x80 \\
\hline
$1 \ll 8$ & $2^{8}$ & 256 & 0x100 \\
\hline
$1 \ll 9$ & $2^{9}$ & 512 & 0x200 \\
\hline
$1 \ll 10$ & $2^{10}$ & 1024 & 0x400 \\
\hline
$1 \ll 11$ & $2^{11}$ & 2048 & 0x800 \\
\hline
$1 \ll 12$ & $2^{12}$ & 4096 & 0x1000 \\
\hline
$1 \ll 13$ & $2^{13}$ & 8192 & 0x2000 \\
\hline
$1 \ll 14$ & $2^{14}$ & 16384 & 0x4000 \\
\hline
$1 \ll 15$ & $2^{15}$ & 32768 & 0x8000 \\
\hline
$1 \ll 16$ & $2^{16}$ & 65536 & 0x10000 \\
\hline
$1 \ll 17$ & $2^{17}$ & 131072 & 0x20000 \\
\hline
$1 \ll 18$ & $2^{18}$ & 262144 & 0x40000 \\
\hline
$1 \ll 19$ & $2^{19}$ & 524288 & 0x80000 \\
\hline
$1 \ll 20$ & $2^{20}$ & 1048576 & 0x100000 \\
\hline
$1 \ll 21$ & $2^{21}$ & 2097152 & 0x200000 \\
\hline
$1 \ll 22$ & $2^{22}$ & 4194304 & 0x400000 \\
\hline
$1 \ll 23$ & $2^{23}$ & 8388608 & 0x800000 \\
\hline
$1 \ll 24$ & $2^{24}$ & 16777216 & 0x1000000 \\
\hline
$1 \ll 25$ & $2^{25}$ & 33554432 & 0x2000000 \\
\hline
$1 \ll 26$ & $2^{26}$ & 67108864 & 0x4000000 \\
\hline
$1 \ll 27$ & $2^{27}$ & 134217728 & 0x8000000 \\
\hline
$1 \ll 28$ & $2^{28}$ & 268435456 & 0x10000000 \\
\hline
$1 \ll 29$ & $2^{29}$ & 536870912 & 0x20000000 \\
\hline
$1 \ll 30$ & $2^{30}$ & 1073741824 & 0x40000000 \\
\hline
$1 \ll 31$ & $2^{31}$ & 2147483648 & 0x80000000 \\
\hline
\end{tabular}
\end{center}
\normalsize
Diese Konstanten (Bitmasken) tauchen im Code oft auf und ein Reverse Engineer
muss in der Lage sein, sie schnell und sicher zu erkennen.

% TBT
Es dazu jedoch nicht notwendig, die Dezimalzahlen (Zweierpotenzen) größer
65535 auswendig zu kennen. Die hexadezimalen Zahlen sind leicht zu merken.

Die Konstanten werden häufig verwendet um Flags einzelnen Bits zuzuordnen. 
Hier ist zum Beispiel ein Auszug aus \TT{ssl\_private.h} aus dem Quellcode von
Apache 2.4.6:

\begin{lstlisting}[style=customc]
/**
 * Define the SSL options
 */
#define SSL_OPT_NONE           (0)
#define SSL_OPT_RELSET         (1<<0)
#define SSL_OPT_STDENVVARS     (1<<1)
#define SSL_OPT_EXPORTCERTDATA (1<<3)
#define SSL_OPT_FAKEBASICAUTH  (1<<4)
#define SSL_OPT_STRICTREQUIRE  (1<<5)
#define SSL_OPT_OPTRENEGOTIATE (1<<6)
#define SSL_OPT_LEGACYDNFORMAT (1<<7)
\end{lstlisting}

Zurück zu unserem Beispiel.

Das Makro \TT{IS\_SET} prüft auf Anwesenheit von Bits in $a$.
\myindex{x86!\Instructions!AND}

Das Makro \TT{IS\_SET} entspricht dabei dem logischen (\IT{AND})
und gibt 0 zurück, wenn das entsprechende Bit nicht gesetzt ist, oder die
Bitmaske, wenn das Bit gesetzt ist.
Der Operator \IT{if()} wird in \CCpp ausgeführt, wenn der boolesche Ausdruck
nicht null ist (er könnte sogar 123456 sein), weshalb es meistens richtig
funktioniert.


% subsections
\subsubsection{x86}

\myparagraph{MSVC}

Kompilieren wir das Beispiel:

\lstinputlisting[caption=MSVC 2008,style=customasmx86]{patterns/13_arrays/1_simple/simple_msvc.asm}

\myindex{x86!\Instructions!SHL}
Soweit nichts Außergewöhnliches, nur zwei Schleifen: die erste füllt mit Werten auf und die zweite gibt Werte aus.
% TBT
Der Befehl \TT{shl ecx, 1} wird für die Multiplikation mit 2 in \ECX verwendet; mehr dazu unten~\myref{SHR}.

Auf dem Stack werden 80 Bytes für das Array reserviert: 20 Elemente von je 4 Byte.

\clearpage
Untersuchen wir dieses Beispiel in \olly.
\myindex{\olly}

Wir erkennen wie das Array befüllt wird:

jedes Element ist ein 32-Bit-Wort vom Typ \Tint und der Wert ist der Index multipliziert mit 2:

\begin{figure}[H]
\centering
\myincludegraphics{patterns/13_arrays/1_simple/olly.png}
\caption{\olly: nach dem Füllen des Arrays}
\label{fig:array_simple_olly}
\end{figure}
Da sich dieses Array auf dem Stack befindet, finden wir dort alle seine 20 Elemente.

\myparagraph{GCC}

Hier ist was GCC 4.4.1 erzeugt:

\lstinputlisting[caption=GCC 4.4.1,style=customasmx86]{patterns/13_arrays/1_simple/simple_gcc.asm}
Die Variable $a$ ist übrigens vom Typ \IT{int*} (Pointer auf \Tint{})--man kann einen Pointer auf ein Array an eine
andere Funktion übergeben, aber es ist richtiger zu sagen, dass der Pointer auf das erste Element des Arrays übergeben
wird. (Die Adressen der übrigen Elemente werden in bekannter Weise berechnet.)

Wenn man diesen Pointer mittels \IT{a[idx]} indiziert, wird \IT{idx} zum Pointer addiert und das dort abgelegte Element
(auf das der berechnete Pointer zeigt) wird zurückgegeben.

Ein interessantes Beispiel: ein String wie \IT{\q{string}} ist ein Array von Chars und hat den Typ \IT{const
char[]}.

Auch auf diesen Pointer kann ein Index angewendet werden.

Das ist der Grund warum es es möglich ist, Dinge wie \TT{\q{string}[i]} zu schreiben--es handelt sich dabei um einen
korrekten \CCpp Ausdruck!


\input{patterns/14_bitfields/4_popcnt/x64_DE}
\subsubsection{ARM}

\myparagraph{\OptimizingKeilVI (\ThumbMode)}

\lstinputlisting[style=customasmARM]{patterns/04_scanf/1_simple/ARM_IDA.lst}

\myindex{\CLanguageElements!\Pointers}
Damit \scanf Elemente einlesen kann, benötigt die Funktion einen Paramter--einen Pointer vom Typ \Tint.
\Tint hat die Größe 32 Bit, wir benötigen also 4 Byte, um den Wert im Speicher abzulegen, und passt daher genau in ein 32-Bit-Register.
\myindex{IDA!var\_?}
Auf dem Stack wird Platz für die lokalen Variable \GTT{x} reserviert und IDA bezeichnet diese Variable mit \IT{var\_8}. 
Eigentlich ist aber an dieser Stelle gar nicht notwendig, Platz auf dem Stack zu reservieren, da \ac{SP} (\gls{stack pointer} 
bereits auf die Adresse zeigt und auch direkt verwendet werden kann.

Der Wert von \ac{SP} wird also in das \Reg{1} Register kopiert und zusammen mit dem Formatierungsstring an \scanf übergeben.

% TBT here
%\INS{PUSH/POP} instructions behaves differently in ARM than in x86 (it's the other way around).
%They are synonyms to \INS{STM/STMDB/LDM/LDMIA} instructions.
%And \INS{PUSH} instruction first writes a value into the stack, \IT{and then} subtracts \ac{SP} by 4.
%\INS{POP} first adds 4 to \ac{SP}, \IT{and then} reads a value from the stack.
%Hence, after \INS{PUSH}, \ac{SP} points to an unused space in stack.
%It is used by \scanf, and by \printf after.

%\INS{LDMIA} means \IT{Load Multiple Registers Increment address After each transfer}.
%\INS{STMDB} means \IT{Store Multiple Registers Decrement address Before each transfer}.

\myindex{ARM!\Instructions!LDR}
Später wird mithilfe des \INS{LDR} Befehls dieser Wert vom Stack in das \Reg{1} Register verschoben um an \printf übergeben werden zu können.

\myparagraph{ARM64}

\lstinputlisting[caption=\NonOptimizing GCC 4.9.1 ARM64,numbers=left,style=customasmARM]{patterns/04_scanf/1_simple/ARM64_GCC491_O0_DE.s}

Im Stack Frame werden 32 Byte reserviert, was deutlich mehr als benötigt ist. Vielleicht handelt es sich um eine Frage des Aligning (dt. Angleichens) von Speicheradressen.
Der interessanteste Teil ist, im Stack Frame einen Platz für die Variable $x$ zu finden (Zeile 22).
Warum 28? Irgendwie hat der Compiler entschieden die Variable am Ende des Stack Frames anstatt an dessen Beginn abzulegen.
Die Adresse wird an \scanf übergeben; diese Funktion speichert den Userinput an der genannten Adresse im Speicher.
Es handelt sich hier um einen 32-Bit-Wert vom Typ \Tint. 
Der Wert wird in Zeile 27 abgeholt und dann an \printf übergeben.



\subsubsection{MIPS}
% FIXME better start at non-optimizing version?
Die Funktion verwendet eine Menge S-Register, die gesichert werden müssen. Das ist der Grund dafür, dass die Werte im
Funktionsprolog gespeichert und im Funktionsepilog wiederhergestellt werden.

\lstinputlisting[caption=\Optimizing GCC 4.4.5
(IDA),style=customasmMIPS]{patterns/13_arrays/1_simple/MIPS_O3_IDA_DE.lst}
Interessant: es gibt zwei Schleifen und die erste benötigt $i$ nicht; sie benötigt nur $i\cdot 2$ (erhöht um 2 bei
jedem Iterationsschritt) und die Adresse im Speicher (erhöht um 4 bei jedem Iterationsschritt).

Wir sehen hier also zwei Variablen: eine (in \$V0), die jedes Mal um 2 erhöht wird, und eine andere (in\$V1), die um 4
erhöht wird.

Die zweite Schleife ist der Ort, an dem \printf aufgerufen wird und dem Benutzer den Wert von $i$ zurückliefert, es gibt
also eine Variable die in \$S0 inkrementiert wird und eine Speicheradresse in \$S1, die jedes Mal um 4 erhöht wird.

% TBT
Das erinnert uns an die Optimierung von Schleifen, die wir früher betrachtet haben: \myref{loop_iterators}.

Das Ziel der Optimierung ist es, die Multiplikationen loszuwerden.

}
\FR{\mysection{\Stack}
\label{sec:stack}
\myindex{\Stack}

La pile est une des structures de données les plus fondamentales en informatique
\footnote{\href{http://go.yurichev.com/17119}{wikipedia.org/wiki/Call\_stack}}.
\ac{AKA} \ac{LIFO}.

Techniquement, il s'agit d'un bloc de mémoire situé dans l'espace d'adressage
d'un processus et qui est utilisé par le registre \ESP en x86, \RSP en x64
ou par le registre \ac{SP} en ARM comme un pointeur dans ce bloc mémoire.

\myindex{ARM!\Instructions!PUSH}
\myindex{ARM!\Instructions!POP}
\myindex{x86!\Instructions!PUSH}
\myindex{x86!\Instructions!POP}
Les instructions d'accès à la pile sont \PUSH et \POP (en x86 ainsi qu'en ARM Thumb-mode).
\PUSH soustrait à \ESP/\RSP/\ac{SP} 4 en mode 32-bit (ou 8 en mode 64-bit) et écrit
ensuite le contenu de l'opérande associé à l'adresse mémoire pointée par \ESP/\RSP/\ac{SP}.

\POP est l'opération inverse: elle récupère la donnée depuis l'adresse mémoire pointée par \ac{SP},
l'écrit dans l'opérande associé (souvent un registre) puis ajoute 4 (ou 8) au \glslink{stack pointer}{pointeur de pile}.

Après une allocation sur la pile, le \glslink{stack pointer}{pointeur de pile} pointe sur le bas de la pile.
\PUSH décrémente le \glslink{stack pointer}{pointeur de pile} et \POP l'incrémente.

Le bas de la pile représente en réalité le début de la mémoire allouée pour
le bloc de pile. Cela semble étrange, mais c'est comme ça.

ARM supporte à la fois les piles ascendantes et descendantes.

\myindex{ARM!\Instructions!STMFD}
\myindex{ARM!\Instructions!LDMFD}
\myindex{ARM!\Instructions!STMED}
\myindex{ARM!\Instructions!LDMED}
\myindex{ARM!\Instructions!STMFA}
\myindex{ARM!\Instructions!LDMFA}
\myindex{ARM!\Instructions!STMEA}
\myindex{ARM!\Instructions!LDMEA}

Par exemple les instructions \ac{STMFD}/\ac{LDMFD}, \ac{STMED}/\ac{LDMED} sont utilisées pour gérer les piles
descendantes (qui grandissent vers le bas en commençant avec une adresse haute et évoluent vers une plus basse).

Les instructions \ac{STMFA}/\ac{LDMFA}, \ac{STMEA}/\ac{LDMEA} sont utilisées pour gérer les piles montantes
(qui grandissent vers les adresses hautes de l'espace d'adressage, en commençant
avec une adresse située en bas de l'espace d'adressage).

% It might be worth mentioning that STMED and STMEA write first,
% and then move the pointer,
% and that LDMED and LDMEA move the pointer first, and then read.
% In other words, ARM not only lets the stack grow in a non-standard direction,
% but also in a non-standard order.
% Maybe this can be in the glossary, which would explain why E stands for "empty".

\subsection{Pourquoi la pile grandit en descendant ?}
\label{stack_grow_backwards}

Intuitivement, on pourrait penser que la pile grandit vers le haut, i.e. vers des
adresses plus élevées, comme n'importe qu'elle autre structure de données.

La raison pour laquelle la pile grandit vers le bas est probablement historique.
Dans le passé, les ordinateurs étaient énormes et occupaient des pièces entières,
il était facile de diviser la mémoire en deux parties, une pour le \gls{heap} et
une pour la pile.
Évidemment, on ignorait quelle serait la taille du \gls{heap} et de la pile durant
l'exécution du programme, donc cette solution était la plus simple possible.

\input{patterns/02_stack/stack_and_heap}

Dans \RitchieThompsonUNIX on peut lire:

\begin{framed}
\begin{quotation}
The user-core part of an image is divided into three logical segments. The program text segment begins at location 0 in the virtual address space. During execution, this segment is write-protected and a single copy of it is shared among all processes executing the same program. At the first 8K byte boundary above the program text segment in the virtual address space begins a nonshared, writable data segment, the size of which may be extended by a system call. Starting at the highest address in the virtual address space is a pile segment, which automatically grows downward as the hardware's pile pointer fluctuates.
\end{quotation}
\end{framed}

Cela nous rappelle comment certains étudiants prennent des notes pour deux cours différents dans
un seul et même cahier en prenant un cours d'un côté du cahier, et l'autre cours de l'autre côté.
Les notes de cours finissent par se rencontrer à un moment dans le cahier quand il n'y a plus de place.

% I think if we want to expand on this analogy,
% one might remember that the line number increases as as you go down a page.
% So when you decrease the address when pushing to the stack, visually,
% the stack does grow upwards.
% Of course, the problem is that in most human languages,
% just as with computers,
% we write downwards, so this direction is what makes buffer overflows so messy.

\subsection{Quel est le rôle de la pile ?}

% subsections
\input{patterns/02_stack/01_saving_ret_addr_FR}
\input{patterns/02_stack/02_args_passing_FR}
\input{patterns/02_stack/03_local_vars_FR}
\mysection{\oracle}
\label{oracle}

% sections
\EN{\input{examples/oracle/1_version_EN}}\RU{\input{examples/oracle/1_version_RU}}
\EN{\input{examples/oracle/2_ksmlru_EN}}\RU{\input{examples/oracle/2_ksmlru_RU}}
\EN{\input{examples/oracle/3_timer_EN}}\RU{\input{examples/oracle/3_timer_RU}}


\input{patterns/02_stack/05_SEH}
\input{patterns/02_stack/06_BO_protection}

\subsubsection{Dé-allocation automatique de données dans la pile}

Peut-être que la raison pour laquelle les variables locales et les enregistrements SEH sont stockés dans la
pile est qu'ils sont automatiquement libérés quand la fonction se termine en utilisant simplement une
instruction pour corriger la position du pointeur de pile (souvent \ADD).
Les arguments de fonction sont aussi désalloués automatiquement à la fin de la fonction.
À l'inverse, toutes les données allouées sur le \IT{heap} doivent être désallouées de façon explicite.

% sections
\input{patterns/02_stack/07_layout_FR}
\mysection{\oracle}
\label{oracle}

% sections
\EN{\input{examples/oracle/1_version_EN}}\RU{\input{examples/oracle/1_version_RU}}
\EN{\input{examples/oracle/2_ksmlru_EN}}\RU{\input{examples/oracle/2_ksmlru_RU}}
\EN{\input{examples/oracle/3_timer_EN}}\RU{\input{examples/oracle/3_timer_RU}}


\input{patterns/02_stack/exercises}
}
\ITA{\subsection{\IT{LD\_PRELOAD} hack in Linux}

\myindex{LD\_PRELOAD}
\label{ld_preload}

Questo ci permette di caricare le nostre librerie dinamiche prima delle altre, anche quelle del sistema, come libc.so.6.

Questo a sua volta ci permette di \q{sostituire} la funzione che abbiamo scritto con quella nelle librerie del sistema.
Ad esempio, è facile intercettare tutte le chiamate da  
time(), read(), write(), etc. \\
\\
\myindex{uptime}
Proviamo ad ingannare l'utility \IT{uptime}.
Come sappiamo, essa ci dice da quanto tempo il computer sta lavorando.
\myindex{strace}
Con l'aiuto di strace(\myref{strace}), è possibile osservare che l'utility prende questa informazione dal file \TT{/proc/uptime}:

\begin{lstlisting}
$ strace uptime 
...
open("/proc/uptime", O_RDONLY)          = 3
lseek(3, 0, SEEK_SET)                   = 0
read(3, "416166.86 414629.38\n", 2047)  = 20
...
\end{lstlisting}

Questo non è un file presente su disco ma uno virtuale, il suo contenuto è generato al volo nel Linux kernel.
Contiene due numeri:

\begin{lstlisting}
$ cat /proc/uptime
416690.91 415152.03
\end{lstlisting}

Da Wikipedia possiamo imparare che
\footnote{\href{http://go.yurichev.com/17043}{wikipedia}}:

\begin{framed}
\begin{quotation}
Il primo numero è il numero totale di secondi che il sistema è acceso.
Il secondo numero è la quantità di tempo che la macchina è rimasta in attesa (idle), in secondi.
\end{quotation}
\end{framed}

\myindex{\CStandardLibrary!open()}
\myindex{\CStandardLibrary!read()}
\myindex{\CStandardLibrary!close()}

Proviamo a scrivere la nostra libreria dinamica con le funzioni open(), read(), e close().

Come prima cosa, la funzione open() confronterà il nome del file da aprire con con quello che ci serve,
se l'esito è positivo, scriverà il descrittore del file aperto.

In secondo luogo, la funzione read(), se chiamata per tale descrittore del file, sostituirà l'output, altrimenti chiamerà 
la funzione read() originale dalla libreria libc.so.6.
E quindi la funzione close(), chiuderà il file che abbiamo utilizzato.

\myindex{dlopen()}
\myindex{dlsym()}

Useremo le funzioni dlopen() e dlsym() per determinare l'indirizzo della funzione originale in libc.so.6.
Dobbiamo usare queste funzioni per passare il controllo alla \q{vera} funzione.

\myindex{\CStandardLibrary!strcmp()}

D'altra parte, se intercettiamo la chiamata a strcmp() e monitoriamo ogni confronto tra stringhe
nel programma, dovremmo implementare una nostra versione di strcmp(), senza usare la funzione originale.
\footnote{Ad esempio, in questo articolo trovi quanto facilmente strcmp() riesce ad intercettare le chiamate
\footnote{\href{http://go.yurichev.com/17143}{yurichev.com}}
written by Yong Huang}

\lstinputlisting[style=customc]{OS/LD_PRELOAD/fool_uptime.c}
( \href{https://github.com/DennisYurichev/RE-for-beginners/blob/master/OS/LD_PRELOAD/fool_uptime.c}{Source code at GitHub} )
% FIXME go.yurichev.com...

Compiliamolo con librerie dinamiche comuni:

\begin{lstlisting}
gcc -fpic -shared -Wall -o fool_uptime.so fool_uptime.c -ldl
\end{lstlisting}

Avviamo \IT{uptime} caricando prima le nostre librerie:

\begin{lstlisting}
LD_PRELOAD=`pwd`/fool_uptime.so uptime
\end{lstlisting}

Osserviamo che:

\begin{lstlisting}
 01:23:02 up 24855 days,  3:14,  3 users,  load average: 0.00, 0.01, 0.05
\end{lstlisting}

Se la variabile d'ambiente \IT{LD\_PRELOAD} punta sempre al nome del file ed al percorso della nostra libreria, 
deve essere per forza avviato per tutti i programmi che andremo ad avviare. \\
\\
Altri esempi:

\begin{itemize}

\item
Semplice intercettazione della funzione strcmp() (Yong Huang) 
\url{http://go.yurichev.com/17143}

\item
Kevin Pulo---Fun with LD\_PRELOAD. Molti esempio ed idee.
\href{http://go.yurichev.com/17145}{yurichev.com}

\item
Funzioni che intercettano file, per la compressione/decompressione di file al volo (zlibc). \url{http://go.yurichev.com/17146}

\end{itemize}
}
\JPN{\subsection{A lot of cases}

\TT{switch()}ステートメントに大量のケースが含まれている場合、コンパイラが多くの \JE/\JNE 命令で大きすぎるコードを
出力することはあまり便利ではありません。

\lstinputlisting[label=switch_lot_c,style=customc]{patterns/08_switch/2_lot/lot.c}

\subsubsection{x86}

\myparagraph{\NonOptimizing MSVC}

We get (MSVC 2010):

\lstinputlisting[caption=MSVC 2010,style=customasmx86]{patterns/08_switch/2_lot/lot_msvc_JPN.asm}

\myindex{jumptable}

ここでは、さまざまな引数を持つ \printf 呼び出しのセットを見ていきます。
すべては、プロセスのメモリだけでなく、コンパイラによって割り当てられた内部シンボリックラベルも持っています。
これらのラベルはすべて \TT{\$LN11@f} 内部テーブルにも記載されています。

関数の開始時に、 $a$ が4より大きい場合、制御フローはラベル \TT{\$LN1@f}に渡されます。
引数 \TT{'something unknown'}をとって \printf が呼び出されます。

しかし、 $a$ の値が4以下の場合は、4を乗算して \TT{\$LN11@f}テーブルアドレスで加算します。 
これはテーブル内のアドレスがどのように構築され、必要な要素を正確に指し示すものです。 
たとえば、 $a$ が2に等しいとしましょう。$2*4 = 8$(すべてのテーブル要素は
32ビットプロセスのアドレスなので、すべての要素が4バイト幅です)。
\TT{\$LN11@f}テーブルのアドレス+ 8は\TT{\$LN4@f}ラベルが格納されているテーブル要素です。
\JMP はテーブルから\TT{\$LN4@f}アドレスを取り出し、それにジャンプします。

このテーブルはしばしば\IT{jumptable} または \IT{branch table}\footnote{The whole method was once called 
\IT{computed GOTO} in early versions of Fortran:
\href{http://go.yurichev.com/17122}{wikipedia}.
Not quite relevant these days, but what a term!}と呼ばれます。

それから、対応する \printf は引数 \TT{'two'}で呼び出されます。
実際、\TT{jmp DWORD PTR \$LN11@f[ecx*4]}命令は\IT{jump to the DWORD that is stored at address} \TT{\$LN11@f + ecx * 4}

\TT{npad}(\myref{sec:npad})は、4バイト(または16バイト)の境界に整列したアドレスに格納されるように次のラベルを整列するアセンブリ言語マクロです。
これは、メモリバス、キャッシュメモリなどを介してメモリから32ビット値をフェッチすることができるため、
プロセッサが整列している場合にはより効果的な方法でプロセッサに非常に適しています。

\input{patterns/08_switch/2_lot/olly_JPN}

\myparagraph{\NonOptimizing GCC}
\label{switch_lot_GCC}

GCC 4.4.1が生成するものを見てみましょう:

\lstinputlisting[caption=GCC 4.4.1,style=customasmx86]{patterns/08_switch/2_lot/lot_gcc.asm}

\myindex{x86!\Registers!JMP}

引数\TT{arg\_0}は2ビット左にシフトすることで4倍されます
(これは4倍の乗算とほぼ同じです)。~(\myref{SHR})
\TT{off\_804855C}配列からラベルのアドレスを取り出し、 \EAX に格納してから、\TT{JMP EAX}が実際のジャンプを行います。


\subsubsection{ARM: \OptimizingKeilVI (\ARMMode)}
\label{sec:SwitchARMLot}

\lstinputlisting[caption=\OptimizingKeilVI (\ARMMode),style=customasmARM]{patterns/08_switch/2_lot/lot_ARM_ARM_O3.asm}

このコードでは、すべての命令の固定サイズが4バイトのARMモード機能を使用しています。

$a$ の最大値は4で、それ以上の値を指定すると、\IT{<<something unknown\textbackslash{}n>>}文字列が
出力されることに注意しましょう。

\myindex{ARM!\Instructions!CMP}
\myindex{ARM!\Instructions!ADDCC}
最初の\TT{CMP R0, \#5}命令は、 $a$ の入力値を5と比較します。

\footnote{ADD---addition}
次の\TT{ADDCC PC, PC,R0, LSL \#2}命令は、$R0 < 5$(\IT{CC=Carry clear / Less than})の場合にのみ実行されます。
したがって、\TT{ADDCC}がトリガしない場合($R0 \geq 5$の場合)、\IT{default\_case}ラベルにジャンプします。

しかし$R0 < 5$と\TT{ADDCC}がトリガされた場合、次のことが起こります:

\Reg{0}の値には4が掛けられます。
実際、命令のサフィックスの\TT{LSL \#2}は\q{2ビット左シフト}の略です。
しかし、セクション\q{\ShiftsSectionName}の~(\myref{division_by_shifting})で後で見るように、2ビット左シフトは4を乗算するのと同じです。

次に、$R0*4$を\ac{PC}の現在の値に追加し、下にある\TT{B}(\IT{Branch})命令の1つにジャンプします。

\TT{ADDCC}命令の実行時に、\ac{PC}の値は\TT{ADDCC}命令が置かれているアドレス(\TT{0x178})よりも8バイト先(\TT{0x180})であり、
言い換えれば2命令先にあります。

\myindex{ARM!Pipeline}

これはARMプロセッサのパイプラインがどのように動作するかを示しています。
\TT{ADDCC}が実行されると、現時点でプロセッサは次の命令の後に命令を処理し始めているので、
\ac{PC}がそこを指しているのはそのためです。 
これは覚えておく必要があります。

$a=0$ の場合、\ac{PC}の値に加算され、\ac{PC}の実際の値は\ac{PC}(8バイト先)に書き込まれ、
\IT{loc\_180}というラベルへのジャンプが起こります。これは、\TT{ADDCC}命令の先の8バイト先です。

$a=1$ の場合、\ac{PC}には $PC+8+a*4 = PC+8+1*4 = PC+12 = 0x184$ が書き込まれます。
\IT{loc\_184}というラベルが付いたアドレスです。

1を $a$ に加えるごとに、結果の\ac{PC}は4ずつ増加します。

4はARMモードの命令長であり、各\TT{B}命令の長さ4でそれらは5つあります。

これらの5つの\TT{B}命令のそれぞれは、制御を\IT{switch()}にプログラムされたものにさらに渡します。

対応する文字列のポインタローディングが発生します。

\subsubsection{ARM: \OptimizingKeilVI (\ThumbMode)}

\lstinputlisting[caption=\OptimizingKeilVI (\ThumbMode),style=customasmARM]{patterns/08_switch/2_lot/lot_ARM_thumb_O3.asm}

\myindex{ARM!\ThumbMode}
\myindex{ARM!\ThumbTwoMode}

ThumbモードとThumb-2モードのすべての命令が同じサイズであることを確認することはできません。
これらのモードでは、x86の場合と同様に、命令の長さが可変であるといえます。

\myindex{jumptable}

したがって、そこにあるケースの数(デフォルトケースを含まない)に関する情報と、
対応するケースでコントロールを渡す必要があるラベルを持つそれぞれのオフセットが含まれている特別なテーブルが追加されています。

\myindex{ARM!Mode switching}
\myindex{ARM!\Instructions!BX}

\IT{\_\_ARM\_common\_switch8\_thumb}という名前のテーブルと
パスコントロールを扱うために特別な関数がここにあります。
\TT{BX PC}で始まり、その機能はプロセッサをARMモードに切り替えることです。
次に、テーブル処理の機能が表示されます。

今ここで説明するにはあまりにも進んでいるので、省略しましょう。
% TODO explain it...

\myindex{ARM!\Registers!Link Register}

関数が\ac{LR}レジスタをテーブルへのポインタとして使用することは興味深いことです。

実際、この関数を呼び出した後、\ac{LR}にはテーブルが始まる\TT{BL \_\_ARM\_common\_switch8\_thumb}命令の後のアドレスが入ります。

また、コードを再利用するために別の関数として生成されるので、
コンパイラはすべてのswitch()文に対して同じコードを生成しないことにも注意してください。

\IDA はそれをサービス関数とテーブルとして認識し、\TT{jumptable 000000FA case 0}のような
ラベルのコメントを追加します。

\subsubsection{MIPS}

\lstinputlisting[caption=\Optimizing GCC 4.4.5 (IDA),style=customasmMIPS]{patterns/08_switch/2_lot/MIPS_O3_IDA_JPN.lst}

\myindex{MIPS!\Instructions!SLTIU}

私たちの新しい命令は \INS{SLTIU} です(\q{Set on Less Than Immediate Unsigned})。
\myindex{MIPS!\Instructions!SLTU}

\INS{SLTU}と同じですが、\q{I}は\q{immediate}を表します。
つまり、命令自体に数値を指定する必要があります。

\myindex{MIPS!\Instructions!BNEZ}
\INS{BNEZ} は \q{Branch if Not Equal to Zero}です。

コードは他の\ac{ISA}に似ています。
\myindex{MIPS!\Instructions!SLL}
\INS{SLL} (\q{Shift Word Left Logical})は4を掛けます。

結局のところ、MIPSは32ビットCPUなので、\IT{jumptable}のすべてのアドレスは32ビットのものです。


\subsubsection{\Conclusion{}}

\IT{switch()} の大まかなスケルトン:

% TODO: ARM, MIPS skeleton
\lstinputlisting[caption=x86,style=customasmx86]{patterns/08_switch/2_lot/skel1_JPN.lst}

ジャンプテーブルのアドレスへのジャンプはこの命令で用いて実装されるでしょう:\TT{JMP jump\_table[REG*4]}
もしくはx64では\TT{JMP jump\_table[REG*8]} 。

\IT{jumptable}は単にポインタの配列で、後で説明します:\myref{array_of_pointers_to_strings}
}

% TODO What's the difference between 3 and 4? Seems to be the same...
% it is fallthrough from 3 to 4 :) --DY
\EN{\mysection{Returning Values}
\label{ret_val_func}

Another simple function is the one that simply returns a constant value:

\lstinputlisting[caption=\EN{\CCpp Code},style=customc]{patterns/011_ret/1.c}

Let's compile it.

\subsection{x86}

Here's what both the GCC and MSVC compilers produce (with optimization) on the x86 platform:

\lstinputlisting[caption=\Optimizing GCC/MSVC (\assemblyOutput),style=customasmx86]{patterns/011_ret/1.s}

\myindex{x86!\Instructions!RET}
There are just two instructions: the first places the value 123 into the \EAX register,
which is used by convention for storing the return
value, and the second one is \RET, which returns execution to the \gls{caller}.

The caller will take the result from the \EAX register.

\subsection{ARM}

There are a few differences on the ARM platform:

\lstinputlisting[caption=\OptimizingKeilVI (\ARMMode) ASM Output,style=customasmARM]{patterns/011_ret/1_Keil_ARM_O3.s}

ARM uses the register \Reg{0} for returning the results of functions, so 123 is copied into \Reg{0}.

\myindex{ARM!\Instructions!MOV}
\myindex{x86!\Instructions!MOV}
It is worth noting that \MOV is a misleading name for the instruction in both the x86 and ARM \ac{ISA}s.

The data is not in fact \IT{moved}, but \IT{copied}.

\subsection{MIPS}

\label{MIPS_leaf_function_ex1}

The GCC assembly output below lists registers by number:

\lstinputlisting[caption=\Optimizing GCC 4.4.5 (\assemblyOutput),style=customasmMIPS]{patterns/011_ret/MIPS.s}

\dots while \IDA does it by their pseudo names:

\lstinputlisting[caption=\Optimizing GCC 4.4.5 (IDA),style=customasmMIPS]{patterns/011_ret/MIPS_IDA.lst}

The \$2 (or \$V0) register is used to store the function's return value.
\myindex{MIPS!\Pseudoinstructions!LI}
\INS{LI} stands for ``Load Immediate'' and is the MIPS equivalent to \MOV.

\myindex{MIPS!\Instructions!J}
The other instruction is the jump instruction (J or JR) which returns the execution flow to the \gls{caller}.

\myindex{MIPS!Branch delay slot}
You might be wondering why the positions of the load instruction (LI) and the jump instruction (J or JR) are swapped. This is due to a \ac{RISC} feature called ``branch delay slot''.

The reason this happens is a quirk in the architecture of some RISC \ac{ISA}s and isn't important for our
purposes---we must simply keep in mind that in MIPS, the instruction following a jump or branch instruction
is executed \IT{before} the jump/branch instruction itself.

As a consequence, branch instructions always swap places with the instruction executed immediately beforehand.

In practice, functions which merely return 1 (\IT{true}) or 0 (\IT{false}) are very frequent.

The smallest ever of the standard UNIX utilities, \IT{/bin/true} and \IT{/bin/false} return 0 and 1 respectively, as an exit code.
(Zero as an exit code usually means success, non-zero means error.)
}
\RU{\mysection{Оптимизации циклов}

% subsections:
\subsection{Странная оптимизация циклов}

Это самая простая (из всех возможных) реализация memcpy():

\begin{lstlisting}[style=customc]
void memcpy (unsigned char* dst, unsigned char* src, size_t cnt)
{
	size_t i;
	for (i=0; i<cnt; i++)
		dst[i]=src[i];
};
\end{lstlisting}

Как минимум MSVC 6.0 из конца 90-х вплоть до MSVC 2013 может выдавать вот такой странный код (этот листинг создан MSVC 2013
x86):

\lstinputlisting[style=customasmx86]{advanced/500_loop_optimizations/1_1_RU.lst}

Это всё странно, потому что как люди работают с двумя указателями? Они сохраняют два адреса в двух регистрах или двух
ячейках памяти.
Компилятор MSVC в данном случае сохраняет два указателя как один указатель (\IT{скользящий dst} в \EAX)
и разницу между указателями \IT{src} и \IT{dst} (она остается неизменной во время исполнения цикла, в \ESI).
\myindex{\CLanguageElements!ptrdiff\_t}
(Кстати, это тот редкий случай, когда можно использовать тип ptrdiff\_t.)
Когда нужно загрузить байт из \IT{src}, он загружается на \IT{diff + скользящий dst} и сохраняет байт просто на
\IT{скользящем dst}.

Должно быть это какой-то трюк для оптимизации. Но я переписал эту ф-цию так:

\lstinputlisting[style=customasmx86]{advanced/500_loop_optimizations/1_2.lst}

\dots и она работает также быстро как и \IT{соптимизированная} версия на моем Intel Xeon E31220 @ 3.10GHz.
Может быть, эта оптимизация предназначалась для более старых x86-процессоров 90-х, т.к., этот трюк использует
как минимум древний MS VC 6.0?

Есть идеи?

\myindex{Hex-Rays}
Hex-Rays 2.2 не распознает такие шаблонные фрагменты кода (будем надеятся, это временно?):

\begin{lstlisting}[style=customc]
void __cdecl f1(char *dst, char *src, size_t size)
{
  size_t counter; // edx@1
  char *sliding_dst; // eax@2
  char tmp; // cl@3

  counter = size;
  if ( size )
  {
    sliding_dst = dst;
    do
    {
      tmp = (sliding_dst++)[src - dst];         // разница (src-dst) вычисляется один раз, перед телом цикла
      *(sliding_dst - 1) = tmp;
      --counter;
    }
    while ( counter );
  }
}
\end{lstlisting}

Тем не менее, этот трюк часто используется в MSVC (и не только в самодельных ф-циях \IT{memcpy()}, но также и во многих
циклах, работающих с двумя или более массивами), так что для реверс-инжиниров стоит помнить об этом.

% <!-- As of why writting occurred after <b>dst</b> incrementing? -->


\subsection{Возврат строки}

Классическая ошибка из \RobPikePractice{}:

\begin{lstlisting}[style=customc]
#include <stdio.h>

char* amsg(int n, char* s)
{
        char buf[100];

        sprintf (buf, "error %d: %s\n", n, s) ;

        return buf;
};

int main()
{
        printf ("%s\n", amsg (1234, "something wrong!"));
};
\end{lstlisting}

Она упадет.
В начале, попытаемся понять, почему.

Это состояние стека перед возвратом из amsg():

% FIXME! TikZ or whatever
\begin{lstlisting}
§(низкие адреса)§

§[amsg(): 100 байт]§
§[RA]                               <- текущий SP§
§[два аргумента amsg]§
§[что-то еще]§
§[локальные переменные main()]§

§(высокие адреса)§
\end{lstlisting}

Когда управление возвращается из amsg() в \main, пока всё хорошо.
Но когда \printf вызывается из \main, который, в свою очередь, использует стек для своих нужд, затирая 100-байтный буфер.
В лучшем случае, будет выведен случайный мусор.

Трудно поверить, но я знаю, как это исправить:

\begin{lstlisting}[style=customc]
#include <stdio.h>

char* amsg(int n, char* s)
{
        char buf[100];

        sprintf (buf, "error %d: %s\n", n, s) ;

        return buf;
};

char* interim (int n, char* s)
{
        char large_buf[8000];
        // используем локальный массив.
        // а иначе компилятор выбросит его при оптимизации, как неиспользуемый.
        large_buf[0]=0;
        return amsg (n, s);
};

int main()
{
        printf ("%s\n", interim (1234, "something wrong!"));
};
\end{lstlisting}

Это заработает если скомпилировано в MSVC 2013 без оптимизаций и с опцией \TT{/GS-}\footnote{Выключить защиту от переполнения буфера}.
MSVC предупредит: ``warning C4172: returning address of local variable or temporary'', но код запустится и сообщение выведется.
Посмотрим состояние стека в момент, когда amsg() возвращает управление в interim():

\begin{lstlisting}
§(низкие адреса)§

§[amsg(): 100 байт]§
§[RA]                                      <- текущий SP§
§[два аргумента amsg()]§
§[вледения interim(), включая 8000 байт]§
§[еще что-то]§
§[локальные переменные main()]§

§(высокие адреса)§
\end{lstlisting}

Теперь состояние стека на момент, когда interim() возвращает управление в \main{}:

\begin{lstlisting}
§(низкие адреса)§

§[amsg(): 100 байт]§
§[RA]§
§[два аргумента amsg()]§
§[вледения interim(), включая 8000 байт]§
§[еще что-то]                              <- текущий SP§
§[локальные переменные main()]§

§(высокие адреса)§
\end{lstlisting}

Так что когда \main вызывает \printf, он использует стек в месте, где выделен буфер в interim(),
и не затирает 100 байт с сообщение об ошибке внутри, потому что 8000 байт (или может быть меньше) это достаточно для всего,
что делает \printf и другие нисходящие ф-ции!

Это также может сработать, если между ними много ф-ций, например:
\main $\rightarrow$ f1() $\rightarrow$ f2() $\rightarrow$ f3() ... $\rightarrow$ amsg(),
и тогда результат amsg() используется в \main.
Дистанция между \ac{SP} в \main и адресом буфера \TT{buf[]} должна быть достаточно длинной.

Вот почему такие ошибки опасны: иногда ваш код работает (и бага прячется незамеченной). иногда нет.
\label{heisenbug}
\myindex{Хейзенбаги}
Такие баги в шутку называют хейзенбаги или шрёдинбаги\footnote{\url{https://en.wikipedia.org/wiki/Heisenbug}}.



}
\DE{\subsection{Gesetzte Bits zählen}
Hier ist ein einfaches Beispiel einer Funktion, die die Anzahl der gesetzten
Bits in einem Eingabewert zählt.

Diese Operation wird auch \q{population count}\footnote{moderne x86 CPUs
(die SSE4 unterstützen) haben zu diesem Zweck sogar einen eigenen POPCNT Befehl}
genannt.

\lstinputlisting[style=customc]{patterns/14_bitfields/4_popcnt/shifts.c}
In dieser Schleife wird der Wert von $i$ schrittweise von 0 bis 31 erhöht,
sodass der Ausdruck $1 \ll i$ von 1 bis \TT{0x80000000} zählt.
In natürlicher Sprache würden wir diese Operation als \IT{verschiebe 1 um n
Bits nach links} beschreiben.
Mit anderen Worten: Der Ausdruck $1 \ll i$ erzeugt alle möglichen Bitpositionen
in einer 32-Bit-Zahl.
Das freie Bit auf der rechten Seite wird jeweils gelöscht.

\label{2n_numbers_table}
Hier ist eine Tabelle mit allen Werten von $1 \ll i$ 
für $i=0 \ldots 31$:

\small
\begin{center}
\begin{tabular}{ | l | l | l | l | }
\hline
\HeaderColor \CCpp Ausdruck & 
\HeaderColor Zweierpotenz & 
\HeaderColor Dezimalzahl & 
\HeaderColor Hexadezimalzahl \\
\hline
$1 \ll 0$ & $2^{0}$ & 1 & 1 \\
\hline
$1 \ll 1$ & $2^{1}$ & 2 & 2 \\
\hline
$1 \ll 2$ & $2^{2}$ & 4 & 4 \\
\hline
$1 \ll 3$ & $2^{3}$ & 8 & 8 \\
\hline
$1 \ll 4$ & $2^{4}$ & 16 & 0x10 \\
\hline
$1 \ll 5$ & $2^{5}$ & 32 & 0x20 \\
\hline
$1 \ll 6$ & $2^{6}$ & 64 & 0x40 \\
\hline
$1 \ll 7$ & $2^{7}$ & 128 & 0x80 \\
\hline
$1 \ll 8$ & $2^{8}$ & 256 & 0x100 \\
\hline
$1 \ll 9$ & $2^{9}$ & 512 & 0x200 \\
\hline
$1 \ll 10$ & $2^{10}$ & 1024 & 0x400 \\
\hline
$1 \ll 11$ & $2^{11}$ & 2048 & 0x800 \\
\hline
$1 \ll 12$ & $2^{12}$ & 4096 & 0x1000 \\
\hline
$1 \ll 13$ & $2^{13}$ & 8192 & 0x2000 \\
\hline
$1 \ll 14$ & $2^{14}$ & 16384 & 0x4000 \\
\hline
$1 \ll 15$ & $2^{15}$ & 32768 & 0x8000 \\
\hline
$1 \ll 16$ & $2^{16}$ & 65536 & 0x10000 \\
\hline
$1 \ll 17$ & $2^{17}$ & 131072 & 0x20000 \\
\hline
$1 \ll 18$ & $2^{18}$ & 262144 & 0x40000 \\
\hline
$1 \ll 19$ & $2^{19}$ & 524288 & 0x80000 \\
\hline
$1 \ll 20$ & $2^{20}$ & 1048576 & 0x100000 \\
\hline
$1 \ll 21$ & $2^{21}$ & 2097152 & 0x200000 \\
\hline
$1 \ll 22$ & $2^{22}$ & 4194304 & 0x400000 \\
\hline
$1 \ll 23$ & $2^{23}$ & 8388608 & 0x800000 \\
\hline
$1 \ll 24$ & $2^{24}$ & 16777216 & 0x1000000 \\
\hline
$1 \ll 25$ & $2^{25}$ & 33554432 & 0x2000000 \\
\hline
$1 \ll 26$ & $2^{26}$ & 67108864 & 0x4000000 \\
\hline
$1 \ll 27$ & $2^{27}$ & 134217728 & 0x8000000 \\
\hline
$1 \ll 28$ & $2^{28}$ & 268435456 & 0x10000000 \\
\hline
$1 \ll 29$ & $2^{29}$ & 536870912 & 0x20000000 \\
\hline
$1 \ll 30$ & $2^{30}$ & 1073741824 & 0x40000000 \\
\hline
$1 \ll 31$ & $2^{31}$ & 2147483648 & 0x80000000 \\
\hline
\end{tabular}
\end{center}
\normalsize
Diese Konstanten (Bitmasken) tauchen im Code oft auf und ein Reverse Engineer
muss in der Lage sein, sie schnell und sicher zu erkennen.

% TBT
Es dazu jedoch nicht notwendig, die Dezimalzahlen (Zweierpotenzen) größer
65535 auswendig zu kennen. Die hexadezimalen Zahlen sind leicht zu merken.

Die Konstanten werden häufig verwendet um Flags einzelnen Bits zuzuordnen. 
Hier ist zum Beispiel ein Auszug aus \TT{ssl\_private.h} aus dem Quellcode von
Apache 2.4.6:

\begin{lstlisting}[style=customc]
/**
 * Define the SSL options
 */
#define SSL_OPT_NONE           (0)
#define SSL_OPT_RELSET         (1<<0)
#define SSL_OPT_STDENVVARS     (1<<1)
#define SSL_OPT_EXPORTCERTDATA (1<<3)
#define SSL_OPT_FAKEBASICAUTH  (1<<4)
#define SSL_OPT_STRICTREQUIRE  (1<<5)
#define SSL_OPT_OPTRENEGOTIATE (1<<6)
#define SSL_OPT_LEGACYDNFORMAT (1<<7)
\end{lstlisting}

Zurück zu unserem Beispiel.

Das Makro \TT{IS\_SET} prüft auf Anwesenheit von Bits in $a$.
\myindex{x86!\Instructions!AND}

Das Makro \TT{IS\_SET} entspricht dabei dem logischen (\IT{AND})
und gibt 0 zurück, wenn das entsprechende Bit nicht gesetzt ist, oder die
Bitmaske, wenn das Bit gesetzt ist.
Der Operator \IT{if()} wird in \CCpp ausgeführt, wenn der boolesche Ausdruck
nicht null ist (er könnte sogar 123456 sein), weshalb es meistens richtig
funktioniert.


% subsections
\subsubsection{x86}

\myparagraph{MSVC}

Kompilieren wir das Beispiel:

\lstinputlisting[caption=MSVC 2008,style=customasmx86]{patterns/13_arrays/1_simple/simple_msvc.asm}

\myindex{x86!\Instructions!SHL}
Soweit nichts Außergewöhnliches, nur zwei Schleifen: die erste füllt mit Werten auf und die zweite gibt Werte aus.
% TBT
Der Befehl \TT{shl ecx, 1} wird für die Multiplikation mit 2 in \ECX verwendet; mehr dazu unten~\myref{SHR}.

Auf dem Stack werden 80 Bytes für das Array reserviert: 20 Elemente von je 4 Byte.

\clearpage
Untersuchen wir dieses Beispiel in \olly.
\myindex{\olly}

Wir erkennen wie das Array befüllt wird:

jedes Element ist ein 32-Bit-Wort vom Typ \Tint und der Wert ist der Index multipliziert mit 2:

\begin{figure}[H]
\centering
\myincludegraphics{patterns/13_arrays/1_simple/olly.png}
\caption{\olly: nach dem Füllen des Arrays}
\label{fig:array_simple_olly}
\end{figure}
Da sich dieses Array auf dem Stack befindet, finden wir dort alle seine 20 Elemente.

\myparagraph{GCC}

Hier ist was GCC 4.4.1 erzeugt:

\lstinputlisting[caption=GCC 4.4.1,style=customasmx86]{patterns/13_arrays/1_simple/simple_gcc.asm}
Die Variable $a$ ist übrigens vom Typ \IT{int*} (Pointer auf \Tint{})--man kann einen Pointer auf ein Array an eine
andere Funktion übergeben, aber es ist richtiger zu sagen, dass der Pointer auf das erste Element des Arrays übergeben
wird. (Die Adressen der übrigen Elemente werden in bekannter Weise berechnet.)

Wenn man diesen Pointer mittels \IT{a[idx]} indiziert, wird \IT{idx} zum Pointer addiert und das dort abgelegte Element
(auf das der berechnete Pointer zeigt) wird zurückgegeben.

Ein interessantes Beispiel: ein String wie \IT{\q{string}} ist ein Array von Chars und hat den Typ \IT{const
char[]}.

Auch auf diesen Pointer kann ein Index angewendet werden.

Das ist der Grund warum es es möglich ist, Dinge wie \TT{\q{string}[i]} zu schreiben--es handelt sich dabei um einen
korrekten \CCpp Ausdruck!


\input{patterns/14_bitfields/4_popcnt/x64_DE}
\subsubsection{ARM}

\myparagraph{\OptimizingKeilVI (\ThumbMode)}

\lstinputlisting[style=customasmARM]{patterns/04_scanf/1_simple/ARM_IDA.lst}

\myindex{\CLanguageElements!\Pointers}
Damit \scanf Elemente einlesen kann, benötigt die Funktion einen Paramter--einen Pointer vom Typ \Tint.
\Tint hat die Größe 32 Bit, wir benötigen also 4 Byte, um den Wert im Speicher abzulegen, und passt daher genau in ein 32-Bit-Register.
\myindex{IDA!var\_?}
Auf dem Stack wird Platz für die lokalen Variable \GTT{x} reserviert und IDA bezeichnet diese Variable mit \IT{var\_8}. 
Eigentlich ist aber an dieser Stelle gar nicht notwendig, Platz auf dem Stack zu reservieren, da \ac{SP} (\gls{stack pointer} 
bereits auf die Adresse zeigt und auch direkt verwendet werden kann.

Der Wert von \ac{SP} wird also in das \Reg{1} Register kopiert und zusammen mit dem Formatierungsstring an \scanf übergeben.

% TBT here
%\INS{PUSH/POP} instructions behaves differently in ARM than in x86 (it's the other way around).
%They are synonyms to \INS{STM/STMDB/LDM/LDMIA} instructions.
%And \INS{PUSH} instruction first writes a value into the stack, \IT{and then} subtracts \ac{SP} by 4.
%\INS{POP} first adds 4 to \ac{SP}, \IT{and then} reads a value from the stack.
%Hence, after \INS{PUSH}, \ac{SP} points to an unused space in stack.
%It is used by \scanf, and by \printf after.

%\INS{LDMIA} means \IT{Load Multiple Registers Increment address After each transfer}.
%\INS{STMDB} means \IT{Store Multiple Registers Decrement address Before each transfer}.

\myindex{ARM!\Instructions!LDR}
Später wird mithilfe des \INS{LDR} Befehls dieser Wert vom Stack in das \Reg{1} Register verschoben um an \printf übergeben werden zu können.

\myparagraph{ARM64}

\lstinputlisting[caption=\NonOptimizing GCC 4.9.1 ARM64,numbers=left,style=customasmARM]{patterns/04_scanf/1_simple/ARM64_GCC491_O0_DE.s}

Im Stack Frame werden 32 Byte reserviert, was deutlich mehr als benötigt ist. Vielleicht handelt es sich um eine Frage des Aligning (dt. Angleichens) von Speicheradressen.
Der interessanteste Teil ist, im Stack Frame einen Platz für die Variable $x$ zu finden (Zeile 22).
Warum 28? Irgendwie hat der Compiler entschieden die Variable am Ende des Stack Frames anstatt an dessen Beginn abzulegen.
Die Adresse wird an \scanf übergeben; diese Funktion speichert den Userinput an der genannten Adresse im Speicher.
Es handelt sich hier um einen 32-Bit-Wert vom Typ \Tint. 
Der Wert wird in Zeile 27 abgeholt und dann an \printf übergeben.



\subsubsection{MIPS}
% FIXME better start at non-optimizing version?
Die Funktion verwendet eine Menge S-Register, die gesichert werden müssen. Das ist der Grund dafür, dass die Werte im
Funktionsprolog gespeichert und im Funktionsepilog wiederhergestellt werden.

\lstinputlisting[caption=\Optimizing GCC 4.4.5
(IDA),style=customasmMIPS]{patterns/13_arrays/1_simple/MIPS_O3_IDA_DE.lst}
Interessant: es gibt zwei Schleifen und die erste benötigt $i$ nicht; sie benötigt nur $i\cdot 2$ (erhöht um 2 bei
jedem Iterationsschritt) und die Adresse im Speicher (erhöht um 4 bei jedem Iterationsschritt).

Wir sehen hier also zwei Variablen: eine (in \$V0), die jedes Mal um 2 erhöht wird, und eine andere (in\$V1), die um 4
erhöht wird.

Die zweite Schleife ist der Ort, an dem \printf aufgerufen wird und dem Benutzer den Wert von $i$ zurückliefert, es gibt
also eine Variable die in \$S0 inkrementiert wird und eine Speicheradresse in \$S1, die jedes Mal um 4 erhöht wird.

% TBT
Das erinnert uns an die Optimierung von Schleifen, die wir früher betrachtet haben: \myref{loop_iterators}.

Das Ziel der Optimierung ist es, die Multiplikationen loszuwerden.

}
\FR{\mysection{\Stack}
\label{sec:stack}
\myindex{\Stack}

La pile est une des structures de données les plus fondamentales en informatique
\footnote{\href{http://go.yurichev.com/17119}{wikipedia.org/wiki/Call\_stack}}.
\ac{AKA} \ac{LIFO}.

Techniquement, il s'agit d'un bloc de mémoire situé dans l'espace d'adressage
d'un processus et qui est utilisé par le registre \ESP en x86, \RSP en x64
ou par le registre \ac{SP} en ARM comme un pointeur dans ce bloc mémoire.

\myindex{ARM!\Instructions!PUSH}
\myindex{ARM!\Instructions!POP}
\myindex{x86!\Instructions!PUSH}
\myindex{x86!\Instructions!POP}
Les instructions d'accès à la pile sont \PUSH et \POP (en x86 ainsi qu'en ARM Thumb-mode).
\PUSH soustrait à \ESP/\RSP/\ac{SP} 4 en mode 32-bit (ou 8 en mode 64-bit) et écrit
ensuite le contenu de l'opérande associé à l'adresse mémoire pointée par \ESP/\RSP/\ac{SP}.

\POP est l'opération inverse: elle récupère la donnée depuis l'adresse mémoire pointée par \ac{SP},
l'écrit dans l'opérande associé (souvent un registre) puis ajoute 4 (ou 8) au \glslink{stack pointer}{pointeur de pile}.

Après une allocation sur la pile, le \glslink{stack pointer}{pointeur de pile} pointe sur le bas de la pile.
\PUSH décrémente le \glslink{stack pointer}{pointeur de pile} et \POP l'incrémente.

Le bas de la pile représente en réalité le début de la mémoire allouée pour
le bloc de pile. Cela semble étrange, mais c'est comme ça.

ARM supporte à la fois les piles ascendantes et descendantes.

\myindex{ARM!\Instructions!STMFD}
\myindex{ARM!\Instructions!LDMFD}
\myindex{ARM!\Instructions!STMED}
\myindex{ARM!\Instructions!LDMED}
\myindex{ARM!\Instructions!STMFA}
\myindex{ARM!\Instructions!LDMFA}
\myindex{ARM!\Instructions!STMEA}
\myindex{ARM!\Instructions!LDMEA}

Par exemple les instructions \ac{STMFD}/\ac{LDMFD}, \ac{STMED}/\ac{LDMED} sont utilisées pour gérer les piles
descendantes (qui grandissent vers le bas en commençant avec une adresse haute et évoluent vers une plus basse).

Les instructions \ac{STMFA}/\ac{LDMFA}, \ac{STMEA}/\ac{LDMEA} sont utilisées pour gérer les piles montantes
(qui grandissent vers les adresses hautes de l'espace d'adressage, en commençant
avec une adresse située en bas de l'espace d'adressage).

% It might be worth mentioning that STMED and STMEA write first,
% and then move the pointer,
% and that LDMED and LDMEA move the pointer first, and then read.
% In other words, ARM not only lets the stack grow in a non-standard direction,
% but also in a non-standard order.
% Maybe this can be in the glossary, which would explain why E stands for "empty".

\subsection{Pourquoi la pile grandit en descendant ?}
\label{stack_grow_backwards}

Intuitivement, on pourrait penser que la pile grandit vers le haut, i.e. vers des
adresses plus élevées, comme n'importe qu'elle autre structure de données.

La raison pour laquelle la pile grandit vers le bas est probablement historique.
Dans le passé, les ordinateurs étaient énormes et occupaient des pièces entières,
il était facile de diviser la mémoire en deux parties, une pour le \gls{heap} et
une pour la pile.
Évidemment, on ignorait quelle serait la taille du \gls{heap} et de la pile durant
l'exécution du programme, donc cette solution était la plus simple possible.

\input{patterns/02_stack/stack_and_heap}

Dans \RitchieThompsonUNIX on peut lire:

\begin{framed}
\begin{quotation}
The user-core part of an image is divided into three logical segments. The program text segment begins at location 0 in the virtual address space. During execution, this segment is write-protected and a single copy of it is shared among all processes executing the same program. At the first 8K byte boundary above the program text segment in the virtual address space begins a nonshared, writable data segment, the size of which may be extended by a system call. Starting at the highest address in the virtual address space is a pile segment, which automatically grows downward as the hardware's pile pointer fluctuates.
\end{quotation}
\end{framed}

Cela nous rappelle comment certains étudiants prennent des notes pour deux cours différents dans
un seul et même cahier en prenant un cours d'un côté du cahier, et l'autre cours de l'autre côté.
Les notes de cours finissent par se rencontrer à un moment dans le cahier quand il n'y a plus de place.

% I think if we want to expand on this analogy,
% one might remember that the line number increases as as you go down a page.
% So when you decrease the address when pushing to the stack, visually,
% the stack does grow upwards.
% Of course, the problem is that in most human languages,
% just as with computers,
% we write downwards, so this direction is what makes buffer overflows so messy.

\subsection{Quel est le rôle de la pile ?}

% subsections
\input{patterns/02_stack/01_saving_ret_addr_FR}
\input{patterns/02_stack/02_args_passing_FR}
\input{patterns/02_stack/03_local_vars_FR}
\mysection{\oracle}
\label{oracle}

% sections
\EN{\input{examples/oracle/1_version_EN}}\RU{\input{examples/oracle/1_version_RU}}
\EN{\input{examples/oracle/2_ksmlru_EN}}\RU{\input{examples/oracle/2_ksmlru_RU}}
\EN{\input{examples/oracle/3_timer_EN}}\RU{\input{examples/oracle/3_timer_RU}}


\input{patterns/02_stack/05_SEH}
\input{patterns/02_stack/06_BO_protection}

\subsubsection{Dé-allocation automatique de données dans la pile}

Peut-être que la raison pour laquelle les variables locales et les enregistrements SEH sont stockés dans la
pile est qu'ils sont automatiquement libérés quand la fonction se termine en utilisant simplement une
instruction pour corriger la position du pointeur de pile (souvent \ADD).
Les arguments de fonction sont aussi désalloués automatiquement à la fin de la fonction.
À l'inverse, toutes les données allouées sur le \IT{heap} doivent être désallouées de façon explicite.

% sections
\input{patterns/02_stack/07_layout_FR}
\mysection{\oracle}
\label{oracle}

% sections
\EN{\input{examples/oracle/1_version_EN}}\RU{\input{examples/oracle/1_version_RU}}
\EN{\input{examples/oracle/2_ksmlru_EN}}\RU{\input{examples/oracle/2_ksmlru_RU}}
\EN{\input{examples/oracle/3_timer_EN}}\RU{\input{examples/oracle/3_timer_RU}}


\input{patterns/02_stack/exercises}
}
\ITA{\subsection{\IT{LD\_PRELOAD} hack in Linux}

\myindex{LD\_PRELOAD}
\label{ld_preload}

Questo ci permette di caricare le nostre librerie dinamiche prima delle altre, anche quelle del sistema, come libc.so.6.

Questo a sua volta ci permette di \q{sostituire} la funzione che abbiamo scritto con quella nelle librerie del sistema.
Ad esempio, è facile intercettare tutte le chiamate da  
time(), read(), write(), etc. \\
\\
\myindex{uptime}
Proviamo ad ingannare l'utility \IT{uptime}.
Come sappiamo, essa ci dice da quanto tempo il computer sta lavorando.
\myindex{strace}
Con l'aiuto di strace(\myref{strace}), è possibile osservare che l'utility prende questa informazione dal file \TT{/proc/uptime}:

\begin{lstlisting}
$ strace uptime 
...
open("/proc/uptime", O_RDONLY)          = 3
lseek(3, 0, SEEK_SET)                   = 0
read(3, "416166.86 414629.38\n", 2047)  = 20
...
\end{lstlisting}

Questo non è un file presente su disco ma uno virtuale, il suo contenuto è generato al volo nel Linux kernel.
Contiene due numeri:

\begin{lstlisting}
$ cat /proc/uptime
416690.91 415152.03
\end{lstlisting}

Da Wikipedia possiamo imparare che
\footnote{\href{http://go.yurichev.com/17043}{wikipedia}}:

\begin{framed}
\begin{quotation}
Il primo numero è il numero totale di secondi che il sistema è acceso.
Il secondo numero è la quantità di tempo che la macchina è rimasta in attesa (idle), in secondi.
\end{quotation}
\end{framed}

\myindex{\CStandardLibrary!open()}
\myindex{\CStandardLibrary!read()}
\myindex{\CStandardLibrary!close()}

Proviamo a scrivere la nostra libreria dinamica con le funzioni open(), read(), e close().

Come prima cosa, la funzione open() confronterà il nome del file da aprire con con quello che ci serve,
se l'esito è positivo, scriverà il descrittore del file aperto.

In secondo luogo, la funzione read(), se chiamata per tale descrittore del file, sostituirà l'output, altrimenti chiamerà 
la funzione read() originale dalla libreria libc.so.6.
E quindi la funzione close(), chiuderà il file che abbiamo utilizzato.

\myindex{dlopen()}
\myindex{dlsym()}

Useremo le funzioni dlopen() e dlsym() per determinare l'indirizzo della funzione originale in libc.so.6.
Dobbiamo usare queste funzioni per passare il controllo alla \q{vera} funzione.

\myindex{\CStandardLibrary!strcmp()}

D'altra parte, se intercettiamo la chiamata a strcmp() e monitoriamo ogni confronto tra stringhe
nel programma, dovremmo implementare una nostra versione di strcmp(), senza usare la funzione originale.
\footnote{Ad esempio, in questo articolo trovi quanto facilmente strcmp() riesce ad intercettare le chiamate
\footnote{\href{http://go.yurichev.com/17143}{yurichev.com}}
written by Yong Huang}

\lstinputlisting[style=customc]{OS/LD_PRELOAD/fool_uptime.c}
( \href{https://github.com/DennisYurichev/RE-for-beginners/blob/master/OS/LD_PRELOAD/fool_uptime.c}{Source code at GitHub} )
% FIXME go.yurichev.com...

Compiliamolo con librerie dinamiche comuni:

\begin{lstlisting}
gcc -fpic -shared -Wall -o fool_uptime.so fool_uptime.c -ldl
\end{lstlisting}

Avviamo \IT{uptime} caricando prima le nostre librerie:

\begin{lstlisting}
LD_PRELOAD=`pwd`/fool_uptime.so uptime
\end{lstlisting}

Osserviamo che:

\begin{lstlisting}
 01:23:02 up 24855 days,  3:14,  3 users,  load average: 0.00, 0.01, 0.05
\end{lstlisting}

Se la variabile d'ambiente \IT{LD\_PRELOAD} punta sempre al nome del file ed al percorso della nostra libreria, 
deve essere per forza avviato per tutti i programmi che andremo ad avviare. \\
\\
Altri esempi:

\begin{itemize}

\item
Semplice intercettazione della funzione strcmp() (Yong Huang) 
\url{http://go.yurichev.com/17143}

\item
Kevin Pulo---Fun with LD\_PRELOAD. Molti esempio ed idee.
\href{http://go.yurichev.com/17145}{yurichev.com}

\item
Funzioni che intercettano file, per la compressione/decompressione di file al volo (zlibc). \url{http://go.yurichev.com/17146}

\end{itemize}
}
\JPN{\subsection{あるブロックに複数の\IT{case}文があるとき}

よく用いられる構成があります:単一ブロックにいくつか\IT{case}ステートメントがあります:

\lstinputlisting[style=customc]{patterns/08_switch/3_several_cases/several_cases.c}

可能性のあるケースごとにブロックを生成するのは無駄です。
通常は、各ブロックに何らかのディスパッチャーを加えたものを生成します。

\subsubsection{MSVC}

\lstinputlisting[caption=\Optimizing MSVC 2010,numbers=left,style=customasmx86]{patterns/08_switch/3_several_cases/several_cases_MSVC_2010_Ox_JPN.asm}

最初のテーブル(\TT{\$LN10@f})はインデックステーブルで、2番目のテーブル(\TT{\$LN11@f})はブロックへのポインタの配列です。

まず、入力値がインデックステーブルのインデックスとして使用されます(13行目)。

表の値の短い凡例は次のとおりです。
0は最初の\IT{case}ブロックです(値1,2,7,10の場合)。
1は2番目の値(値3,4,5)です。
2は3番目の値(値8,9,21)です。
3は4番目の値(値22)です。
4はデフォルトブロック用です。

コードポインタの2番目のテーブルのインデックスを取得し、それにジャンプします(14行目)。

注目すべき点は、入力値0の場合がないことです。

そのため、10行目の \DEC 命令が表示され、$a=1$にテーブル要素を割り当てる必要がないため、
$a=1$でテーブルが開始されます。

これはよく用いられるパターンです。

それでなぜこれが経済的なのでしょうか?
以前はブロックポインタで構成された1つのテーブルだけで、
それを作ることができないのはなぜですか?(\myref{switch_lot_GCC})
その理由は、インデックステーブルの要素が8ビットで、よりコンパクトなためです。

\subsubsection{GCC}

GCCはすでに述べた方法で(\myref{switch_lot_GCC})、ポインタのテーブルを1つだけ使用して仕事をしています。

\subsubsection{ARM64: \Optimizing GCC 4.9.1}

入力値が0の場合にトリガされるコードはないので、GCCはジャンプテーブルをよりコンパクトにしようとし、
入力値として1から開始します。

ARM64用のGCC 4.9.1は、より巧妙なトリックを使用します。
すべてのオフセットを8ビットのバイトとしてエンコードできます。

すべてのARM64命令のサイズが4バイトであることを思い出してみましょう。

GCCは、私の小さな例のすべてのオフセットがお互いに非常に近いという事実を利用しています。
ジャンプテーブルは1バイトで構成されています。

\lstinputlisting[caption=\Optimizing GCC 4.9.1 ARM64,style=customasmARM]{patterns/08_switch/3_several_cases/ARM64_GCC491_O3_JPN.s}

この例をオブジェクトファイルにコンパイルし、 \IDA で開きましょう。 ここにジャンプテーブルがあります:

\lstinputlisting[caption=jumptable in IDA,style=customasmARM]{patterns/08_switch/3_several_cases/ARM64_GCC491_O3_IDA.lst}

したがって、1の場合、9は4で乗算され、\TT{Lrtx4}ラベルのアドレスに追加されます。

22の場合、0には4が掛けられ、結果は0になります。

\TT{Lrtx4}ラベルの直後に\TT{L7}ラベルがあります。このラベルでは、\q{22}を出力するコードを見つけることができます。

コードセグメントにはジャンプテーブルはありません。別の.rodataセクションに割り当てられています
(コードセクションに配置する特別な必要はありません)。

負のバイト(0xF7)もあり、\q{default}文字列(\TT{.L2})を出力するコードにジャンプするために使用されます。
}


\EN{\mysection{Returning Values}
\label{ret_val_func}

Another simple function is the one that simply returns a constant value:

\lstinputlisting[caption=\EN{\CCpp Code},style=customc]{patterns/011_ret/1.c}

Let's compile it.

\subsection{x86}

Here's what both the GCC and MSVC compilers produce (with optimization) on the x86 platform:

\lstinputlisting[caption=\Optimizing GCC/MSVC (\assemblyOutput),style=customasmx86]{patterns/011_ret/1.s}

\myindex{x86!\Instructions!RET}
There are just two instructions: the first places the value 123 into the \EAX register,
which is used by convention for storing the return
value, and the second one is \RET, which returns execution to the \gls{caller}.

The caller will take the result from the \EAX register.

\subsection{ARM}

There are a few differences on the ARM platform:

\lstinputlisting[caption=\OptimizingKeilVI (\ARMMode) ASM Output,style=customasmARM]{patterns/011_ret/1_Keil_ARM_O3.s}

ARM uses the register \Reg{0} for returning the results of functions, so 123 is copied into \Reg{0}.

\myindex{ARM!\Instructions!MOV}
\myindex{x86!\Instructions!MOV}
It is worth noting that \MOV is a misleading name for the instruction in both the x86 and ARM \ac{ISA}s.

The data is not in fact \IT{moved}, but \IT{copied}.

\subsection{MIPS}

\label{MIPS_leaf_function_ex1}

The GCC assembly output below lists registers by number:

\lstinputlisting[caption=\Optimizing GCC 4.4.5 (\assemblyOutput),style=customasmMIPS]{patterns/011_ret/MIPS.s}

\dots while \IDA does it by their pseudo names:

\lstinputlisting[caption=\Optimizing GCC 4.4.5 (IDA),style=customasmMIPS]{patterns/011_ret/MIPS_IDA.lst}

The \$2 (or \$V0) register is used to store the function's return value.
\myindex{MIPS!\Pseudoinstructions!LI}
\INS{LI} stands for ``Load Immediate'' and is the MIPS equivalent to \MOV.

\myindex{MIPS!\Instructions!J}
The other instruction is the jump instruction (J or JR) which returns the execution flow to the \gls{caller}.

\myindex{MIPS!Branch delay slot}
You might be wondering why the positions of the load instruction (LI) and the jump instruction (J or JR) are swapped. This is due to a \ac{RISC} feature called ``branch delay slot''.

The reason this happens is a quirk in the architecture of some RISC \ac{ISA}s and isn't important for our
purposes---we must simply keep in mind that in MIPS, the instruction following a jump or branch instruction
is executed \IT{before} the jump/branch instruction itself.

As a consequence, branch instructions always swap places with the instruction executed immediately beforehand.

In practice, functions which merely return 1 (\IT{true}) or 0 (\IT{false}) are very frequent.

The smallest ever of the standard UNIX utilities, \IT{/bin/true} and \IT{/bin/false} return 0 and 1 respectively, as an exit code.
(Zero as an exit code usually means success, non-zero means error.)
}
\RU{\mysection{Оптимизации циклов}

% subsections:
\subsection{Странная оптимизация циклов}

Это самая простая (из всех возможных) реализация memcpy():

\begin{lstlisting}[style=customc]
void memcpy (unsigned char* dst, unsigned char* src, size_t cnt)
{
	size_t i;
	for (i=0; i<cnt; i++)
		dst[i]=src[i];
};
\end{lstlisting}

Как минимум MSVC 6.0 из конца 90-х вплоть до MSVC 2013 может выдавать вот такой странный код (этот листинг создан MSVC 2013
x86):

\lstinputlisting[style=customasmx86]{advanced/500_loop_optimizations/1_1_RU.lst}

Это всё странно, потому что как люди работают с двумя указателями? Они сохраняют два адреса в двух регистрах или двух
ячейках памяти.
Компилятор MSVC в данном случае сохраняет два указателя как один указатель (\IT{скользящий dst} в \EAX)
и разницу между указателями \IT{src} и \IT{dst} (она остается неизменной во время исполнения цикла, в \ESI).
\myindex{\CLanguageElements!ptrdiff\_t}
(Кстати, это тот редкий случай, когда можно использовать тип ptrdiff\_t.)
Когда нужно загрузить байт из \IT{src}, он загружается на \IT{diff + скользящий dst} и сохраняет байт просто на
\IT{скользящем dst}.

Должно быть это какой-то трюк для оптимизации. Но я переписал эту ф-цию так:

\lstinputlisting[style=customasmx86]{advanced/500_loop_optimizations/1_2.lst}

\dots и она работает также быстро как и \IT{соптимизированная} версия на моем Intel Xeon E31220 @ 3.10GHz.
Может быть, эта оптимизация предназначалась для более старых x86-процессоров 90-х, т.к., этот трюк использует
как минимум древний MS VC 6.0?

Есть идеи?

\myindex{Hex-Rays}
Hex-Rays 2.2 не распознает такие шаблонные фрагменты кода (будем надеятся, это временно?):

\begin{lstlisting}[style=customc]
void __cdecl f1(char *dst, char *src, size_t size)
{
  size_t counter; // edx@1
  char *sliding_dst; // eax@2
  char tmp; // cl@3

  counter = size;
  if ( size )
  {
    sliding_dst = dst;
    do
    {
      tmp = (sliding_dst++)[src - dst];         // разница (src-dst) вычисляется один раз, перед телом цикла
      *(sliding_dst - 1) = tmp;
      --counter;
    }
    while ( counter );
  }
}
\end{lstlisting}

Тем не менее, этот трюк часто используется в MSVC (и не только в самодельных ф-циях \IT{memcpy()}, но также и во многих
циклах, работающих с двумя или более массивами), так что для реверс-инжиниров стоит помнить об этом.

% <!-- As of why writting occurred after <b>dst</b> incrementing? -->


\subsection{Возврат строки}

Классическая ошибка из \RobPikePractice{}:

\begin{lstlisting}[style=customc]
#include <stdio.h>

char* amsg(int n, char* s)
{
        char buf[100];

        sprintf (buf, "error %d: %s\n", n, s) ;

        return buf;
};

int main()
{
        printf ("%s\n", amsg (1234, "something wrong!"));
};
\end{lstlisting}

Она упадет.
В начале, попытаемся понять, почему.

Это состояние стека перед возвратом из amsg():

% FIXME! TikZ or whatever
\begin{lstlisting}
§(низкие адреса)§

§[amsg(): 100 байт]§
§[RA]                               <- текущий SP§
§[два аргумента amsg]§
§[что-то еще]§
§[локальные переменные main()]§

§(высокие адреса)§
\end{lstlisting}

Когда управление возвращается из amsg() в \main, пока всё хорошо.
Но когда \printf вызывается из \main, который, в свою очередь, использует стек для своих нужд, затирая 100-байтный буфер.
В лучшем случае, будет выведен случайный мусор.

Трудно поверить, но я знаю, как это исправить:

\begin{lstlisting}[style=customc]
#include <stdio.h>

char* amsg(int n, char* s)
{
        char buf[100];

        sprintf (buf, "error %d: %s\n", n, s) ;

        return buf;
};

char* interim (int n, char* s)
{
        char large_buf[8000];
        // используем локальный массив.
        // а иначе компилятор выбросит его при оптимизации, как неиспользуемый.
        large_buf[0]=0;
        return amsg (n, s);
};

int main()
{
        printf ("%s\n", interim (1234, "something wrong!"));
};
\end{lstlisting}

Это заработает если скомпилировано в MSVC 2013 без оптимизаций и с опцией \TT{/GS-}\footnote{Выключить защиту от переполнения буфера}.
MSVC предупредит: ``warning C4172: returning address of local variable or temporary'', но код запустится и сообщение выведется.
Посмотрим состояние стека в момент, когда amsg() возвращает управление в interim():

\begin{lstlisting}
§(низкие адреса)§

§[amsg(): 100 байт]§
§[RA]                                      <- текущий SP§
§[два аргумента amsg()]§
§[вледения interim(), включая 8000 байт]§
§[еще что-то]§
§[локальные переменные main()]§

§(высокие адреса)§
\end{lstlisting}

Теперь состояние стека на момент, когда interim() возвращает управление в \main{}:

\begin{lstlisting}
§(низкие адреса)§

§[amsg(): 100 байт]§
§[RA]§
§[два аргумента amsg()]§
§[вледения interim(), включая 8000 байт]§
§[еще что-то]                              <- текущий SP§
§[локальные переменные main()]§

§(высокие адреса)§
\end{lstlisting}

Так что когда \main вызывает \printf, он использует стек в месте, где выделен буфер в interim(),
и не затирает 100 байт с сообщение об ошибке внутри, потому что 8000 байт (или может быть меньше) это достаточно для всего,
что делает \printf и другие нисходящие ф-ции!

Это также может сработать, если между ними много ф-ций, например:
\main $\rightarrow$ f1() $\rightarrow$ f2() $\rightarrow$ f3() ... $\rightarrow$ amsg(),
и тогда результат amsg() используется в \main.
Дистанция между \ac{SP} в \main и адресом буфера \TT{buf[]} должна быть достаточно длинной.

Вот почему такие ошибки опасны: иногда ваш код работает (и бага прячется незамеченной). иногда нет.
\label{heisenbug}
\myindex{Хейзенбаги}
Такие баги в шутку называют хейзенбаги или шрёдинбаги\footnote{\url{https://en.wikipedia.org/wiki/Heisenbug}}.



}
\DE{\subsection{Gesetzte Bits zählen}
Hier ist ein einfaches Beispiel einer Funktion, die die Anzahl der gesetzten
Bits in einem Eingabewert zählt.

Diese Operation wird auch \q{population count}\footnote{moderne x86 CPUs
(die SSE4 unterstützen) haben zu diesem Zweck sogar einen eigenen POPCNT Befehl}
genannt.

\lstinputlisting[style=customc]{patterns/14_bitfields/4_popcnt/shifts.c}
In dieser Schleife wird der Wert von $i$ schrittweise von 0 bis 31 erhöht,
sodass der Ausdruck $1 \ll i$ von 1 bis \TT{0x80000000} zählt.
In natürlicher Sprache würden wir diese Operation als \IT{verschiebe 1 um n
Bits nach links} beschreiben.
Mit anderen Worten: Der Ausdruck $1 \ll i$ erzeugt alle möglichen Bitpositionen
in einer 32-Bit-Zahl.
Das freie Bit auf der rechten Seite wird jeweils gelöscht.

\label{2n_numbers_table}
Hier ist eine Tabelle mit allen Werten von $1 \ll i$ 
für $i=0 \ldots 31$:

\small
\begin{center}
\begin{tabular}{ | l | l | l | l | }
\hline
\HeaderColor \CCpp Ausdruck & 
\HeaderColor Zweierpotenz & 
\HeaderColor Dezimalzahl & 
\HeaderColor Hexadezimalzahl \\
\hline
$1 \ll 0$ & $2^{0}$ & 1 & 1 \\
\hline
$1 \ll 1$ & $2^{1}$ & 2 & 2 \\
\hline
$1 \ll 2$ & $2^{2}$ & 4 & 4 \\
\hline
$1 \ll 3$ & $2^{3}$ & 8 & 8 \\
\hline
$1 \ll 4$ & $2^{4}$ & 16 & 0x10 \\
\hline
$1 \ll 5$ & $2^{5}$ & 32 & 0x20 \\
\hline
$1 \ll 6$ & $2^{6}$ & 64 & 0x40 \\
\hline
$1 \ll 7$ & $2^{7}$ & 128 & 0x80 \\
\hline
$1 \ll 8$ & $2^{8}$ & 256 & 0x100 \\
\hline
$1 \ll 9$ & $2^{9}$ & 512 & 0x200 \\
\hline
$1 \ll 10$ & $2^{10}$ & 1024 & 0x400 \\
\hline
$1 \ll 11$ & $2^{11}$ & 2048 & 0x800 \\
\hline
$1 \ll 12$ & $2^{12}$ & 4096 & 0x1000 \\
\hline
$1 \ll 13$ & $2^{13}$ & 8192 & 0x2000 \\
\hline
$1 \ll 14$ & $2^{14}$ & 16384 & 0x4000 \\
\hline
$1 \ll 15$ & $2^{15}$ & 32768 & 0x8000 \\
\hline
$1 \ll 16$ & $2^{16}$ & 65536 & 0x10000 \\
\hline
$1 \ll 17$ & $2^{17}$ & 131072 & 0x20000 \\
\hline
$1 \ll 18$ & $2^{18}$ & 262144 & 0x40000 \\
\hline
$1 \ll 19$ & $2^{19}$ & 524288 & 0x80000 \\
\hline
$1 \ll 20$ & $2^{20}$ & 1048576 & 0x100000 \\
\hline
$1 \ll 21$ & $2^{21}$ & 2097152 & 0x200000 \\
\hline
$1 \ll 22$ & $2^{22}$ & 4194304 & 0x400000 \\
\hline
$1 \ll 23$ & $2^{23}$ & 8388608 & 0x800000 \\
\hline
$1 \ll 24$ & $2^{24}$ & 16777216 & 0x1000000 \\
\hline
$1 \ll 25$ & $2^{25}$ & 33554432 & 0x2000000 \\
\hline
$1 \ll 26$ & $2^{26}$ & 67108864 & 0x4000000 \\
\hline
$1 \ll 27$ & $2^{27}$ & 134217728 & 0x8000000 \\
\hline
$1 \ll 28$ & $2^{28}$ & 268435456 & 0x10000000 \\
\hline
$1 \ll 29$ & $2^{29}$ & 536870912 & 0x20000000 \\
\hline
$1 \ll 30$ & $2^{30}$ & 1073741824 & 0x40000000 \\
\hline
$1 \ll 31$ & $2^{31}$ & 2147483648 & 0x80000000 \\
\hline
\end{tabular}
\end{center}
\normalsize
Diese Konstanten (Bitmasken) tauchen im Code oft auf und ein Reverse Engineer
muss in der Lage sein, sie schnell und sicher zu erkennen.

% TBT
Es dazu jedoch nicht notwendig, die Dezimalzahlen (Zweierpotenzen) größer
65535 auswendig zu kennen. Die hexadezimalen Zahlen sind leicht zu merken.

Die Konstanten werden häufig verwendet um Flags einzelnen Bits zuzuordnen. 
Hier ist zum Beispiel ein Auszug aus \TT{ssl\_private.h} aus dem Quellcode von
Apache 2.4.6:

\begin{lstlisting}[style=customc]
/**
 * Define the SSL options
 */
#define SSL_OPT_NONE           (0)
#define SSL_OPT_RELSET         (1<<0)
#define SSL_OPT_STDENVVARS     (1<<1)
#define SSL_OPT_EXPORTCERTDATA (1<<3)
#define SSL_OPT_FAKEBASICAUTH  (1<<4)
#define SSL_OPT_STRICTREQUIRE  (1<<5)
#define SSL_OPT_OPTRENEGOTIATE (1<<6)
#define SSL_OPT_LEGACYDNFORMAT (1<<7)
\end{lstlisting}

Zurück zu unserem Beispiel.

Das Makro \TT{IS\_SET} prüft auf Anwesenheit von Bits in $a$.
\myindex{x86!\Instructions!AND}

Das Makro \TT{IS\_SET} entspricht dabei dem logischen (\IT{AND})
und gibt 0 zurück, wenn das entsprechende Bit nicht gesetzt ist, oder die
Bitmaske, wenn das Bit gesetzt ist.
Der Operator \IT{if()} wird in \CCpp ausgeführt, wenn der boolesche Ausdruck
nicht null ist (er könnte sogar 123456 sein), weshalb es meistens richtig
funktioniert.


% subsections
\subsubsection{x86}

\myparagraph{MSVC}

Kompilieren wir das Beispiel:

\lstinputlisting[caption=MSVC 2008,style=customasmx86]{patterns/13_arrays/1_simple/simple_msvc.asm}

\myindex{x86!\Instructions!SHL}
Soweit nichts Außergewöhnliches, nur zwei Schleifen: die erste füllt mit Werten auf und die zweite gibt Werte aus.
% TBT
Der Befehl \TT{shl ecx, 1} wird für die Multiplikation mit 2 in \ECX verwendet; mehr dazu unten~\myref{SHR}.

Auf dem Stack werden 80 Bytes für das Array reserviert: 20 Elemente von je 4 Byte.

\clearpage
Untersuchen wir dieses Beispiel in \olly.
\myindex{\olly}

Wir erkennen wie das Array befüllt wird:

jedes Element ist ein 32-Bit-Wort vom Typ \Tint und der Wert ist der Index multipliziert mit 2:

\begin{figure}[H]
\centering
\myincludegraphics{patterns/13_arrays/1_simple/olly.png}
\caption{\olly: nach dem Füllen des Arrays}
\label{fig:array_simple_olly}
\end{figure}
Da sich dieses Array auf dem Stack befindet, finden wir dort alle seine 20 Elemente.

\myparagraph{GCC}

Hier ist was GCC 4.4.1 erzeugt:

\lstinputlisting[caption=GCC 4.4.1,style=customasmx86]{patterns/13_arrays/1_simple/simple_gcc.asm}
Die Variable $a$ ist übrigens vom Typ \IT{int*} (Pointer auf \Tint{})--man kann einen Pointer auf ein Array an eine
andere Funktion übergeben, aber es ist richtiger zu sagen, dass der Pointer auf das erste Element des Arrays übergeben
wird. (Die Adressen der übrigen Elemente werden in bekannter Weise berechnet.)

Wenn man diesen Pointer mittels \IT{a[idx]} indiziert, wird \IT{idx} zum Pointer addiert und das dort abgelegte Element
(auf das der berechnete Pointer zeigt) wird zurückgegeben.

Ein interessantes Beispiel: ein String wie \IT{\q{string}} ist ein Array von Chars und hat den Typ \IT{const
char[]}.

Auch auf diesen Pointer kann ein Index angewendet werden.

Das ist der Grund warum es es möglich ist, Dinge wie \TT{\q{string}[i]} zu schreiben--es handelt sich dabei um einen
korrekten \CCpp Ausdruck!


\input{patterns/14_bitfields/4_popcnt/x64_DE}
\subsubsection{ARM}

\myparagraph{\OptimizingKeilVI (\ThumbMode)}

\lstinputlisting[style=customasmARM]{patterns/04_scanf/1_simple/ARM_IDA.lst}

\myindex{\CLanguageElements!\Pointers}
Damit \scanf Elemente einlesen kann, benötigt die Funktion einen Paramter--einen Pointer vom Typ \Tint.
\Tint hat die Größe 32 Bit, wir benötigen also 4 Byte, um den Wert im Speicher abzulegen, und passt daher genau in ein 32-Bit-Register.
\myindex{IDA!var\_?}
Auf dem Stack wird Platz für die lokalen Variable \GTT{x} reserviert und IDA bezeichnet diese Variable mit \IT{var\_8}. 
Eigentlich ist aber an dieser Stelle gar nicht notwendig, Platz auf dem Stack zu reservieren, da \ac{SP} (\gls{stack pointer} 
bereits auf die Adresse zeigt und auch direkt verwendet werden kann.

Der Wert von \ac{SP} wird also in das \Reg{1} Register kopiert und zusammen mit dem Formatierungsstring an \scanf übergeben.

% TBT here
%\INS{PUSH/POP} instructions behaves differently in ARM than in x86 (it's the other way around).
%They are synonyms to \INS{STM/STMDB/LDM/LDMIA} instructions.
%And \INS{PUSH} instruction first writes a value into the stack, \IT{and then} subtracts \ac{SP} by 4.
%\INS{POP} first adds 4 to \ac{SP}, \IT{and then} reads a value from the stack.
%Hence, after \INS{PUSH}, \ac{SP} points to an unused space in stack.
%It is used by \scanf, and by \printf after.

%\INS{LDMIA} means \IT{Load Multiple Registers Increment address After each transfer}.
%\INS{STMDB} means \IT{Store Multiple Registers Decrement address Before each transfer}.

\myindex{ARM!\Instructions!LDR}
Später wird mithilfe des \INS{LDR} Befehls dieser Wert vom Stack in das \Reg{1} Register verschoben um an \printf übergeben werden zu können.

\myparagraph{ARM64}

\lstinputlisting[caption=\NonOptimizing GCC 4.9.1 ARM64,numbers=left,style=customasmARM]{patterns/04_scanf/1_simple/ARM64_GCC491_O0_DE.s}

Im Stack Frame werden 32 Byte reserviert, was deutlich mehr als benötigt ist. Vielleicht handelt es sich um eine Frage des Aligning (dt. Angleichens) von Speicheradressen.
Der interessanteste Teil ist, im Stack Frame einen Platz für die Variable $x$ zu finden (Zeile 22).
Warum 28? Irgendwie hat der Compiler entschieden die Variable am Ende des Stack Frames anstatt an dessen Beginn abzulegen.
Die Adresse wird an \scanf übergeben; diese Funktion speichert den Userinput an der genannten Adresse im Speicher.
Es handelt sich hier um einen 32-Bit-Wert vom Typ \Tint. 
Der Wert wird in Zeile 27 abgeholt und dann an \printf übergeben.



\subsubsection{MIPS}
% FIXME better start at non-optimizing version?
Die Funktion verwendet eine Menge S-Register, die gesichert werden müssen. Das ist der Grund dafür, dass die Werte im
Funktionsprolog gespeichert und im Funktionsepilog wiederhergestellt werden.

\lstinputlisting[caption=\Optimizing GCC 4.4.5
(IDA),style=customasmMIPS]{patterns/13_arrays/1_simple/MIPS_O3_IDA_DE.lst}
Interessant: es gibt zwei Schleifen und die erste benötigt $i$ nicht; sie benötigt nur $i\cdot 2$ (erhöht um 2 bei
jedem Iterationsschritt) und die Adresse im Speicher (erhöht um 4 bei jedem Iterationsschritt).

Wir sehen hier also zwei Variablen: eine (in \$V0), die jedes Mal um 2 erhöht wird, und eine andere (in\$V1), die um 4
erhöht wird.

Die zweite Schleife ist der Ort, an dem \printf aufgerufen wird und dem Benutzer den Wert von $i$ zurückliefert, es gibt
also eine Variable die in \$S0 inkrementiert wird und eine Speicheradresse in \$S1, die jedes Mal um 4 erhöht wird.

% TBT
Das erinnert uns an die Optimierung von Schleifen, die wir früher betrachtet haben: \myref{loop_iterators}.

Das Ziel der Optimierung ist es, die Multiplikationen loszuwerden.

}
\FR{\mysection{\Stack}
\label{sec:stack}
\myindex{\Stack}

La pile est une des structures de données les plus fondamentales en informatique
\footnote{\href{http://go.yurichev.com/17119}{wikipedia.org/wiki/Call\_stack}}.
\ac{AKA} \ac{LIFO}.

Techniquement, il s'agit d'un bloc de mémoire situé dans l'espace d'adressage
d'un processus et qui est utilisé par le registre \ESP en x86, \RSP en x64
ou par le registre \ac{SP} en ARM comme un pointeur dans ce bloc mémoire.

\myindex{ARM!\Instructions!PUSH}
\myindex{ARM!\Instructions!POP}
\myindex{x86!\Instructions!PUSH}
\myindex{x86!\Instructions!POP}
Les instructions d'accès à la pile sont \PUSH et \POP (en x86 ainsi qu'en ARM Thumb-mode).
\PUSH soustrait à \ESP/\RSP/\ac{SP} 4 en mode 32-bit (ou 8 en mode 64-bit) et écrit
ensuite le contenu de l'opérande associé à l'adresse mémoire pointée par \ESP/\RSP/\ac{SP}.

\POP est l'opération inverse: elle récupère la donnée depuis l'adresse mémoire pointée par \ac{SP},
l'écrit dans l'opérande associé (souvent un registre) puis ajoute 4 (ou 8) au \glslink{stack pointer}{pointeur de pile}.

Après une allocation sur la pile, le \glslink{stack pointer}{pointeur de pile} pointe sur le bas de la pile.
\PUSH décrémente le \glslink{stack pointer}{pointeur de pile} et \POP l'incrémente.

Le bas de la pile représente en réalité le début de la mémoire allouée pour
le bloc de pile. Cela semble étrange, mais c'est comme ça.

ARM supporte à la fois les piles ascendantes et descendantes.

\myindex{ARM!\Instructions!STMFD}
\myindex{ARM!\Instructions!LDMFD}
\myindex{ARM!\Instructions!STMED}
\myindex{ARM!\Instructions!LDMED}
\myindex{ARM!\Instructions!STMFA}
\myindex{ARM!\Instructions!LDMFA}
\myindex{ARM!\Instructions!STMEA}
\myindex{ARM!\Instructions!LDMEA}

Par exemple les instructions \ac{STMFD}/\ac{LDMFD}, \ac{STMED}/\ac{LDMED} sont utilisées pour gérer les piles
descendantes (qui grandissent vers le bas en commençant avec une adresse haute et évoluent vers une plus basse).

Les instructions \ac{STMFA}/\ac{LDMFA}, \ac{STMEA}/\ac{LDMEA} sont utilisées pour gérer les piles montantes
(qui grandissent vers les adresses hautes de l'espace d'adressage, en commençant
avec une adresse située en bas de l'espace d'adressage).

% It might be worth mentioning that STMED and STMEA write first,
% and then move the pointer,
% and that LDMED and LDMEA move the pointer first, and then read.
% In other words, ARM not only lets the stack grow in a non-standard direction,
% but also in a non-standard order.
% Maybe this can be in the glossary, which would explain why E stands for "empty".

\subsection{Pourquoi la pile grandit en descendant ?}
\label{stack_grow_backwards}

Intuitivement, on pourrait penser que la pile grandit vers le haut, i.e. vers des
adresses plus élevées, comme n'importe qu'elle autre structure de données.

La raison pour laquelle la pile grandit vers le bas est probablement historique.
Dans le passé, les ordinateurs étaient énormes et occupaient des pièces entières,
il était facile de diviser la mémoire en deux parties, une pour le \gls{heap} et
une pour la pile.
Évidemment, on ignorait quelle serait la taille du \gls{heap} et de la pile durant
l'exécution du programme, donc cette solution était la plus simple possible.

\input{patterns/02_stack/stack_and_heap}

Dans \RitchieThompsonUNIX on peut lire:

\begin{framed}
\begin{quotation}
The user-core part of an image is divided into three logical segments. The program text segment begins at location 0 in the virtual address space. During execution, this segment is write-protected and a single copy of it is shared among all processes executing the same program. At the first 8K byte boundary above the program text segment in the virtual address space begins a nonshared, writable data segment, the size of which may be extended by a system call. Starting at the highest address in the virtual address space is a pile segment, which automatically grows downward as the hardware's pile pointer fluctuates.
\end{quotation}
\end{framed}

Cela nous rappelle comment certains étudiants prennent des notes pour deux cours différents dans
un seul et même cahier en prenant un cours d'un côté du cahier, et l'autre cours de l'autre côté.
Les notes de cours finissent par se rencontrer à un moment dans le cahier quand il n'y a plus de place.

% I think if we want to expand on this analogy,
% one might remember that the line number increases as as you go down a page.
% So when you decrease the address when pushing to the stack, visually,
% the stack does grow upwards.
% Of course, the problem is that in most human languages,
% just as with computers,
% we write downwards, so this direction is what makes buffer overflows so messy.

\subsection{Quel est le rôle de la pile ?}

% subsections
\input{patterns/02_stack/01_saving_ret_addr_FR}
\input{patterns/02_stack/02_args_passing_FR}
\input{patterns/02_stack/03_local_vars_FR}
\mysection{\oracle}
\label{oracle}

% sections
\EN{\input{examples/oracle/1_version_EN}}\RU{\input{examples/oracle/1_version_RU}}
\EN{\input{examples/oracle/2_ksmlru_EN}}\RU{\input{examples/oracle/2_ksmlru_RU}}
\EN{\input{examples/oracle/3_timer_EN}}\RU{\input{examples/oracle/3_timer_RU}}


\input{patterns/02_stack/05_SEH}
\input{patterns/02_stack/06_BO_protection}

\subsubsection{Dé-allocation automatique de données dans la pile}

Peut-être que la raison pour laquelle les variables locales et les enregistrements SEH sont stockés dans la
pile est qu'ils sont automatiquement libérés quand la fonction se termine en utilisant simplement une
instruction pour corriger la position du pointeur de pile (souvent \ADD).
Les arguments de fonction sont aussi désalloués automatiquement à la fin de la fonction.
À l'inverse, toutes les données allouées sur le \IT{heap} doivent être désallouées de façon explicite.

% sections
\input{patterns/02_stack/07_layout_FR}
\mysection{\oracle}
\label{oracle}

% sections
\EN{\input{examples/oracle/1_version_EN}}\RU{\input{examples/oracle/1_version_RU}}
\EN{\input{examples/oracle/2_ksmlru_EN}}\RU{\input{examples/oracle/2_ksmlru_RU}}
\EN{\input{examples/oracle/3_timer_EN}}\RU{\input{examples/oracle/3_timer_RU}}


\input{patterns/02_stack/exercises}
}
\ITA{\subsection{\IT{LD\_PRELOAD} hack in Linux}

\myindex{LD\_PRELOAD}
\label{ld_preload}

Questo ci permette di caricare le nostre librerie dinamiche prima delle altre, anche quelle del sistema, come libc.so.6.

Questo a sua volta ci permette di \q{sostituire} la funzione che abbiamo scritto con quella nelle librerie del sistema.
Ad esempio, è facile intercettare tutte le chiamate da  
time(), read(), write(), etc. \\
\\
\myindex{uptime}
Proviamo ad ingannare l'utility \IT{uptime}.
Come sappiamo, essa ci dice da quanto tempo il computer sta lavorando.
\myindex{strace}
Con l'aiuto di strace(\myref{strace}), è possibile osservare che l'utility prende questa informazione dal file \TT{/proc/uptime}:

\begin{lstlisting}
$ strace uptime 
...
open("/proc/uptime", O_RDONLY)          = 3
lseek(3, 0, SEEK_SET)                   = 0
read(3, "416166.86 414629.38\n", 2047)  = 20
...
\end{lstlisting}

Questo non è un file presente su disco ma uno virtuale, il suo contenuto è generato al volo nel Linux kernel.
Contiene due numeri:

\begin{lstlisting}
$ cat /proc/uptime
416690.91 415152.03
\end{lstlisting}

Da Wikipedia possiamo imparare che
\footnote{\href{http://go.yurichev.com/17043}{wikipedia}}:

\begin{framed}
\begin{quotation}
Il primo numero è il numero totale di secondi che il sistema è acceso.
Il secondo numero è la quantità di tempo che la macchina è rimasta in attesa (idle), in secondi.
\end{quotation}
\end{framed}

\myindex{\CStandardLibrary!open()}
\myindex{\CStandardLibrary!read()}
\myindex{\CStandardLibrary!close()}

Proviamo a scrivere la nostra libreria dinamica con le funzioni open(), read(), e close().

Come prima cosa, la funzione open() confronterà il nome del file da aprire con con quello che ci serve,
se l'esito è positivo, scriverà il descrittore del file aperto.

In secondo luogo, la funzione read(), se chiamata per tale descrittore del file, sostituirà l'output, altrimenti chiamerà 
la funzione read() originale dalla libreria libc.so.6.
E quindi la funzione close(), chiuderà il file che abbiamo utilizzato.

\myindex{dlopen()}
\myindex{dlsym()}

Useremo le funzioni dlopen() e dlsym() per determinare l'indirizzo della funzione originale in libc.so.6.
Dobbiamo usare queste funzioni per passare il controllo alla \q{vera} funzione.

\myindex{\CStandardLibrary!strcmp()}

D'altra parte, se intercettiamo la chiamata a strcmp() e monitoriamo ogni confronto tra stringhe
nel programma, dovremmo implementare una nostra versione di strcmp(), senza usare la funzione originale.
\footnote{Ad esempio, in questo articolo trovi quanto facilmente strcmp() riesce ad intercettare le chiamate
\footnote{\href{http://go.yurichev.com/17143}{yurichev.com}}
written by Yong Huang}

\lstinputlisting[style=customc]{OS/LD_PRELOAD/fool_uptime.c}
( \href{https://github.com/DennisYurichev/RE-for-beginners/blob/master/OS/LD_PRELOAD/fool_uptime.c}{Source code at GitHub} )
% FIXME go.yurichev.com...

Compiliamolo con librerie dinamiche comuni:

\begin{lstlisting}
gcc -fpic -shared -Wall -o fool_uptime.so fool_uptime.c -ldl
\end{lstlisting}

Avviamo \IT{uptime} caricando prima le nostre librerie:

\begin{lstlisting}
LD_PRELOAD=`pwd`/fool_uptime.so uptime
\end{lstlisting}

Osserviamo che:

\begin{lstlisting}
 01:23:02 up 24855 days,  3:14,  3 users,  load average: 0.00, 0.01, 0.05
\end{lstlisting}

Se la variabile d'ambiente \IT{LD\_PRELOAD} punta sempre al nome del file ed al percorso della nostra libreria, 
deve essere per forza avviato per tutti i programmi che andremo ad avviare. \\
\\
Altri esempi:

\begin{itemize}

\item
Semplice intercettazione della funzione strcmp() (Yong Huang) 
\url{http://go.yurichev.com/17143}

\item
Kevin Pulo---Fun with LD\_PRELOAD. Molti esempio ed idee.
\href{http://go.yurichev.com/17145}{yurichev.com}

\item
Funzioni che intercettano file, per la compressione/decompressione di file al volo (zlibc). \url{http://go.yurichev.com/17146}

\end{itemize}
}
\JPN{\subsection{フォールスルー}

\TT{switch()}演算子の別のポピュラーな使い方は\q{フォールスルー}です。
単純なサンプルがあります。\footnote{\url{https://github.com/azonalon/prgraas/blob/master/prog1lib/lecture_examples/is_whitespace.c}からコピーペースト}:

\lstinputlisting[numbers=left,style=customc]{patterns/08_switch/4_fallthrough/fallthrough1.c}

やや難しいものをLinuxカーネル\footnote{\url{https://github.com/torvalds/linux/blob/master/drivers/media/dvb-frontends/lgdt3306a.c}からコピーペースト}:

\lstinputlisting[numbers=left,style=customc]{patterns/08_switch/4_fallthrough/fallthrough2.c}

\lstinputlisting[caption=Optimizing GCC 5.4.0 x86,numbers=left,style=customasmx86]{patterns/08_switch/4_fallthrough/fallthrough2.s}

関数の入力に3250という数字がある場合、\TT{.L5}ラベルを得ることができます。
しかし、我々は反対側からこのラベルに行くことができます:
\printf 呼び出しと\TT{.L5}ラベルの間にはジャンプがないことがわかります。

\IT{switch()}文がバグの原因となることが理解できます。
\IT{break}を1つ忘れるとはあなたの\IT{switch()}文を\IT{フォールスルー}に変換し、1つのブロックの代わりにいくつかのブロックが実行されます。
}


\subsection{\Exercises}

\subsubsection{\Exercise \#1}
\label{exercise_switch_1}

\RU{Вполне возможно переделать пример на Си в листинге \myref{switch_lot_c} так, чтобы при компиляции
получалось даже ещё меньше кода, но работать всё будет точно так же.
Попробуйте этого добиться.}
\EN{It's possible to rework the C example in \myref{switch_lot_c} in such way that the compiler
can produce even smaller code, but will work just the same.
Try to achieve it.}
\DE{Der C-Code des Beispiels in \myref{switch_lot_c} soll so neu geschrieben werden, dass der Compiler die gleiche
Funktionalität in noch kürzerem Code erreichen kann.}
\FR{Il est possible de modifier l'exemple en C de \myref{switch_lot_c} de telle sorte
que le compilateur produise un code plus concis, mais qui fonctionne toujours pareil.}
\ITA{E' possibile riscrivere l'esempio C da \myref{switch_lot_c} in modo tale che il compilatore riesca a produrre codice ancora più breve e che funzioni allo stesso modo. Prova a farlo.}
\PLph{}
\JPN{コンパイラがより小さなコードを生成することができるように\myref{switch_lot_c}のCの例を
修正することは可能ですが、まったく同じように動作します。
やってみてください。}


% \RU{Подсказка}\EN{Hint}: \printf \EN{may be called only from a single place}\RU{вполне может 
% вызываться только из одного места}.
