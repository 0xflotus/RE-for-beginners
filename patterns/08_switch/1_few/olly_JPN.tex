\clearpage
\mysubparagraph{\olly}

この例は扱いにくいので、 \olly でトレースしてみましょう。

\olly はそのようなswitch()構文を検出することができ、有用なコメントを追加することができます。
\EAX の値は最初は2で、それは関数への入力値です:

\begin{figure}[H]
\centering
\myincludegraphics{patterns/08_switch/1_few/olly1.png}
\caption{\olly: \EAX 
は最初の(そして唯一の)関数への引数を含んでいます}
\label{fig:switch_few_olly1}
\end{figure}

\clearpage
0は \EAX から2を引いた値です。
もちろん、 \EAX にはまだ2が入っています。
しかし、 \ZF フラグは0になり、結果の値がゼロでないことを示します。

\begin{figure}[H]
\centering
\myincludegraphics{patterns/08_switch/1_few/olly2.png}
\caption{\olly: \SUB の実行}
\label{fig:switch_few_olly2}
\end{figure}

\clearpage
\DEC が実行され、 \EAX には1が入ります。
しかし1はゼロではないので、 \ZF フラグはまだ0です:

\begin{figure}[H]
\centering
\myincludegraphics{patterns/08_switch/1_few/olly3.png}
\caption{\olly: 最初の \DEC 実行}
\label{fig:switch_few_olly3}
\end{figure}

\clearpage
次の \DEC が実行されます。
\EAX は最終的に0になり、結果がゼロであるため \ZF フラグが設定されます。

\begin{figure}[H]
\centering
\myincludegraphics{patterns/08_switch/1_few/olly4.png}
\caption{\olly: 2回目の \DEC 実行}
\label{fig:switch_few_olly4}
\end{figure}

\olly は、このジャンプが今行われることを示しています。

\clearpage
\q{two}という文字列へのポインタが今スタックに書き込まれます:

\begin{figure}[H]
\centering
\myincludegraphics{patterns/08_switch/1_few/olly5.png}
\caption{\olly: 
文字列へのポインタは、最初の引数の場所に書き込まれる}
\label{fig:switch_few_olly5}
\end{figure}

% TODO: homogenize numbers
% now they are inconsistent: sometimes plain text, sometimes in math mode
% some kind of \expr{} both for numbers and expressions? --DY
注意:関数の現在の引数は2であり、2はスタックに\TT{0x001EF850}のアドレスにあります。

\clearpage
\MOV はアドレス\TT{0x001EF850}の文字列にポインタを書き込みます(スタックウィンドウを参照)。
その後、ジャンプが発生します。
これはMSVCR100.DLLの \printf 関数の最初の命令です(この例は/MDスイッチでコンパイルされています)。

\begin{figure}[H]
\centering
\myincludegraphics{patterns/08_switch/1_few/olly6.png}
\caption{\olly: MSVCR100.DLLでの \printf の最初の命令}
\label{fig:switch_few_olly6}
\end{figure}

今や \printf は\TT{0x00FF3010}の文字列を唯一の引数として扱い、文字列を出力します。

\clearpage
これが \printf の最後の命令です。

\begin{figure}[H]
\centering
\myincludegraphics{patterns/08_switch/1_few/olly7.png}
\caption{\olly: MSVCR100.DLLの \printf の最後の命令}
\label{fig:switch_few_olly7}
\end{figure}

文字列 \q{two} はコンソールウィンドウに表示されます。

\clearpage
F7またはF8を押して(\stepover) リターンすると\dots \ttf ではなく、 \main にいきます。

\begin{figure}[H]
\centering
\myincludegraphics{patterns/08_switch/1_few/olly8.png}
\caption{\olly: \main へのリターン}
\label{fig:switch_few_olly8}
\end{figure}

はい、 \printf の中心から \main に直接ジャンプしました。
なぜならスタックの\ac{RA}は \ttf ではなく、 \main の場所を指しているからです。
\CALL \TT{0x00FF1000}は \ttf を呼び出した実際の命令です。
