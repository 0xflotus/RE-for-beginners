\mysection{Prologo y epilogo de funciones}
\label{sec:prologepilog}
\myindex{\ESph{}} % Function epilogue
\myindex{\ESph{}} % Function prologue

El prologo de una funcion es una secuencia de instrucciones al inicio de esta.
Por lo general luce mas o menos como el siguiente fragmento de codigo:

\begin{lstlisting}[style=customasmx86]
    push    ebp
    mov     ebp, esp
    sub     esp, X
\end{lstlisting}

Lo que estas instrucciones hacen es: guardan el valor en el registro \EBP, establece el valor del registro \EBP al valor del registro \ESP y luego asigna espacio en la pila para variables locales.

El valor de \EBP permanece igual durante el periodo de ejecucion de la funcion, y es usado para variables locales y acceso a los argumentos.
Para los mismos fines uno puede usar \ESP, pero como este cambia con el tiempo, este enfoque no es muy conveniente.

El epilogo de la funcion libera el espacio asignado en la pila, coloca el valor del registro \EBP vuelta su estado inicial y retorna el control de flujo a la la funcion llamada:

\begin{lstlisting}[style=customasmx86]
    mov    esp, ebp
    pop    ebp
    ret    0
\end{lstlisting}

Los prologos y epilogos de funciones usualmente son detectados por desensambladores para la delimitacion de funciones.

\subsection{\Recursion}
\myindex{\Recursion}

Epilogos y prologos pueden afectar negativamente el rendimiento de la recursion.
Mas acerca de la recursion en este libro: \myref{Recursion_and_tail_call}.

