\subsubsection{ARM}

\myparagraph{\OptimizingXcodeIV (\ARMMode)}

\lstinputlisting[caption=\OptimizingXcodeIV (\ARMMode),style=customasmARM]{patterns/12_FPU/3_comparison/ARM/Xcode_ARM_JPN.lst}

\myindex{ARM!\Registers!APSR}
\myindex{ARM!\Registers!FPSCR}
A very simple case.
The input values are placed into the \GTT{D17} and \GTT{D16} registers and then compared using the \INS{VCMPE} instruction.

Just like in the x86 coprocessor, the ARM coprocessor has its own status and flags register (\ac{FPSCR}),
since there is a necessity to store coprocessor-specific flags.
% TODO -> расписать регистр по битам
\myindex{ARM!\Instructions!VMRS}
And just like in x86, there are no conditional jump instruction in ARM, 
that can check bits in the status register of the coprocessor. 
So there is \INS{VMRS}, which copies 4 bits (N, Z, C, V) from the coprocessor status word into bits of the \IT{general} status register (\ac{APSR}).

\myindex{ARM!\Instructions!VMOVGT}
\INS{VMOVGT} is the analog of the \INS{MOVGT}, 
instruction for D-registers, it executes if one operand is greater than the other while comparing (\IT{GT---Greater Than}). 

If it gets executed, the value of $a$ is to be written into \GTT{D16} (that is currently stored in \GTT{D17}).
Otherwise the value of $b$ stays in the \GTT{D16} register.

\myindex{ARM!\Instructions!VMOV}

The penultimate instruction \INS{VMOV} prepares the value in the \GTT{D16} register for returning it via the \Reg{0} and \Reg{1}
register pair.

\myparagraph{\OptimizingXcodeIV (\ThumbTwoMode)}

\begin{lstlisting}[caption=\OptimizingXcodeIV (\ThumbTwoMode),style=customasmARM]
VMOV            D16, R2, R3 ; b
VMOV            D17, R0, R1 ; a
VCMPE.F64       D17, D16
VMRS            APSR_nzcv, FPSCR
IT GT 
VMOVGT.F64      D16, D17
VMOV            R0, R1, D16
BX              LR
\end{lstlisting}

Almost the same as in the previous example, however slightly different.
As we already know, many instructions in ARM mode can be supplemented by condition predicate.
But there is no such thing in Thumb mode. 
There is no space in the 16-bit instructions for 4 more bits in which conditions can be encoded.

\myindex{ARM!\ThumbTwoMode}

However, Thumb-2 was extended to make it possible to specify predicates to old Thumb instructions.
Here, in the \IDA-generated listing, we see the \INS{VMOVGT} instruction, as in previous example.

In fact, the usual \INS{VMOV} is encoded there, but \IDA adds the \GTT{-GT} suffix to it, 
since there is a \INS{IT GT} instruction placed right before it.

\label{ARM_Thumb_IT}
\myindex{ARM!\Instructions!IT}
\myindex{ARM!if-then block}
The \INS{IT} instruction defines a so-called \IT{if-then block}. 

After the instruction it is possible to place up to 4 instructions, 
each of them has a predicate suffix.
In our example, \INS{IT GT} implies that the next instruction is to be executed, if the \IT{GT} (\IT{Greater Than}) condition is true.

\myindex{Angry Birds}
Here is a more complex code fragment, by the way, from Angry Birds (for iOS):

\begin{lstlisting}[caption=Angry Birds Classic,style=customasmARM]
...
ITE NE
VMOVNE          R2, R3, D16
VMOVEQ          R2, R3, D17
BLX             _objc_msgSend ; not suffixed
...
\end{lstlisting}

\INS{ITE} stands for \IT{if-then-else} 

and it encodes suffixes for the next two instructions.

The first instruction executes if the condition encoded in \INS{ITE} (\IT{NE, not equal}) is true at, and the second---if the condition is not true.
(The inverse condition of \GTT{NE} is \GTT{EQ} (\IT{equal})).

The instruction followed after the second \INS{VMOV} (or \INS{VMOVEQ}) is a normal one, not suffixed (\INS{BLX}).

\myindex{Angry Birds}
One more that's slightly harder, which is also from Angry Birds:

\begin{lstlisting}[caption=Angry Birds Classic,style=customasmARM]
...
ITTTT EQ
MOVEQ           R0, R4
ADDEQ           SP, SP, #0x20
POPEQ.W         {R8,R10}
POPEQ           {R4-R7,PC}
BLX             ___stack_chk_fail ; not suffixed
...
\end{lstlisting}

Four \q{T} symbols in the instruction mnemonic mean that the four subsequent instructions are to be executed if the condition is true.

That's why \IDA adds the \GTT{-EQ} suffix to each one of them. 

And if there was, for example, \INS{ITEEE EQ} (\IT{if-then-else-else-else}), 
then the suffixes would have been set as follows:

\begin{lstlisting}
-EQ
-NE
-NE
-NE
\end{lstlisting}

\myindex{Angry Birds}
Another fragment from Angry Birds:

\begin{lstlisting}[caption=Angry Birds Classic,style=customasmARM]
...
CMP.W           R0, #0xFFFFFFFF
ITTE LE
SUBLE.W         R10, R0, #1
NEGLE           R0, R0
MOVGT           R10, R0
MOVS            R6, #0         ; not suffixed
CBZ             R0, loc_1E7E32 ; not suffixed
...
\end{lstlisting}

\INS{ITTE} (\IT{if-then-then-else}) 

implies that the 1st and 2nd instructions are to be executed if the \GTT{LE} (\IT{Less or Equal})
condition is true, and the 3rd---if the inverse condition (\GTT{GT}---\IT{Greater Than}) 
is true.

Compilers usually don't generate all possible combinations.
\myindex{Angry Birds}

For example, in the mentioned Angry Birds game (\IT{classic} version for iOS)
only these variants of the \INS{IT} instruction are used: 
\INS{IT}, \INS{ITE}, \INS{ITT}, \INS{ITTE}, \INS{ITTT}, \INS{ITTTT}.
\myindex{\GrepUsage}
How to learn this?
In \IDA, it is possible to produce listing files, so it was created with an option to show 4 bytes for each opcode.
Then, knowing the high part of the 16-bit opcode (\INS{IT} is \GTT{0xBF}), we do the following using \GTT{grep}:

\begin{lstlisting}
cat AngryBirdsClassic.lst | grep " BF" | grep "IT" > results.lst
\end{lstlisting}

\myindex{ARM!\ThumbTwoMode}

By the way, if you program in ARM assembly language manually for Thumb-2 mode, 
and you add conditional suffixes,
the assembler will add the \INS{IT} instructions automatically with the required flags where it is necessary.

\myparagraph{\NonOptimizingXcodeIV (\ARMMode)}

\begin{lstlisting}[caption=\NonOptimizingXcodeIV (\ARMMode),style=customasmARM]
b               = -0x20
a               = -0x18
val_to_return   = -0x10
saved_R7        = -4

                STR             R7, [SP,#saved_R7]!
                MOV             R7, SP
                SUB             SP, SP, #0x1C
                BIC             SP, SP, #7
                VMOV            D16, R2, R3
                VMOV            D17, R0, R1
                VSTR            D17, [SP,#0x20+a]
                VSTR            D16, [SP,#0x20+b]
                VLDR            D16, [SP,#0x20+a]
                VLDR            D17, [SP,#0x20+b]
                VCMPE.F64       D16, D17
                VMRS            APSR_nzcv, FPSCR
                BLE             loc_2E08
                VLDR            D16, [SP,#0x20+a]
                VSTR            D16, [SP,#0x20+val_to_return]
                B               loc_2E10

loc_2E08
                VLDR            D16, [SP,#0x20+b]
                VSTR            D16, [SP,#0x20+val_to_return]

loc_2E10
                VLDR            D16, [SP,#0x20+val_to_return]
                VMOV            R0, R1, D16
                MOV             SP, R7
                LDR             R7, [SP+0x20+b],#4
                BX              LR
\end{lstlisting}

Almost the same as we already saw, 
but there is too much redundant code because the $a$ and $b$ variables are stored in the local stack, as well
as the return value.

\myparagraph{\OptimizingKeilVI (\ThumbMode)}

\begin{lstlisting}[caption=\OptimizingKeilVI (\ThumbMode),style=customasmARM]
                PUSH    {R3-R7,LR}
                MOVS    R4, R2
                MOVS    R5, R3
                MOVS    R6, R0
                MOVS    R7, R1
                BL      __aeabi_cdrcmple
                BCS     loc_1C0
                MOVS    R0, R6
                MOVS    R1, R7
                POP     {R3-R7,PC}

loc_1C0
                MOVS    R0, R4
                MOVS    R1, R5
                POP     {R3-R7,PC}
\end{lstlisting}


Keil doesn't generate FPU-instructions since it cannot rely on them being
supported on the target CPU, and it cannot be done by straightforward bitwise comparing.
%TODO1: why?
So it calls an external library function to do the comparison: \GTT{\_\_aeabi\_cdrcmple}. 
\myindex{ARM!\Instructions!BCS}

N.B. The result of the comparison is to be left in the flags by this function, so the following
\INS{BCS} (\IT{Carry set---Greater than or equal})
instruction can work without any additional code.

