\myparagraph{\NonOptimizing MSVC}

MSVC 2010は以下のコードを生成します。

\lstinputlisting[caption=\NonOptimizing MSVC 2010,style=customasmx86]{patterns/12_FPU/3_comparison/x86/MSVC/MSVC_JPN.asm}

\myindex{x86!\Instructions!FLD}

\FLD は\GTT{\_b}を\ST{0}にロードします。

\label{Czero_etc}
\newcommand{\Czero}{\GTT{C0}\xspace}
\newcommand{\Ctwo}{\GTT{C2}\xspace}
\newcommand{\Cthree}{\GTT{C3}\xspace}
\newcommand{\CThreeBits}{\Cthree/\Ctwo/\Czero}

\myindex{x86!\Instructions!FCOMP}

\FCOMP は\ST{0}の値と\GTT{\_a}の値を比較し、
それに応じてFPUステータスワードレジスタの \CThreeBits ビットを設定します。
これは、FPUの現在の状態を反映する16ビットのレジスタです。

ビットがセットされると、 \FCOMP 命令はスタックから1つの変数もポップします。
これは、値を比較してスタックを同じ状態にしておく \FCOM とは区別されます。

残念ながら、インテルP6 
\footnote{インテルP6はPentium Pro、Pentium IIなどです。}
より前のCPUには、 \CThreeBits ビットをチェックする条件付きジャンプ命令はありません。
おそらく、それは歴史の問題です。(思い起こしてみてください:FPUは過去に別のチップでした)

インテルP6で始まる最新のCPUは、\FCOMI/\FCOMIP/\FUCOMI/\FUCOMIP 命令を持っていて、
同じことをしますが、 \ZF/\PF/\CF CPUフラグを変更します。

\myindex{x86!\Instructions!FNSTSW}

\FNSTSW 命令は状態レジスタであるFPUを \AX にコピーします。 
\CThreeBits ビットは14/10/8の位置に配置され、
\AX レジスタの同じ位置にあり、 \AX{}~---\AH{} の上位部分に配置されます。

\begin{itemize}
\item If $b>a$ in our example, then \CThreeBits bits are to be set as following: 0, 0, 0.
\item If $a>b$, then the bits are: 0, 0, 1.
\item If $a=b$, then the bits are: 1, 0, 0.
\item

If the result is unordered (in case of error), then the set bits are: 1, 1, 1.
\end{itemize}
% TODO: table here?

\begin{itemize}
\item この例では $b>a$ の場合、 \CThreeBits ビットは0,0,0と設定します。
\item $a>b$ の場合、ビットは0,0,1です。
\item $a=b$ の場合、ビットは1,0,0です。
\item

結果が順序付けられていない場合(エラーの場合)、セットされたビットは1,1,1,1です。
\end{itemize}
% TODO: table here?

これは、 \CThreeBits ビットが \AX レジスタにどのように配置されるかを示しています。

\input{C3_in_AX}

これは、 \CThreeBits ビットが \AH レジスタにどのように配置されるかを示しています。

\input{C3_in_AH}

\INS{test ah, 5}\footnote{5=101b}の実行後、
\Czero と \Ctwo ビット(0と2の位置)のみが考慮され、他のビットはすべて無視されます。

\label{parity_flag}
\myindex{x86!\Registers!\Flags!Parity flag}

さて、\IT{パリティーフラグ}と注目すべきもう1つの歴史的基礎についてお話しましょう。

このフラグは、最後の計算結果の1の数が偶数の場合は1に設定され、奇数の場合は0に設定されます。

Wikipedia\footnote{\href{http://go.yurichev.com/17131}{wikipedia.org/wiki/Parity\_flag}}を見てみましょう:

\begin{framed}
\begin{quotation}
パリティフラグをテストする一般的な理由の1つに、無関係なFPUフラグをチェックすることがあります。 FPUには4つの条件フラグ
(C0~C3)がありますが、直接テストすることはできず、最初にフラグレジスタにコピーする必要があります。 
これが起こると、C0はキャリーフラグに、C2はパリティフラグに、C3はゼロフラグに置かれます。 
C2フラグは、例えば比較できない浮動小数点値(NaNまたはサポートされていない形式)がFUCOM命令と比較されます。
\end{quotation}
\end{framed}

Wikipediaで述べられているように、パリティフラグはFPUコードで使用されることがあります。

\myindex{x86!\Instructions!JP}

\Czero と \Ctwo の両方が0に設定されている場合、 \PF フラグは1に設定されます。その場合、
後続の \JP (\IT{jump if PF==1})が実行されます。 
いろいろな場合の \CThreeBits の値を思い出すと、
条件ジャンプ \JP は、 $b>a$ または $a=b$ の場合に実行されます。
(\INS{test ah, 5}命令によってクリアされているので、 \Cthree ビットはここでは考慮されていません)

それ以降はすべて簡単です。 
条件付きジャンプが実行された場合、
\FLD は\ST{0}の\GTT{\_b}の値をロードし、
実行されていなければ\GTT{\_a}の値をロードします。

\mysubparagraph{\Ctwo? のチェックは?}

\Ctwo フラグはエラー(\gls{NaN}など)の場合に設定されますが、コードではチェックされません。

プログラマがFPUエラーを気にする場合は、チェックを追加する必要があります。

\clearpage
\mysubparagraph{First \olly example: a=1.2 and b=3.4}
\myindex{\olly}

Let's load the example into \olly:

\begin{figure}[H]
\centering
\myincludegraphics{patterns/12_FPU/3_comparison/x86/MSVC/olly1_1.png}
\caption{\olly: first \FLD has been executed}
\label{fig:FPU_comparison_case1_olly1}
\end{figure}

Current arguments of the function: $a=1.2$ and $b=3.4$ (We can see them in the stack: two pairs of 32-bit values).
$b$ (3.4) is already loaded in \ST{0}.
Now \FCOMP is being executed. 
\olly shows the second \FCOMP argument, which is in stack right now.

\clearpage
\FCOMP has been executed:

\begin{figure}[H]
\centering
\myincludegraphics{patterns/12_FPU/3_comparison/x86/MSVC/olly1_2.png}
\caption{\olly: \FCOMP has been executed}
\label{fig:FPU_comparison_case1_olly2}
\end{figure}

We see the state of the \ac{FPU}'s condition flags: all zeros.
The popped value is reflected as \ST{7}, it was written earlier about reason for this: 
\myref{FPU_is_rather_circular_buffer}.

\clearpage
\FNSTSW has been executed:
\begin{figure}[H]
\centering
\myincludegraphics{patterns/12_FPU/3_comparison/x86/MSVC/olly1_3.png}
\caption{\olly: \FNSTSW has been executed}
\label{fig:FPU_comparison_case1_olly3}
\end{figure}

We see that the \GTT{AX} register contain zeros: indeed, all condition flags are zero.
(\olly disassembles the \FNSTSW instruction as \INS{FSTSW}---they are synonyms).

\clearpage
\TEST has been executed:

\begin{figure}[H]
\centering
\myincludegraphics{patterns/12_FPU/3_comparison/x86/MSVC/olly1_4.png}
\caption{\olly: \TEST has been executed}
\label{fig:FPU_comparison_case1_olly4}
\end{figure}

The \GTT{PF} flag is set to 1.

Indeed: the number of bits set in 0 is 0 and 0 is an even number.
\olly disassembles \INS{JP} as \ac{JPE}---they are synonyms.
And it is about to trigger now.

\clearpage
\ac{JPE} triggered, \FLD loads the value of $b$ (3.4) in \ST{0}:

\begin{figure}[H]
\centering
\myincludegraphics{patterns/12_FPU/3_comparison/x86/MSVC/olly1_5.png}
\caption{\olly: the second \FLD has been executed}
\label{fig:FPU_comparison_case1_olly5}
\end{figure}

The function finishes its work.

\clearpage
\mysubparagraph{Second \olly example: a=5.6 and b=-4}

Let's load example into \olly:

\begin{figure}[H]
\centering
\myincludegraphics{patterns/12_FPU/3_comparison/x86/MSVC/olly2_1.png}
\caption{\olly: first \FLD executed}
\label{fig:FPU_comparison_case2_olly1}
\end{figure}

Current function arguments: $a=5.6$ and $b=-4$.
$b$ (-4) is already loaded in \ST{0}.
\FCOMP about to execute now. 
\olly shows the second \FCOMP argument, which is in stack right now.

\clearpage
\FCOMP executed:

\begin{figure}[H]
\centering
\myincludegraphics{patterns/12_FPU/3_comparison/x86/MSVC/olly2_2.png}
\caption{\olly: \FCOMP executed}
\label{fig:FPU_comparison_case2_olly2}
\end{figure}

We see the state of the \ac{FPU}'s condition flags: all zeros except \Czero.

\clearpage
\FNSTSW executed:

\begin{figure}[H]
\centering
\myincludegraphics{patterns/12_FPU/3_comparison/x86/MSVC/olly2_3.png}
\caption{\olly: \FNSTSW executed}
\label{fig:FPU_comparison_case2_olly3}
\end{figure}

We see that the \GTT{AX} register contains \GTT{0x100}: the \Czero flag is at the 8th bit.

\clearpage
\TEST executed:

\begin{figure}[H]
\centering
\myincludegraphics{patterns/12_FPU/3_comparison/x86/MSVC/olly2_4.png}
\caption{\olly: \TEST executed}
\label{fig:FPU_comparison_case2_olly4}
\end{figure}

The \GTT{PF}  flag is cleared.
Indeed: 

the count of bits set in \GTT{0x100} is 1 and 1 is an odd number.
\ac{JPE} is being skipped now.

\clearpage
\ac{JPE} hasn't been triggered, so \FLD loads the value of $a$ (5.6) in \ST{0}:

\begin{figure}[H]
\centering
\myincludegraphics{patterns/12_FPU/3_comparison/x86/MSVC/olly2_5.png}
\caption{\olly: second \FLD executed}
\label{fig:FPU_comparison_case2_olly5}
\end{figure}

The function finishes its work.

