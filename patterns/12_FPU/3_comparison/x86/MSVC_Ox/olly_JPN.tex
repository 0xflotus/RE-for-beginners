\clearpage
\mysubparagraph{最初の \olly の例: a=1.2 と b=3.4}
\myindex{\olly}

\FLD が両方とも実行されます。

\begin{figure}[H]
\centering
\myincludegraphics{patterns/12_FPU/3_comparison/x86/MSVC_Ox/olly1_1.png}
\caption{\olly: \FLD が両方とも実行される}
\label{fig:FPU_comparison_Ox_case1_olly1}
\end{figure}

\FCOM が実行されます。
\olly は便利なことに、\ST{0} と \ST{1}の内容を表示します。

\clearpage
\FCOM が実行されます。

\begin{figure}[H]
\centering
\myincludegraphics{patterns/12_FPU/3_comparison/x86/MSVC_Ox/olly1_2.png}
\caption{\olly: \FCOM が実行される}
\label{fig:FPU_comparison_Ox_case1_olly2}
\end{figure}

\Czero がセットされ、他の条件フラグはすべてクリアされます。

\clearpage
\FNSTSW が実行され、 \GTT{AX}=0x3100 になります。

\begin{figure}[H]
\centering
\myincludegraphics{patterns/12_FPU/3_comparison/x86/MSVC_Ox/olly1_3.png}
\caption{\olly: \FNSTSW が実行される}
\label{fig:FPU_comparison_Ox_case1_olly3}
\end{figure}

\clearpage
\TEST が実行されます。

\begin{figure}[H]
\centering
\myincludegraphics{patterns/12_FPU/3_comparison/x86/MSVC_Ox/olly1_4.png}
\caption{\olly: \TEST が実行される}
\label{fig:FPU_comparison_Ox_case1_olly4}
\end{figure}

ZF=0の場合、条件ジャンプは実行されます。

\clearpage
\INS{FSTP ST} (または \FSTP \ST{0})が実行されます。1.2がスタックからポップされ、3.4がスタックのトップに残ります。

\begin{figure}[H]
\centering
\myincludegraphics{patterns/12_FPU/3_comparison/x86/MSVC_Ox/olly1_5.png}
\caption{\olly: \FSTP が実行される}
\label{fig:FPU_comparison_Ox_case1_olly5}
\end{figure}

\INS{FSTP ST} を見てみます。

命令は値を1つFPUスタックからポップするだけのように働きます。

\clearpage
\mysubparagraph{次の \olly の例: a=5.6 と b=-4}

\FLD が両方とも実行されます。

\begin{figure}[H]
\centering
\myincludegraphics{patterns/12_FPU/3_comparison/x86/MSVC_Ox/olly2_1.png}
\caption{\olly: \FLD が両方とも実行される}
\label{fig:FPU_comparison_Ox_case2_olly1}
\end{figure}

\FLD が実行されます。

\clearpage
\FCOM が実行されます。

\begin{figure}[H]
\centering
\myincludegraphics{patterns/12_FPU/3_comparison/x86/MSVC_Ox/olly2_2.png}
\caption{\olly: \FCOM が終了する}
\label{fig:FPU_comparison_Ox_case2_olly2}
\end{figure}

条件フラグはすべてクリアされます。

\clearpage
\FNSTSW が完了し、 \GTT{AX}=0x3000 になります。

\begin{figure}[H]
\centering
\myincludegraphics{patterns/12_FPU/3_comparison/x86/MSVC_Ox/olly2_3.png}
\caption{\olly: \FNSTSW が実行される}
\label{fig:FPU_comparison_Ox_case2_olly3}
\end{figure}

\clearpage
\TEST が実行されます。

\begin{figure}[H]
\centering
\myincludegraphics{patterns/12_FPU/3_comparison/x86/MSVC_Ox/olly2_4.png}
\caption{\olly: \TEST が実行される}
\label{fig:FPU_comparison_Ox_case2_olly4}
\end{figure}

ZF=1の場合、ジャンプは発生しません。

\clearpage
\FSTP \ST{1}が実行されます。5.6はFPUスタックのトップにあります。

\begin{figure}[H]
\centering
\myincludegraphics{patterns/12_FPU/3_comparison/x86/MSVC_Ox/olly2_5.png}
\caption{\olly: \FSTP が実行される}
\label{fig:FPU_comparison_Ox_case2_olly5}
\end{figure}

\FSTP \ST{1}命令が以下のように動作することがわかります。値がスタックのトップの残り、\ST{1}がクリアされる。
