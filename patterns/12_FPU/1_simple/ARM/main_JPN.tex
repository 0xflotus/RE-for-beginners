\subsubsection{ARM: \OptimizingXcodeIV (\ARMMode)}

ARMが標準化された浮動小数点サポートを得るまで、いくつかのプロセッサーメーカーは独自の命令拡張を追加しました。 
次に、VFP(\IT{Vector Floating Point})を標準化しました。

x86との重要な違いの1つは、ARMではスタックがなく、
レジスタだけで動作するということです。

\lstinputlisting[label=ARM_leaf_example10,caption=\OptimizingXcodeIV (\ARMMode),style=customasmARM]{patterns/12_FPU/1_simple/ARM/Xcode_ARM_O3_JPN.asm}

\myindex{ARM!D-\registers{}}
\myindex{ARM!S-\registers{}}

そこで、ここではDの接頭辞を使用して新しいレジスタをいくつか見ていきます。

これらは64ビットレジスタで、32個あり、浮動小数点数(double)とSIMD(ARMではNEONと呼ばれます)
の両方に使用できます。

32ビットの32ビットSレジスタもあり、単精度浮動小数点数
(浮動小数点数)として使用されます。

暗記するのは簡単です.Dレジスタは倍精度の数値用であり、Sレジスタは単精度の数値です。
詳細は:\myref{ARM_VFP_registers}

両方の定数(3.14と4.1)はIEEE 754形式でメモリに格納されます。

\myindex{ARM!\Instructions!VLDR}
\myindex{ARM!\Instructions!VMOV}
\INS{VLDR}と\INS{VMOV}は、簡単に推測できるように、\INS{LDR}命令と \MOV 命令に似ていますが、
Dレジスタで動作します。

これらの命令は、Dレジスタと同様に、浮動小数点数だけでなく、
SIMD(NEON)演算にも使用でき、これもすぐに表示されることに注意してください。

引数はRレジスタを介して共通の方法で関数に渡されますが、
倍精度の各数値のサイズは64ビットなので、各レジスタを渡すには2つのRレジスタが必要です。

\INS{VMOV D17, R0, R1}は\Reg{0}と\Reg{1}から2つの32ビット値を1つの64ビット値に合成し、
\GTT{D17}に保存します。

\INS{VMOV R0, R1, D16}は逆の演算です。\GTT{D16}にあったものは、
\Reg{0}と\Reg{1}の2つのレジスタに分割されます。
これは、格納に64ビット必要な倍精度数が\Reg{0}と\Reg{1}に返されるためです。

\myindex{ARM!\Instructions!VDIV}
\myindex{ARM!\Instructions!VMUL}
\myindex{ARM!\Instructions!VADD}
\INS{VDIV}、\INS{VMUL}、\INS{VADD}はそれぞれ\gls{quotient}、\gls{product}、
和を計算する浮動小数点数を処理する命令です。

Thumb-2のコードは同じです。

\subsubsection{ARM: \OptimizingKeilVI (\ThumbMode)}

\lstinputlisting[style=customasmARM]{patterns/12_FPU/1_simple/ARM/Keil_O3_thumb_JPN.asm}

KeilはFPUまたはNEONをサポートしていないプロセッサ用のコードを生成しました。

倍精度浮動小数点数は、汎用Rレジスタを介して渡され、
FPU命令の代わりに浮動小数点数の乗算、除算、加算をエミュレートするサービスライブラリ関数
(\GTT{\_\_aeabi\_dmul}、 \GTT{\_\_aeabi\_ddiv}、 \GTT{\_\_aeabi\_dadd}など)が呼び出されます。

もちろん、それはFPUコプロセッサよりも遅いですが、何もないよりはましです。

ところで、同様のFPUエミュレートライブラリは、コプロセッサが貴重で高価で、
高価なコンピュータにしかインストールされていなかったx86の世界で非常に人気がありました。

\myindex{ARM!soft float}
\myindex{ARM!armel}
\myindex{ARM!armhf}
\myindex{ARM!hard float}

FPUコプロセッサエミュレーションは、ARMワールドでは\IT{ソフトフロート}または\IT{armel}(\IT{エミュレーション})と呼ばれ、
コプロセッサのFPU命令はハードフロートまたは\IT{armhf}と呼ばれます。

\iffalse
% TODO разобраться...
\myindex{Raspberry Pi}

For example, the Linux kernel for Raspberry Pi is compiled in two variants.

In the \IT{soft float} case, arguments are passed via R-registers, and in the \IT{hard float} case---via D-registers.

And that is what stops you from using armhf-libraries from armel-code or vice versa,
and that is why all the code in Linux distributions must be compiled according to a single convention.
\fi

\subsubsection{ARM64: \Optimizing GCC (Linaro) 4.9}

とってもコンパクトなコードです。

\lstinputlisting[caption=\Optimizing GCC (Linaro) 4.9,style=customasmARM]{patterns/12_FPU/1_simple/ARM/ARM64_GCC_O3_JPN.s}

\subsubsection{ARM64: \NonOptimizing GCC (Linaro) 4.9}

\lstinputlisting[caption=\NonOptimizing GCC (Linaro) 4.9,style=customasmARM]{patterns/12_FPU/1_simple/ARM/ARM64_GCC_O0_JPN.s}

\NonOptimizing GCCはもっと冗長です。

いくつかの明確に冗長なコード(最後の2つの\INS{FMOV}命令)を含む、不要な値のシャッフルが多くあります。 
おそらく、GCC 4.9はまだARM64コードを生成するのに適していません。

注目すべきことは、ARM64には64ビットのレジスタがあり、Dレジスタには64ビットのレジスタも含まれているということです。

したがって、コンパイラはローカルスタックではなく\ac{GPR}に \Tdouble 型の値を自由に保存できます。 
これは32ビットCPUでは不可能です。

また、エクササイズとして、\INS{FMADD}のような新しい命令を導入することなく、
この機能を手動で最適化してみることができます。
