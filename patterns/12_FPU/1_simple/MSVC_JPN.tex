\myparagraph{MSVC}

Compile it in MSVC 2010:

\lstinputlisting[caption=MSVC 2010: \ttf{},style=customasmx86]{patterns/12_FPU/1_simple/MSVC_EN.asm}

\FLD takes 8 bytes from stack and loads the number into the \ST{0} register, automatically converting 
it into the internal 80-bit format (\IT{extended precision}).

\myindex{x86!\Instructions!FDIV}

\FDIV divides the value in \ST{0} by the number stored at address \\
\GTT{\_\_real@40091eb851eb851f}~---the value 3.14 is encoded there. 
The assembly syntax doesn't support floating point numbers, so 
what we see here is the hexadecimal representation of 3.14 in 64-bit IEEE 754 format.

After the execution of \FDIV \ST{0} holds the \gls{quotient}.

\myindex{x86!\Instructions!FDIVP}

By the way, there is also the \FDIVP instruction, which divides \ST{1} by \ST{0}, 
popping both these values from stack and then pushing the result. 
If you know the Forth language\FNURLFORTH,
you can quickly understand that this is a stack machine\FNURLSTACK.

The subsequent \FLD instruction pushes the value of $b$ into the stack.

After that, the quotient is placed in \ST{1}, and \ST{0} has the value of $b$.

\myindex{x86!\Instructions!FMUL}

The next \FMUL instruction does multiplication: $b$ from \ST{0} is multiplied by value at\\
\GTT{\_\_real@4010666666666666} (the number 4.1 is there) and leaves the result in the \ST{0} register.

\myindex{x86!\Instructions!FADDP}

The last \FADDP instruction adds the two values at top of stack, storing the result in \ST{1} 
and then popping the value of \ST{0}, thereby leaving the result at the top of the stack, in \ST{0}.

The function must return its result in the \ST{0} register, 
so there are no any other instructions except the function epilogue after \FADDP.

\clearpage
\myparagraph{\Optimizing MSVC + \olly}
\myindex{\olly}

We can try this (optimized) example in \olly.  Here is the first iteration:

\begin{figure}[H]
\centering
\myincludegraphics{patterns/10_strings/1_strlen/olly1.png}
\caption{\olly: first iteration start}
\label{fig:strlen_olly_1}
\end{figure}

We see that \olly found a loop and, for convenience, \IT{wrapped} its instructions in brackets.
By clicking the right button on \EAX, we can choose 
\q{Follow in Dump} and the memory window scrolls to the right place.
Here we can see the string \q{hello!} in memory.
There is at least
one zero byte after it and then random garbage.

If \olly sees a register with a valid address in it, that points to some string, 
it is shown as a string.

\clearpage
Let's press F8 (\stepover) a few times, to get to the start of the body of the loop:

\begin{figure}[H]
\centering
\myincludegraphics{patterns/10_strings/1_strlen/olly2.png}
\caption{\olly: second iteration start}
\label{fig:strlen_olly_2}
\end{figure}

We see that \EAX contains the address of the second character in the string.

\clearpage

We have to press F8 enough number of times in order to escape from the loop:

\begin{figure}[H]
\centering
\myincludegraphics{patterns/10_strings/1_strlen/olly3.png}
\caption{\olly: pointers difference to be calculated now}
\label{fig:strlen_olly_3}
\end{figure}

We see that \EAX now contains the address of zero byte that's right after the string plus 1 (because INC EAX was executed regardless of whether
we exit from the loop or not).
Meanwhile, \EDX hasn't changed,
so it still pointing to the start of the string.

The difference between these two addresses is being calculated now.

\clearpage
The \SUB instruction just got executed:

\begin{figure}[H]
\centering
\myincludegraphics{patterns/10_strings/1_strlen/olly4.png}
\caption{\olly: \EAX to be decremented now}
\label{fig:strlen_olly_4}
\end{figure}

The difference of pointers is in the \EAX register now---7.
Indeed, the length of the \q{hello!} string is 6, 
but with the zero byte included---7.
But \TT{strlen()} must return the number of non-zero characters in the string.
So the decrement executes and then the function returns.

