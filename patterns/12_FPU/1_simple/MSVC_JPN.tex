\myparagraph{MSVC}

MSVC 2010でコンパイルしましょう。

\lstinputlisting[caption=MSVC 2010: \ttf{},style=customasmx86]{patterns/12_FPU/1_simple/MSVC_JPN.asm}

\FLD はスタックから8バイトを取り出し、その数値を\ST{0}レジスタにロードし、内部80ビットフォーマット
(\IT{拡張精度})に自動的に変換します。

\myindex{x86!\Instructions!FDIV}

\FDIV は、\ST{0}の値をアドレス\GTT{\_\_real@40091eb851eb851f}~に格納された数値で除算します。
値3.14はそこにエンコードされます。
アセンブリ構文は浮動小数点数をサポートしていないので、64ビットIEEE 754形式での3.14の16進表現です。

\FDIV \ST{0}の実行後に\gls{quotient}が保持されます。

\myindex{x86!\Instructions!FDIVP}

ちなみに、 \FDIVP 命令もあります。これは、\ST{1}を\ST{0}で除算し、
これらの値をスタックからポップし、その結果をプッシュします。
あなたがForth言語\FNURLFORTH を知っていれば、
すぐにこれがスタックマシン\FNURLSTACK であることがわかります。

後続の \FLD 命令は、 $b$ の値をスタックにプッシュします。

その後、商は\ST{1}に置かれ、\ST{0}は $b$ の値を持ちます。

\myindex{x86!\Instructions!FMUL}

次の \FMUL 命令は乗算を行います。\ST{0}の $b$ は\GTT{\_\_real@4010666666666666}
(そこには4.1が入る)の値で乗算され、結果は\ST{0}レジスタに残ります。

\myindex{x86!\Instructions!FADDP}

最後の \FADDP 命令は、スタックの先頭に2つの値を加算し、結果を\ST{1}に格納した後、
\ST{0}の値をポップし、\ST{0}のスタックの先頭に結果を残します。

関数はその結果を\ST{0}レジスタに戻す必要があるため、
\FADDP 後の関数エピローグ以外の命令はありません。

\clearpage
\myparagraph{MSVC + \olly}
\myindex{\olly}

2組の32ビットワードは、スタック内で赤色でマークされます。
各ペアはIEEE 754形式の倍数で、 \main から渡されます。

最初の \FLD がスタックからどのように値($1.2$)をロードし、それを\ST{0}に入れるかを確認します。

\begin{figure}[H]
\centering
\myincludegraphics{patterns/12_FPU/1_simple/olly1.png}
\caption{\olly: 最初の \FLD が実行された}
\label{fig:FPU_simple_olly_1}
\end{figure}

64ビットIEEE 754浮動小数点から80ビット(FPU内部で使用される)への避けられない変換エラーのため、
ここでは1.299に近い1.1999\ldots が見られます。

\EIP は次の命令(\FDIV)を指すようになり、メモリから倍精度浮動小数点数(定数)がロードされます。
便宜上、 \olly は3.14を示します。

\clearpage
さらにトレースしましょう。 
\FDIV が実行されましたが、\ST{0}に0.382\ldots
(\gls{quotient})が含まれています。

\begin{figure}[H]
\centering
\myincludegraphics{patterns/12_FPU/1_simple/olly2.png}
\caption{\olly: \FDIV が実行された}
\label{fig:FPU_simple_olly_2}
\end{figure}

\clearpage
3番目のステップ:次の \FLD が実行され、
\ST{0}に3.4をロードします(ここでは、およそ3.39999\ldots の値が確認できます)。

\begin{figure}[H]
\centering
\myincludegraphics{patterns/12_FPU/1_simple/olly3.png}
\caption{\olly: 2回目の \FLD が実行された}
\label{fig:FPU_simple_olly_3}
\end{figure}

同時に、\gls{quotient}は\ST{1}に\IT{プッシュ}されます。
今、 \EIP は次の命令 \FMUL を指しています。
これは、 \olly が示すメモリから定数4.1をロードします。

\clearpage
次: \FMUL が実行されたので、\gls{product}は\ST{0}になります。

\begin{figure}[H]
\centering
\myincludegraphics{patterns/12_FPU/1_simple/olly4.png}
\caption{\olly: \FMUL が実行された}
\label{fig:FPU_simple_olly_4}
\end{figure}

\clearpage
次に、 \FADDP が実行され、加算結果が\ST{0}になり、\ST{1}がクリアされます。

\begin{figure}[H]
\centering
\myincludegraphics{patterns/12_FPU/1_simple/olly5.png}
\caption{\olly: \FADDP が実行された}
\label{fig:FPU_simple_olly_5}
\end{figure}

関数はその値を\ST{0}に戻すため、結果は\ST{0}に残ります。

\main は後でこの値をレジスタから取得します。

13.93\ldots 値は現在\ST{7}に位置しています。
どうしてでしょうか?

\label{FPU_is_rather_circular_buffer}

この本の中で少し前に読んだことがあるように、\ac{FPU}レジスタはスタックです。\myref{FPU_is_stack}
しかしこれは単純化されています。

説明したように\IT{ハードウェアで}実装されているとしたら、
プッシュとポップ中に7つのレジスタのすべての内容を隣接するレジスタに移動(またはコピー)する必要があります。

現実には、\ac{FPU}は8つのレジスタと、現在の\q{トップ・オブ・スタック}であるレジスタ番号を含む
ポインタ(\GTT{TOP}と呼ばれる)とを持ちます。

値がスタックにプッシュされると、\GTT{TOP}は次に使用可能なレジスタをポイントし、
そのレジスタに値が書き込まれます。

値がポップされると、プロシージャは元に戻されますが、解放されたレジスタはクリアされません
(クリアされる可能性がありますが、パフォーマンスが低下する可能性があります)。
それがここにある理由です。

\FADDP はスタックに合計を保存した後、要素を1つポップしたと言えるでしょう。

しかし、実際には、この命令は合計を保存してから\GTT{TOP}にシフトします。

より正確には、\ac{FPU}のレジスタは循環バッファです。

