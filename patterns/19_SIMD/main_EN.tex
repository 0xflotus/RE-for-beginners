\mysection{SIMD}

\label{SIMD_x86}
\ac{SIMD} is an acronym: \IT{Single Instruction, Multiple Data}.

As its name implies, it processes multiple data using only one instruction.

Like the \ac{FPU}, that \ac{CPU} subsystem looks like a separate processor inside x86.

\myindex{x86!MMX}

SIMD began as MMX in x86. 8 new 64-bit registers appeared: MM0-MM7.

Each MMX register can hold 2 32-bit values, 4 16-bit values or 8 bytes.
For example, it is possible to add 8 8-bit values (bytes) simultaneously by adding two values in MMX registers.

One simple example is a graphics editor that represents an image as a two dimensional array.
When the user changes the brightness of the image, the editor must add or subtract a coefficient to/from each pixel value.
For the sake of brevity if we say that the image is grayscale and each pixel is defined by one 8-bit byte, then it is possible
to change the brightness of 8 pixels simultaneously.

By the way, this is the reason why the \IT{saturation} instructions are present in SIMD.

When the user changes the brightness in the graphics editor, overflow and underflow are not desirable, 
so there are addition instructions in SIMD which are not adding anything if the maximum value is reached, etc.

When MMX appeared, these registers were actually located in the FPU's registers. 
It was possible to use either FPU or MMX at the same time. One might think that Intel saved on transistors,
but in fact the reason of such symbiosis was simpler~---older \ac{OS}es that are not aware 
of the additional CPU registers would not save them at the context switch, 
but saving the FPU registers.
Thus, MMX-enabled CPU + old \ac{OS} + process utilizing MMX features will still work.

\myindex{x86!SSE}
\myindex{x86!SSE2}
SSE---is extension of the SIMD registers to 128 bits, now separate from the FPU.

\myindex{x86!AVX}
AVX---another extension, to 256 bits.

Now about practical usage.

Of course, this is memory copy routines (\TT{memcpy}), memory comparing (\TT{memcmp}) and so on.

\myindex{DES}

One more example: the DES encryption algorithm takes a 64-bit block and a 56-bit key, encrypt the block and produces a 64-bit result.
The DES algorithm may be considered as a very large electronic circuit, with wires and AND/OR/NOT gates.

\label{bitslicedes}
\newcommand{\URLBS}{\url{http://go.yurichev.com/17329}}

Bitslice DES\footnote{\URLBS}~---is the idea of processing groups of blocks and keys simultaneously.
Let's say, variable of type \IT{unsigned int} on x86 can hold up to 32 bits, so it is possible to store there
intermediate results for 32 block-key pairs simultaneously, using 64+56 variables of type \IT{unsigned int}.

\myindex{\oracle}
There is an utility to brute-force \oracle passwords/hashes (ones based on DES),
using slightly modified bitslice DES algorithm for SSE2 and AVX---now it is possible to encrypt 128 
or 256 block-keys pairs simultaneously.

\url{http://go.yurichev.com/17313}

% sections
\input{patterns/19_SIMD/vectorization_EN.tex}
\input{patterns/19_SIMD/strlen_EN.tex}

