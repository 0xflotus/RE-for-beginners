\mysection{SIMD}

\label{SIMD_x86}
\ac{SIMD} ist ein Akronym: \IT{Single Instruction, Multiple Data}.
Wie der Name schon sagt, verarbeitet SIMD mehrere Daten in nur einem Befehl.

Wie die \ac{FPU} sieht das \ac{CPU}-Subsystem wie ein separater Prozessor innerhalb von x86 aus.

\myindex{x86!MMX}

SIMD begann als MMX in x86. 8 neue 64-Bit-Register erschienen: MM0-MM7.

Jedes MMX Register kann 2 32-Bit-Werte, 4 16-Bit-Werte oder 8 Byte aufnehmen.
Es ist zum Beispiel möglich, 8 8-Bit-Werte gleichzeitig zu addieren, indem zwei Werte in MMX Registern addiert werden.

Ein einfaches Beispiel ist ein Graphikeditor, der ein Bild als ein zweidimensionales Array darstellt.
Wenn der User die Helligkeit des Bildes verändert, muss der Editor einen bestimmten Koeffizienten zu jedem Pixelwert
addieren bzw. von ihm subtrahieren.
Wenn wir um es einfach zu halten annehmen, dass das Bild in schwarzweiß ist und jeder Pixel durch ein Byte definiert
ist, dann ist es möglich die Helligkeit von 8 Pixeln gleichzeitig zu verändern.

Das ist übrigens der Grund dafür, dass es in SIMB die Sättigungsbefehle gibt.

Wenn der User die Helligkeit im Graphikeditor verändert, sind Überlauf und Unterlauf nicht erwünscht, weshalb es in SIMD
Additionsbefehle gibt, die nicht weiter addieren, wenn der Maximalwert bereits erreicht its, etc.

Als MMX erschien befanden sich diese Register räumlich innerhalb der FPU Register.
Es war möglich entweder die FPU oder MMX zur gleichen Zeit zu benutzen. Man könnte meinen, dass Intel dieses Layout
gewählt hat um Transistoren zu sparen, aber der wirkliche Grund für diese Symbiose ist trivialer~---ältere
Betriebssysteme, die die zusätzlichen CPU Register nicht erwarteten, hätten deren Inhalte nicht gesichert, sicherten
aber sehr wohl den Inhalt der FPU Register.
Dadurch funktioniert die Kombination aus MMX CPU und altem Betriebssystem, wenn MMX Features verwendet werden. 

\myindex{x86!SSE}
\myindex{x86!SSE2}
SSE ist eine Erweriterung der SIMB Register auf 128 Bit. Die Register sind hier von der FPU getrennt.

\myindex{x86!AVX}
AVX ist eine andere Erweiterung--auf 256 Bits.
Kommen wir nun zur Verwendung in der Praxis.
Natürlich gibt es Funktionen zum Kopieren (\TT{memcpy}) oder Vergleichen (\TT{memcmp}) von Speicherblöcken etc.

\myindex{DES}
Ein weiteres Beispiel: der DES-Verschlüsselungsalgorithmus nimmt einen 64-Bit-Block und einen 56-Bit-Key, verschlüsselt
den Block und erzeugt ein 64-Bit-Ergebnis. 
Der DES-Algorithmus kann als sehr großer Schaltkreis aus Drähten und AND/OR/NOT-Gattern aufgefasst werden.

\label{bitslicedes}
\newcommand{\URLBS}{\url{http://go.yurichev.com/17329}}
Hinter Bitslice DES\footnote{\URLBS}~---steckt die Idee, mehrere Gruppen von Blöcken und Schlüsseln simultan zu
verarbeiten. Eine Variable vom Typ \IT{unsigned int} umfasst in x86 beispielsweise 32 Bit, sodass es möglich ist, in ihr
die Zwischenergebnisse für 32 Block-Schlüssel-Paare gleichzeitig zu speichern, indem $64+56$ Variablen vom Typ
\IT{unsigned int} verwendet werden.

\myindex{\oracle}
Es gibt es Tool um \oracle Passwörter und Hashes (die auf DES basieren) per Brute-Force zu knacken. Hierbei wird ein nur
wenig veränderter Bitslice-DES-Algorithmus für SSE2 und AVX verwendet---nun ist es möglich, 128 oder 256
Block-Schlüssel-Paare simultan zu verschlüsseln.

\url{http://go.yurichev.com/17313}

% sections
\input{patterns/19_SIMD/vectorization_DE.tex}
\input{patterns/19_SIMD/strlen_DE.tex}

