\mysection{SIMD}

\label{SIMD_x86}
\ac{SIMD} это акроним: \IT{Single Instruction, Multiple Data}.

Как можно судить по названию, это обработка множества данных исполняя только одну инструкцию.

Как и \ac{FPU}, эта подсистема процессора выглядит так же отдельным процессором внутри x86.

\myindex{x86!MMX}
SIMD в x86 начался с MMX. Появилось 8 64-битных регистров MM0-MM7.

Каждый MMX-регистр может содержать 2 32-битных значения, 4 16-битных или же 8 байт. 
Например, складывая значения двух MMX-регистров, можно складывать одновременно 8 8-битных значений.

Простой пример, это некий графический редактор, который хранит открытое изображение как двумерный массив. 
Когда пользователь меняет яркость изображения, редактору нужно, например, прибавить некий коэффициент 
ко всем пикселям, или отнять. 
Для простоты можно представить, что изображение у нас бело-серо-черное и каждый пиксель занимает один байт, 
то с помощью MMX можно менять яркость сразу у восьми пикселей.

Кстати, вот причина почему в SIMD присутствуют инструкции с \IT{насыщением} (\IT{saturation}).

Когда пользователь в графическом редакторе изменяет яркость, переполнение и антипереполнение (\IT{underflow})
не нужны, так что в SIMD имеются, например, инструкции сложения, которые ничего не будут прибавлять
если максимальное значение уже достигнуто, итд.

Когда MMX только появилось, эти регистры на самом деле располагались в FPU-регистрах. 
Можно было использовать 
либо FPU либо MMX в одно и то же время. Можно подумать, что Intel решило немного сэкономить на транзисторах, 
но на самом деле причина такого симбиоза проще ~--- более старая \ac{OS} не знающая о дополнительных 
регистрах процессора не будет сохранять их во время переключения задач, а вот регистры FPU сохранять будет. 
Таким образом, процессор с MMX + старая \ac{OS} + задача, использующая возможности MMX = все 
это может работать вместе.

\myindex{x86!SSE}
\myindex{x86!SSE2}
SSE --- это расширение регистров до 128 бит, теперь уже отдельно от FPU.

\myindex{x86!AVX}
AVX --- расширение регистров до 256 бит.

Немного о практическом применении.

Конечно же, это копирование блоков в памяти (\TT{memcpy}), сравнение (\TT{memcmp}), и подобное.

\myindex{DES}
Еще пример: имеется алгоритм шифрования DES, который берет 64-битный блок, 56-битный ключ, 
шифрует блок с ключом и образуется 64-битный результат.
Алгоритм DES можно легко представить в виде очень большой электронной цифровой схемы, 
с проводами, элементами И, ИЛИ, НЕ.

\label{bitslicedes}
\newcommand{\URLBS}{\url{http://go.yurichev.com/17329}}

Идея bitslice DES\footnote{\URLBS} ~--- это обработка сразу группы блоков и ключей одновременно. 
Скажем, на x86 переменная типа \IT{unsigned int} вмещает в себе 32 бита, так что там можно хранить 
промежуточные результаты сразу для 32-х блоков-ключей, используя 64+56 переменных типа \IT{unsigned int}.

\myindex{\oracle}
Существует утилита для перебора паролей/хешей \oracle (которые основаны на алгоритме DES), 
реализующая алгоритм bitslice DES для SSE2 и AVX --- и теперь возможно шифровать одновременно 
128 или 256 блоков-ключей:

\url{http://go.yurichev.com/17313}

% sections
\input{patterns/19_SIMD/vectorization_RU.tex}
\input{patterns/19_SIMD/strlen_RU.tex}

