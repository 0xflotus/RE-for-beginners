% TODO to be resynced with EN version
\mysection{La función más simple}

La función más simple posible es sin duda uno que simplemente devuelve un valor constante:

Ejemplo:

\lstinputlisting[caption=\ESph{},style=customc]{patterns/011_ret/1.c}

Vamos a compilar!

\subsection{x86}

Esto es lo que optimiza el GCC  y los compiladores de MSVC los cuales se producen en la plataforma x86:

\lstinputlisting[caption=\Optimizing GCC/MSVC (\assemblyOutput),style=customasmx86]{patterns/011_ret/1.s}

\myindex{x86!\Instructions!RET}
Hay solo dos instrucciones: los primeros lugares el valor 123 en el registro \EAX, que se utiliza por convención para almacenar el valor de retorno y el segundo es \RET, que devuelve la ejecución al llamado.

La persona que llama tomar el resultado del registro \EAX.

\subsection{ARM}

Hay algunas diferencias en la plataforma ARM:

\lstinputlisting[caption=\OptimizingKeilVI (\ARMMode) ASM Output,style=customasmARM]{patterns/011_ret/1_Keil_ARM_O3.s}

ARM utiliza el registro \Reg{0} para el retorno de los resultados de las funciones, por lo que se copia 123 en \Reg{0}.

La dirección que retorna no es guardada en la pila local de la ARM \ac{ISA}, sino más bien en el registro de enlace,
entonces la instrucción \INS{BX LR} provoca que la ejecución de un salto a la dirección eficazmente retornando la ejecución a donde fue llamada(caller ``nuestro llamador'').

\myindex{ARM!\Instructions!MOV}
\myindex{x86!\Instructions!MOV}
Vale la pena señalar que \MOV es un nombre engañoso para la instrucción tanto en x86 y ARM \ac{ISA}s.

Los datos de hecho, no se \IT{mueven}, sino que se \IT{copian}.

\subsection{MIPS}

\label{MIPS_leaf_function_ex1}
Hay dos convenciones de nomenclatura utilizadas en el mundo de MIPS al nombrar registros: por número (de \$0 a \$31) o por seudónimo (\$V0, \$A0, etc.).

El ensamblado de GCC produce una salida en listando los registros por numeros:

\lstinputlisting[caption=\Optimizing GCC 4.4.5 (\assemblyOutput),style=customasmMIPS]{patterns/011_ret/MIPS.s}

\dots Mientras que \IDA lo hace por sus seudónimos:

\lstinputlisting[caption=\Optimizing GCC 4.4.5 (IDA),style=customasmMIPS]{patterns/011_ret/MIPS_IDA.lst}

El \$2 (o \$V0) registro es usado para almacenar el valor devuelto por la función.
\myindex{MIPS!\Pseudoinstructions!LI}
\INS{LI} significa ``Load Immediate'' (``carga inmediata'') y es el MIPS equivalente a \MOV.

\myindex{MIPS!\Instructions!J}
La otra instrucción es el jump (``salto'') que es  (J o JR) el cual retorna la ejecución fluida para el llamado, da un salto a la dirección del registro \$31 (o \$RA).

Este es el registro análoga a \ac{LR} en ARM.

\myindex{MIPS!Branch delay slot}
Es posible que se pregunte por qué posiciones de la instrucción de la carga (LI) y el salto instrucciones (J y JR) se intercambian.Esto es debido a una característica llamada RISC (``ranura de retardo rama'').

La razón por la que esto sucede es una peculiaridad en la arquitectura RISC de algunas ISA y no es importante para nuestros propósitos --– sólo tenemos que recordar que en MIPS, la instrucción que sigue a un salto o instrucción de salto se ejecuta antes de la instrucción / rama salto en sí.

Como consecuencia, las instrucciones de ramificación siempre intercambiar lugares con la instrucción que debe ser ejecutado de antemano.

\subsubsection{Una nota acerca de la instrucción MIPS nombres / Registro}

Registros y nombre de instrucciones en el mundo de MIPS tradicionalmente se escriben en minúsculas.
Sin embargo, en aras de la  coherencia, que me quedo con el uso de letras mayúsculas, ya que es la convención seguida por el resto de las ISAs destacados aquí.

% TBT

