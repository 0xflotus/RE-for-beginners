% TODO to be resynced with EN version
\mysection{De meest eenvoudige functie}

De meest eenvoudige functie is er een die simpelweg een constante waarde terug geeft:

Hier is ze:

\lstinputlisting[caption=\NLph{},style=customc]{patterns/011_ret/1.c}

Compileren maar!

\subsection{x86}

Hier zie je wat zowel de optimaliserende GCC als MSVC compilers produceren op een x86 platform:

\lstinputlisting[caption=\Optimizing GCC/MSVC (\assemblyOutput),style=customasmx86]{patterns/011_ret/1.s}

\myindex{x86!\Instructions!RET}
Er zijn slechts twee instructies: de eerste plaatst de waarde 123 in het \EAX register, hetwelk volgens de conventies gebruikt wordt om de return value op te slaan en de tweede is \RET, wat de programma-uitvoering terug geeft aan de \gls{caller}.

De caller zal dan het resultaat uit het \EAX register halen.

\subsection{ARM}

Er zijn een aantal verschillen met het ARM platform:

\lstinputlisting[caption=\OptimizingKeilVI (\ARMMode) ASM Output,style=customasmARM]{patterns/011_ret/1_Keil_ARM_O3.s}

ARM gebruikt het register \Reg{0} om de resultaten van functies te returnen. 123 wordt dus gekopieerd in \Reg{0}.

Het return address wordt niet bijgehouden op de local stack in het ARM \ac{ISA}, maar eerder in het link register, de \INS{BX LR} instructie zorgt er dus voor dat de programma-executie naar dat adres springt, en op die manier de uitvoering terug geeft aan de \gls{caller}.

\myindex{ARM!\Instructions!MOV}
\myindex{x86!\Instructions!MOV}
Het is een vermelding waard dat \MOV een misleidende naam is voor de instructie in zowel x86 als ARM \ac{ISA}s.

De data wordt feitelijk niet \IT{verplaatst}, maar \IT{gekopieerd}.

\subsection{MIPS}

\label{MIPS_leaf_function_ex1}
In de wereld van MIPS zijn er twee afspraken mbt de benaming wanneer het op registers aankomt:
genummerd (van \$0 tot \$31) of per pseudoniem (\$V0, \$A0, etc.).

De GCC assembly output hieronder lijst de registers genummerd op:

\lstinputlisting[caption=\Optimizing GCC 4.4.5 (\assemblyOutput),style=customasmMIPS]{patterns/011_ret/MIPS.s}

\dots terwijl \IDA dit doet per pseudoniem:

\lstinputlisting[caption=\Optimizing GCC 4.4.5 (IDA),style=customasmMIPS]{patterns/011_ret/MIPS_IDA.lst}

Het \$2 (of \$V0) register wordt gebruikt om de return value van de functie in op te slaan.
\myindex{MIPS!\Pseudoinstructions!LI}
\INS{LI} staat voor ``Load Immediate'' en is het MIPS equivalent voor MOV.

\myindex{MIPS!\Instructions!J}
De andere instructie is de jump instructie (J of JR), dewelke de uitvoering van het programma terug geeft aan de \gls{caller}, door een jump naar het adres in het \$31 (of \$RA) register.

Dit register staat analoog tot het \ac{LR} in ARM.

\myindex{MIPS!Branch delay slot}
Je kan je wel afvragen waarom de posities van de load instructie (LI) en de jump instructie (J of JR) omgewisseld zijn. Dit komt door een \ac{RISC} feature genaamd ``branch delay slot''.

De reden dat dit gebeurd is een kronkel in de architectuur van sommige RISC \ac{ISA}s en is niet belangrijk voor onze doeleinden - we dienen enkel te onthouden dat in MIPS, de instructie die volgt op een jump of branch instructie uitgevoerd wordt \IT{voor} de jump/branch instructie zelf.

Bijgevolg verwisselen branch instructies steeds met de instructie die op voorhand moet worden uitgevoerd.

\subsubsection{Nota over MIPS instructies/register benaming}

Register- en instructienamen in de wereld van MIPS zijn traditioneel geschreven in kleine letters.
Echter, om alles consistent te houden, zullen wij gebruik blijven maken van hoofdletters, aangezien dit de gevolgde afspraken zijn die gevolgd worden door alle andere \ac{ISA}s in dit boek.

% TBT

