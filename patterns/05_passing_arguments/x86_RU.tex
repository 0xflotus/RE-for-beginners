\subsection{x86}

\subsubsection{MSVC}

Рассмотрим пример, скомпилированный в (MSVC 2010 Express):

\lstinputlisting[label=src:passing_arguments_ex_MSVC_cdecl,caption=MSVC 2010 Express,style=customasmx86]{patterns/05_passing_arguments/msvc_RU.asm}

\myindex{x86!\Registers!EBP}
Итак, здесь видно: в функции \main заталкиваются три числа в стек и вызывается функция \TT{f(int,int,int)}.
 
Внутри \ttf доступ к аргументам, также как и к локальным переменным, происходит через макросы: 
\TT{\_a\$ = 8}, но разница в том, что эти смещения со знаком \IT{плюс}, 
таким образом если прибавить макрос \TT{\_a\$} к указателю на \EBP, то адресуется \IT{внешняя} 
часть \glslink{stack frame}{фрейма} стека относительно \EBP.

\myindex{x86!\Instructions!IMUL}
\myindex{x86!\Instructions!ADD}
Далее всё более-менее просто: значение $a$ помещается в \EAX. 
Далее \EAX умножается при помощи инструкции \IMUL на то, что лежит в \TT{\_b}, 
и в \EAX остается \glslink{product}{произведение} этих двух значений.

Далее к регистру \EAX прибавляется то, что лежит в \TT{\_c}.

Значение из \EAX никуда не нужно перекладывать, оно уже лежит где надо. 
Возвращаем управление вызывающей функции~--- она возьмет значение из \EAX и отправит его в \printf.

\clearpage
\myparagraph{\Optimizing MSVC + \olly}
\myindex{\olly}

Можем попробовать этот (соптимизированный) пример в \olly.  Вот самая первая итерация:

\begin{figure}[H]
\centering
\myincludegraphics{patterns/10_strings/1_strlen/olly1.png}
\caption{\olly: начало первой итерации}
\label{fig:strlen_olly_1}
\end{figure}

Видно, что \olly обнаружил цикл и, для удобства, \IT{свернул} инструкции тела цикла в скобке.

Нажав правой кнопкой на \EAX, можно выбрать \q{Follow in Dump} 
и позиция в окне памяти будет как раз там, где надо.

Здесь мы видим в памяти строку \q{hello!}.
После неё имеется как минимум 1 нулевой байт, затем случайный мусор.
Если \olly видит, что в регистре содержится адрес какой-то строки, он показывает эту строку.

\clearpage
Нажмем F8 (\stepover) столько раз, чтобы текущий адрес снова был в начале тела цикла:

\begin{figure}[H]
\centering
\myincludegraphics{patterns/10_strings/1_strlen/olly2.png}
\caption{\olly: начало второй итерации}
\label{fig:strlen_olly_2}
\end{figure}

Видно, что \EAX уже содержит адрес второго символа в строке.

\clearpage
Будем нажимать F8 достаточное количество раз, чтобы выйти из цикла:

\begin{figure}[H]
\centering
\myincludegraphics{patterns/10_strings/1_strlen/olly3.png}
\caption{\olly: сейчас будет вычисление разницы указателей}
\label{fig:strlen_olly_3}
\end{figure}

Увидим, что \EAX теперь содержит адрес нулевого байта, следующего сразу за строкой плюс 1 (потому что INC EAX исполнился вне зависимости
от того, выходим мы из цикла, или нет).

А \EDX так и не менялся~--- он всё ещё указывает на начало строки.
Здесь сейчас будет вычисляться разница между этими двумя адресами.

\clearpage
Инструкция \SUB исполнилась:

\begin{figure}[H]
\centering
\myincludegraphics{patterns/10_strings/1_strlen/olly4.png}
\caption{\olly: сейчас будет декремент \EAX}
\label{fig:strlen_olly_4}
\end{figure}

Разница указателей сейчас в регистре \EAX~--- 7.

Действительно, длина строки \q{hello!}~--- 6, 
но вместе с нулевым байтом --- 7.
Но \TT{strlen()} должна возвращать количество ненулевых символов в строке.
Так что сейчас будет исполняться декремент и выход из функции.



\subsubsection{GCC}

Скомпилируем то же в GCC 4.4.1 и посмотрим результат в \IDA:

\lstinputlisting[caption=GCC 4.4.1,style=customasmx86]{patterns/05_passing_arguments/gcc_RU.asm}

Практически то же самое, если не считать мелких отличий описанных ранее.

После вызова обоих функций \glslink{stack pointer}{указатель стека} не возвращается назад, 
потому что предпоследняя инструкция \TT{LEAVE} (\myref{x86_ins:LEAVE}) делает это за один раз, в конце исполнения.

