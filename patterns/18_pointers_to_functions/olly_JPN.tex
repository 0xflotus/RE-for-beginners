\clearpage
\subsubsection{MSVC + \olly}
\myindex{\olly}

例を \olly にロードし、 \comp にブレークポイントを設定しましょう。
最初の \comp 呼び出しで値がどのように比較されるかを見ることができます。

\begin{figure}[H]
\centering
\myincludegraphics{patterns/18_pointers_to_functions/olly1.png}
\caption{\olly: \comp の最初の呼び出し}
\label{fig:qsort_olly1}
\end{figure}

便宜上、 \olly はコードウィンドウの下のウィンドウに比較値を表示します。
また、\ac{SP}が\ac{RA}を指していることがわかります( \qsort 関数は\TT{MSVCR100.DLL}にあります)。

\clearpage
\TT{RETN}命令までトレースし(F8)、もう一度F8を押すと、 \qsort 関数に戻ります。

\begin{figure}[H]
\centering
\myincludegraphics{patterns/18_pointers_to_functions/olly2.png}
\caption{\olly: \comp 呼び出し直後の \qsort のコード}
\label{fig:qsort_olly2}
\end{figure}

それは比較関数への呼び出しでした。

\clearpage
これは \comp の2回目の呼び出しの瞬間のスクリーンショットでもあり、比較する必要がある値は異なります。

\begin{figure}[H]
\centering
\myincludegraphics{patterns/18_pointers_to_functions/olly3.png}
\caption{\olly: \comp の2回目の呼び出し}
\label{fig:qsort_olly3}
\end{figure}
