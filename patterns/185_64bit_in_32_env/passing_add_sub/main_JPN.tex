\subsection{Arguments passing, addition, subtraction}

\lstinputlisting[style=customc]{patterns/185_64bit_in_32_env/passing_add_sub/1.c}

\subsubsection{x86}

\lstinputlisting[caption=\Optimizing MSVC 2012 /Ob1,style=customasmx86]{patterns/185_64bit_in_32_env/passing_add_sub/1_MSVC.asm}

\GTT{f\_add\_test()}関数では、各64ビット値が2つの32ビット値を使用して渡されることを確認できます。
上位部分が最初に、次に下位部分になります。

足し算と引き算もペアで行われます。

\myindex{x86!\Instructions!ADC}
さらに、下位32ビット部分が最初に追加されます。 
加算中にキャリーが発生した場合は、\TT{CF}フラグが設定されます。

次の\INS{ADC}命令は、値の上位部分を加算し、 $CF=1$ の場合は1を加算します。

\myindex{x86!\Instructions!SBB}
減算もペアで行われます。 
最初の \SUB は、後続の \INS{SBB} 命令でチェックされるCFフラグをオンにすることもできます。
キャリーフラグがオンの場合は、結果から1も減算されます。

\GTT{f\_add()}関数の結果がどのように \printf{} に渡されるのかを理解するのは簡単です。

\lstinputlisting[caption=GCC 4.8.1 -O1 -fno-inline,style=customasmx86]{patterns/185_64bit_in_32_env/passing_add_sub/1_GCC.asm}

GCCのコードも同様です。

\subsubsection{ARM}

\lstinputlisting[caption=\OptimizingKeilVI (\ARMMode),style=customasmARM]{patterns/185_64bit_in_32_env/passing_add_sub/Keil_ARM_O3.s}

\myindex{ARM!\Instructions!ADDS}
\myindex{ARM!\Instructions!SUBS}
\myindex{ARM!\Instructions!ADC}
\myindex{ARM!\Instructions!SBC}

最初の64ビット値は\Reg{0}と\Reg{1}のレジスタペアに渡され、2番目の値は\Reg{2}と\Reg{3}のレジスタペアに渡されます。 
ARMには\INS{ADC}命令(キャリーフラグをカウントする)と\INS{SBC}(\q{subtract with carry})もあります。
重要なこと:下位部分が加算/減算されるとき、-S接尾辞付きの\INS{ADDS} および \INS{SUBS}命令が使用されます。 
-S接尾辞は\q{set flags}をあらわし、flags(特にキャリーフラグ)は、結果として生じる\INS{ADC}/\INS{SBC}命令が確実に必要とするものです。
そうでなければ、接尾辞-Sを付けずに命令を実行します( \ADD および \SUB )。

\subsubsection{MIPS}

\lstinputlisting[caption=\Optimizing GCC 4.4.5 (IDA),style=customasmMIPS]{patterns/185_64bit_in_32_env/passing_add_sub/MIPS_O3_IDA_JPN.lst}

MIPSにはフラグレジスタがないため、算術演算の実行後にそのような情報は存在しません。
そのため、\INS{ADC} や \INS{SBB}ような命令はありません。 
キャリーフラグが設定されるかどうかを知るために、デスティネーションレジスタを1または0に
設定する比較{\INS{SLTU}命令を使用)も行われます。
その後、この1または0が最終結果に加算または減算されます。

