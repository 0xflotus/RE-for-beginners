\subsection{64ビットの値を返す}

\lstinputlisting[style=customc]{patterns/185_64bit_in_32_env/ret/0.c}

\subsubsection{x86}

32ビット環境では、64ビットの値は \EDX{}:\EAX{} レジスタペアを使って関数から返されます。

\lstinputlisting[caption=\Optimizing MSVC 2010,style=customasmx86]{patterns/185_64bit_in_32_env/ret/0_MSVC_2010_Ox.asm}

\subsubsection{ARM}

64ビットの値は \Reg{0}-\Reg{1} レジスタペアを使って返されます(\Reg{1}は高位の部分を\Reg{0}は低位の部分です)。

\lstinputlisting[caption=\OptimizingKeilVI (\ARMMode),style=customasmARM]{patterns/185_64bit_in_32_env/ret/Keil_ARM_O3.s}

\subsubsection{MIPS}

64ビットの値は\TT{V0}-\TT{V1} (\$2-\$3)レジスタペアを使って返されます(\TT{V0} (\$2)は高位の部分を\TT{V1} (\$3)は低位の部分です)。

\lstinputlisting[caption=\Optimizing GCC 4.4.5 (assembly listing),style=customasmMIPS]{patterns/185_64bit_in_32_env/ret/0_MIPS.s}

\lstinputlisting[caption=\Optimizing GCC 4.4.5 (IDA),style=customasmMIPS]{patterns/185_64bit_in_32_env/ret/0_MIPS_IDA.lst}
