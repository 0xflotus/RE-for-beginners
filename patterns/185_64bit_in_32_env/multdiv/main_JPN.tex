\subsection{乗算、除算}

\lstinputlisting[style=customc]{patterns/185_64bit_in_32_env/multdiv/2.c}

\subsubsection{x86}

\lstinputlisting[caption=\Optimizing MSVC 2013 /Ob1,style=customasmx86]{patterns/185_64bit_in_32_env/multdiv/2_MSVC_JPN.asm}

乗算と除算はより複雑な演算なので、通常、コンパイラはそれを行うライブラリ関数への
呼び出しを埋め込みます。

これらの機能はここに記述されています:\myref{sec:MSVC_library_func}

\lstinputlisting[caption=\Optimizing GCC 4.8.1 -fno-inline,style=customasmx86]{patterns/185_64bit_in_32_env/multdiv/2_GCC_JPN.asm}

GCCは期待どおりに機能しますが、乗算コードは
関数内でインライン化されているため、より効率的になる可能性があります。
GCCには異なる関数名のライブラリあります:\myref{sec:GCC_library_func}

\subsubsection{ARM}

ThumbモードのKeilはライブラリサブルーチン呼び出しを挿入します。

\lstinputlisting[caption=\OptimizingKeilVI (\ThumbMode),style=customasmARM]{patterns/185_64bit_in_32_env/multdiv/Keil_thumb_O3.s}

一方、ARMモードのKeilでは64ビットの乗算コードを生成できます。

\lstinputlisting[caption=\OptimizingKeilVI (\ARMMode),style=customasmARM]{patterns/185_64bit_in_32_env/multdiv/Keil_ARM_O3.s}
% TODO add explanation

\subsubsection{MIPS}

MIPS用に \Optimizing GCC 64ビット乗算コードを生成できますが、64ビット除算用のライブラリルーチンを呼び出す必要があります。

\lstinputlisting[caption=\Optimizing GCC 4.4.5 (IDA),style=customasmMIPS]{patterns/185_64bit_in_32_env/multdiv/MIPS_O3_IDA.lst}

たくさんの\ac{NOP}があります。おそらく乗算命令の後に埋められた遅延スロットです(結局のところ、
それは他の命令より遅いです)。

% TODO add explanation
