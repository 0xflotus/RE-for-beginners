\subsection{右シフト}

\lstinputlisting[style=customc]{patterns/185_64bit_in_32_env/shifting/3.c}

\subsubsection{x86}

\lstinputlisting[caption=\Optimizing MSVC 2012 /Ob1,style=customasmx86]{patterns/185_64bit_in_32_env/shifting/3_MSVC.asm}

\lstinputlisting[caption=\Optimizing GCC 4.8.1 -fno-inline,style=customasmx86]{patterns/185_64bit_in_32_env/shifting/3_GCC.asm}

\myindex{x86!\Instructions!SHRD}

シフトは2つのパスでも発生します:最初に下部がシフトされ、次に上部がシフトされます。 
しかし、下位部分は\INS{SHRD}命令の助けを借りてシフトされ、それは\EAX{}の値を7ビットだけシフトしますが、\EDX{}から
すなわち上位部分から新しいビットを引き出します。 つまり、\TT{EDX:EAX}レジスタのペアからの64ビット値は、全体として7ビットシフトされ、
結果の最下位32ビットが \EAX{} に格納されます。 
上位部分は、より一般的な \SHR{} 命令を使用してシフトされます。実際、上位部分の解放されたビットは
ゼロで埋められなければなりません。

\subsubsection{ARM}

ARMはx86では\INS{SHRD}のような命令を持っていないので、Keilコンパイラはこれを単純なシフトと\INS{OR}演算を使って行うべきです。

\lstinputlisting[caption=\OptimizingKeilVI (\ARMMode),style=customasmARM]{patterns/185_64bit_in_32_env/shifting/Keil_ARM_O3.s}

\lstinputlisting[caption=\OptimizingKeilVI (\ThumbMode),style=customasmARM]{patterns/185_64bit_in_32_env/shifting/Keil_thumb_O3.s}
% TODO add explanation

\subsubsection{MIPS}

MIPS向けのGCCは、KeilがThumbモードで行うのと同じアルゴリズムに従います。

\lstinputlisting[caption=\Optimizing GCC 4.4.5 (IDA),style=customasmMIPS]{patterns/185_64bit_in_32_env/shifting/MIPS_O3_IDA.lst}

% TODO add explanation

