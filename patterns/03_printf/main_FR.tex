\mysection{\PrintfSeveralArgumentsSectionName}

Maintenant, améliorons l'exemple \IT{\HelloWorldSectionName}~(\myref{sec:helloworld})
en remplaçant \printf dans la fonction \main par ceci:

\lstinputlisting[label=hw_c,style=customc]{patterns/03_printf/1.c}

% sections
\mysection{\oracle}
\label{oracle}

% sections
\EN{\input{examples/oracle/1_version_EN}}\RU{\input{examples/oracle/1_version_RU}}
\EN{\input{examples/oracle/2_ksmlru_EN}}\RU{\input{examples/oracle/2_ksmlru_RU}}
\EN{\input{examples/oracle/3_timer_EN}}\RU{\input{examples/oracle/3_timer_RU}}


\mysection{\oracle}
\label{oracle}

% sections
\EN{\input{examples/oracle/1_version_EN}}\RU{\input{examples/oracle/1_version_RU}}
\EN{\input{examples/oracle/2_ksmlru_EN}}\RU{\input{examples/oracle/2_ksmlru_RU}}
\EN{\input{examples/oracle/3_timer_EN}}\RU{\input{examples/oracle/3_timer_RU}}


\mysection{\oracle}
\label{oracle}

% sections
\EN{\input{examples/oracle/1_version_EN}}\RU{\input{examples/oracle/1_version_RU}}
\EN{\input{examples/oracle/2_ksmlru_EN}}\RU{\input{examples/oracle/2_ksmlru_RU}}
\EN{\input{examples/oracle/3_timer_EN}}\RU{\input{examples/oracle/3_timer_RU}}



\subsection{\Conclusion{}}

Voici un schéma grossier de l'appel de la fonction:

\begin{lstlisting}[caption=x86,style=customasmx86]
...
PUSH 3rd argument
PUSH 2nd argument
PUSH 1st argument
CALL fonction
; modifier le pointeur de pile (si besoin)
\end{lstlisting}

\begin{lstlisting}[caption=x64 (MSVC),style=customasmx86]
MOV RCX, 1st argument
MOV RDX, 2nd argument
MOV R8, 3rd argument
MOV R9, 4th argument
...
PUSH 5ème, 6ème argument, etc. (si besoin)
CALL fonction
; modifier le pointeur de pile (si besoin)
\end{lstlisting}

\begin{lstlisting}[caption=x64 (GCC),style=customasmx86]
MOV RDI, 1st argument
MOV RSI, 2nd argument
MOV RDX, 3rd argument
MOV RCX, 4th argument
MOV R8, 5th argument
MOV R9, 6th argument
...
PUSH 7ème, 8ème argument, etc. (si besoin)
CALL fonction
; modifier le pointeur de pile (si besoin)
\end{lstlisting}

\begin{lstlisting}[caption=ARM,style=customasmARM]
MOV R0, 1st argument
MOV R1, 2nd argument
MOV R2, 3rd argument
MOV R3, 4th argument
; passer le 5ème, 6ème argument, etc., dans la pile (si besoin)
BL fonction
; modifier le pointeur de pile (si besoin)
\end{lstlisting}

\begin{lstlisting}[caption=ARM64,style=customasmARM]
MOV X0, 1st argument
MOV X1, 2nd argument
MOV X2, 3rd argument
MOV X3, 4th argument
MOV X4, 5th argument
MOV X5, 6th argument
MOV X6, 7th argument
MOV X7, 8th argument
; passer le 9ème, 10ème argument, etc., dans la pile (si besoin)
BL fonction
; modifier le pointeur de pile (si besoin)
\end{lstlisting}

\myindex{MIPS!O32}
\begin{lstlisting}[caption=MIPS (O32 calling convention),style=customasmMIPS]
LI $4, 1st argument ; AKA $A0
LI $5, 2nd argument ; AKA $A1
LI $6, 3rd argument ; AKA $A2
LI $7, 4th argument ; AKA $A3
; passer le 5ème, 6ème argument, etc., dans la pile (si besoin)
LW temp_reg, adresse de la fonction
JALR temp_reg
\end{lstlisting}

\subsection{À propos}

\myindex{fastcall}
À propos, cette différence dans le passage des arguments entre x86, x64, fastcall, ARM
et MIPS est une bonne illustration du fait que le CPU est inconscient de comment
les arguments sont passés aux fonctions.
Il est aussi possible de créer un hypothétique compilateur capable de passer les
arguments via une structure spéciale sans utiliser du tout la pile.

\myindex{MIPS!O32}
Les registres MIPS \$A0 \dots \$A3 sont appelés comme ceci par commodité (c'est
dans la convention d'appel O32).
Les programmeurs peuvent utiliser n'importe quel autre registre, (bon, peut-être
à l'exception de \$ZERO) pour passer des données ou n'importe quelle autre convention d'appel.

Le \ac{CPU} n'est pas au courant de quoi que ce soit des conventions d'appel.

Nous pouvons aussi nous rappeler comment les débutants en langage d'assemblage passent
les arguments aux autres fonctions: généralement par les registres, sans ordre explicite,
ou même par des variables globales.
Bien sûr, cela fonctionne.

