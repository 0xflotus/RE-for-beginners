\mysection{\PrintfSeveralArgumentsSectionName}

Now let's extend the \IT{\HelloWorldSectionName}~(\myref{sec:helloworld}) example, replacing \printf in
the \main function body with this:

\lstinputlisting[label=hw_c,style=customc]{patterns/03_printf/1.c}

% sections
\mysection{\oracle}
\label{oracle}

% sections
\EN{\input{examples/oracle/1_version_EN}}\RU{\input{examples/oracle/1_version_RU}}
\EN{\input{examples/oracle/2_ksmlru_EN}}\RU{\input{examples/oracle/2_ksmlru_RU}}
\EN{\input{examples/oracle/3_timer_EN}}\RU{\input{examples/oracle/3_timer_RU}}


\mysection{\oracle}
\label{oracle}

% sections
\EN{\input{examples/oracle/1_version_EN}}\RU{\input{examples/oracle/1_version_RU}}
\EN{\input{examples/oracle/2_ksmlru_EN}}\RU{\input{examples/oracle/2_ksmlru_RU}}
\EN{\input{examples/oracle/3_timer_EN}}\RU{\input{examples/oracle/3_timer_RU}}


\mysection{\oracle}
\label{oracle}

% sections
\EN{\input{examples/oracle/1_version_EN}}\RU{\input{examples/oracle/1_version_RU}}
\EN{\input{examples/oracle/2_ksmlru_EN}}\RU{\input{examples/oracle/2_ksmlru_RU}}
\EN{\input{examples/oracle/3_timer_EN}}\RU{\input{examples/oracle/3_timer_RU}}



\subsection{\Conclusion{}}

Here is a rough skeleton of the function call:

\begin{lstlisting}[caption=x86,style=customasmx86]
...
PUSH 3rd argument
PUSH 2nd argument
PUSH 1st argument
CALL function
; modify stack pointer (if needed)
\end{lstlisting}

\begin{lstlisting}[caption=x64 (MSVC),style=customasmx86]
MOV RCX, 1st argument
MOV RDX, 2nd argument
MOV R8, 3rd argument
MOV R9, 4th argument
...
PUSH 5th, 6th argument, etc. (if needed)
CALL function
; modify stack pointer (if needed)
\end{lstlisting}

\begin{lstlisting}[caption=x64 (GCC),style=customasmx86]
MOV RDI, 1st argument
MOV RSI, 2nd argument
MOV RDX, 3rd argument
MOV RCX, 4th argument
MOV R8, 5th argument
MOV R9, 6th argument
...
PUSH 7th, 8th argument, etc. (if needed)
CALL function
; modify stack pointer (if needed)
\end{lstlisting}

\begin{lstlisting}[caption=ARM,style=customasmARM]
MOV R0, 1st argument
MOV R1, 2nd argument
MOV R2, 3rd argument
MOV R3, 4th argument
; pass 5th, 6th argument, etc., in stack (if needed)
BL function
; modify stack pointer (if needed)
\end{lstlisting}

\begin{lstlisting}[caption=ARM64,style=customasmARM]
MOV X0, 1st argument
MOV X1, 2nd argument
MOV X2, 3rd argument
MOV X3, 4th argument
MOV X4, 5th argument
MOV X5, 6th argument
MOV X6, 7th argument
MOV X7, 8th argument
; pass 9th, 10th argument, etc., in stack (if needed)
BL function
; modify stack pointer (if needed)
\end{lstlisting}

\myindex{MIPS!O32}
\begin{lstlisting}[caption=MIPS (O32 calling convention),style=customasmMIPS]
LI $4, 1st argument ; AKA $A0
LI $5, 2nd argument ; AKA $A1
LI $6, 3rd argument ; AKA $A2
LI $7, 4th argument ; AKA $A3
; pass 5th, 6th argument, etc., in stack (if needed)
LW temp_reg, address of function
JALR temp_reg
\end{lstlisting}

\subsection{By the way}

\myindex{fastcall}
By the way, this difference between the arguments passing in x86, x64, 
fastcall, ARM and MIPS is a good illustration of the fact that the CPU is oblivious to how the arguments are passed to functions. 
It is also possible to create a hypothetical compiler able to pass arguments 
via a special structure without using stack at all.

\myindex{MIPS!O32}
MIPS \$A0 \dots \$A3 registers are labeled this way only for convenience (that is in the O32 calling convention).
Programmers may use any other register (well, maybe except \$ZERO) 
to pass data or use any other calling convention.

The \ac{CPU} is not aware of calling conventions whatsoever.

We may also recall how new coming assembly language programmers passing arguments into
other functions: usually via registers, without any explicit order, or even via global variables.
Of course, it works fine.

