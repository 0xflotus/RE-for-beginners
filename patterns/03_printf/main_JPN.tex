\mysection{\PrintfSeveralArgumentsSectionName}

さて、\IT{\HelloWorldSectionName}~(\myref{sec:helloworld})の例では、 
\main 関数本体の \printf を次のように置き換えます。

\lstinputlisting[label=hw_c,style=customc]{patterns/03_printf/1.c}

% sections
\mysection{\oracle}
\label{oracle}

% sections
\EN{\input{examples/oracle/1_version_EN}}\RU{\input{examples/oracle/1_version_RU}}
\EN{\input{examples/oracle/2_ksmlru_EN}}\RU{\input{examples/oracle/2_ksmlru_RU}}
\EN{\input{examples/oracle/3_timer_EN}}\RU{\input{examples/oracle/3_timer_RU}}


\mysection{\oracle}
\label{oracle}

% sections
\EN{\input{examples/oracle/1_version_EN}}\RU{\input{examples/oracle/1_version_RU}}
\EN{\input{examples/oracle/2_ksmlru_EN}}\RU{\input{examples/oracle/2_ksmlru_RU}}
\EN{\input{examples/oracle/3_timer_EN}}\RU{\input{examples/oracle/3_timer_RU}}


\mysection{\oracle}
\label{oracle}

% sections
\EN{\input{examples/oracle/1_version_EN}}\RU{\input{examples/oracle/1_version_RU}}
\EN{\input{examples/oracle/2_ksmlru_EN}}\RU{\input{examples/oracle/2_ksmlru_RU}}
\EN{\input{examples/oracle/3_timer_EN}}\RU{\input{examples/oracle/3_timer_RU}}



\subsection{\Conclusion{}}

関数呼び出しの概略を以下に示します。

\begin{lstlisting}[caption=x86,style=customasmx86]
...
PUSH 3rd argument
PUSH 2nd argument
PUSH 1st argument
CALL function
; スタックポインタを修正する(必要なら)
\end{lstlisting}

\begin{lstlisting}[caption=x64 (MSVC),style=customasmx86]
MOV RCX, 1st argument
MOV RDX, 2nd argument
MOV R8, 3rd argument
MOV R9, 4th argument
...
PUSH 5th, 6th argument, etc. (if needed)
CALL function
; スタックポインタを修正する(必要なら)
\end{lstlisting}

\begin{lstlisting}[caption=x64 (GCC),style=customasmx86]
MOV RDI, 1st argument
MOV RSI, 2nd argument
MOV RDX, 3rd argument
MOV RCX, 4th argument
MOV R8, 5th argument
MOV R9, 6th argument
...
PUSH 7th, 8th argument, etc. (if needed)
CALL function
; スタックポインタを修正する(必要なら)
\end{lstlisting}

\begin{lstlisting}[caption=ARM,style=customasmARM]
MOV R0, 1st argument
MOV R1, 2nd argument
MOV R2, 3rd argument
MOV R3, 4th argument
; 5番目、6番目の引数などをスタックに渡す(必要なら)
BL function
; スタックポインタを修正する(必要なら)
\end{lstlisting}

\begin{lstlisting}[caption=ARM64,style=customasmARM]
MOV X0, 1st argument
MOV X1, 2nd argument
MOV X2, 3rd argument
MOV X3, 4th argument
MOV X4, 5th argument
MOV X5, 6th argument
MOV X6, 7th argument
MOV X7, 8th argument
; 9番目、10番目の引数などをスタックに渡す(必要なら)
BL function
; スタックポインタを修正する(必要なら)
\end{lstlisting}

\myindex{MIPS!O32}
\begin{lstlisting}[caption=MIPS (O32 calling convention),style=customasmMIPS]
LI $4, 1st argument ; AKA $A0
LI $5, 2nd argument ; AKA $A1
LI $6, 3rd argument ; AKA $A2
LI $7, 4th argument ; AKA $A3
; 5番目、6番目の引数などをスタックに渡す(必要なら)
LW temp_reg, address of function
JALR temp_reg
\end{lstlisting}

\subsection{ところで}

\myindex{fastcall}
ところで、x86、x64、fastcall、ARM、およびMIPSで渡される引数の違いは、
CPUが引数を関数にどのように引き渡すかを知らないという事実の良い例です。 
スタックをまったく使用せずに引数を特殊な構造体に渡すことができる
仮想コンパイラを作成することもできます。

\myindex{MIPS!O32}
MIPS \$A0 \dots \$A3 レジスタは、便宜上(O32呼び出し規約にある)このようにラベル付けされています。 
プログラマは、データを渡すために、あるいは他の呼び出し規約を使用するために、
他のレジスタ(おそらく \$ZERO を除く)を使用することができます。

\ac{CPU}は呼び出し規約を認識していません。

他の関数に引数を渡す新しいアセンブリ言語プログラマが、通常はレジスタ経由で、
明示的な順序なしに、あるいはグローバル変数を介して、どのように新しい関数を呼び出すかを思い出すかもしれません。 
もちろん、それは正常に動作します。
