\mysection{\PrintfSeveralArgumentsSectionName}

Agora vamos extender o nosso exemplo \IT{\HelloWorldSectionName}~(\myref{sec:helloworld}),
trocando \printf no corpo da função \main() por isso:

\lstinputlisting[label=hw_c,style=customc]{patterns/03_printf/1.c}

% sections
\mysection{\oracle}
\label{oracle}

% sections
\EN{\input{examples/oracle/1_version_EN}}\RU{\input{examples/oracle/1_version_RU}}
\EN{\input{examples/oracle/2_ksmlru_EN}}\RU{\input{examples/oracle/2_ksmlru_RU}}
\EN{\input{examples/oracle/3_timer_EN}}\RU{\input{examples/oracle/3_timer_RU}}


\mysection{\oracle}
\label{oracle}

% sections
\EN{\input{examples/oracle/1_version_EN}}\RU{\input{examples/oracle/1_version_RU}}
\EN{\input{examples/oracle/2_ksmlru_EN}}\RU{\input{examples/oracle/2_ksmlru_RU}}
\EN{\input{examples/oracle/3_timer_EN}}\RU{\input{examples/oracle/3_timer_RU}}


\mysection{\oracle}
\label{oracle}

% sections
\EN{\input{examples/oracle/1_version_EN}}\RU{\input{examples/oracle/1_version_RU}}
\EN{\input{examples/oracle/2_ksmlru_EN}}\RU{\input{examples/oracle/2_ksmlru_RU}}
\EN{\input{examples/oracle/3_timer_EN}}\RU{\input{examples/oracle/3_timer_RU}}



\subsection{\Conclusion{}}

Aqui está uma estrutura bem rústica da chamada da função

% TODO to be translated to PTBR:
\begin{lstlisting}[caption=x86,style=customasmx86]
...
PUSH 3rd argument
PUSH 2nd argument
PUSH 1st argument
CALL function
; modify stack pointer (if needed)
\end{lstlisting}

\begin{lstlisting}[caption=x64 (MSVC),style=customasmx86]
MOV RCX, 1st argument
MOV RDX, 2nd argument
MOV R8, 3rd argument
MOV R9, 4th argument
...
PUSH 5th, 6th argument, etc. (if needed)
CALL function
; modify stack pointer (if needed)
\end{lstlisting}

\begin{lstlisting}[caption=x64 (GCC),style=customasmx86]
MOV RDI, 1st argument
MOV RSI, 2nd argument
MOV RDX, 3rd argument
MOV RCX, 4th argument
MOV R8, 5th argument
MOV R9, 6th argument
...
PUSH 7th, 8th argument, etc. (if needed)
CALL function
; modify stack pointer (if needed)
\end{lstlisting}

\begin{lstlisting}[caption=ARM,style=customasmARM]
MOV R0, 1st argument
MOV R1, 2nd argument
MOV R2, 3rd argument
MOV R3, 4th argument
; pass 5th, 6th argument, etc., in stack (if needed)
BL function
; modify stack pointer (if needed)
\end{lstlisting}

\begin{lstlisting}[caption=ARM64,style=customasmARM]
MOV X0, 1st argument
MOV X1, 2nd argument
MOV X2, 3rd argument
MOV X3, 4th argument
MOV X4, 5th argument
MOV X5, 6th argument
MOV X6, 7th argument
MOV X7, 8th argument
; pass 9th, 10th argument, etc., in stack (if needed)
BL function
; modify stack pointer (if needed)
\end{lstlisting}

\myindex{MIPS!O32}
\begin{lstlisting}[caption=MIPS (O32 calling convention),style=customasmMIPS]
LI $4, 1st argument ; AKA $A0
LI $5, 2nd argument ; AKA $A1
LI $6, 3rd argument ; AKA $A2
LI $7, 4th argument ; AKA $A3
; pass 5th, 6th argument, etc., in stack (if needed)
LW temp_reg, address of function
JALR temp_reg
\end{lstlisting}

\subsection{A propósito}

\myindex{fastcall}
A propósito, a diferença entre os argumentos passados em x86, x64, fastcall, ARM e MIPS  é uma boa demonstração do fato de como a CPU é indiferente sobre como os argumentos são passados para as funções.
Também é possível criar um compilador hipotético capaz de passar argumentos por alguma outra estrutura especial sem usar a pilha de nenhuma maneira.

\myindex{MIPS!O32}
\PTBRph{}

A \ac{CPU} não está ciente de convenções de chamada de funções.

Agora nós podemos também relembrar de dos programadores novatos de assembly passando argumentos para outras funções:
geralmente via registradores, sem nenhuma sequência explícita, ou mesmo por variáveis globais. Logicamente, também funciona.

