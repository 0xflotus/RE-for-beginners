\mysection{\PrintfSeveralArgumentsSectionName}

Spróbujmy trochę rozszerzyć przykład \IT{\HelloWorldSectionName}~(\myref{sec:helloworld}),
pisząc w funkcji \main:

\lstinputlisting[label=hw_c,style=customc]{patterns/03_printf/1.c}

% sections
\mysection{\oracle}
\label{oracle}

% sections
\EN{\input{examples/oracle/1_version_EN}}\RU{\input{examples/oracle/1_version_RU}}
\EN{\input{examples/oracle/2_ksmlru_EN}}\RU{\input{examples/oracle/2_ksmlru_RU}}
\EN{\input{examples/oracle/3_timer_EN}}\RU{\input{examples/oracle/3_timer_RU}}


\mysection{\oracle}
\label{oracle}

% sections
\EN{\input{examples/oracle/1_version_EN}}\RU{\input{examples/oracle/1_version_RU}}
\EN{\input{examples/oracle/2_ksmlru_EN}}\RU{\input{examples/oracle/2_ksmlru_RU}}
\EN{\input{examples/oracle/3_timer_EN}}\RU{\input{examples/oracle/3_timer_RU}}


\mysection{\oracle}
\label{oracle}

% sections
\EN{\input{examples/oracle/1_version_EN}}\RU{\input{examples/oracle/1_version_RU}}
\EN{\input{examples/oracle/2_ksmlru_EN}}\RU{\input{examples/oracle/2_ksmlru_RU}}
\EN{\input{examples/oracle/3_timer_EN}}\RU{\input{examples/oracle/3_timer_RU}}



\subsection{\Conclusion{}}

Tak wygląda szkielet wywołania funkcji:

\begin{lstlisting}[caption=x86,style=customasmx86]
...
PUSH trzeci argument 
PUSH drugi argument
PUSH pierwszy argument
CALL funkcja
; zmodyfikować wskaźnik stosu jeśli trzeba (§jeśli§ §trzeba§)
\end{lstlisting}

\begin{lstlisting}[caption=x64 (MSVC),style=customasmx86]
MOV RCX, pierwszy argument
MOV RDX, drugi argument
MOV R8, trzeci argument
MOV R9, §4-y argument§
...
PUSH §5-y, 6-y argument, itd (jeśli trzeba)§
CALL funkcja
; §zmodyfikować wskaźnik stosu (jeśli trzeba)§
\end{lstlisting}

\begin{lstlisting}[caption=x64 (GCC),style=customasmx86]
MOV RDI, pierwszy argument
MOV RSI, drugi argument
MOV RDX, trzeci argument
MOV RCX, §czwarty argument§
MOV R8, §5-y argument§
MOV R9, §6-y argument§
...
PUSH §7-y, 8-y argument, itd (jeśli trzeba)§
CALL funkcja
; §zmodyfikować wskaźnik stosu (jeśli trzeba)§
\end{lstlisting}

\begin{lstlisting}[caption=ARM,style=customasmARM]
MOV R0, pierwszy argument
MOV R1, drugi argument
MOV R2, trzeci argument
MOV R3, §czwarty argument§
; §przekazać 5-y, 6-y argument, itd, przez stos (jeśli trzeba)§
BL funkcja
; §zmodyfikować wskaźnik stosu (jeśli trzeba)§
\end{lstlisting}

\begin{lstlisting}[caption=ARM64,style=customasmARM]
MOV X0, pierwszy argument
MOV X1, drugi argument
MOV X2, trzeci argument
MOV X3, §4-y argument§
MOV X4, §5-y argument§
MOV X5, §6-y argument§
MOV X6, §7-y argument§
MOV X7, §8-y argument§
; §przekazać 9-y, 10-y argument, itd, przez stos (jeśli trzeba)§
BL funkcja
; §zmodyfikować wskaźnik stosu (jeśli trzeba)§
\end{lstlisting}

\myindex{MIPS!O32}
\begin{lstlisting}[caption=MIPS,style=customasmMIPS]
LI $4, pierwszy argument ; AKA $A0
LI $5, drugi argument ; AKA $A1
LI $6, trzeci argument ; AKA $A2
LI $7, §4-y argument §; AKA $A3
; §przekazać 5-y, 6-y argument, itd, przez stos (jeśli trzeba)§
LW temp_reg, adress funkcji
JALR temp_reg
\end{lstlisting}

\subsection{A propos}

\myindex{fastcall}
A propos, różnica między przekazywaniem argumentów funkcji w x86, x64, fastcall, ARM i MIPS demonstruje, że procesorowi, ogólnie, jest obojętnie w kjaki sposób będą 
przekazywane parametry funkcji. Można stworzy ć hipotetyczny kompilator, który będzie je przekazywał za pomocą 
wskaźnika na strukturę z prametrami, nie korzystając ze stosu w ogóle.

\myindex{MIPS!O32}
Rejestry \$A0\dots \$A3 w MIPS są nazwane w ten sposób tylko dla wygody (jest to standard przy wywołaniach 032).
Programiści mogą korzystać z jakichkolwiek innych rejestrów (może oprócz \$ZERO) do
przekazywania danych.

Początkujący programiści w asemblerze przekazują parametry do innych funkcji
zwykle przez rejestry, lub nawet przez zmienne globalne.
I wszystko to normalnie działa.


