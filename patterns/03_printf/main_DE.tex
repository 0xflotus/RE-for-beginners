\mysection{\PrintfSeveralArgumentsSectionName}

An dieser Stelle wird das \IT{\HelloWorldSectionName}~(\myref{sec:helloworld})-Beispiel ein
wenig erweitert, indem \printf in der \main-Funktion durch folgendes ersetzt wird:

\lstinputlisting[label=hw_c,style=customc]{patterns/03_printf/1.c}

% sections
\mysection{\oracle}
\label{oracle}

% sections
\EN{\input{examples/oracle/1_version_EN}}\RU{\input{examples/oracle/1_version_RU}}
\EN{\input{examples/oracle/2_ksmlru_EN}}\RU{\input{examples/oracle/2_ksmlru_RU}}
\EN{\input{examples/oracle/3_timer_EN}}\RU{\input{examples/oracle/3_timer_RU}}


\mysection{\oracle}
\label{oracle}

% sections
\EN{\input{examples/oracle/1_version_EN}}\RU{\input{examples/oracle/1_version_RU}}
\EN{\input{examples/oracle/2_ksmlru_EN}}\RU{\input{examples/oracle/2_ksmlru_RU}}
\EN{\input{examples/oracle/3_timer_EN}}\RU{\input{examples/oracle/3_timer_RU}}


\mysection{\oracle}
\label{oracle}

% sections
\EN{\input{examples/oracle/1_version_EN}}\RU{\input{examples/oracle/1_version_RU}}
\EN{\input{examples/oracle/2_ksmlru_EN}}\RU{\input{examples/oracle/2_ksmlru_RU}}
\EN{\input{examples/oracle/3_timer_EN}}\RU{\input{examples/oracle/3_timer_RU}}



\subsection{\Conclusion{}}

Hier ist der grobe Aufbau der Aufruffunktion:

\begin{lstlisting}[caption=x86,style=customasmx86]
...
PUSH Drittes Argument
PUSH Zweites Argument
PUSH Erstes Argument
CALL Funktion
; gegebenenfalls den Stackpointer modifizieren
\end{lstlisting}

\begin{lstlisting}[caption=x64 (MSVC),style=customasmx86]
MOV RCX, Erstes Argument
MOV RDX, Zweites Argument
MOV R8, Drittes Argument
MOV R9, Viertes Argument
...
PUSH fünftes, sechstes Argument, usw. (falls notwendig)
CALL Funktion
; gegebenenfalls den Stackpointer modifizieren
\end{lstlisting}

\begin{lstlisting}[caption=x64 (GCC),style=customasmx86]
MOV RDI, Erstes Argument
MOV RSI, Zweites Argument
MOV RDX, Drittes Argument
MOV RCX, Viertes Argument
MOV R8, Fünftes Argument
MOV R9, Sechstes Argument
...
PUSH Siebtes, Achtes Argument, usw. (falls notwendig)
CALL Funktion
; gegebenenfalls den Stackpointer modifizieren
\end{lstlisting}

\begin{lstlisting}[caption=ARM,style=customasmARM]
MOV R0, Erstes Argument
MOV R1, Zweites Argument
MOV R2, Drittes Argument
MOV R3, Viertes Argument
; Fünftes, Sechstes Argument, usw. auf den Stack (falls notwendig)
BL Funktion
; gegebenenfalls den Stackpointer modifizieren
\end{lstlisting}

\begin{lstlisting}[caption=ARM64,style=customasmARM]
MOV X0, Erstes Argument
MOV X1, Zweites Argument
MOV X2, Drittes Argument
MOV X3, Viertes Argument
MOV X4, Fünftes Argument
MOV X5, Sechstes Argument
MOV X6, Siebtes Argument
MOV X7, Achtes Argument
; Neuntes, Zehntes Argument, usw. auf den Stack (falls notwendig)
BL Funktion
; gegebenenfalls den Stackpointer modifizieren
\end{lstlisting}

\myindex{MIPS!O32}
\begin{lstlisting}[caption=MIPS (O32 calling convention),style=customasmMIPS]
LI $4, Erstes argument ; AKA $A0
LI $5, Zweites argument ; AKA $A1
LI $6, Drittes argument ; AKA $A2
LI $7, Viertes argument ; AKA $A3
; pass Fünftes, Sechstes argument, usw. auf den Stack (falls notwendig)
LW temporäres Register, Adresse der Funktion
JALR temporäres Regist
\end{lstlisting}

\subsection{Übrigens\dots{}}

\myindex{fastcall}
Übrigens ist der Unterschied der Art der Argumenten Übergabe in x86, x64, fastcall, ARM und MIPS eine gute
Darstellung der Tatsache, dass die CPU nicht weiß wie die Argumente an die Funktion übergeben werden.
Es ist auch möglich einen hypothetischen Compiler zu erstellen, der die Möglichkeit hat Argumente mittels
einer speziellen Struktur, ohne den Stack an die Funktionen zu übergebe.

\myindex{MIPS!O32}
MIPS \$A0 \dots \$A3-Register sind aus Bequemlichkeitsgründen auf diese Weise beschriftet (O32 Aufrufkonvention).
Programmierer können auch andere Register (vielleicht außer \$ZERO) nutzen um Daten zu übergeben
oder eine andere Aufrufkonvention zu nutzen.

Die \ac{CPU} hatte jedoch keinerlei Kenntnisse über die Aufrufkonvention.

Man sieht hier auch wie Neulinge der Assemblersprache Argumente an andere Funktionen übergeben:
in der Regel per Register ohne explizite Reihenfolge oder globale Variablen.
Natürlich funktioniert das ebenso gut.
