\mysection[Générateur congruentiel linéaire]{Générateur congruentiel linéaire comme générateur de nombres pseudo-aléatoires}
\myindex{\CStandardLibrary!rand()}
\label{LCG_simple}

Peut-être que le générateur congruentiel linéaire est le moyen le plus simple possible
de générer des nombres aléatoires.

Ce n'est plus très utilisé aujourd'hui\footnote{Le twister de Mersenne est meilleur.},
mais il est si simple (juste une multiplication, une addition et une opération AND)
que nous pouvons l'utiliser comme un exemple.

\lstinputlisting[style=customc]{patterns/145_LCG/rand_FR.c}

Il y a deux fonctions: la première est utilisée pour initialiser l'état interne,
et la seconde est appelée pour générer un nombre pseudo-aléatoire.

Nous voyons que deux constantes sont utilisées dans l'algorithme.
Elles proviennent de
[William H. Press and Saul A. Teukolsky and William T. Vetterling and Brian P. Flannery, \IT{Numerical Recipes}, (2007)].

Définissons-les en utilisant la déclaration \CCpp \TT{\#define}. C'est une macro.

La différence entre une macro \CCpp et une constante est que toutes les macros sont
remplacées par leur valeur par le pré-processeur \CCpp, et qu'elles n'utilisent pas
de mémoire, contrairement aux variables.

Par contre, une constante est une variable en lecture seule.

Il est possible de prendre un pointeur (ou une adresse) d'une variable constante,
mais c'est impossible de faire ça avec une macro.

La dernière opération AND est nécessaire car d'après le standard C \TT{my\_rand()}
doit renvoyer une valeur dans l'intervalle 0..32767.

Si vous voulez obtenir des valeurs pseudo-aléatoires 32-bit, il suffit d'omettre
la dernière opération AND.

\subsection{x86}

\lstinputlisting[caption=MSVC 2013 \Optimizing,style=customasmx86]{patterns/145_LCG/rand_MSVC_2013_x86_Ox.asm}

Nous les voyons ici: les deux constantes sont intégrées dans le code.
Il n'y a pas de mémoire allouée pour elles.

La fonction \TT{my\_srand()} copie juste sa valeur en entrée dans la variable \TT{rand\_state}
interne.

\TT{my\_rand()} la prend, calcule le \TT{rand\_state} suivant, le coupe et le laisse
dans le registre \EAX.

La version non optimisée est plus verbeuse:

\lstinputlisting[caption=MSVC 2013 \NonOptimizing,style=customasmx86]{patterns/145_LCG/rand_MSVC_2013_x86.asm}

\subsection{x64}

La version x64 est essentiellement la même et utilise des registres 32-bit au lieu
de 64-bit (car nous travaillons avec des valeurs de type \Tint ici).

Mais \TT{my\_srand()} prend son argument en entrée dans le registre \ECX plutôt que
sur la pile:

\lstinputlisting[caption=MSVC 2013 x64 \Optimizing,style=customasmx86]{patterns/145_LCG/rand_MSVC_2013_x64_Ox_FR.asm}

Le compilateur GCC génère en grande partie le même code.

\subsection{ARM 32-bit}

\lstinputlisting[caption=\OptimizingKeilVI (\ARMMode),style=customasmARM]{patterns/145_LCG/rand.s_Keil_ARM_O3_FR.s}

Il n'est pas possible d'intégrer une constante 32-bit dans des instructions ARM,
donc Keil doit les stocker à l'extérieur et en outre les charger.
Une chose intéressante est qu'il n'est pas possible non plus d'intégrer la constante
0x7FFF.
Donc ce que fait Keil est de décaler \TT{rand\_state} vers la gauche de 17 bits et
ensuite la décale de 17 bits vers la droite.
Ceci est analogue à la déclaration $(rand\_state \ll 17) \gg 17$ en \CCpp.
Il semble que ça soit une opération inutile, mais ce qu'elle fait est de mettre à
zéro les 17 bits hauts, laissant les 15 bits bas inchangés, et c'est notre but après
tout.\\
\\
Keil \Optimizing pour le mode Thumb génère essentiellement le même code.

\subsection{MIPS}

\lstinputlisting[caption=\Optimizing GCC 4.4.5 (IDA),style=customasmMIPS]{patterns/145_LCG/MIPS_O3_IDA_FR.lst}

Ouah, ici nous ne voyons qu'une seule constante (0x3C6EF35F ou 1013904223).
Où est l'autre (1664525)?

Il semble que la multiplication soit effectuée en utilisant seulement des décalages
et des additions!
Vérifions cette hypothèse:

\lstinputlisting[style=customc]{patterns/145_LCG/test.c}

\lstinputlisting[caption=GCC 4.4.5 \Optimizing (IDA),style=customasmMIPS]{patterns/145_LCG/test_O3_MIPS.lst}

En effet!

\subsubsection{Relocations MIPS}

Nous allons nous concentrer sur comment les opérations comme charger et stocker dans
la mémoire fonctionnent.

Les listings ici sont produits par IDA, qui cache certains détails.

Nous allons lancer objdump deux fois: pour obtenir le listing désassemblé et aussi
la liste des relocations:

\lstinputlisting[caption=GCC 4.4.5 \Optimizing (objdump)]{patterns/145_LCG/MIPS_O3_objdump.txt}

Considérons les deux relocations pour la fonction \TT{my\_srand()}.

La première, pour l'adresse 0 a un type de \TT{R\_MIPS\_HI16} et la seconde pour
l'adresse 8 a un type de \TT{R\_MIPS\_LO16}.

Cela implique que l'adresse du début du segment .bss soit écrite dans les instructions
à l'adresse 0 (partie haute de l'adresse) et 8 (partie basse de l'adresse).

La variable \TT{rand\_state} est au tout début du segment .bss.

Donc nous voyons des zéros dans les opérandes des instructions \LUI et \SW, car il
n'y a encore rien ici--- le compilateur ne sait pas quoi y écrire.

L'éditeur de liens va arranger cela, et la partie haute de l'adresse sera écrite
dans l'opérande de \LUI et la partie basse de l'adresse---dans l'opérande de \SW.

\SW va ajouter la partie basse de l'adresse avec le contenu du registre \$V0 (la
partie haute y est).

C'est la même histoire avec la fonction my\_rand(): la relocation R\_MIPS\_HI16 indique
à l'éditeur de liens d'écrire la partie haute.

Donc la partie haute de l'adresse de la variable rand\_state se trouve dans le registre
\$V1.

L'instruction \LW à l'adresse 0x10 ajoute les parties haute et basse et charge la
valeur de la variable rand\_state dans \$V0.

L'instruction \SW à l'adresse 0x54 fait à nouveau la somme et stocke la nouvelle
valeur dans la variable globale rand\_state.

IDA traite les relocations pendant le chargement, cachant ainsi ces détails, mais
nous devons les garder à l'esprit.

% TODO add example of compiled binary, GDB example, etc...


\subsection{Version thread-safe de l'exemple}

La version thread-safe de l'exemple sera montrée plus tard: \myref{LCG_TLS}.
