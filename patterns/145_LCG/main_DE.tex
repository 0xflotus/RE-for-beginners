\mysection[Linear congruential generator]{Linearer Kongruenzgenerator als Pseudozufallszahlengenerator}
\myindex{\CStandardLibrary!rand()}
\label{LCG_simple}
Ein linearer Kongruenzgenerator ist die wohl einfachste Form der Zufallszahlenerzeugung.

Er wird heutzutage nicht mehr so häufig eingesetzt\footnote{Der Mersenne-Twister ist besser}, kann aber, weil er so
einfach ist (nur eine Multiplikation, eine Addition und eine AND-Operation), dennoch gut als Beispiel dienen.

\lstinputlisting[style=customc]{patterns/145_LCG/rand_DE.c}
Es gibt hier zwei Funktionen: die erste wird verwendet und den internen Zustand zu initialisieren und die zweite wird
zum Erzeugen der Pseudozufallszahlen aufgerufen.

Wir sehen, dass im Algorithmus zwei Konstanten verwendet werden.
Sie stamen aus [William H. Press and Saul A. Teukolsky and William T. Vetterling and Brian P. Flannery, \IT{Numerical Recipes}, (2007)].

Definieren wir sie über den \TT{\#define} \CCpp Ausdruck. Es handelt sich um ein Makro.

Der Unterschied zwischen einem \CCpp Makro und einer Konstanten ist, dass alle Makros durch den \CCpp Präprozessor durch
mit ihrem Wert ersetzt werden und dadurch im Gegensatz zu Variablen keinen Speicherplatz verbrauchen.

Eine Konstante ist im Gegensatz dazu eine nur lesbare Variable.

Es ist möglich einen Pointer (oder eine Adresse) einer Konstanten zu verwenden; das ist mit einem Makro nicht möglich.

Die letzte AND-Operation wird benötigt, da \TT{my\_rand()} gemäß C-Standard einen Wert zwischen 0 und 32767 zurückgeben
muss.

Wenn man 32-Bit-Pseudozufallszahlen benötigt, kann die AND-Operation einfach weggelassen werden.

\subsection{x86}

\lstinputlisting[caption=\Optimizing MSVC 2013,style=customasmx86]{patterns/145_LCG/rand_MSVC_2013_x86_Ox.asm}
Hier sehen wir, dass beide Konstanten in den Code eingebettet wurden. Es wurde kein Speicher für sie reserviert.

Die Funktion \TT{my\_srand()} kopiert ihren Eingabewert in die interne Variable \TT{rand\_state}.

\TT{my\_rand()} nimmt diese, berechnet den nächsten \TT{rand\_state}, schneidet ihn ab und belässt ihn im \EAX Register.

Die nicht optimierte Version ist umfangreicher:

\lstinputlisting[caption=\NonOptimizing MSVC 2013,style=customasmx86]{patterns/145_LCG/rand_MSVC_2013_x86.asm}

\subsection{x64}
Die x64 Version ist größtenteils identisch und verwendet 32-Bit-Register anstelle der 64-Bit-Register (wir arbeiten
hier mit \Tint Werten).

Die Funktion \TT{my\_srand()} nimmt seinen Eingabewert aus dem Register \ECX und nicht vom Stack:

\lstinputlisting[caption=\Optimizing MSVC 2013 x64,style=customasmx86]{patterns/145_LCG/rand_MSVC_2013_x64_Ox_DE.asm}

GCC erzeugt fast den gleichen Code.

\subsection{32-bit ARM}

\lstinputlisting[caption=\OptimizingKeilVI (\ARMMode),style=customasmARM]{patterns/145_LCG/rand.s_Keil_ARM_O3_DE.s}
Es ist nicht möglich 32-Bit-Konstanten in ARM Befehle einzubetten, sodass Keil diese extern speichern und dann
zusätzlich laden muss. Eine interessante Sache ist, dass es ebenfalls nicth möglich ist, die Konstante 0x7FFF
einzubetten.
Was Keil dann tut, ist \IT{rand\_state} um 17 Bits nach links zu verschieben und dann um 17 Bits nach rechts zu
Verchieben.
Die entspricht Ausdruck $(rand\_state \ll 17) \gg 17$ in \CCpp.
Es scheint eine nutzlose Operation zu sein, löscht aber die oberen 17 Bits und lässt die 15 niederen Bits intakt und das
ist genau was wir wollen.\\\\

\Optimizing Keil für Thumb mode erzeugt fast den gleichen Code.

\subsubsection{MIPS}
% FIXME better start at non-optimizing version?
Die Funktion verwendet eine Menge S-Register, die gesichert werden müssen. Das ist der Grund dafür, dass die Werte im
Funktionsprolog gespeichert und im Funktionsepilog wiederhergestellt werden.

\lstinputlisting[caption=\Optimizing GCC 4.4.5
(IDA),style=customasmMIPS]{patterns/13_arrays/1_simple/MIPS_O3_IDA_DE.lst}
Interessant: es gibt zwei Schleifen und die erste benötigt $i$ nicht; sie benötigt nur $i\cdot 2$ (erhöht um 2 bei
jedem Iterationsschritt) und die Adresse im Speicher (erhöht um 4 bei jedem Iterationsschritt).

Wir sehen hier also zwei Variablen: eine (in \$V0), die jedes Mal um 2 erhöht wird, und eine andere (in\$V1), die um 4
erhöht wird.

Die zweite Schleife ist der Ort, an dem \printf aufgerufen wird und dem Benutzer den Wert von $i$ zurückliefert, es gibt
also eine Variable die in \$S0 inkrementiert wird und eine Speicheradresse in \$S1, die jedes Mal um 4 erhöht wird.

% TBT
Das erinnert uns an die Optimierung von Schleifen, die wir früher betrachtet haben: \myref{loop_iterators}.

Das Ziel der Optimierung ist es, die Multiplikationen loszuwerden.


\subsection{Threadsichere Version des Beispiels}
Eine threadsichere Version des Beispiels wird später hier gezeigt:\myref{LCG_TLS}. 
