\subsection{MIPS}

\lstinputlisting[caption=\Optimizing GCC 4.4.5 (IDA),style=customasmMIPS]{patterns/145_LCG/MIPS_O3_IDA_FR.lst}

Ouah, ici nous ne voyons qu'une seule constante (0x3C6EF35F ou 1013904223).
Où est l'autre (1664525)?

Il semble que la multiplication soit effectuée en utilisant seulement des décalages
et des additions!
Vérifions cette hypothèse:

\lstinputlisting[style=customc]{patterns/145_LCG/test.c}

\lstinputlisting[caption=GCC 4.4.5 \Optimizing (IDA),style=customasmMIPS]{patterns/145_LCG/test_O3_MIPS.lst}

En effet!

\subsubsection{Relocations MIPS}

Nous allons nous concentrer sur comment les opérations comme charger et stocker dans
la mémoire fonctionnent.

Les listings ici sont produits par IDA, qui cache certains détails.

Nous allons lancer objdump deux fois: pour obtenir le listing désassemblé et aussi
la liste des relocations:

\lstinputlisting[caption=GCC 4.4.5 \Optimizing (objdump)]{patterns/145_LCG/MIPS_O3_objdump.txt}

Considérons les deux relocations pour la fonction \TT{my\_srand()}.

La première, pour l'adresse 0 a un type de \TT{R\_MIPS\_HI16} et la seconde pour
l'adresse 8 a un type de \TT{R\_MIPS\_LO16}.

Cela implique que l'adresse du début du segment .bss soit écrite dans les instructions
à l'adresse 0 (partie haute de l'adresse) et 8 (partie basse de l'adresse).

La variable \TT{rand\_state} est au tout début du segment .bss.

Donc nous voyons des zéros dans les opérandes des instructions \LUI et \SW, car il
n'y a encore rien ici--- le compilateur ne sait pas quoi y écrire.

L'éditeur de liens va arranger cela, et la partie haute de l'adresse sera écrite
dans l'opérande de \LUI et la partie basse de l'adresse---dans l'opérande de \SW.

\SW va ajouter la partie basse de l'adresse avec le contenu du registre \$V0 (la
partie haute y est).

C'est la même histoire avec la fonction my\_rand(): la relocation R\_MIPS\_HI16 indique
à l'éditeur de liens d'écrire la partie haute.

Donc la partie haute de l'adresse de la variable rand\_state se trouve dans le registre
\$V1.

L'instruction \LW à l'adresse 0x10 ajoute les parties haute et basse et charge la
valeur de la variable rand\_state dans \$V0.

L'instruction \SW à l'adresse 0x54 fait à nouveau la somme et stocke la nouvelle
valeur dans la variable globale rand\_state.

IDA traite les relocations pendant le chargement, cachant ainsi ces détails, mais
nous devons les garder à l'esprit.

% TODO add example of compiled binary, GDB example, etc...
