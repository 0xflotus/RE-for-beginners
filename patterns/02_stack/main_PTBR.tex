\mysection{\Stack}
\label{sec:stack}
\myindex{\Stack}

A pilha é uma das estruturas mais fundamentais na ciência da computação.
\footnote{\href{http://go.yurichev.com/17119}{wikipedia.org/wiki/Call\_stack}}.
\ac{AKA} \ac{LIFO}.

Tecnicamente, é só um bloco de memória junto com os registradores \ESP ou \RSP em x86 e x64, ou o \ac{SP} no ARM, como um ponteiro para aquele bloco.

\myindex{ARM!\Instructions!PUSH}
\myindex{ARM!\Instructions!POP}
\myindex{x86!\Instructions!PUSH}
\myindex{x86!\Instructions!POP}
As instruções mais frequente para o acesso da pilha são \PUSH e \POP (em ambos x86 e x64).
\PUSH subtrai de \ESP/\RSP/\ac{SP} 4 no modo 32-bits (ou 8 no modo 64-bits) e então escreve o conteúdo desse operando único para o endereço de memória apontado por \ESP/\RSP/\ac{SP}.

\POP é a operação reversa: recupera a informação da localização de memória que é apontada por \ac{SP}, 
carrega a mesma no operando da instrução (geralmente um registrador) e então adiciona 4 (ou 8) para o ponteiro da pilha.

Depois da alocação da pilha, o ponteiro aponta para o fundo da pilha.
\PUSH decrementa o ponteiro da pilha e \POP incrementa. O fundo da pilha está na verdade no começo do bloco de memória alocado para ela.
Pode parecer estranho, mas é a maneira como é feita.

ARM: \PTBRph{}

\subsection{Por que a pilha ``cresce'' para trás?}
\label{stack_grow_backwards}

Intuitivamente, nós podemos pensar que a pilha cresce para frente, em direção a endereços mais altos, como qualquer outra estrutura de informação.

O motivo da pilha crescer para trás é provavelmente histórico. Quando os computadores era grandes e ocupavam um cômodo todo, era mais fácil dividir a memória em duas partes, uma para a ‘heap’ e outra para a pilha.
Logicamente, era desconhecido o quão grande a heap e a pilha seriam durante a execução do programa, então essa solução era a mais simples possível.

\input{patterns/02_stack/stack_and_heap}

No \RitchieThompsonUNIX nós podemos ler:

\begin{framed}
\begin{quotation}
A parte relacionada ao usuário é dividida em três segmentos lógicos. O segmento de texto do programa começa na localização 0 no espaço virtual de endereçamento.
Durante a execução, esse segmento é protegido para não ser reescrito e uma única cópia dele é compartilhado entre
todos os processos executando o mesmo programa.
Começando no limite de 8Kbytes acima do segmento de texto do programa no espaço de endereçamento virtual começa um segmento de informação gravável,
não compartilhável e de um tamanho que pode ser extendido por uma chamada do sistema.
Começando no endereço mais alto no espaço de endereçamento virtual está a pilha, que automaticamente cresce para trás conforme o ponteiro da pilha do hardware se altera.
\end{quotation}
\end{framed}

Isso pode ser análogo a como um estudante escreve notas de duas matérias diferentes em um caderno só:
as notas para a primeira matéria são escritas como de costume e as notas para a segunda são escritas do final do caderno,
virando o mesmo. As anotações de uma matéria podem encontrar as da outra no meio, no caso de haver falta de espaço.

\subsection{Para que a pilha é usada?}

% subsections
\input{patterns/02_stack/01_saving_ret_addr}
\EN{\input{patterns/02_stack/02_args_passing_EN}}
\RU{\input{patterns/02_stack/02_args_passing_RU}}
\PTBR{\input{patterns/02_stack/02_args_passing_PTBR}}
\DE{\input{patterns/02_stack/02_args_passing_DE}}
\ITA{\input{patterns/02_stack/02_args_passing_ITA}}
\FR{\input{patterns/02_stack/02_args_passing_FR}}
\JPN{\input{patterns/02_stack/02_args_passing_JPN}}
\PL{\input{patterns/02_stack/02_args_passing_PL}}


\EN{\input{patterns/02_stack/03_local_vars_EN}}
\RU{\input{patterns/02_stack/03_local_vars_RU}}
\PTBR{\input{patterns/02_stack/03_local_vars_PTBR}}
\mysection{\oracle}
\label{oracle}

% sections
\EN{\input{examples/oracle/1_version_EN}}\RU{\input{examples/oracle/1_version_RU}}
\EN{\input{examples/oracle/2_ksmlru_EN}}\RU{\input{examples/oracle/2_ksmlru_RU}}
\EN{\input{examples/oracle/3_timer_EN}}\RU{\input{examples/oracle/3_timer_RU}}


\input{patterns/02_stack/05_SEH}
\input{patterns/02_stack/06_BO_protection}

\subsubsection{\PTBRph{}}

Talvez, o motivo para armazenar variáveis locais e registros SEH na pilha é que eles são desvinculados automaticamente depois do fim da função,
usando somente uma instrução para corrigir o ponteiro da pilha (geralmente é \ADD). Argumentos de funções, como podemos dizer, são
também desalocados automaticamente com o fim da função.
Como contraste, tudo armazenado na memória heap tem de ser desalocado explicitamente.

% sections
\EN{\input{patterns/02_stack/07_layout_EN}}
\RU{\input{patterns/02_stack/07_layout_RU}}
\PTBR{\input{patterns/02_stack/07_layout_PTBR}}
\mysection{\oracle}
\label{oracle}

% sections
\EN{\input{examples/oracle/1_version_EN}}\RU{\input{examples/oracle/1_version_RU}}
\EN{\input{examples/oracle/2_ksmlru_EN}}\RU{\input{examples/oracle/2_ksmlru_RU}}
\EN{\input{examples/oracle/3_timer_EN}}\RU{\input{examples/oracle/3_timer_RU}}


\input{patterns/02_stack/exercises}

