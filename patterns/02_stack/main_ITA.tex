\mysection{\Stack}
\label{sec:stack}
\myindex{\Stack}

Lo stack e' una delle strutture dati piu' importanti in informatica
\footnote{\href{http://go.yurichev.com/17119}{wikipedia.org/wiki/Call\_stack}}.
\ac{AKA} \ac{LIFO}.

Tecnicamente, e' soltanto un blocco di memoria nella memoria di un processo insieme al registro \ESP o \RSP in x86 o x86, o il registro \ac{SP} in ARM, come puntatore all'interno di quel blocco.

\myindex{ARM!\Instructions!PUSH}
\myindex{ARM!\Instructions!POP}
\myindex{x86!\Instructions!PUSH}
\myindex{x86!\Instructions!POP}
Le istruzioni di accesso allo stack piu' usate sono \PUSH e \POP (sia in x86 che in ARM Thumb-mode).
\PUSH sottrae da \ESP/\RSP/\ac{SP} 4 in modalita' 32-bit (oppur 8 in modalita' 64-bit) e scrive successivamente il contenuto del suo unico operando nell'indirizzo di memoria puntato da \ESP/\RSP/\ac{SP}.

\POP e' l'operazione inversa: recupera il dato dalla memoria a cui punta \ac{SP}, lo carica nell'operando dell'istruzione (di solito un registro)
e successivamente aggiunge 4 (o 8) allo \gls{stack pointer}.

A seguito dell'allocazione dello stack, lo \gls{stack pointer} punta alla base (fondo) dello stack.
\PUSH decrementa lo \gls{stack pointer} e \POP lo incrementa.
La base dello stack e' in realta' all'inizio della memoria allocata per il blocco (porzione) dello stack. Sembra strano, ma e' cosi'.

ARM supporta stack decrescenti e crescenti.

\myindex{ARM!\Instructions!STMFD}
\myindex{ARM!\Instructions!LDMFD}
\myindex{ARM!\Instructions!STMED}
\myindex{ARM!\Instructions!LDMED}
\myindex{ARM!\Instructions!STMFA}
\myindex{ARM!\Instructions!LDMFA}
\myindex{ARM!\Instructions!STMEA}
\myindex{ARM!\Instructions!LDMEA}

Ad esempio le istruzioni \ac{STMFD}/\ac{LDMFD}, \ac{STMED}/\ac{LDMED} sono fatte per operare con uno stack decrescente (che cresce verso il basso, inizia con un indirizzo alto e prosegue verso il basso).
Le istruzioni \ac{STMFA}/\ac{LDMFA}, \ac{STMEA}/\ac{LDMEA} sono fatte per operare con uno stack crescente (che cresce verso l'alto, da un indirizzo basso verso uno piu alto).

% It might be worth mentioning that STMED and STMEA write first,
% and then move the pointer,
% and that LDMED and LDMEA move the pointer first, and then read.
% In other words, ARM not only lets the stack grow in a non-standard direction,
% but also in a non-standard order.
% Maybe this can be in the glossary, which would explain why E stands for "empty".

\subsection{Perche' lo stack cresce al contrario?}
\label{stack_grow_backwards}

Intuitivamente potremmo pensare che lo stack cresca verso l'alto, ovvero verso indirizzi piu' alti, come qualunque altra struttura dati.

La ragione per cui lo stack cresce verso il basso e' probabilmente di natura storica.
Quando i computer erano talmente grandi da occupare un'intera stanza, era facile dividere la memoria in due parti, una per lo 
\gls{heap} e l'altra per lo stack.
Ovviamente non era possibile sapere a priori quanto sarebbero stati grandi lo stack e lo \gls{heap} durante l'esecuzione di un programma,
e questa soluzione era la piu' semplice.

\input{patterns/02_stack/stack_and_heap}

In \RitchieThompsonUNIX possiamo leggere:

\begin{framed}
\begin{quotation}
The user-core part of an image is divided into three logical segments.
The program text segment begins at location 0 in the virtual address space.
During execution, this segment is write-protected and a single copy of it is shared among all processes executing the same program.
At the first 8K byte boundary above the program text segment in the virtual address space begins a nonshared, writable data segment, the size of which may be extended by a system call.
Starting at the highest address in the virtual address space is a stack segment, which automatically grows downward as the hardware's stack pointer fluctuates.

Il nucleo utente di una immagine e' diviso in tre segmenti logici.
Il segmento text del programma inizia in posizione 0 nel virtual address space.
Durante l'esecuzione questo segmento viene protetto da scrittura, ed una sua singola copia viene condivisa tra i processi che eseguono lo stesso programma.
Al primo limite di 8K byte dopra il segmento text del programma, nel virtual address space comincia un segmento dati scrivibile, non condiviso, le cui dimensioni possono essere estese da una chiamata di sistema.
A partire dall'indirizzo piu' alto nel virtual address space c'e' lo stack segment, che automaticammente cresce verso il basso al variare dello stack pointer hardware.
\end{quotation}
\end{framed}

Questo ricorda molto come alcuni studenti utilizzino lo stesso quaderno per prendere appunti di due diverse materie:
gli appunti per la prima materia sono scritti normalmente, e quelli della seconda materia sono scritti a partire dalla fine del quaderno, capovolgendolo.
Le note si potrebbero "incontrare" da qualche parte in mezzo al quaderno, nel caso in cui non ci sia abbastanza spazio libero.

% I think if we want to expand on this analogy,
% one might remember that the line number increases as as you go down a page.
% So when you decrease the address when pushing to the stack, visually,
% the stack does grow upwards.
% Of course, the problem is that in most human languages,
% just as with computers,
% we write downwards, so this direction is what makes buffer overflows so messy.

\subsection{Per cosa viene usato lo stack?}

% subsections
\input{patterns/02_stack/01_saving_ret_addr}
\EN{\input{patterns/02_stack/02_args_passing_EN}}
\RU{\input{patterns/02_stack/02_args_passing_RU}}
\PTBR{\input{patterns/02_stack/02_args_passing_PTBR}}
\DE{\input{patterns/02_stack/02_args_passing_DE}}
\ITA{\input{patterns/02_stack/02_args_passing_ITA}}
\FR{\input{patterns/02_stack/02_args_passing_FR}}
\JPN{\input{patterns/02_stack/02_args_passing_JPN}}
\PL{\input{patterns/02_stack/02_args_passing_PL}}


\EN{\input{patterns/02_stack/03_local_vars_EN}}
\RU{\input{patterns/02_stack/03_local_vars_RU}}
\PTBR{\input{patterns/02_stack/03_local_vars_PTBR}}
\mysection{\oracle}
\label{oracle}

% sections
\EN{\input{examples/oracle/1_version_EN}}\RU{\input{examples/oracle/1_version_RU}}
\EN{\input{examples/oracle/2_ksmlru_EN}}\RU{\input{examples/oracle/2_ksmlru_RU}}
\EN{\input{examples/oracle/3_timer_EN}}\RU{\input{examples/oracle/3_timer_RU}}


\input{patterns/02_stack/05_SEH}
\input{patterns/02_stack/06_BO_protection}

\subsubsection{Deallocazione automatica dei dati nello stack}

Probabilmente la ragione per cui si memorizzano nello stack le variabili locali e i record SEH deriva dal fatto che questi dati vengono "liberati" automaticamente all'uscita dalla funzione,
usando soltanto un'istruzione per correggere lo stack pointer (spesso e' \ADD).
Si puo' dire che anche gli argomenti delle funzioni sono deallocati automaticamente alla fine della funzione.
Invece, qualunque altra cosa memorizzata nello \IT{heap} deve essere deallocata esplicitamente.

% sections
\EN{\input{patterns/02_stack/07_layout_EN}}
\RU{\input{patterns/02_stack/07_layout_RU}}
\PTBR{\input{patterns/02_stack/07_layout_PTBR}}
\mysection{\oracle}
\label{oracle}

% sections
\EN{\input{examples/oracle/1_version_EN}}\RU{\input{examples/oracle/1_version_RU}}
\EN{\input{examples/oracle/2_ksmlru_EN}}\RU{\input{examples/oracle/2_ksmlru_RU}}
\EN{\input{examples/oracle/3_timer_EN}}\RU{\input{examples/oracle/3_timer_RU}}


\input{patterns/02_stack/exercises}
