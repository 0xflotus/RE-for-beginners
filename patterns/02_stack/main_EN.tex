\mysection{\Stack}
\label{sec:stack}
\myindex{\Stack}

The stack is one of the most fundamental data structures in computer science
\footnote{\href{http://go.yurichev.com/17119}{wikipedia.org/wiki/Call\_stack}}.
\ac{AKA} \ac{LIFO}.

Technically, it is just a block of memory in process memory along with the \ESP or \RSP register in x86 or x64, or the \ac{SP} register in ARM, as a pointer within that block.

\myindex{ARM!\Instructions!PUSH}
\myindex{ARM!\Instructions!POP}
\myindex{x86!\Instructions!PUSH}
\myindex{x86!\Instructions!POP}
The most frequently used stack access instructions are \PUSH and \POP (in both x86 and ARM Thumb-mode). 
\PUSH subtracts from \ESP/\RSP/\ac{SP} 4 in 32-bit mode (or 8 in 64-bit mode) and then writes the contents of its sole operand to the memory address pointed by \ESP/\RSP/\ac{SP}.

\POP is the reverse operation: retrieve the data from the memory location that \ac{SP} points to, 
load it into the instruction operand (often a register) and then add 4 (or 8) to the \gls{stack pointer}.

After stack allocation, the \gls{stack pointer} points at the bottom of the stack.
\PUSH decreases the \gls{stack pointer} and \POP increases it.
The bottom of the stack is actually at the beginning of the memory allocated for the stack block. It seems strange, but that's the way it is.

ARM supports both descending and ascending stacks.

\myindex{ARM!\Instructions!STMFD}
\myindex{ARM!\Instructions!LDMFD}
\myindex{ARM!\Instructions!STMED}
\myindex{ARM!\Instructions!LDMED}
\myindex{ARM!\Instructions!STMFA}
\myindex{ARM!\Instructions!LDMFA}
\myindex{ARM!\Instructions!STMEA}
\myindex{ARM!\Instructions!LDMEA}

For example the \ac{STMFD}/\ac{LDMFD}, \ac{STMED}/\ac{LDMED} instructions are intended to deal with a descending stack (grows downwards, starting with a high address and progressing to a lower one).
The \ac{STMFA}/\ac{LDMFA}, \ac{STMEA}/\ac{LDMEA} instructions are intended to deal with an ascending stack (grows upwards, starting from a low address and progressing to a higher one).

% It might be worth mentioning that STMED and STMEA write first,
% and then move the pointer,
% and that LDMED and LDMEA move the pointer first, and then read.
% In other words, ARM not only lets the stack grow in a non-standard direction,
% but also in a non-standard order.
% Maybe this can be in the glossary, which would explain why E stands for "empty".

\subsection{Why does the stack grow backwards?}
\label{stack_grow_backwards}

Intuitively, we might think that the stack grows upwards, i.e. towards higher addresses, like any other data structure.

The reason that the stack grows backward is probably historical.
When the computers were big and occupied a whole room, it was easy to divide memory into two parts, one for the \gls{heap} and one for the stack.
Of course, it was unknown how big the \gls{heap} and the stack would be during program execution, so this solution was the simplest possible.

\input{patterns/02_stack/stack_and_heap}

In \RitchieThompsonUNIX we can read:

\begin{framed}
\begin{quotation}
The user-core part of an image is divided into three logical segments. The program text segment begins at location 0 in the virtual address space. During execution, this segment is write-protected and a single copy of it is shared among all processes executing the same program. At the first 8K byte boundary above the program text segment in the virtual address space begins a nonshared, writable data segment, the size of which may be extended by a system call. Starting at the highest address in the virtual address space is a stack segment, which automatically grows downward as the hardware's stack pointer fluctuates.
\end{quotation}
\end{framed}

This reminds us how some students write two lecture notes using only one notebook:
notes for the first lecture are written as usual, 
and notes for the second one are written from the end of notebook, by flipping it.
Notes may meet each other somewhere in between, in case of lack of free space.

% I think if we want to expand on this analogy,
% one might remember that the line number increases as as you go down a page.
% So when you decrease the address when pushing to the stack, visually,
% the stack does grow upwards.
% Of course, the problem is that in most human languages,
% just as with computers,
% we write downwards, so this direction is what makes buffer overflows so messy.

\subsection{What is the stack used for?}

% subsections
\input{patterns/02_stack/01_saving_ret_addr}
\EN{\input{patterns/02_stack/02_args_passing_EN}}
\RU{\input{patterns/02_stack/02_args_passing_RU}}
\PTBR{\input{patterns/02_stack/02_args_passing_PTBR}}
\DE{\input{patterns/02_stack/02_args_passing_DE}}
\ITA{\input{patterns/02_stack/02_args_passing_ITA}}
\FR{\input{patterns/02_stack/02_args_passing_FR}}
\JPN{\input{patterns/02_stack/02_args_passing_JPN}}
\PL{\input{patterns/02_stack/02_args_passing_PL}}


\EN{\input{patterns/02_stack/03_local_vars_EN}}
\RU{\input{patterns/02_stack/03_local_vars_RU}}
\PTBR{\input{patterns/02_stack/03_local_vars_PTBR}}
\mysection{\oracle}
\label{oracle}

% sections
\EN{\input{examples/oracle/1_version_EN}}\RU{\input{examples/oracle/1_version_RU}}
\EN{\input{examples/oracle/2_ksmlru_EN}}\RU{\input{examples/oracle/2_ksmlru_RU}}
\EN{\input{examples/oracle/3_timer_EN}}\RU{\input{examples/oracle/3_timer_RU}}


\input{patterns/02_stack/05_SEH}
\input{patterns/02_stack/06_BO_protection}

\subsubsection{Automatic deallocation of data in stack}

Perhaps the reason for storing local variables and SEH records in the stack is that they are freed automatically upon function exit,
using just one instruction to correct the stack pointer (it is often \ADD).
Function arguments, as we could say, are also deallocated automatically at the end of function.
In contrast, everything stored in the \IT{heap} must be deallocated explicitly.

% sections
\EN{\input{patterns/02_stack/07_layout_EN}}
\RU{\input{patterns/02_stack/07_layout_RU}}
\PTBR{\input{patterns/02_stack/07_layout_PTBR}}
\mysection{\oracle}
\label{oracle}

% sections
\EN{\input{examples/oracle/1_version_EN}}\RU{\input{examples/oracle/1_version_RU}}
\EN{\input{examples/oracle/2_ksmlru_EN}}\RU{\input{examples/oracle/2_ksmlru_RU}}
\EN{\input{examples/oracle/3_timer_EN}}\RU{\input{examples/oracle/3_timer_RU}}


\input{patterns/02_stack/exercises}
