\mysection{\Stack}
\label{sec:stack}
\myindex{\Stack}

Stos w informatyce jest jedną z najbardziej fundamentalnych struktur danych
\footnote{\href{http://go.yurichev.com/17119}{wikipedia.org/wiki/Call\_stack}}.
\ac{AKA} \ac{LIFO}.

Technicznie rzecz biorąc, jest to tylko blok pamięci w pamięci procesora + rejestr \ESP w x86 lub \RSP w x64, lub \ac{SP} w ARM, który wskazuje obszar gdzieś w granicach tego bloku.

\myindex{ARM!\Instructions!PUSH}
\myindex{ARM!\Instructions!POP}
\myindex{x86!\Instructions!PUSH}
\myindex{x86!\Instructions!POP}
Najczęsciej wykorzystywanymi instrukcjami do operowania na stosie są \PUSH i \POP (w x86 i Thumb-trybie ARM). 
\PUSH zmniejsza \ESP/\RSP/\ac{SP} o 4 w trybie 32-bitowym (lub o 8 w 64-bitowym),
następnie zapisuje pod adres, na który wskazuje \ESP/\RSP/\ac{SP}, zawartość swojego operandu.

\POP jest odwrotną operacją- najpierw zdejmuje ze \glslink{stack pointer}{wskaźnika stosu} wartość i umieszcza ją do operandu 
(który często jest rejestrem), a następnie zwiększa wskaźnik o 4 (lub 8).

Przed alokacją pamięci na stosie \glslink{stack pointer}{rejestr-wskaźnik} wskazuje na koniec stosu.
Koniec stosu znajduje się na początku zaalokowanego bloku pamięci, przeznaczonego na stos. Może zabrzmieć to dziwnie, ale tak to działa.
\PUSH zmniejsza \glslink{stack pointer}{rejestr-wskaźnik}, а \POP~--- zwiększa.

W procesorze ARM jest wsparcie dla stosów zarówno rosnących w dół, jak i rosnącyh w górę.

\myindex{ARM!\Instructions!STMFD}
\myindex{ARM!\Instructions!LDMFD}
\myindex{ARM!\Instructions!STMED}
\myindex{ARM!\Instructions!LDMED}
\myindex{ARM!\Instructions!STMFA}
\myindex{ARM!\Instructions!LDMFA}
\myindex{ARM!\Instructions!STMEA}
\myindex{ARM!\Instructions!LDMEA}

Na przykład, instrukcje \ac{STMFD}/\ac{LDMFD}, \ac{STMED}/\ac{LDMED} są przeznaczone dla stosu malejącego (rośnie w dół, zaczynając od adresów wysokich, do adresów niskich).\\
Natomiast instrukcje \ac{STMFA}/\ac{LDMFA}, \ac{STMEA}/\ac{LDMEA} są przeznaczone dla stosu rosnącego (rośnie w górę, zaczynając od niskich adresów, kończąc na adresach wysokich).

% It might be worth mentioning that STMED and STMEA write first,
% and then move the pointer,
% and that LDMED and LDMEA move the pointer first, and then read.
% In other words, ARM not only lets the stack grow in a non-standard direction,
% but also in a non-standard order.
% Maybe this can be in the glossary, which would explain why E stands for "empty".

\subsection{Dlaczego stos rośnie wstecznie?}
\label{stack_grow_backwards}

Intuicyjnie moglibyśmy pomyśleć, że, jak i każda inna struktura danych, stos mogłby rosnąć w górę, tzn. w kierunku zwiększenia adresów.

Powód, dlaczego stos rośnie w dół, jest najprawdobodobniej historyczny.
Kiedy komputery były duże i zajmowały cały pokój, można było bardzo łatwo rozdzielić segment na dwa obszary: dla \glslink{heap}{kopca} i dla stosu.
Z góry nie było wiadomo, jak duża może być \glslink{heap}{sterta} lub stos, dlatego takie rozwiązanie było najbardziej logiczne.

\input{patterns/02_stack/stack_and_heap}

W \RitchieThompsonUNIX można przeczytać:

\begin{framed}
\begin{quotation}
The user-core part of an image is divided into three logical segments. The program text segment begins at location 0 in the virtual address space. During execution, this segment is write-protected and a single copy of it is shared among all processes executing the same program. At the first 8K byte boundary above the program text segment in the virtual address space begins a nonshared, writable data segment, the size of which may be extended by a system call. Starting at the highest address in the virtual address space is a stack segment, which automatically grows downward as the hardware's stack pointer fluctuates.
\end{quotation}
\end{framed}

To trochę przypomina podejście studenta,
który pisze dwie osobne lektury w jednym zeszycie:
pierwsza lektura jest pisana jak zwykle od początku zeszytu, a druga jest pisana od końca zeszytu.
Lektury mogą się "spotkać" gdzieś na środku zeszytu, jeśli zabraknie miejsca.

% I think if we want to expand on this analogy,
% one might remember that the line number increases as as you go down a page.
% So when you decrease the address when pushing to the stack, visually,
% the stack does grow upwards.
% Of course, the problem is that in most human languages,
% just as with computers,
% we write downwards, so this direction is what makes buffer overflows so messy.

\subsection{Do jakich celów służy stos?}

% subsections
\input{patterns/02_stack/01_saving_ret_addr}
\EN{\input{patterns/02_stack/02_args_passing_EN}}
\RU{\input{patterns/02_stack/02_args_passing_RU}}
\PTBR{\input{patterns/02_stack/02_args_passing_PTBR}}
\DE{\input{patterns/02_stack/02_args_passing_DE}}
\ITA{\input{patterns/02_stack/02_args_passing_ITA}}
\FR{\input{patterns/02_stack/02_args_passing_FR}}
\JPN{\input{patterns/02_stack/02_args_passing_JPN}}
\PL{\input{patterns/02_stack/02_args_passing_PL}}


\EN{\input{patterns/02_stack/03_local_vars_EN}}
\RU{\input{patterns/02_stack/03_local_vars_RU}}
\PTBR{\input{patterns/02_stack/03_local_vars_PTBR}}
\PL{\input{patterns/02_stack/03_local_vars_PL}}
\mysection{\oracle}
\label{oracle}

% sections
\EN{\input{examples/oracle/1_version_EN}}\RU{\input{examples/oracle/1_version_RU}}
\EN{\input{examples/oracle/2_ksmlru_EN}}\RU{\input{examples/oracle/2_ksmlru_RU}}
\EN{\input{examples/oracle/3_timer_EN}}\RU{\input{examples/oracle/3_timer_RU}}


\input{patterns/02_stack/05_SEH}
\input{patterns/02_stack/06_BO_protection}

\subsubsection{Automatyczne zwalnianie danych na stosie}
Możliwym powodem przechowywania zmiennych lokalnych i rekordów SEH na stosie jest to, że kiedy funkcja zakończy działanie są one automatycznie zwalniane ze stosu używając tylko jednej instrukcji w celu przywrócenia poprzedniego stanu stosu (często jest to instrukcja ADD). Argumenty funkcji także są automatycznie zwalniane z pamięci pod koniec funkcji. Natomiast wszystko co jest przechowywane na stercie(\IT{heap}) trzeba zwalniać jawnie.

% sections
\EN{\input{patterns/02_stack/07_layout_EN}}
\RU{\input{patterns/02_stack/07_layout_RU}}
\PTBR{\input{patterns/02_stack/07_layout_PTBR}}
\PL{\input{patterns/02_stack/07_layout_PL}}
\mysection{\oracle}
\label{oracle}

% sections
\EN{\input{examples/oracle/1_version_EN}}\RU{\input{examples/oracle/1_version_RU}}
\EN{\input{examples/oracle/2_ksmlru_EN}}\RU{\input{examples/oracle/2_ksmlru_RU}}
\EN{\input{examples/oracle/3_timer_EN}}\RU{\input{examples/oracle/3_timer_RU}}


\input{patterns/02_stack/exercises}


