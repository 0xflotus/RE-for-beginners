\subsection{Exemple de générateur de nombre pseudo-aléatoire revisité}
\label{FPU_PRNG_SIMD}

Revoyons l'exemple de \q{générateur de nombre pseudo-aléatoire} \lstref{FPU_PRNG}.

Si nous compilons ceci en MSVC 2012, il va utiliser les instructions SIMD pour le
FPU.

\lstinputlisting[caption=MSVC 2012 \Optimizing,style=customasmx86]{patterns/205_floating_SIMD/FPU_PRNG/MSVC2012_Ox_Ob0_FR.asm}

% FIXME1 rewrite!

Toutes les instructions ont le suffixe -SS, qui signifie \q{Scalar Single} (scalaire simple).

\q{Scalar} (scalaire) implique qu'une seule valeur est stockée dans le registre.

\q{Single} (simple\footnote{pour simple précision}) signifie un type de donnée \Tfloat.

