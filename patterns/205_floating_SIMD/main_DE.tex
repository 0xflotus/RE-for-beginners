% FIXME1 divide this file into separate ones...
\mysection{Arbeiten mit Fließkommazahlen und SIMD}

\label{floating_SIMD}
\myindex{IEEE 754}
\myindex{SIMD}
\myindex{SSE}
\myindex{SSE2}
Natürlich verblieb die \ac{FPU} in x86-kompatiblen Prozessoren als die \ac{SIMD} Erweiterungen hinzugefügt wurden.

Die \ac{SIMD} Erweiterungen (SSE2) bieten einen einfacheren Weg um mit Fließkommazahlen zu arbeiten.

Das Zahlenformat bleibt dabei das gleich (IEEE 754).

\myindex{x86-64}
Moderne Compiler (inklusive derer für x86-64) verwenden normalerweise \ac{SIMD} Befehle anstatt FPU-Befehle.

Wir werden hier die Beispiele aus dem Abschnitt über die FPU recyclen: \myref{sec:FPU}.

\subsection{Ein einfaches Beispiel}

\lstinputlisting[style=customc]{patterns/12_FPU/1_simple/simple.c}

\subsubsection{x64}

\lstinputlisting[caption=\Optimizing MSVC 2012 x64,style=customasmx86]{patterns/205_floating_SIMD/simple_MSVC_2012_x64_Ox.asm}
Die eingegebenen Fließkommawerte werden in die Register \XMM{0}-\XMM{3} übergeben und der Rest über den
Stackfootnote{\href{http://go.yurichev.com/17263}{MSDN: Parameterübergabe}}.

$a$ wird nach \XMM{0} übergeben und $b$ nach \XMM{1}.

Die XMM Register sind 128 Bit breit (wie wir bereits aus dem Abschnitt über \ac{SIMD} wissen: \myref{SIMD_x86}), aber
die \Tdouble Werte umfassen nur 64 Bit, sodass nur die niedere Hälfte der Register benötigt wird.

\myindex{x86!\Instructions!DIVSD}
\TT{DIVSD} ist ein SSE-Befehl, der für  
\q{Divide Scalar Double-Precision Floating-Point Values} steht;
er teilt einen \Tdouble Wert durch einen anderen, wobei beide in den niederen Hälften der Operanden gespeichert sind. 

Die Konstanten werden durch den Compiler im IEEE 754 Format kodiert.

\myindex{x86!\Instructions!MULSD}
\myindex{x86!\Instructions!ADDSD}
\TT{MULSD} und \TT{ADDSD} funktionieren genauso und führen Multiplikation und Addition durch.

Das Ergebnis der Funktionsausführung ist von Typ \Tdouble und wird im \XMM{0} Register abgelegt.\\\\
So arbeitet der nicht optimierende MSVC:

\lstinputlisting[caption=MSVC 2012 x64,style=customasmx86]{patterns/205_floating_SIMD/simple_MSVC_2012_x64.asm}

\myindex{Shadow space}
Ein wenig redundant. Die Eingabewerte bzw. deren niedere Registerhälften, d.h. die 64-Bit-Werte vom Typ \Tdouble, werden
im \q{shadow space}(\myref{shadow_space}) gespeichert.
GCC erzeugt identischen Code.

\subsubsection{x86}
Kompilieren wir das Beispiel für x86. Obwohl der Code für x86 erzeugt wird, verwendet MSVC 2012 SSE2 Befehle:

\lstinputlisting[caption=\NonOptimizing MSVC 2012 x86,style=customasmx86]{patterns/205_floating_SIMD/simple_MSVC_2012_x86.asm}

\lstinputlisting[caption=\Optimizing MSVC 2012 x86,style=customasmx86]{patterns/205_floating_SIMD/simple_MSVC_2012_x86_Ox.asm}
Es ist fast der gleiche Code, aber es gibt einige Unterschiede in Bezug auf die Aufrufkonventionen:
1) die Argumente werden nicht über die XMM Register, sondern über den Stack übergeben, wie in den FPU Beispielen
(\myref{sec:FPU});
2) das Ergebnis der Funktion wird über \ST{0} zurückgegeben---um dies zu erreichen wird es (durch die lokale Variable
\TT{tv}) aus einem der XMM Register nach \ST{0} kopiert.

\clearpage
Untersuchen wir das optimierte Beispiel mit \olly:

\begin{figure}[H]
\centering
\myincludegraphics{patterns/205_floating_SIMD/simple_olly1.png}
\caption{\olly: \TT{MOVSD} lädt den Wert von $a$ nach \XMM{1}}
\label{fig:FPU_SIMD_simple_olly1}
\end{figure}

\clearpage
\begin{figure}[H]
\centering
\myincludegraphics{patterns/205_floating_SIMD/simple_olly2.png}
\caption{\olly: \TT{DIVSD} hat den Quotienten berechnet und in \XMM{1} gespeichert} 
\label{fig:FPU_SIMD_simple_olly2}
\end{figure}

\clearpage
\begin{figure}[H]
\centering
\myincludegraphics{patterns/205_floating_SIMD/simple_olly3.png}
\caption{\olly: \TT{MULSD} hat das Produkt berechnet und in \XMM{0} gespeichert}
\label{fig:FPU_SIMD_simple_olly3}
\end{figure}

\clearpage
\begin{figure}[H]
\centering
\myincludegraphics{patterns/205_floating_SIMD/simple_olly4.png}
\caption{\olly: \TT{ADDSD} addiert den Wert in \XMM{0} zu \XMM{1}}
\label{fig:FPU_SIMD_simple_olly4}
\end{figure}

\clearpage
\begin{figure}[H]
\centering
\myincludegraphics{patterns/205_floating_SIMD/simple_olly5.png}
\caption{\olly: \FLD lässt das Funktionsergebnis in \ST{0}}
\label{fig:FPU_SIMD_simple_olly5}
\end{figure}
Wir sehen, dass \olly die XMM Register als Paare von \Tdouble Zahlen anzeigt, aber nur der niedere Teil davon verwendet
wird.

Offenbar zeigt \olly sie in diesem Format an, weil die SSE2 Befehle (die mit dem Suffix \TT{-SD}) jetzt ausgeführt
werden.

Natürlich ist es auch möglich das Registerformat zu ändern und sich die Inhalte als 4 \Tfloat-Zahlen oder nur als 16
Byte anzeigen zu lassen.

\clearpage
\subsection{Fließkommazahlen als Argumente übergeben}

\lstinputlisting[style=customc]{patterns/12_FPU/2_passing_floats/pow.c}
Sie werden über die niederen Hälften der Register \XMM{0}-\XMM{3} übergeben.

\lstinputlisting[caption=\Optimizing MSVC 2012 x64,style=customasmx86]{patterns/205_floating_SIMD/pow_MSVC_2012_x64_Ox.asm}

\myindex{x86!\Instructions!MOVSD}
\myindex{x86!\Instructions!MOVSDX}
Es gibt in Intel keinen \TT{MOVSDX} Befehl und AMD-Handbücher (\myref{x86_manuals}) bezeichnen ihn nur mit \TT{MOVSD}.
Es gibt insgesamt zwei Befehle, die sich in x86 denselben Namen teilen (für den anderen siehe: \myref{REP_MOVSx}).
Offenbar wollten die Microsoft Entwickler hier aufräumen und haben ihn deshalb in \TT{MOVSDX} umbenannt.
Er lädt lediglich einen Wert in die niedere Hälfte eines XMM Registers.

\TT{pow()} nimmt Argumente aus \XMM{0} und \XMM{1} und gibt das Ergebnis in \XMM{0} zurück.
Es wird dann für \printf nach \RDX verschoben.
Der Grund dafür ist möglicherweise, dass \printf eine Funktion mit einer variablen Anzahl an Argumenten ist.

\lstinputlisting[caption=\Optimizing GCC 4.4.6
x64,style=customasmx86]{patterns/205_floating_SIMD/pow_GCC446_x64_O3_DE.s}
GCC erzeugt klarer strukturierten Output.
Der Wert für \printf wird in \XMM{0} übergeben.
Hier ist übrigens ein Fall, in dem 1 nach \EAX für \printf geschrieben wird um anzuzeigen, dass ein Argument in den
Vektorregistern übergeben wird, genau wie es der Standard \SysVABI verlangt.

\subsection{Beispiel mit Vergleich}

\lstinputlisting[style=customc]{patterns/12_FPU/3_comparison/d_max.c}

\subsubsection{x64}

\lstinputlisting[caption=\Optimizing MSVC 2012 x64,style=customasmx86]{patterns/205_floating_SIMD/d_max_MSVC_2012_x64_Ox.asm}

\Optimizing MSVC erzeugt leicht verständlichen Code.

\myindex{x86!\Instructions!COMISD}
\TT{COMISD} bedeutet \q{Compare Scalar Ordered Double-Precision Floating-Point 
Values and Set EFLAGS}. Das beschreibt ziemlich genau, was der Befehl tatsächlich tut.\\
\\
\NonOptimizing MSVC erzeugt Code mit mehr Redundanzen, der aber immer noch gut verständlich ist:

\lstinputlisting[caption=MSVC 2012 x64,style=customasmx86]{patterns/205_floating_SIMD/d_max_MSVC_2012_x64.asm}

\myindex{x86!\Instructions!MAXSD}
GCC 4.4.6 hat mehr Optimierungen durchgeführt und den Befehl \TT{MAXSD} (\q{Return Maximum Scalar 
Double-Precision Floating-Point Value}) verwendet, der einfach den größten Wert auswählt!

\lstinputlisting[caption=\Optimizing GCC 4.4.6 x64,style=customasmx86]{patterns/205_floating_SIMD/d_max_GCC446_x64_O3.s}

\clearpage
\subsubsection{x86}
Kompilieren wir dieses Beispiel in MSVC 2012 mit aktivierter Optimierung:

\lstinputlisting[caption=\Optimizing MSVC 2012 x86,style=customasmx86]{patterns/205_floating_SIMD/d_max_MSVC_2012_x86_Ox.asm}
Fast identisch, nur dass die Werte $a$ und $b$ vom Stack geholt werden und das Funktionsergebnis in \ST{0} gelassen
wird.

Wenn wir dieses Beispiel in \olly laden, erkennen wir, wie der Befehl \TT{COMISD} Werte vergleicht und die \CF und \PF
Flags setzt bzw. löscht:

\begin{figure}[H]
\centering
\myincludegraphics{patterns/205_floating_SIMD/d_max_olly.png}
\caption{\olly: \TT{COMISD} hat die \CF und \PF Flags verändert}
\label{fig:FPU_SIMD_d_max_olly}
\end{figure}

\subsection{Berechnen der Maschinengenauigkeit: x64 und SIMD}
\label{machine_epsilon_x64_and_SIMD}
Betrachten wir erneut das Beispiel zur Berechnung der Maschinengenauigkeit für \Tdouble
\lstref{machine_epsilon_double_c}.

Wir kompilieren es jetzt für x64:

\lstinputlisting[caption=\Optimizing MSVC 2012 x64,style=customasmx86]{patterns/205_floating_SIMD/epsilon_double_MSVC_2012_x64_Ox.asm}
Es gibt keinen Weg 1 zum einem Wert in einem 128-Bit XMM Register zu addieren, also muss der Wert im Speicher abgelegt
werden.

Es gibt zwar den Befehl \INS{ADDSD} (\IT{Add Scalar Double-Precision Floating-Point Values}), der einen Wert zu niederen
64-Bit-Hälfte eines XMM Registers addieren kann und die höheren Bits ignoriert, aber MSVC 2012 scheint an dieser Stelle
nicht gut genug zu sein, um diese Möglichkeit zu erkennen\footnote{Als Übung können Sie versuchen, den Code so
umzugestalten, dass der lokale Stack nicht mehr verwendet wird}.

Nichtsdestotrotz wird der Wert dann wieder in ein XMM Register geladen und eine Subtraktion wird durchgeführt.
\INS{SUBSD} steht für \q{Subtract Scalar Double-Precision Floating-Point Values}, d.h. er arbeitet nur auf dem niederen
64-Bit-Teil des 128-Bit XMM Registers.
Das Ergebnis wird in das XMM0-Register zurückgegeben.

\subsection{Gesetzte Bits zählen}
Hier ist ein einfaches Beispiel einer Funktion, die die Anzahl der gesetzten
Bits in einem Eingabewert zählt.

Diese Operation wird auch \q{population count}\footnote{moderne x86 CPUs
(die SSE4 unterstützen) haben zu diesem Zweck sogar einen eigenen POPCNT Befehl}
genannt.

\lstinputlisting[style=customc]{patterns/14_bitfields/4_popcnt/shifts.c}
In dieser Schleife wird der Wert von $i$ schrittweise von 0 bis 31 erhöht,
sodass der Ausdruck $1 \ll i$ von 1 bis \TT{0x80000000} zählt.
In natürlicher Sprache würden wir diese Operation als \IT{verschiebe 1 um n
Bits nach links} beschreiben.
Mit anderen Worten: Der Ausdruck $1 \ll i$ erzeugt alle möglichen Bitpositionen
in einer 32-Bit-Zahl.
Das freie Bit auf der rechten Seite wird jeweils gelöscht.

\label{2n_numbers_table}
Hier ist eine Tabelle mit allen Werten von $1 \ll i$ 
für $i=0 \ldots 31$:

\small
\begin{center}
\begin{tabular}{ | l | l | l | l | }
\hline
\HeaderColor \CCpp Ausdruck & 
\HeaderColor Zweierpotenz & 
\HeaderColor Dezimalzahl & 
\HeaderColor Hexadezimalzahl \\
\hline
$1 \ll 0$ & $2^{0}$ & 1 & 1 \\
\hline
$1 \ll 1$ & $2^{1}$ & 2 & 2 \\
\hline
$1 \ll 2$ & $2^{2}$ & 4 & 4 \\
\hline
$1 \ll 3$ & $2^{3}$ & 8 & 8 \\
\hline
$1 \ll 4$ & $2^{4}$ & 16 & 0x10 \\
\hline
$1 \ll 5$ & $2^{5}$ & 32 & 0x20 \\
\hline
$1 \ll 6$ & $2^{6}$ & 64 & 0x40 \\
\hline
$1 \ll 7$ & $2^{7}$ & 128 & 0x80 \\
\hline
$1 \ll 8$ & $2^{8}$ & 256 & 0x100 \\
\hline
$1 \ll 9$ & $2^{9}$ & 512 & 0x200 \\
\hline
$1 \ll 10$ & $2^{10}$ & 1024 & 0x400 \\
\hline
$1 \ll 11$ & $2^{11}$ & 2048 & 0x800 \\
\hline
$1 \ll 12$ & $2^{12}$ & 4096 & 0x1000 \\
\hline
$1 \ll 13$ & $2^{13}$ & 8192 & 0x2000 \\
\hline
$1 \ll 14$ & $2^{14}$ & 16384 & 0x4000 \\
\hline
$1 \ll 15$ & $2^{15}$ & 32768 & 0x8000 \\
\hline
$1 \ll 16$ & $2^{16}$ & 65536 & 0x10000 \\
\hline
$1 \ll 17$ & $2^{17}$ & 131072 & 0x20000 \\
\hline
$1 \ll 18$ & $2^{18}$ & 262144 & 0x40000 \\
\hline
$1 \ll 19$ & $2^{19}$ & 524288 & 0x80000 \\
\hline
$1 \ll 20$ & $2^{20}$ & 1048576 & 0x100000 \\
\hline
$1 \ll 21$ & $2^{21}$ & 2097152 & 0x200000 \\
\hline
$1 \ll 22$ & $2^{22}$ & 4194304 & 0x400000 \\
\hline
$1 \ll 23$ & $2^{23}$ & 8388608 & 0x800000 \\
\hline
$1 \ll 24$ & $2^{24}$ & 16777216 & 0x1000000 \\
\hline
$1 \ll 25$ & $2^{25}$ & 33554432 & 0x2000000 \\
\hline
$1 \ll 26$ & $2^{26}$ & 67108864 & 0x4000000 \\
\hline
$1 \ll 27$ & $2^{27}$ & 134217728 & 0x8000000 \\
\hline
$1 \ll 28$ & $2^{28}$ & 268435456 & 0x10000000 \\
\hline
$1 \ll 29$ & $2^{29}$ & 536870912 & 0x20000000 \\
\hline
$1 \ll 30$ & $2^{30}$ & 1073741824 & 0x40000000 \\
\hline
$1 \ll 31$ & $2^{31}$ & 2147483648 & 0x80000000 \\
\hline
\end{tabular}
\end{center}
\normalsize
Diese Konstanten (Bitmasken) tauchen im Code oft auf und ein Reverse Engineer
muss in der Lage sein, sie schnell und sicher zu erkennen.

% TBT
Es dazu jedoch nicht notwendig, die Dezimalzahlen (Zweierpotenzen) größer
65535 auswendig zu kennen. Die hexadezimalen Zahlen sind leicht zu merken.

Die Konstanten werden häufig verwendet um Flags einzelnen Bits zuzuordnen. 
Hier ist zum Beispiel ein Auszug aus \TT{ssl\_private.h} aus dem Quellcode von
Apache 2.4.6:

\begin{lstlisting}[style=customc]
/**
 * Define the SSL options
 */
#define SSL_OPT_NONE           (0)
#define SSL_OPT_RELSET         (1<<0)
#define SSL_OPT_STDENVVARS     (1<<1)
#define SSL_OPT_EXPORTCERTDATA (1<<3)
#define SSL_OPT_FAKEBASICAUTH  (1<<4)
#define SSL_OPT_STRICTREQUIRE  (1<<5)
#define SSL_OPT_OPTRENEGOTIATE (1<<6)
#define SSL_OPT_LEGACYDNFORMAT (1<<7)
\end{lstlisting}

Zurück zu unserem Beispiel.

Das Makro \TT{IS\_SET} prüft auf Anwesenheit von Bits in $a$.
\myindex{x86!\Instructions!AND}

Das Makro \TT{IS\_SET} entspricht dabei dem logischen (\IT{AND})
und gibt 0 zurück, wenn das entsprechende Bit nicht gesetzt ist, oder die
Bitmaske, wenn das Bit gesetzt ist.
Der Operator \IT{if()} wird in \CCpp ausgeführt, wenn der boolesche Ausdruck
nicht null ist (er könnte sogar 123456 sein), weshalb es meistens richtig
funktioniert.


% subsections
\subsubsection{x86}

\myparagraph{MSVC}

Kompilieren wir das Beispiel:

\lstinputlisting[caption=MSVC 2008,style=customasmx86]{patterns/13_arrays/1_simple/simple_msvc.asm}

\myindex{x86!\Instructions!SHL}
Soweit nichts Außergewöhnliches, nur zwei Schleifen: die erste füllt mit Werten auf und die zweite gibt Werte aus.
% TBT
Der Befehl \TT{shl ecx, 1} wird für die Multiplikation mit 2 in \ECX verwendet; mehr dazu unten~\myref{SHR}.

Auf dem Stack werden 80 Bytes für das Array reserviert: 20 Elemente von je 4 Byte.

\clearpage
Untersuchen wir dieses Beispiel in \olly.
\myindex{\olly}

Wir erkennen wie das Array befüllt wird:

jedes Element ist ein 32-Bit-Wort vom Typ \Tint und der Wert ist der Index multipliziert mit 2:

\begin{figure}[H]
\centering
\myincludegraphics{patterns/13_arrays/1_simple/olly.png}
\caption{\olly: nach dem Füllen des Arrays}
\label{fig:array_simple_olly}
\end{figure}
Da sich dieses Array auf dem Stack befindet, finden wir dort alle seine 20 Elemente.

\myparagraph{GCC}

Hier ist was GCC 4.4.1 erzeugt:

\lstinputlisting[caption=GCC 4.4.1,style=customasmx86]{patterns/13_arrays/1_simple/simple_gcc.asm}
Die Variable $a$ ist übrigens vom Typ \IT{int*} (Pointer auf \Tint{})--man kann einen Pointer auf ein Array an eine
andere Funktion übergeben, aber es ist richtiger zu sagen, dass der Pointer auf das erste Element des Arrays übergeben
wird. (Die Adressen der übrigen Elemente werden in bekannter Weise berechnet.)

Wenn man diesen Pointer mittels \IT{a[idx]} indiziert, wird \IT{idx} zum Pointer addiert und das dort abgelegte Element
(auf das der berechnete Pointer zeigt) wird zurückgegeben.

Ein interessantes Beispiel: ein String wie \IT{\q{string}} ist ein Array von Chars und hat den Typ \IT{const
char[]}.

Auch auf diesen Pointer kann ein Index angewendet werden.

Das ist der Grund warum es es möglich ist, Dinge wie \TT{\q{string}[i]} zu schreiben--es handelt sich dabei um einen
korrekten \CCpp Ausdruck!


\input{patterns/14_bitfields/4_popcnt/x64_DE}
\subsubsection{ARM}

\myparagraph{\OptimizingKeilVI (\ThumbMode)}

\lstinputlisting[style=customasmARM]{patterns/04_scanf/1_simple/ARM_IDA.lst}

\myindex{\CLanguageElements!\Pointers}
Damit \scanf Elemente einlesen kann, benötigt die Funktion einen Paramter--einen Pointer vom Typ \Tint.
\Tint hat die Größe 32 Bit, wir benötigen also 4 Byte, um den Wert im Speicher abzulegen, und passt daher genau in ein 32-Bit-Register.
\myindex{IDA!var\_?}
Auf dem Stack wird Platz für die lokalen Variable \GTT{x} reserviert und IDA bezeichnet diese Variable mit \IT{var\_8}. 
Eigentlich ist aber an dieser Stelle gar nicht notwendig, Platz auf dem Stack zu reservieren, da \ac{SP} (\gls{stack pointer} 
bereits auf die Adresse zeigt und auch direkt verwendet werden kann.

Der Wert von \ac{SP} wird also in das \Reg{1} Register kopiert und zusammen mit dem Formatierungsstring an \scanf übergeben.

% TBT here
%\INS{PUSH/POP} instructions behaves differently in ARM than in x86 (it's the other way around).
%They are synonyms to \INS{STM/STMDB/LDM/LDMIA} instructions.
%And \INS{PUSH} instruction first writes a value into the stack, \IT{and then} subtracts \ac{SP} by 4.
%\INS{POP} first adds 4 to \ac{SP}, \IT{and then} reads a value from the stack.
%Hence, after \INS{PUSH}, \ac{SP} points to an unused space in stack.
%It is used by \scanf, and by \printf after.

%\INS{LDMIA} means \IT{Load Multiple Registers Increment address After each transfer}.
%\INS{STMDB} means \IT{Store Multiple Registers Decrement address Before each transfer}.

\myindex{ARM!\Instructions!LDR}
Später wird mithilfe des \INS{LDR} Befehls dieser Wert vom Stack in das \Reg{1} Register verschoben um an \printf übergeben werden zu können.

\myparagraph{ARM64}

\lstinputlisting[caption=\NonOptimizing GCC 4.9.1 ARM64,numbers=left,style=customasmARM]{patterns/04_scanf/1_simple/ARM64_GCC491_O0_DE.s}

Im Stack Frame werden 32 Byte reserviert, was deutlich mehr als benötigt ist. Vielleicht handelt es sich um eine Frage des Aligning (dt. Angleichens) von Speicheradressen.
Der interessanteste Teil ist, im Stack Frame einen Platz für die Variable $x$ zu finden (Zeile 22).
Warum 28? Irgendwie hat der Compiler entschieden die Variable am Ende des Stack Frames anstatt an dessen Beginn abzulegen.
Die Adresse wird an \scanf übergeben; diese Funktion speichert den Userinput an der genannten Adresse im Speicher.
Es handelt sich hier um einen 32-Bit-Wert vom Typ \Tint. 
Der Wert wird in Zeile 27 abgeholt und dann an \printf übergeben.



\subsubsection{MIPS}
% FIXME better start at non-optimizing version?
Die Funktion verwendet eine Menge S-Register, die gesichert werden müssen. Das ist der Grund dafür, dass die Werte im
Funktionsprolog gespeichert und im Funktionsepilog wiederhergestellt werden.

\lstinputlisting[caption=\Optimizing GCC 4.4.5
(IDA),style=customasmMIPS]{patterns/13_arrays/1_simple/MIPS_O3_IDA_DE.lst}
Interessant: es gibt zwei Schleifen und die erste benötigt $i$ nicht; sie benötigt nur $i\cdot 2$ (erhöht um 2 bei
jedem Iterationsschritt) und die Adresse im Speicher (erhöht um 4 bei jedem Iterationsschritt).

Wir sehen hier also zwei Variablen: eine (in \$V0), die jedes Mal um 2 erhöht wird, und eine andere (in\$V1), die um 4
erhöht wird.

Die zweite Schleife ist der Ort, an dem \printf aufgerufen wird und dem Benutzer den Wert von $i$ zurückliefert, es gibt
also eine Variable die in \$S0 inkrementiert wird und eine Speicheradresse in \$S1, die jedes Mal um 4 erhöht wird.

% TBT
Das erinnert uns an die Optimierung von Schleifen, die wir früher betrachtet haben: \myref{loop_iterators}.

Das Ziel der Optimierung ist es, die Multiplikationen loszuwerden.



\subsection{Zusammenfassung}
In den Beispielen hier wird nur die niedere Hälfte der XMM Register verwendet, um eine Zahl im IEEE 754 Format zu
speichern.

Im Prinzip arbeiten alle Befehle, die um den Präfix \TT{-SD} (\q{Scalar Double-Precision}) ergänzt wurden, mit
Fließkommazahlen im IEEE 754 Format, die in der niederen 64-Bit-Hälfte eines XMM Registers gespeichert werden.

Dies ist einfacher als in der FPU, da die SIMD sich in weniger chaotischer Weise als die FPU entwickelt hat.
Das Stackregister Modell wird nicht verwendet.

\myindex{x86!\Instructions!ADDSS}
\myindex{x86!\Instructions!MOVSS}
\myindex{x86!\Instructions!COMISS}
% TODO1: do this!
Wenn man in diesen Beispielen \Tdouble durch \Tfloat ersetzen würde, würden die gleichen Befehle verwendet werden, aber
jeweils mit \TT{-SS}(\q{Scalar Single-Precision}) Präfix, z.B. \TT{MOVSS}, \TT{COMISS}, \TT{ADDSS}, etc.

\q{Scalar} bedeutet, dass die SIMD Register nur einen anstatt mehreren Werten enthalten.

Befehle, die mit mehreren Werte in einem Register gleichzeitig arbeiten, haben ein \q{Packed} in ihrem Namen.

Unnötig extra zu erwähnen, dass die SSE2 Befehle mit 64-Bit IEEE 743 Werten (\Tdouble) arbeiten, wohingegen die interne
Repräsentation von Fließkommazahlen in der FPU 80-Bit-Zahlen verwendet.

Die FPU wird deshalb manchmal weniger Rundungsfehler machen und als Konsequenz daraus möglicherweise präzisere
Berechnungsergebnisse liefern.
