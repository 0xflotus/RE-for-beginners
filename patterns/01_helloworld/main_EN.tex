\mysection{\HelloWorldSectionName}
\label{sec:helloworld}

Let's use the famous example from the book [\KRBook]:

\lstinputlisting[caption=\CCpp Code,style=customc]{patterns/01_helloworld/hw.c}

\subsection{x86}

\input{patterns/01_helloworld/MSVC_x86}
\input{patterns/01_helloworld/GCC_x86}
\input{patterns/01_helloworld/string_patching_EN}

\subsection{x86-64}
\input{patterns/01_helloworld/MSVC_x64}
\input{patterns/01_helloworld/GCC_x64}
\input{patterns/01_helloworld/address_patching_EN}

\input{patterns/01_helloworld/GCC_one_more}
\mysection{\oracle}
\label{oracle}

% sections
\EN{\input{examples/oracle/1_version_EN}}\RU{\input{examples/oracle/1_version_RU}}
\EN{\input{examples/oracle/2_ksmlru_EN}}\RU{\input{examples/oracle/2_ksmlru_RU}}
\EN{\input{examples/oracle/3_timer_EN}}\RU{\input{examples/oracle/3_timer_RU}}


\mysection{\oracle}
\label{oracle}

% sections
\EN{\input{examples/oracle/1_version_EN}}\RU{\input{examples/oracle/1_version_RU}}
\EN{\input{examples/oracle/2_ksmlru_EN}}\RU{\input{examples/oracle/2_ksmlru_RU}}
\EN{\input{examples/oracle/3_timer_EN}}\RU{\input{examples/oracle/3_timer_RU}}



\subsection{\Conclusion{}}

The main difference between x86/ARM and x64/ARM64 code is that the pointer to the string is now 64-bits in length.
Indeed, modern \ac{CPU}s are now 64-bit due to both the reduced cost of memory and the greater demand for it by modern applications. 
We can add much more memory to our computers than 32-bit pointers are able to address.
As such, all pointers are now 64-bit.

% sections
\input{patterns/01_helloworld/exercises}
