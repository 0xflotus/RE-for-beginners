\mysection{\HelloWorldSectionName}
\label{sec:helloworld}

Utilizziamo il famoso esempio dal libro [\KRBook]:

\lstinputlisting[style=customc]{patterns/01_helloworld/hw.c}

\subsection{x86}

\input{patterns/01_helloworld/MSVC_x86}
\input{patterns/01_helloworld/GCC_x86}
% \input{patterns/01_helloworld/string_patching_EN} % TODO translate

\subsection{x86-64}
\input{patterns/01_helloworld/MSVC_x64}
\input{patterns/01_helloworld/GCC_x64}
% \input{patterns/01_helloworld/address_patching_EN} % TODO translate

\input{patterns/01_helloworld/GCC_one_more}
\mysection{\oracle}
\label{oracle}

% sections
\EN{\input{examples/oracle/1_version_EN}}\RU{\input{examples/oracle/1_version_RU}}
\EN{\input{examples/oracle/2_ksmlru_EN}}\RU{\input{examples/oracle/2_ksmlru_RU}}
\EN{\input{examples/oracle/3_timer_EN}}\RU{\input{examples/oracle/3_timer_RU}}


\mysection{\oracle}
\label{oracle}

% sections
\EN{\input{examples/oracle/1_version_EN}}\RU{\input{examples/oracle/1_version_RU}}
\EN{\input{examples/oracle/2_ksmlru_EN}}\RU{\input{examples/oracle/2_ksmlru_RU}}
\EN{\input{examples/oracle/3_timer_EN}}\RU{\input{examples/oracle/3_timer_RU}}



\subsection{\Conclusion{}}

La differenza principale tra il codice x86/ARM e x64/ARM64 è che il puntatore alla stringa è adesso lungo 64 bit.
Infatti, le moderne \ac{CPU} sono ora a 64-bit grazie ai costi ridotti della memoria e alla sua grande richiesta da parte delle applicazioni moderne. 
Possiamo aggiungere ai nostri computer più memoria di quanto i puntatori a 32-bit siano in grado di indirizzare.  
Di conseguenza, tutti i puntatori sono oggi a 64-bit.

% sections
\input{patterns/01_helloworld/exercises}
