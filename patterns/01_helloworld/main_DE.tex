\mysection{\HelloWorldSectionName}
\label{sec:helloworld}

Beginnen wir mit dem berühmten Beispiel aus dem Buch [\KRBook]:

\lstinputlisting[style=customc]{patterns/01_helloworld/hw.c}

\subsection{x86}

\input{patterns/01_helloworld/MSVC_x86}
\input{patterns/01_helloworld/GCC_x86}
\input{patterns/01_helloworld/string_patching_DE}

\subsection{x86-64}
\input{patterns/01_helloworld/MSVC_x64}
\input{patterns/01_helloworld/GCC_x64}
\input{patterns/01_helloworld/address_patching_DE}

\input{patterns/01_helloworld/GCC_one_more_DE}
\mysection{\oracle}
\label{oracle}

% sections
\EN{\input{examples/oracle/1_version_EN}}\RU{\input{examples/oracle/1_version_RU}}
\EN{\input{examples/oracle/2_ksmlru_EN}}\RU{\input{examples/oracle/2_ksmlru_RU}}
\EN{\input{examples/oracle/3_timer_EN}}\RU{\input{examples/oracle/3_timer_RU}}


\mysection{\oracle}
\label{oracle}

% sections
\EN{\input{examples/oracle/1_version_EN}}\RU{\input{examples/oracle/1_version_RU}}
\EN{\input{examples/oracle/2_ksmlru_EN}}\RU{\input{examples/oracle/2_ksmlru_RU}}
\EN{\input{examples/oracle/3_timer_EN}}\RU{\input{examples/oracle/3_timer_RU}}



\subsection{\Conclusion{}}

Der Hauptunterschied zwischen x86/ARM and x64/ARM64-Code ist das der Zeiger auf den String 64 Bit lang ist.
Moderne \ac{CPU}s haben eine 64-Bit-Architektur um Speicherkosten zu reduzieren und den höheren Bedarf
aktueller Anwendungen erfüllen zu können.
Es ist möglich sehr viel mehr Speicher in dem Computer zu verwenden als 32-Bit-Zeiger adressieren können.
Aus diesem Grund sind alle Zeiger 64 Bit lang.

% sections
\input{patterns/01_helloworld/exercises}

