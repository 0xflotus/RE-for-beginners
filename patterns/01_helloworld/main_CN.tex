\mysection{\HelloWorldSectionName}
\label{sec:helloworld}

我们来使用[\KRBook]中的著名例子吧:

\lstinputlisting[style=customc]{patterns/01_helloworld/hw.c}

\subsection{x86}

\input{patterns/01_helloworld/MSVC_x86}
\input{patterns/01_helloworld/GCC_x86}
% \input{patterns/01_helloworld/string_patching_EN} % TODO translate

\subsection{x86-64}
\input{patterns/01_helloworld/MSVC_x64}
\input{patterns/01_helloworld/GCC_x64}
% \input{patterns/01_helloworld/address_patching_EN} % TODO translate

\input{patterns/01_helloworld/GCC_one_more}
\mysection{\oracle}
\label{oracle}

% sections
\EN{\input{examples/oracle/1_version_EN}}\RU{\input{examples/oracle/1_version_RU}}
\EN{\input{examples/oracle/2_ksmlru_EN}}\RU{\input{examples/oracle/2_ksmlru_RU}}
\EN{\input{examples/oracle/3_timer_EN}}\RU{\input{examples/oracle/3_timer_RU}}


\mysection{\oracle}
\label{oracle}

% sections
\EN{\input{examples/oracle/1_version_EN}}\RU{\input{examples/oracle/1_version_RU}}
\EN{\input{examples/oracle/2_ksmlru_EN}}\RU{\input{examples/oracle/2_ksmlru_RU}}
\EN{\input{examples/oracle/3_timer_EN}}\RU{\input{examples/oracle/3_timer_RU}}



\subsection{\Conclusion{}}

x86/ARM架构和x64/ARM64架构,在代码上的主要区别是,后者对于字符串的指针大小是64位长度的。
事实上,由于内存价格走低,以及更多现代应用程序对64位架构的需求增大,目前的\ac{CPU}大多是64位的。
由于64位指针比32位指针的寻址空间大了很多,我们可以向计算机中添加更多的内存。
基于以上几点,目前(64位架构)所有的指针都是64位的。

% sections
\input{patterns/01_helloworld/exercises}

