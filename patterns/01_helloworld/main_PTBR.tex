\mysection{\HelloWorldSectionName}
\label{sec:helloworld}

Vamos usar o famoso exemplo do livro [\KRBook]:

\lstinputlisting[style=customc]{patterns/01_helloworld/hw.c}

\subsection{x86}

\input{patterns/01_helloworld/MSVC_x86}
\input{patterns/01_helloworld/GCC_x86}
% \input{patterns/01_helloworld/string_patching_EN} % TODO translate

\subsection{x86-64}
\input{patterns/01_helloworld/MSVC_x64}
\input{patterns/01_helloworld/GCC_x64}
% \input{patterns/01_helloworld/address_patching_EN} % TODO translate

\input{patterns/01_helloworld/GCC_one_more}
\mysection{\oracle}
\label{oracle}

% sections
\EN{\input{examples/oracle/1_version_EN}}\RU{\input{examples/oracle/1_version_RU}}
\EN{\input{examples/oracle/2_ksmlru_EN}}\RU{\input{examples/oracle/2_ksmlru_RU}}
\EN{\input{examples/oracle/3_timer_EN}}\RU{\input{examples/oracle/3_timer_RU}}


\mysection{\oracle}
\label{oracle}

% sections
\EN{\input{examples/oracle/1_version_EN}}\RU{\input{examples/oracle/1_version_RU}}
\EN{\input{examples/oracle/2_ksmlru_EN}}\RU{\input{examples/oracle/2_ksmlru_RU}}
\EN{\input{examples/oracle/3_timer_EN}}\RU{\input{examples/oracle/3_timer_RU}}



\subsection{\Conclusion{}}

A principal diferença entre os códigos em x86/ARM e x64/ARM64 é que o ponteiro para a string é agora 64-bits de tamanho.
De fato, \ac{CPU}s modernas agora são de 64-bits devido a redução do custo da memória e a demanda mais alta devido a aplicações mais modernas.
Nós podemos adicionar muito mais memória nos nossos computadores do que ponteiros de 32-bits são capazes de endereçar.
Como tal, todos os ponteiros são agora 64-bits.

% sections
\input{patterns/01_helloworld/exercises}

