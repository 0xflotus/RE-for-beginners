\mysection{\HelloWorldSectionName}
\label{sec:helloworld}

We bekijken het beroemde voorbeeld uit het boek [\KRBook]:

\lstinputlisting[style=customc]{patterns/01_helloworld/hw.c}

\subsection{x86}

\input{patterns/01_helloworld/MSVC_x86}
\input{patterns/01_helloworld/GCC_x86}
% \input{patterns/01_helloworld/string_patching_EN} % TODO translate

\subsection{x86-64}
\input{patterns/01_helloworld/MSVC_x64}
\input{patterns/01_helloworld/GCC_x64}
% \input{patterns/01_helloworld/address_patching_EN} % TODO translate

\input{patterns/01_helloworld/GCC_one_more}
\mysection{\oracle}
\label{oracle}

% sections
\EN{\input{examples/oracle/1_version_EN}}\RU{\input{examples/oracle/1_version_RU}}
\EN{\input{examples/oracle/2_ksmlru_EN}}\RU{\input{examples/oracle/2_ksmlru_RU}}
\EN{\input{examples/oracle/3_timer_EN}}\RU{\input{examples/oracle/3_timer_RU}}


\mysection{\oracle}
\label{oracle}

% sections
\EN{\input{examples/oracle/1_version_EN}}\RU{\input{examples/oracle/1_version_RU}}
\EN{\input{examples/oracle/2_ksmlru_EN}}\RU{\input{examples/oracle/2_ksmlru_RU}}
\EN{\input{examples/oracle/3_timer_EN}}\RU{\input{examples/oracle/3_timer_RU}}



\subsection{\Conclusion{}}

Het grootste verschil tussen x86/ARM en x64/ARM64 code is dat de pointer naar de string nu 64-bits in lengte is.
De meeste moderne \ac{CPU}s zijn tegenwoordig 64-bit wegens zowel de verminderde gebruik van geheugen, als de grote vraag ervoor door moderne applicaties.
We kunnen hierdoor veel meer geheugen aan onze computers toevoegen dan dat 32-bit pointers kunnen aanspreken.
Bijgevolg zijn alle pointers nu 64-bit.

% sections
\input{patterns/01_helloworld/exercises}
