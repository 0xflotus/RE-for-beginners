\mysection{\HelloWorldSectionName}
\label{sec:helloworld}

[\KRBook]という本の有名な例を使ってみましょう

\lstinputlisting[caption=\CCpp Code,style=customc]{patterns/01_helloworld/hw.c}

\subsection{x86}

\input{patterns/01_helloworld/MSVC_x86}
\input{patterns/01_helloworld/GCC_x86}
\input{patterns/01_helloworld/string_patching_JPN}

\subsection{x86-64}
\input{patterns/01_helloworld/MSVC_x64}
\input{patterns/01_helloworld/GCC_x64}
\input{patterns/01_helloworld/address_patching_JPN}

\input{patterns/01_helloworld/GCC_one_more}
\mysection{\oracle}
\label{oracle}

% sections
\EN{\input{examples/oracle/1_version_EN}}\RU{\input{examples/oracle/1_version_RU}}
\EN{\input{examples/oracle/2_ksmlru_EN}}\RU{\input{examples/oracle/2_ksmlru_RU}}
\EN{\input{examples/oracle/3_timer_EN}}\RU{\input{examples/oracle/3_timer_RU}}


\mysection{\oracle}
\label{oracle}

% sections
\EN{\input{examples/oracle/1_version_EN}}\RU{\input{examples/oracle/1_version_RU}}
\EN{\input{examples/oracle/2_ksmlru_EN}}\RU{\input{examples/oracle/2_ksmlru_RU}}
\EN{\input{examples/oracle/3_timer_EN}}\RU{\input{examples/oracle/3_timer_RU}}



\subsection{\Conclusion{}}

x86/ARMとx64/ARM64コードの主な違いは、文字列へのポインタが64ビット長になったことです。
確かに、現代の \ac{CPU} は64ビットになりました。これは、現代のアプリケーションではメモリの節約と大きな需要の両方があるからです。
私たちは32ビットポインタよりもはるかに多くのメモリをコンピュータに追加することができます。
そのため、すべてのポインタは64ビットになりました。

% sections
\input{patterns/01_helloworld/exercises}
