\subsection{Exemple de générateur de nombres pseudo-aléatoires}
\label{FPU_PRNG}

Si nous avons besoin de nombres aléatoires à virgule flottante entre 0 et 1, le plus
simple est d'utiliser un \ac{PRNG} comme le Twister de Mersenne.
Il produit une valeur aléatoire non signée sur 32-bit (en d'autres mots, il produit
une valeur 32-bit aléatoire).
Puis, nous pouvons transformer cette valeur en \Tfloat et le diviser par \GTT{RAND\_MAX}
(\GTT{0xFFFFFFFF} dans notre cas)---nous obtenons une valeur dans l'intervalle 0..1.

Mais nous savons que la division est lente.
Aussi, nous aimerions utiliser le moins d'opérations FPU possible.
Peut-on se passer de la division?

\myindex{IEEE 754}

Rappelons-nous en quoi consiste un nombre en virgule flottante: un bit de signe,
un significande et un exposant.
Nous n'avons qu'à stocker des bits aléatoires dans toute le significande pour obtenir
un nombre réel aléatoire!

L'exposant ne peut pas être zéro (le nombre flottant est dénormalisé dans ce cas),
donc nous stockons 0b01111111 dans l'exposant---ce qui signifie que l'exposant est
1.
Ensuite nous remplissons le significande avec des bits aléatoires, mettons le signe à
0 (ce qui indique un nombre positif) et voilà.
Les nombres générés sont entre 1 et 2, donc nous devons soustraire 1.

\newcommand{\URLXOR}{\url{http://go.yurichev.com/17308}}

Un générateur congruentiel linéaire de nombres aléatoire très simple est utilisé dans
mon exemple\footnote{l'idée a été prise de: \URLXOR}, il produit des nombres 32-bit.
Le \ac{PRNG} est initialisé avec le temps courant au format UNIX timestamp.

Ici, nous représentons un type \Tfloat comme une \IT{union}---c'est la construction \CCpp qui nous
permet d'interpréter un bloc de mémoire sous différents types.
Dans notre cas, nous pouvons créer une variable de type \IT{union} et y accéder comme
si c'est un \Tfloat ou un \IT{uint32\_t}.
On peut dire que c'est juste un hack. Un sale.

% WTF?

Le code du \ac{PRNG} entier est le même que celui que nous avons déjà considéré: \myref{LCG_simple}.
Donc la forme compilée du code est omise.

\lstinputlisting[style=customc]{patterns/17_unions/FPU_PRNG/FPU_PRNG_FR.cpp}

\subsubsection{x86}

\lstinputlisting[caption=MSVC 2010 \Optimizing,style=customasmx86]{patterns/17_unions/FPU_PRNG/MSVC2010_Ox_Ob0_FR.asm}

Les noms de fonctions sont étranges ici car cet exemple a été compilé en tant que
C++ et ceci est la modification des noms en C++, nous en parlerons plus loin: \myref{namemangling}.
Si nous compilons ceci avec MSVC 2012, il utilise des instructions SIMD pour le FPU,
pour en savoir plus: \myref{FPU_PRNG_SIMD}.

\subsubsection{MIPS}

\lstinputlisting[caption=GCC 4.4.5 \Optimizing,style=customasmMIPS]{patterns/17_unions/FPU_PRNG/MIPS_O3_IDA_FR.lst}

Il y a aussi une instruction \INS{LUI} inutile, ajoutée pour quelque étrange raison.
Nous avons déjà considéré cet artefact plus tôt: \myref{MIPS_FPU_LUI}.

\subsubsection{ARM (\ARMMode)}

\lstinputlisting[caption=GCC 4.6.3 \Optimizing (IDA),style=customasmARM]{patterns/17_unions/FPU_PRNG/raspberry_GCC_O3_IDA_FR.lst}

\myindex{objdump}
\myindex{binutils}
\myindex{IDA}

Nous allons faire un dump avec objdump et nous allons voir que les instructions FPU
ont un nom différent que dans \IDA.
Apparemment, les développeurs de IDA et binutils ont utilisés des manuels différents?
Peut-être qu'il serait bon de connaître les deux variantes de noms des instructions.

\lstinputlisting[caption=GCC 4.6.3 \Optimizing (objdump),style=customasmARM]{patterns/17_unions/FPU_PRNG/raspberry_GCC_O3_objdump.lst}

Les instructions en 0x5c dans \TT{float\_rand()} et en 0x38 dans \main sont du bruit
(pseudo-)aléatoire.

