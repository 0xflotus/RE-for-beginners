\mysection{\RU{Объединения (union)}\EN{Unions}\DE{Unions}\FR{Unions}\JPN{共用体}}

\EN{\CCpp \IT{union} is mostly used for interpreting a variable (or memory block) of one data type as a variable of another data type.}
\DE{Die \\Cpp \IT{union} wird hauptsächlich verwendet um eine Variable (oder einen Speicherblock) eines Datentyps als
Variable eines anderen Datentyps zu interpretieren.}
\RU{\IT{union} в \CCpp используется в основном для интерпретации переменной (или блока памяти) одного типа как переменной другого типа.}
\FR{Les \IT{unions} en \CCpp sont utilisées principalement pour interpréter une variable
(ou un bloc de mémoire) d'un type de données comme une variable d'un autre type de données.}
\JPN{\CCpp \IT{共用体}は、あるデータ型の変数(またはメモリブロック)を別のデータ型の変数として解釈するために使用されます。}

% sections
\EN{\subsection{Pseudo-random number generator example}
\label{FPU_PRNG}

If we need float random numbers between 0 and 1, the simplest thing is to use a \ac{PRNG} like
the Mersenne twister. 
It produces random unsigned 32-bit values (in other words, it produces random 32 bits).
Then we can transform this value to \Tfloat and then
divide it by \GTT{RAND\_MAX} (\GTT{0xFFFFFFFF} in our case)---we getting a value in the 0..1 interval.

But as we know, division is slow.
Also, we would like to issue as few FPU operations as possible.
Can we get rid of the division?

\myindex{IEEE 754}

Let's recall what a floating point number consists of: sign bit, significand bits and exponent bits.
We just have to store random bits in all significand bits to get a random float number!

The exponent cannot be zero (the floating number is denormalized in this case),
so we are storing 0b01111111
to exponent---this means that the exponent is 1. 
Then we filling the significand with random bits, set the sign bit to
0 (which means a positive number) and voilà.
The generated numbers is to be between 1 and 2, so we must also subtract 1.

\newcommand{\URLXOR}{\url{http://go.yurichev.com/17308}}

A very simple linear congruential random numbers generator is used in my 
example\footnote{the idea was taken from: \URLXOR}, it produces 32-bit numbers. 
The \ac{PRNG} is initialized with the current time in UNIX timestamp format.

Here we represent the \Tfloat type as an \IT{union}---it is the \CCpp construction that enables us
to interpret a piece of memory as different types.
In our case, we are able to create a variable
of type \IT{union} and then access to it as it is \Tfloat or as it is \IT{uint32\_t}. 
It can be said, it is just a hack. A dirty one.

% WTF?

The integer \ac{PRNG} code is the same as we already considered: \myref{LCG_simple}.
So this code in compiled form is omitted.

\lstinputlisting[style=customc]{patterns/17_unions/FPU_PRNG/FPU_PRNG_EN.cpp}

\subsubsection{x86}

\lstinputlisting[caption=\Optimizing MSVC 2010,style=customasmx86]{patterns/17_unions/FPU_PRNG/MSVC2010_Ox_Ob0_EN.asm}

Function names are so strange here because this example was compiled as C++ and this is name mangling in C++,
we will talk about it later: \myref{namemangling}.
If we compile this in MSVC 2012, it uses the SIMD instructions for the FPU, read more about it here: \myref{FPU_PRNG_SIMD}.

\iffalse
A BUG HERE
\subsubsection{MIPS}

\lstinputlisting[caption=\Optimizing GCC 4.4.5,style=customasmMIPS]{patterns/17_unions/FPU_PRNG/MIPS_O3_IDA_EN.lst}

There is also an useless \INS{LUI} instruction added for some weird reason.
We considered this artifact earlier: \myref{MIPS_FPU_LUI}.
\fi

\subsubsection{ARM (\ARMMode)}

\lstinputlisting[caption=\Optimizing GCC 4.6.3 (IDA),style=customasmARM]{patterns/17_unions/FPU_PRNG/raspberry_GCC_O3_IDA_EN.lst}

\myindex{objdump}
\myindex{binutils}
\myindex{IDA}

We'll also make a dump in objdump and we'll see that the FPU instructions have different names than in \IDA.
Apparently, IDA and binutils developers used different manuals?
Perhaps it would be good to know both instruction name variants.

\lstinputlisting[caption=\Optimizing GCC 4.6.3 (objdump),style=customasmARM]{patterns/17_unions/FPU_PRNG/raspberry_GCC_O3_objdump.lst}

The instructions at 0x5c in \TT{float\_rand()} and at 0x38 in \main are (pseudo-)random noise.

}
\RU{\subsection{Пример генератора случайных чисел}
\label{FPU_PRNG}

Если нам нужны случайные значения с плавающей запятой в интервале от 0 до 1, самое простое это взять
\ac{PRNG} вроде Mersenne twister.
Он выдает случайные беззнаковые 32-битные числа (иными словами, он выдает 32 случайных бита).
Затем мы можем преобразовать это число в \Tfloat и затем разделить на \\
\GTT{RAND\_MAX} (\GTT{0xFFFFFFFF} в данном случае) --- полученное число будет в интервале от 0 до 1.

Но как известно, операция деления --- это медленная операция. 
Да и вообще хочется избежать лишних операций с FPU.
Сможем ли мы избежать деления?

\myindex{IEEE 754}
Вспомним состав числа с плавающей запятой: это бит знака, биты мантиссы и биты экспоненты. 
Для получения случайного числа, нам нужно просто заполнить случайными битами все биты мантиссы!

Экспонента не может быть нулевой (иначе число с плавающей точкой будет денормализованным), 
так что в эти биты мы запишем 0b01111111 --- это будет означать что экспонента равна единице.
Далее заполняем мантиссу случайными битами, знак оставляем в виде 0 (что значит наше число положительное), и вуаля.
Генерируемые числа будут в интервале от 1 до 2, так что нам еще нужно будет отнять единицу.

\newcommand{\URLXOR}{\url{http://go.yurichev.com/17308}}

В моем примере\footnote{идея взята здесь: \URLXOR} 
применяется очень простой линейный конгруэнтный генератор случайных чисел, выдающий 32-битные числа.
Генератор инициализируется текущим временем в стиле UNIX.

Далее, тип \Tfloat представляется в виде \IT{union} --- это конструкция \CCpp позволяющая 
интерпретировать часть памяти по-разному. В нашем случае, мы можем создать переменную типа \IT{union} 
и затем обращаться к ней как к \Tfloat или как к \IT{uint32\_t}. Можно сказать, что это хак, причем грязный.

% WTF?

Код целочисленного \ac{PRNG} точно такой же, как мы уже рассматривали ранее: \myref{LCG_simple}.
Так что и в скомпилированном виде этот код будет опущен.

\lstinputlisting[style=customc]{patterns/17_unions/FPU_PRNG/FPU_PRNG_RU.cpp}

\subsubsection{x86}

\lstinputlisting[caption=\Optimizing MSVC 2010,style=customasmx86]{patterns/17_unions/FPU_PRNG/MSVC2010_Ox_Ob0_RU.asm}

Имена функций такие странные, потому что этот пример был скомпилирован как Си++, и это манглинг имен в Си++, мы будем рассматривать это позже: \myref{namemangling}.

Если скомпилировать это в MSVC 2012, компилятор будет использовать SIMD-инструкции для FPU, читайте об этом здесь:

\myref{FPU_PRNG_SIMD}.

\iffalse
A BUG HERE
\subsubsection{MIPS}

\lstinputlisting[caption=\Optimizing GCC 4.4.5,style=customasmMIPS]{patterns/17_unions/FPU_PRNG/MIPS_O3_IDA_RU.lst}

Здесь снова зачем-то добавлена инструкция \INS{LUI}, которая ничего не делает.
Мы уже рассматривали этот артефакт ранее: \myref{MIPS_FPU_LUI}.
\fi

\subsubsection{ARM (\ARMMode)}

\lstinputlisting[caption=\Optimizing GCC 4.6.3 (IDA),style=customasmARM]{patterns/17_unions/FPU_PRNG/raspberry_GCC_O3_IDA_RU.lst}

\myindex{objdump}
\myindex{binutils}
\myindex{IDA}
Мы также сделаем дамп в objdump и увидим, что FPU-инструкции имеют немного другие имена чем в \IDA.
Наверное, разработчики IDA и binutils пользовались разной документацией?
Должно быть, будет полезно знать оба варианта названий инструкций.

\lstinputlisting[caption=\Optimizing GCC 4.6.3 (objdump),style=customasmARM]{patterns/17_unions/FPU_PRNG/raspberry_GCC_O3_objdump.lst}

Инструкции по адресам 0x5c в \TT{float\_rand()} и 0x38 в main() это (псевдо-)случайный мусор.

}
\DE{\subsection{Pseudozufallszahlengenerator Beispiel}
\label{FPU_PRNG}
Wenn wir Zufallszahlen vom Typ \Tfloat zwischen 0 und 1 brauchen, ist die einfachste Möglichkeit einen \ac{PRNG} wie den
Mersenne-Twister zu verwenden.
Er produziert vorzeichenlose 32-Bit-Werte (mit anderen Worten: er erzeugt 32 zufällige Bits).
Diesen Wert können wir in einen \Tfloat umwandeln und dann durch \GTT{RAND\_MAX} (\GTT{0xFFFFFFFF} in unserem Fall)
teilen---wir erhalten einen Wert im Intervall 0..1.

Wie wir jedoch wissen, ist die Division langsam.
Auch würden wir gerne so wenig FPU Operation wie möglich verwenden.
Deshalb fragen wir uns, ob wir die Division loswerden können.

\myindex{IEEE 754}
Erinnern wir uns an den Aufbau einer Fließkommazahl: sie besteht aus einem Vorzeichenbit, Bits im Signifikanden und Bits
im Exponenten.
Wir müssen also lediglich Zufallsbits in allen Bits des Signifikanden speichern um einen zufällige Fließkommazahl zu
erhalten!

Der Exponent kann nicht null sein (in diesem Fall ist die Fließkommazahl denormalisiert); also speichern wir 0b01111111
im Exponenten---das bedeutet, dass der Exponent 1 ist.
Dann füllen wir den Signifikanden mit Zufallsbits, setzen das Vorzeichenbit auf 0 (entspricht einer positiven Zahl) und
voilà.
Die erzeugte Zahl liegt zwischen 1 und 2; wir müssen also am Ende noch 1 abziehen.

\newcommand{\URLXOR}{\url{http://go.yurichev.com/17308}}
Ein sehr einfacher linearer Kongruenzgenerator für Zufallszahlen wird in meinem Beispiel\footnote{die Idee stammt von:
\URLXOR} vorgestellt: er erzeugt 32-Bit-Zahlen.
Der \ac{PRNG} wird mit der aktuellen Zeit um UNIX-Timestamp-Format initialisiert.

Wir stellen hier den Typ \Tfloat als \IT{union} dar--das ist die \CCpp Konstruktion, die es uns ermöglicht, einen
Speicherblock als unterschiedliche Typen aufzufassen.
In unserem Fall sind wir in der Lage eine Variable vom Typ \IT{union} zu erzeugen und dann auf diese wie auf einen
\Tfloat oder \IT{uint32\_t} zuzugreifen.
Man kann sagen, dass es sich dabei um einen Hack, d.h. Trick handelt. Sogar um einen sehr schmutzigen.

% WTF?
Der \ac{PRNG} Code für Integer ist der bereits betrachtete: \myref{LCG_simple}.
Deshalb verzichten wir an dieser Stelle auf ein erneutes Listing des kompilierten Codes.

\lstinputlisting[style=customc]{patterns/17_unions/FPU_PRNG/FPU_PRNG_EN.cpp}

\subsubsection{x86}

\lstinputlisting[caption=\Optimizing MSVC 2010,style=customasmx86]{patterns/17_unions/FPU_PRNG/MSVC2010_Ox_Ob0_DE.asm}
Die Namen der Funktionen sind hier so seltsam, weil das Beispiel als C++ kompiliert wurde, und hier name mangling in C++
vorliegt. Dies werden wir später besprechen: \myref{namemangling}.
Wenn wir das Beispiel mit MSVC 2012 kompilieren, verwendet es SIMD Befehle für die FPU; mehr dazu unter:
\myref{FPU_PRNG_SIMD}.

\iffalse
A BUG HERE
\subsubsection{MIPS}

\lstinputlisting[caption=\Optimizing GCC 4.4.5,style=customasmMIPS]{patterns/17_unions/FPU_PRNG/MIPS_O3_IDA_DE.lst}
Hier wurde auch ein unnützer \INS{LUI} Befehl aus unerfindlichen Gründen hinzugefügt.
Wir haben solch ein Artefakt schon früher betrachtet: \myref{MIPS_FPU_LUI}.
\fi

\subsubsection{ARM (\ARMMode)}

\lstinputlisting[caption=\Optimizing GCC 4.6.3
(IDA),style=customasmARM]{patterns/17_unions/FPU_PRNG/raspberry_GCC_O3_IDA_DE.lst}

\myindex{objdump}
\myindex{binutils}
\myindex{IDA}
Wir ziehen auch einen Dump in objdump und shen, dass die FPU Befehle andere Namen als in \IDA haben.
Offenbar haben die Entwickler von \IDA und binutils unterschiedliche Handbücher verwendet.
Möglicherweise ist es hilfreich, beide Varianten der Befehlsnamen zu kennen.

\lstinputlisting[caption=\Optimizing GCC 4.6.3 (objdump),style=customasmARM]{patterns/17_unions/FPU_PRNG/raspberry_GCC_O3_objdump.lst}

Die Befehle an den Stellen 0x5c in \TT{float\_rand()} und 0x38 in \main sind (Pseudo-)Zufallsrauschen.

}
\FR{\subsection{Exemple de générateur de nombres pseudo-aléatoires}
\label{FPU_PRNG}

Si nous avons besoin de nombres aléatoires à virgule flottante entre 0 et 1, le plus
simple est d'utiliser un \ac{PRNG} comme le Twister de Mersenne.
Il produit une valeur aléatoire non signée sur 32-bit (en d'autres mots, il produit
une valeur 32-bit aléatoire).
Puis, nous pouvons transformer cette valeur en \Tfloat et le diviser par \GTT{RAND\_MAX}
(\GTT{0xFFFFFFFF} dans notre cas)---nous obtenons une valeur dans l'intervalle 0..1.

Mais nous savons que la division est lente.
Aussi, nous aimerions utiliser le moins d'opérations FPU possible.
Peut-on se passer de la division?

\myindex{IEEE 754}

Rappelons-nous en quoi consiste un nombre en virgule flottante: un bit de signe,
un significande et un exposant.
Nous n'avons qu'à stocker des bits aléatoires dans toute le significande pour obtenir
un nombre réel aléatoire!

L'exposant ne peut pas être zéro (le nombre flottant est dénormalisé dans ce cas),
donc nous stockons 0b01111111 dans l'exposant---ce qui signifie que l'exposant est
1.
Ensuite nous remplissons le significande avec des bits aléatoires, mettons le signe à
0 (ce qui indique un nombre positif) et voilà.
Les nombres générés sont entre 1 et 2, donc nous devons soustraire 1.

\newcommand{\URLXOR}{\url{http://go.yurichev.com/17308}}

Un générateur congruentiel linéaire de nombres aléatoire très simple est utilisé dans
mon exemple\footnote{l'idée a été prise de: \URLXOR}, il produit des nombres 32-bit.
Le \ac{PRNG} est initialisé avec le temps courant au format UNIX timestamp.

Ici, nous représentons un type \Tfloat comme une \IT{union}---c'est la construction \CCpp qui nous
permet d'interpréter un bloc de mémoire sous différents types.
Dans notre cas, nous pouvons créer une variable de type \IT{union} et y accéder comme
si c'est un \Tfloat ou un \IT{uint32\_t}.
On peut dire que c'est juste un hack. Un sale.

% WTF?

Le code du \ac{PRNG} entier est le même que celui que nous avons déjà considéré: \myref{LCG_simple}.
Donc la forme compilée du code est omise.

\lstinputlisting[style=customc]{patterns/17_unions/FPU_PRNG/FPU_PRNG_FR.cpp}

\subsubsection{x86}

\lstinputlisting[caption=MSVC 2010 \Optimizing,style=customasmx86]{patterns/17_unions/FPU_PRNG/MSVC2010_Ox_Ob0_FR.asm}

Les noms de fonctions sont étranges ici car cet exemple a été compilé en tant que
C++ et ceci est la modification des noms en C++, nous en parlerons plus loin: \myref{namemangling}.
Si nous compilons ceci avec MSVC 2012, il utilise des instructions SIMD pour le FPU,
pour en savoir plus: \myref{FPU_PRNG_SIMD}.

\subsubsection{MIPS}

\lstinputlisting[caption=GCC 4.4.5 \Optimizing,style=customasmMIPS]{patterns/17_unions/FPU_PRNG/MIPS_O3_IDA_FR.lst}

Il y a aussi une instruction \INS{LUI} inutile, ajoutée pour quelque étrange raison.
Nous avons déjà considéré cet artefact plus tôt: \myref{MIPS_FPU_LUI}.

\subsubsection{ARM (\ARMMode)}

\lstinputlisting[caption=GCC 4.6.3 \Optimizing (IDA),style=customasmARM]{patterns/17_unions/FPU_PRNG/raspberry_GCC_O3_IDA_FR.lst}

\myindex{objdump}
\myindex{binutils}
\myindex{IDA}

Nous allons faire un dump avec objdump et nous allons voir que les instructions FPU
ont un nom différent que dans \IDA.
Apparemment, les développeurs de IDA et binutils ont utilisés des manuels différents?
Peut-être qu'il serait bon de connaître les deux variantes de noms des instructions.

\lstinputlisting[caption=GCC 4.6.3 \Optimizing (objdump),style=customasmARM]{patterns/17_unions/FPU_PRNG/raspberry_GCC_O3_objdump.lst}

Les instructions en 0x5c dans \TT{float\_rand()} et en 0x38 dans \main sont du bruit
(pseudo-)aléatoire.

}
\JPN{\subsection{擬似乱数生成器の例}
\label{FPU_PRNG}

0と1の間の浮動小数点の乱数が必要な場合、最も簡単なのはメルセンヌツイスターのような
\ac{PRNG}を使うことです。
ランダムな符号なし32ビット値を生成します(つまり、ランダム32ビットを生成します)。
この値をfloatに変換し、
\GTT{RAND\_MAX}(ここでは\GTT{0xFFFFFFFF})で割ります。我々は0..1の間で値を取得します。

しかし知ってのとおり、除算は遅いです。
また、できるだけ少ないFPU演算で実行したいと考えています。
私たちは除算を取り除くことができるでしょうか?

\myindex{IEEE 754}

浮動小数点数が符号ビット、仮数ビット、指数ビットからなるものを思い出してみましょう。
ランダムな浮動小数点数を得るには、すべての仮数ビットにランダムなビットを格納するだけです。

指数部はゼロではありません(浮動小数点はこの場合非正規化されています)ので、
指数部に0b01111111
を格納しています。指数部が1であることを意味します。
次に、仮数部をランダムビットで埋め、符号ビットを0に設定する(正の数)と出来上がり。
生成される数は1と2の間にあるので、1を減算する必要があります。

\newcommand{\URLXOR}{\url{http://go.yurichev.com/17308}}

私の例では、非常に単純な線形合同乱数ジェネレータが使用され、
\footnote{アイデアは以下から取りました: \URLXOR} これは32ビットの数値を生成します。 
\ac{PRNG}はUNIXのタイムスタンプ形式で現在の時刻で初期化されます。

ここでは \Tfloat 型を\IT{union}として表します。これは、メモリの種類を
異なる型として解釈できる \CCpp 構造です。
私たちの場合、\IT{union}型の変数を作成し、
それを \Tfloat  または \IT{uint32\_t} のようにアクセスすることができます。
それはまさに汚いハックだと言えるでしょう。

% WTF?

整数\ac{PRNG}コードは、すでに検討しているものと同じです:\myref{LCG_simple}
このコードはコンパイルされた形式では省略されています。

\lstinputlisting[style=customc]{patterns/17_unions/FPU_PRNG/FPU_PRNG_JPN.cpp}

\subsubsection{x86}

\lstinputlisting[caption=\Optimizing MSVC 2010,style=customasmx86]{patterns/17_unions/FPU_PRNG/MSVC2010_Ox_Ob0_JPN.asm}

この例はC++としてコンパイルされており、これはC++での名前の変換であるため、ここでは関数名が非常に奇妙です。
これについては後で説明します:\myref{namemangling}
これをMSVC 2012でコンパイルすると、FPU用のSIMD命令が使用されます。詳細については、こちらを参照してください:\myref{FPU_PRNG_SIMD}

\iffalse
A BUG HERE
\subsubsection{MIPS}

\lstinputlisting[caption=\Optimizing GCC 4.4.5,style=customasmMIPS]{patterns/17_unions/FPU_PRNG/MIPS_O3_IDA_JPN.lst}

いくつかの奇妙な理由のために追加された無駄な\INS{LUI}命令もあります。 
このアーティファクトを以前検討しました:\myref{MIPS_FPU_LUI}
\fi

\subsubsection{ARM (\ARMMode)}

\lstinputlisting[caption=\Optimizing GCC 4.6.3 (IDA),style=customasmARM]{patterns/17_unions/FPU_PRNG/raspberry_GCC_O3_IDA_JPN.lst}

\myindex{objdump}
\myindex{binutils}
\myindex{IDA}

また、objdumpにダンプを作成し、FPU命令の名前が \IDA とは異なることを確認します。 
見たところ、IDAとbinutilsの開発者は異なるマニュアルを使ったのでしょうか? 
おそらく、両方の命令の変種を知っておくとよいでしょう。

\lstinputlisting[caption=\Optimizing GCC 4.6.3 (objdump),style=customasmARM]{patterns/17_unions/FPU_PRNG/raspberry_GCC_O3_objdump.lst}

\TT{float\_rand()}の0x5cと \main の0x38の命令は、(疑似)乱数ノイズです。
}


\EN{\mysection{Returning Values}
\label{ret_val_func}

Another simple function is the one that simply returns a constant value:

\lstinputlisting[caption=\EN{\CCpp Code},style=customc]{patterns/011_ret/1.c}

Let's compile it.

\subsection{x86}

Here's what both the GCC and MSVC compilers produce (with optimization) on the x86 platform:

\lstinputlisting[caption=\Optimizing GCC/MSVC (\assemblyOutput),style=customasmx86]{patterns/011_ret/1.s}

\myindex{x86!\Instructions!RET}
There are just two instructions: the first places the value 123 into the \EAX register,
which is used by convention for storing the return
value, and the second one is \RET, which returns execution to the \gls{caller}.

The caller will take the result from the \EAX register.

\subsection{ARM}

There are a few differences on the ARM platform:

\lstinputlisting[caption=\OptimizingKeilVI (\ARMMode) ASM Output,style=customasmARM]{patterns/011_ret/1_Keil_ARM_O3.s}

ARM uses the register \Reg{0} for returning the results of functions, so 123 is copied into \Reg{0}.

\myindex{ARM!\Instructions!MOV}
\myindex{x86!\Instructions!MOV}
It is worth noting that \MOV is a misleading name for the instruction in both the x86 and ARM \ac{ISA}s.

The data is not in fact \IT{moved}, but \IT{copied}.

\subsection{MIPS}

\label{MIPS_leaf_function_ex1}

The GCC assembly output below lists registers by number:

\lstinputlisting[caption=\Optimizing GCC 4.4.5 (\assemblyOutput),style=customasmMIPS]{patterns/011_ret/MIPS.s}

\dots while \IDA does it by their pseudo names:

\lstinputlisting[caption=\Optimizing GCC 4.4.5 (IDA),style=customasmMIPS]{patterns/011_ret/MIPS_IDA.lst}

The \$2 (or \$V0) register is used to store the function's return value.
\myindex{MIPS!\Pseudoinstructions!LI}
\INS{LI} stands for ``Load Immediate'' and is the MIPS equivalent to \MOV.

\myindex{MIPS!\Instructions!J}
The other instruction is the jump instruction (J or JR) which returns the execution flow to the \gls{caller}.

\myindex{MIPS!Branch delay slot}
You might be wondering why the positions of the load instruction (LI) and the jump instruction (J or JR) are swapped. This is due to a \ac{RISC} feature called ``branch delay slot''.

The reason this happens is a quirk in the architecture of some RISC \ac{ISA}s and isn't important for our
purposes---we must simply keep in mind that in MIPS, the instruction following a jump or branch instruction
is executed \IT{before} the jump/branch instruction itself.

As a consequence, branch instructions always swap places with the instruction executed immediately beforehand.

In practice, functions which merely return 1 (\IT{true}) or 0 (\IT{false}) are very frequent.

The smallest ever of the standard UNIX utilities, \IT{/bin/true} and \IT{/bin/false} return 0 and 1 respectively, as an exit code.
(Zero as an exit code usually means success, non-zero means error.)
}
\RU{\mysection{Оптимизации циклов}

% subsections:
\subsection{Странная оптимизация циклов}

Это самая простая (из всех возможных) реализация memcpy():

\begin{lstlisting}[style=customc]
void memcpy (unsigned char* dst, unsigned char* src, size_t cnt)
{
	size_t i;
	for (i=0; i<cnt; i++)
		dst[i]=src[i];
};
\end{lstlisting}

Как минимум MSVC 6.0 из конца 90-х вплоть до MSVC 2013 может выдавать вот такой странный код (этот листинг создан MSVC 2013
x86):

\lstinputlisting[style=customasmx86]{advanced/500_loop_optimizations/1_1_RU.lst}

Это всё странно, потому что как люди работают с двумя указателями? Они сохраняют два адреса в двух регистрах или двух
ячейках памяти.
Компилятор MSVC в данном случае сохраняет два указателя как один указатель (\IT{скользящий dst} в \EAX)
и разницу между указателями \IT{src} и \IT{dst} (она остается неизменной во время исполнения цикла, в \ESI).
\myindex{\CLanguageElements!ptrdiff\_t}
(Кстати, это тот редкий случай, когда можно использовать тип ptrdiff\_t.)
Когда нужно загрузить байт из \IT{src}, он загружается на \IT{diff + скользящий dst} и сохраняет байт просто на
\IT{скользящем dst}.

Должно быть это какой-то трюк для оптимизации. Но я переписал эту ф-цию так:

\lstinputlisting[style=customasmx86]{advanced/500_loop_optimizations/1_2.lst}

\dots и она работает также быстро как и \IT{соптимизированная} версия на моем Intel Xeon E31220 @ 3.10GHz.
Может быть, эта оптимизация предназначалась для более старых x86-процессоров 90-х, т.к., этот трюк использует
как минимум древний MS VC 6.0?

Есть идеи?

\myindex{Hex-Rays}
Hex-Rays 2.2 не распознает такие шаблонные фрагменты кода (будем надеятся, это временно?):

\begin{lstlisting}[style=customc]
void __cdecl f1(char *dst, char *src, size_t size)
{
  size_t counter; // edx@1
  char *sliding_dst; // eax@2
  char tmp; // cl@3

  counter = size;
  if ( size )
  {
    sliding_dst = dst;
    do
    {
      tmp = (sliding_dst++)[src - dst];         // разница (src-dst) вычисляется один раз, перед телом цикла
      *(sliding_dst - 1) = tmp;
      --counter;
    }
    while ( counter );
  }
}
\end{lstlisting}

Тем не менее, этот трюк часто используется в MSVC (и не только в самодельных ф-циях \IT{memcpy()}, но также и во многих
циклах, работающих с двумя или более массивами), так что для реверс-инжиниров стоит помнить об этом.

% <!-- As of why writting occurred after <b>dst</b> incrementing? -->


\subsection{Возврат строки}

Классическая ошибка из \RobPikePractice{}:

\begin{lstlisting}[style=customc]
#include <stdio.h>

char* amsg(int n, char* s)
{
        char buf[100];

        sprintf (buf, "error %d: %s\n", n, s) ;

        return buf;
};

int main()
{
        printf ("%s\n", amsg (1234, "something wrong!"));
};
\end{lstlisting}

Она упадет.
В начале, попытаемся понять, почему.

Это состояние стека перед возвратом из amsg():

% FIXME! TikZ or whatever
\begin{lstlisting}
§(низкие адреса)§

§[amsg(): 100 байт]§
§[RA]                               <- текущий SP§
§[два аргумента amsg]§
§[что-то еще]§
§[локальные переменные main()]§

§(высокие адреса)§
\end{lstlisting}

Когда управление возвращается из amsg() в \main, пока всё хорошо.
Но когда \printf вызывается из \main, который, в свою очередь, использует стек для своих нужд, затирая 100-байтный буфер.
В лучшем случае, будет выведен случайный мусор.

Трудно поверить, но я знаю, как это исправить:

\begin{lstlisting}[style=customc]
#include <stdio.h>

char* amsg(int n, char* s)
{
        char buf[100];

        sprintf (buf, "error %d: %s\n", n, s) ;

        return buf;
};

char* interim (int n, char* s)
{
        char large_buf[8000];
        // используем локальный массив.
        // а иначе компилятор выбросит его при оптимизации, как неиспользуемый.
        large_buf[0]=0;
        return amsg (n, s);
};

int main()
{
        printf ("%s\n", interim (1234, "something wrong!"));
};
\end{lstlisting}

Это заработает если скомпилировано в MSVC 2013 без оптимизаций и с опцией \TT{/GS-}\footnote{Выключить защиту от переполнения буфера}.
MSVC предупредит: ``warning C4172: returning address of local variable or temporary'', но код запустится и сообщение выведется.
Посмотрим состояние стека в момент, когда amsg() возвращает управление в interim():

\begin{lstlisting}
§(низкие адреса)§

§[amsg(): 100 байт]§
§[RA]                                      <- текущий SP§
§[два аргумента amsg()]§
§[вледения interim(), включая 8000 байт]§
§[еще что-то]§
§[локальные переменные main()]§

§(высокие адреса)§
\end{lstlisting}

Теперь состояние стека на момент, когда interim() возвращает управление в \main{}:

\begin{lstlisting}
§(низкие адреса)§

§[amsg(): 100 байт]§
§[RA]§
§[два аргумента amsg()]§
§[вледения interim(), включая 8000 байт]§
§[еще что-то]                              <- текущий SP§
§[локальные переменные main()]§

§(высокие адреса)§
\end{lstlisting}

Так что когда \main вызывает \printf, он использует стек в месте, где выделен буфер в interim(),
и не затирает 100 байт с сообщение об ошибке внутри, потому что 8000 байт (или может быть меньше) это достаточно для всего,
что делает \printf и другие нисходящие ф-ции!

Это также может сработать, если между ними много ф-ций, например:
\main $\rightarrow$ f1() $\rightarrow$ f2() $\rightarrow$ f3() ... $\rightarrow$ amsg(),
и тогда результат amsg() используется в \main.
Дистанция между \ac{SP} в \main и адресом буфера \TT{buf[]} должна быть достаточно длинной.

Вот почему такие ошибки опасны: иногда ваш код работает (и бага прячется незамеченной). иногда нет.
\label{heisenbug}
\myindex{Хейзенбаги}
Такие баги в шутку называют хейзенбаги или шрёдинбаги\footnote{\url{https://en.wikipedia.org/wiki/Heisenbug}}.



}
\DE{\subsection{Gesetzte Bits zählen}
Hier ist ein einfaches Beispiel einer Funktion, die die Anzahl der gesetzten
Bits in einem Eingabewert zählt.

Diese Operation wird auch \q{population count}\footnote{moderne x86 CPUs
(die SSE4 unterstützen) haben zu diesem Zweck sogar einen eigenen POPCNT Befehl}
genannt.

\lstinputlisting[style=customc]{patterns/14_bitfields/4_popcnt/shifts.c}
In dieser Schleife wird der Wert von $i$ schrittweise von 0 bis 31 erhöht,
sodass der Ausdruck $1 \ll i$ von 1 bis \TT{0x80000000} zählt.
In natürlicher Sprache würden wir diese Operation als \IT{verschiebe 1 um n
Bits nach links} beschreiben.
Mit anderen Worten: Der Ausdruck $1 \ll i$ erzeugt alle möglichen Bitpositionen
in einer 32-Bit-Zahl.
Das freie Bit auf der rechten Seite wird jeweils gelöscht.

\label{2n_numbers_table}
Hier ist eine Tabelle mit allen Werten von $1 \ll i$ 
für $i=0 \ldots 31$:

\small
\begin{center}
\begin{tabular}{ | l | l | l | l | }
\hline
\HeaderColor \CCpp Ausdruck & 
\HeaderColor Zweierpotenz & 
\HeaderColor Dezimalzahl & 
\HeaderColor Hexadezimalzahl \\
\hline
$1 \ll 0$ & $2^{0}$ & 1 & 1 \\
\hline
$1 \ll 1$ & $2^{1}$ & 2 & 2 \\
\hline
$1 \ll 2$ & $2^{2}$ & 4 & 4 \\
\hline
$1 \ll 3$ & $2^{3}$ & 8 & 8 \\
\hline
$1 \ll 4$ & $2^{4}$ & 16 & 0x10 \\
\hline
$1 \ll 5$ & $2^{5}$ & 32 & 0x20 \\
\hline
$1 \ll 6$ & $2^{6}$ & 64 & 0x40 \\
\hline
$1 \ll 7$ & $2^{7}$ & 128 & 0x80 \\
\hline
$1 \ll 8$ & $2^{8}$ & 256 & 0x100 \\
\hline
$1 \ll 9$ & $2^{9}$ & 512 & 0x200 \\
\hline
$1 \ll 10$ & $2^{10}$ & 1024 & 0x400 \\
\hline
$1 \ll 11$ & $2^{11}$ & 2048 & 0x800 \\
\hline
$1 \ll 12$ & $2^{12}$ & 4096 & 0x1000 \\
\hline
$1 \ll 13$ & $2^{13}$ & 8192 & 0x2000 \\
\hline
$1 \ll 14$ & $2^{14}$ & 16384 & 0x4000 \\
\hline
$1 \ll 15$ & $2^{15}$ & 32768 & 0x8000 \\
\hline
$1 \ll 16$ & $2^{16}$ & 65536 & 0x10000 \\
\hline
$1 \ll 17$ & $2^{17}$ & 131072 & 0x20000 \\
\hline
$1 \ll 18$ & $2^{18}$ & 262144 & 0x40000 \\
\hline
$1 \ll 19$ & $2^{19}$ & 524288 & 0x80000 \\
\hline
$1 \ll 20$ & $2^{20}$ & 1048576 & 0x100000 \\
\hline
$1 \ll 21$ & $2^{21}$ & 2097152 & 0x200000 \\
\hline
$1 \ll 22$ & $2^{22}$ & 4194304 & 0x400000 \\
\hline
$1 \ll 23$ & $2^{23}$ & 8388608 & 0x800000 \\
\hline
$1 \ll 24$ & $2^{24}$ & 16777216 & 0x1000000 \\
\hline
$1 \ll 25$ & $2^{25}$ & 33554432 & 0x2000000 \\
\hline
$1 \ll 26$ & $2^{26}$ & 67108864 & 0x4000000 \\
\hline
$1 \ll 27$ & $2^{27}$ & 134217728 & 0x8000000 \\
\hline
$1 \ll 28$ & $2^{28}$ & 268435456 & 0x10000000 \\
\hline
$1 \ll 29$ & $2^{29}$ & 536870912 & 0x20000000 \\
\hline
$1 \ll 30$ & $2^{30}$ & 1073741824 & 0x40000000 \\
\hline
$1 \ll 31$ & $2^{31}$ & 2147483648 & 0x80000000 \\
\hline
\end{tabular}
\end{center}
\normalsize
Diese Konstanten (Bitmasken) tauchen im Code oft auf und ein Reverse Engineer
muss in der Lage sein, sie schnell und sicher zu erkennen.

% TBT
Es dazu jedoch nicht notwendig, die Dezimalzahlen (Zweierpotenzen) größer
65535 auswendig zu kennen. Die hexadezimalen Zahlen sind leicht zu merken.

Die Konstanten werden häufig verwendet um Flags einzelnen Bits zuzuordnen. 
Hier ist zum Beispiel ein Auszug aus \TT{ssl\_private.h} aus dem Quellcode von
Apache 2.4.6:

\begin{lstlisting}[style=customc]
/**
 * Define the SSL options
 */
#define SSL_OPT_NONE           (0)
#define SSL_OPT_RELSET         (1<<0)
#define SSL_OPT_STDENVVARS     (1<<1)
#define SSL_OPT_EXPORTCERTDATA (1<<3)
#define SSL_OPT_FAKEBASICAUTH  (1<<4)
#define SSL_OPT_STRICTREQUIRE  (1<<5)
#define SSL_OPT_OPTRENEGOTIATE (1<<6)
#define SSL_OPT_LEGACYDNFORMAT (1<<7)
\end{lstlisting}

Zurück zu unserem Beispiel.

Das Makro \TT{IS\_SET} prüft auf Anwesenheit von Bits in $a$.
\myindex{x86!\Instructions!AND}

Das Makro \TT{IS\_SET} entspricht dabei dem logischen (\IT{AND})
und gibt 0 zurück, wenn das entsprechende Bit nicht gesetzt ist, oder die
Bitmaske, wenn das Bit gesetzt ist.
Der Operator \IT{if()} wird in \CCpp ausgeführt, wenn der boolesche Ausdruck
nicht null ist (er könnte sogar 123456 sein), weshalb es meistens richtig
funktioniert.


% subsections
\subsubsection{x86}

\myparagraph{MSVC}

Kompilieren wir das Beispiel:

\lstinputlisting[caption=MSVC 2008,style=customasmx86]{patterns/13_arrays/1_simple/simple_msvc.asm}

\myindex{x86!\Instructions!SHL}
Soweit nichts Außergewöhnliches, nur zwei Schleifen: die erste füllt mit Werten auf und die zweite gibt Werte aus.
% TBT
Der Befehl \TT{shl ecx, 1} wird für die Multiplikation mit 2 in \ECX verwendet; mehr dazu unten~\myref{SHR}.

Auf dem Stack werden 80 Bytes für das Array reserviert: 20 Elemente von je 4 Byte.

\clearpage
Untersuchen wir dieses Beispiel in \olly.
\myindex{\olly}

Wir erkennen wie das Array befüllt wird:

jedes Element ist ein 32-Bit-Wort vom Typ \Tint und der Wert ist der Index multipliziert mit 2:

\begin{figure}[H]
\centering
\myincludegraphics{patterns/13_arrays/1_simple/olly.png}
\caption{\olly: nach dem Füllen des Arrays}
\label{fig:array_simple_olly}
\end{figure}
Da sich dieses Array auf dem Stack befindet, finden wir dort alle seine 20 Elemente.

\myparagraph{GCC}

Hier ist was GCC 4.4.1 erzeugt:

\lstinputlisting[caption=GCC 4.4.1,style=customasmx86]{patterns/13_arrays/1_simple/simple_gcc.asm}
Die Variable $a$ ist übrigens vom Typ \IT{int*} (Pointer auf \Tint{})--man kann einen Pointer auf ein Array an eine
andere Funktion übergeben, aber es ist richtiger zu sagen, dass der Pointer auf das erste Element des Arrays übergeben
wird. (Die Adressen der übrigen Elemente werden in bekannter Weise berechnet.)

Wenn man diesen Pointer mittels \IT{a[idx]} indiziert, wird \IT{idx} zum Pointer addiert und das dort abgelegte Element
(auf das der berechnete Pointer zeigt) wird zurückgegeben.

Ein interessantes Beispiel: ein String wie \IT{\q{string}} ist ein Array von Chars und hat den Typ \IT{const
char[]}.

Auch auf diesen Pointer kann ein Index angewendet werden.

Das ist der Grund warum es es möglich ist, Dinge wie \TT{\q{string}[i]} zu schreiben--es handelt sich dabei um einen
korrekten \CCpp Ausdruck!


\input{patterns/14_bitfields/4_popcnt/x64_DE}
\subsubsection{ARM}

\myparagraph{\OptimizingKeilVI (\ThumbMode)}

\lstinputlisting[style=customasmARM]{patterns/04_scanf/1_simple/ARM_IDA.lst}

\myindex{\CLanguageElements!\Pointers}
Damit \scanf Elemente einlesen kann, benötigt die Funktion einen Paramter--einen Pointer vom Typ \Tint.
\Tint hat die Größe 32 Bit, wir benötigen also 4 Byte, um den Wert im Speicher abzulegen, und passt daher genau in ein 32-Bit-Register.
\myindex{IDA!var\_?}
Auf dem Stack wird Platz für die lokalen Variable \GTT{x} reserviert und IDA bezeichnet diese Variable mit \IT{var\_8}. 
Eigentlich ist aber an dieser Stelle gar nicht notwendig, Platz auf dem Stack zu reservieren, da \ac{SP} (\gls{stack pointer} 
bereits auf die Adresse zeigt und auch direkt verwendet werden kann.

Der Wert von \ac{SP} wird also in das \Reg{1} Register kopiert und zusammen mit dem Formatierungsstring an \scanf übergeben.

% TBT here
%\INS{PUSH/POP} instructions behaves differently in ARM than in x86 (it's the other way around).
%They are synonyms to \INS{STM/STMDB/LDM/LDMIA} instructions.
%And \INS{PUSH} instruction first writes a value into the stack, \IT{and then} subtracts \ac{SP} by 4.
%\INS{POP} first adds 4 to \ac{SP}, \IT{and then} reads a value from the stack.
%Hence, after \INS{PUSH}, \ac{SP} points to an unused space in stack.
%It is used by \scanf, and by \printf after.

%\INS{LDMIA} means \IT{Load Multiple Registers Increment address After each transfer}.
%\INS{STMDB} means \IT{Store Multiple Registers Decrement address Before each transfer}.

\myindex{ARM!\Instructions!LDR}
Später wird mithilfe des \INS{LDR} Befehls dieser Wert vom Stack in das \Reg{1} Register verschoben um an \printf übergeben werden zu können.

\myparagraph{ARM64}

\lstinputlisting[caption=\NonOptimizing GCC 4.9.1 ARM64,numbers=left,style=customasmARM]{patterns/04_scanf/1_simple/ARM64_GCC491_O0_DE.s}

Im Stack Frame werden 32 Byte reserviert, was deutlich mehr als benötigt ist. Vielleicht handelt es sich um eine Frage des Aligning (dt. Angleichens) von Speicheradressen.
Der interessanteste Teil ist, im Stack Frame einen Platz für die Variable $x$ zu finden (Zeile 22).
Warum 28? Irgendwie hat der Compiler entschieden die Variable am Ende des Stack Frames anstatt an dessen Beginn abzulegen.
Die Adresse wird an \scanf übergeben; diese Funktion speichert den Userinput an der genannten Adresse im Speicher.
Es handelt sich hier um einen 32-Bit-Wert vom Typ \Tint. 
Der Wert wird in Zeile 27 abgeholt und dann an \printf übergeben.



\subsubsection{MIPS}
% FIXME better start at non-optimizing version?
Die Funktion verwendet eine Menge S-Register, die gesichert werden müssen. Das ist der Grund dafür, dass die Werte im
Funktionsprolog gespeichert und im Funktionsepilog wiederhergestellt werden.

\lstinputlisting[caption=\Optimizing GCC 4.4.5
(IDA),style=customasmMIPS]{patterns/13_arrays/1_simple/MIPS_O3_IDA_DE.lst}
Interessant: es gibt zwei Schleifen und die erste benötigt $i$ nicht; sie benötigt nur $i\cdot 2$ (erhöht um 2 bei
jedem Iterationsschritt) und die Adresse im Speicher (erhöht um 4 bei jedem Iterationsschritt).

Wir sehen hier also zwei Variablen: eine (in \$V0), die jedes Mal um 2 erhöht wird, und eine andere (in\$V1), die um 4
erhöht wird.

Die zweite Schleife ist der Ort, an dem \printf aufgerufen wird und dem Benutzer den Wert von $i$ zurückliefert, es gibt
also eine Variable die in \$S0 inkrementiert wird und eine Speicheradresse in \$S1, die jedes Mal um 4 erhöht wird.

% TBT
Das erinnert uns an die Optimierung von Schleifen, die wir früher betrachtet haben: \myref{loop_iterators}.

Das Ziel der Optimierung ist es, die Multiplikationen loszuwerden.

}
\FR{\subsection{Calcul de l'epsilon de la machine}

L'epsilon de la machine est la plus petite valeur avec laquelle le \ac{FPU} peut
travailler.
Plus il y a de bits alloués pour représenter un nombre en virgule flottante, plus
l'epsilon est petit,
C'est $2^{-23} = 1.19e-07$ pour les \Tfloat et $2^{-52} = 2.22e-16$ pour les \Tdouble.
Voir aussi: \href{http://link.yurichev.com/17367}{l'article de Wikipédia}.%
% TODO recheck values
% I get
% 1.19209e-07
% 2.22045e-16

Il intéressant de voir comme il est facile de calculer l'epsilon de la machine:

\lstinputlisting[style=customc]{patterns/17_unions/epsilon/float.c}

Ce que l'on fait ici est simplement de traiter la partie fractionnaire du nombre
au format IEEE 754 comme un entier et de lui ajouter 1.
Le nombre flottant en résultant est égal à $starting\_value+machine\_epsilon$, donc
il suffit de soustraire $starting\_value$ (en utilisant l'arithmétique flottante)
pour mesurer ce que la différence d'un bit représente dans un nombre flottant simple
précision(\Tfloat).
L' \IT{union} permet ici d'accéder au nombre IEEE 754 comme à un entier normal.
Lui ajouter 1 ajoute en fait 1 au \IT{significande} du nombre, toutefois, inutile
de dire, un débordement est possible, qui ajouterait 1 à l'exposant.

\subsubsection{x86}

\lstinputlisting[caption=\Optimizing MSVC 2010,style=customasmx86]{patterns/17_unions/epsilon/float_MSVC_2010_Ox_FR.asm}

La seconde instruction \INS{FST} est redondante: il n'est pas nécessaire de stocker
la valeur en entrée à la même place (le compilateur a décidé d'allouer la variable
$v$ à la même place dans la pile locale que l'argument en entrée).
Puis elle est incrémentée avec \INS{INC}, puisque c'est une variable entière normale.
Ensuite elle est chargée dans le FPU comme un nombre IEEE 754 32-bit, \INS{FSUBR}
fait le reste du travail et la valeur résultante est stockée dans \TT{ST0}.
La dernière paire d'instructions \INS{FSTP}/\INS{FLD} est redondante, mais le compilateur
n'a pas optimisé le code.

\subsubsection{ARM64}

Étendons notre exemple à 64-bit:

\lstinputlisting[label=machine_epsilon_double_c,style=customc]{patterns/17_unions/epsilon/double.c}

ARM64 n'a pas d'instruction qui peut ajouter un nombre a un D-registre FPU, donc
la valeur en entrée (qui provient du registre x64 \TT{D0}) est d'abord copiée dans
le \ac{GPR}, incrémentée, copiée dans le registre FPU \TT{D1}, et puis la soustraction
est faite.

\lstinputlisting[caption=GCC 4.9 ARM64 \Optimizing,style=customasmARM]{patterns/17_unions/epsilon/double_GCC49_ARM64_O3_FR.s}

Voir aussi cet exemple compilé pour x64 avec instructions SIMD: \myref{machine_epsilon_x64_and_SIMD}.

\subsubsection{MIPS}

\myindex{MIPS!\Instructions!MTC1}

Il y a ici la nouvelle instruction \INS{MTC1} (\q{Move To Coprocessor 1}), elle transfère
simplement des données vers les registres du FPU.

\lstinputlisting[caption=GCC 4.4.5 \Optimizing (IDA),style=customasmMIPS]{patterns/17_unions/epsilon/MIPS_O3_IDA.lst}

\subsubsection{\Conclusion}

Il est difficile de dire si quelqu'un pourrait avoir besoin de cette astuce dans
du code réel, mais comme cela a été mentionné plusieurs fois dans ce livre, cet exemple
est utile pour expliquer le format IEEE 754 et les \IT{union}s en \CCpp.
}
\JPN{\subsection{計算機イプシロンを計算する}

計算機イプシロンは、\ac{FPU}が使用できる最小の値です。 
浮動小数点数に割り当てられるビットが多いほど、計算機イプシロンは小さくなります。
\Tfloat では $2^{-23} = 1.19e-07$ で、 \Tdouble では $2^{-52} = 2.22e-16$ です。
\href{http://link.yurichev.com/17367}{Wikipediaの記事} も参照してください。
% TODO recheck values

興味深いことに、計算機イプシロンを計算するのはとても簡単です。

\lstinputlisting[style=customc]{patterns/17_unions/epsilon/float.c}

ここで行うことは、IEEE 754形式の数の小数部分を整数として扱い、それに1を加えることです。
結果の浮動小数点数は $starting\_value+machine\_epsilon$ に等しいので、
測定するために(浮動小数点演算を使用して)開始値を減算する必要があります。
1ビットが単精度(\Tfloat)にどのように反映されるかを測定します。
\IT{共用体}は、ここでは通常の整数としてIEEE 754形式の数にアクセスする方法として機能します。 
1を加えることは実際には数の\IT{小数}部分に1を加えますが、言うまでもなく、
オーバーフローは可能であり、指数部分に1を加えることになります。

\subsubsection{x86}

\lstinputlisting[caption=\Optimizing MSVC 2010,style=customasmx86]{patterns/17_unions/epsilon/float_MSVC_2010_Ox_JPN.asm}

2番目の\INS{FST}命令は冗長です。入力値を同じ場所に格納する必要はありません
(コンパイラは、ローカルスタックの入力引数と同じ位置に $v$ 変数を割り当てることにしました)。
それは通常の整数の変数なので、\INS{INC}でインクリメントされます。 
その後、32ビットのIEEE 754形式の数としてFPUにロードされ、\INS{FSUBR}が残りのジョブを実行し、
結果の値が\TT{ST0}に格納されます。 
最後の\INS{FSTP}/\INS{FLD}命令ペアは冗長ですが、コンパイラは最適化しませんでした。

\subsubsection{ARM64}

例を64ビットに拡張してみましょう。

\lstinputlisting[label=machine_epsilon_double_c,style=customc]{patterns/17_unions/epsilon/double.c}

ARM64にはFPUのDレジスタに数値を加算する命令がないため、
入力値(\TT{D0}に入力されたもの)が最初に\ac{GPR}にコピーされ、
インクリメントされ、FPUレジスタの\TT{D1}にコピーされてから減算が行われます。

\lstinputlisting[caption=\Optimizing GCC 4.9 ARM64,style=customasmARM]{patterns/17_unions/epsilon/double_GCC49_ARM64_O3_JPN.s}

SIMD命令を使用してx64用にコンパイルされたこの例も参照してください:\myref{machine_epsilon_x64_and_SIMD}

\subsubsection{MIPS}

\myindex{MIPS!\Instructions!MTC1}

ここでの新しい命令は\INS{MTC1}(\q{Move To Coprocessor 1})です。\ac{GPR}からFPUのレジスタへデータを転送するだけです。

\lstinputlisting[caption=\Optimizing GCC 4.4.5 (IDA),style=customasmMIPS]{patterns/17_unions/epsilon/MIPS_O3_IDA.lst}

\subsubsection{\Conclusion}

誰かがこのトリックを実際のコードで必要とするかどうかはわかりにくいですが、
この本で何度も述べたように、この例は
IEEE 754形式と \CCpp の\IT{共用体}を説明するのに役立ちます。
}

\EN{\mysection{FSCALE replacement}
\myindex{x86!\Instructions!FSCALE}

Agner Fog in his \IT{Optimizing subroutines in assembly language / An optimization guide for x86 platforms} work
\footnote{\url{http://www.agner.org/optimize/optimizing_assembly.pdf}} states that \INS{FSCALE} \ac{FPU} instruction
(calculating $2^n$) may be slow on many CPUs, and he offers faster replacement.

Here is my translation of his assembly code to \CCpp:

\lstinputlisting[style=customc]{patterns/17_unions/FSCALE.c}

\INS{FSCALE} instruction may be faster in your environment, but still, it's a good example of \IT{union}'s and the fact
that exponent is stored in $2^n$ form,
so an input $n$ value is shifted to the exponent in IEEE 754 encoded number.
Then exponent is then corrected with addition of 0x3f800000 or 0x3ff0000000000000.

The same can be done without shift using \IT{struct}, but internally, shift operations still occurred.

}
\DE{\mysection{FSCALE Ersatz}
\myindex{x86!\Instructions!FSCALE}
Agner Fog schreibt in seiner Abhandlung \IT{Optimizing subroutines in assembly language / An optimization guide for x86
platforms} \footnote{\url{http://www.agner.org/optimize/optimizing_assembly.pdf}} , dass der Befehl \INS{FSCALE}
\ac{FPU} (der $2^n$ berechnet) auf vielen CPUs langsam ist und bietet einen schnelleren Ersatz an.

Hier ist meine Übersetzung von seinem Assemblercode in \CCpp:

\lstinputlisting[style=customc]{patterns/17_unions/FSCALE.c}
Der Befehl \INS{FSCALE} kann zwar in bestimmten Umgebungen schneller sein, ist aber vor allem ein gutes Beispiel für
\IT{unions} und die Tatsache, dass der Exponent in der Form $2^n$ gespeichert wird, sodass ein Eingabewert $n$ zum
Exponenten nach IEEE 754 Standard verschoben wird.
Der Exponent wird dann durch Addition von 0x3f800000 oder 0x3ff0000000000000 korrigiert.

Das gleiche kann ohne Verschiebung durch ein \IT{struct} erreicht werden, aber intern werden stets Schiebebefehle
verwendet.
}
\FR{\mysection{Remplacement de FSCALE}
\myindex{x86!\Instructions!FSCALE}

Agner Fog dans son travail\footnote{\url{http://www.agner.org/optimize/optimizing_assembly.pdf}}
\IT{Optimizing subroutines in assembly language / An optimization guide for x86 platforms}
indique que l'instruction \ac{FPU} \INS{FSCALE} (qui calcule $2^n$) peut être lente
sur de nombreux CPUs, et propose un remplacement plus rapide.

Voici ma conversion de son code assembleur en \CCpp:

\lstinputlisting[style=customc]{patterns/17_unions/FSCALE.c}

L'instruction \INS{FSCALE} peut être plus rapide dans votre environnement, mais néanmoins,
c'est un bon exemple d'\IT{union} et du fait que l'exposant est stocké sous la forme
$2^n$, donc une valeur $n$ en entrée est décalée à l'exposant dans le nombre encodé
en IEEE 754.
Ensuite, l'exposant est corrigé avec l'ajout de 0x3f800000 ou de 0x3ff0000000000000.

La même chose peut être faite sans décalage utilisant \IT{struct}, mais en interne,
l'opération de décalage aura toujours lieu.

}
\JPN{\mysection{FSCALE replacement}
\myindex{x86!\Instructions!FSCALE}

Agner Fog氏による\IT{Optimizing subroutines in assembly language / An optimization guide for x86 platforms}では
\footnote{\url{http://www.agner.org/optimize/optimizing_assembly.pdf}}、多くのCPUでは\INS{FSCALE} \ac{FPU}命令
($2^n$の計算)が遅くなる可能性があると述べ、より速いものを提案しています。

これが私のアセンブリコードの \CCpp への翻訳です。

\lstinputlisting[style=customc]{patterns/17_unions/FSCALE.c}

\INS{FSCALE}命令はあなたの環境ではより速いかもしれませんが、それでも、それは\IT{共用体}の良い例であり、
指数が$2^n$形式で格納されるという事実です。
そのため、入力された$n$の値はIEEE 754形式で符号化された数の指数にシフトされます。
その後、0x3f800000または0x3ff0000000000000を追加して指数を補正します。

\IT{構造体}を使用してシフトなしで同じことを実行できますが、それでも内部ではシフト操作が発生しました。
}

\subsection{\RU{Быстрое вычисление квадратного корня}\EN{Fast square root calculation}\DE{Schnelle Berechnung der
Quadratwurzel}\JPN{Fast square root calculation}}

\RU{Вот где еще можно на практике применить трактовку типа \Tfloat как целочисленного, это быстрое вычисление квадратного корня.}%
\EN{Another well-known algorithm where \Tfloat is interpreted as integer is fast calculation of square root.}
\DE{Ein anderer bekannter Algorithmus, in dem \Tfloat als \Tint interpretiert wird, ist die schnelle Berechnung einer
Quadratwurzel.}
\FR{Un autre algorithme connu où un \Tfloat est interprété comme un entier est celui
de calcul rapide de racine carrée.}
\JPN{\Tfloat が整数として解釈される別のよく知られたアルゴリズムは平方根の高速計算です。}

\begin{lstlisting}[caption=\DE{Quellcode stammt aus der Wikipedia}\EN{The source code is taken from
Wikipedia}\RU{Исходный код взят из Wikipedia}\FR{Le code source provient de Wikipedia}\JPN{ソースコードはウィキペディアから取りました}:
\url{http://go.yurichev.com/17364},style=customc] /* Assumes that float is in the IEEE 754 single precision floating point format
 * and that int is 32 bits. */
float sqrt_approx(float z)
{
    int val_int = *(int*)&z; /* Same bits, but as an int */
    /*
     * To justify the following code, prove that
     *
     * ((((val_int / 2^m) - b) / 2) + b) * 2^m = ((val_int - 2^m) / 2) + ((b + 1) / 2) * 2^m)
     *
     * where
     *
     * b = exponent bias
     * m = number of mantissa bits
     *
     * .
     */
 
    val_int -= 1 << 23; /* Subtract 2^m. */
    val_int >>= 1; /* Divide by 2. */
    val_int += 1 << 29; /* Add ((b + 1) / 2) * 2^m. */
 
    return *(float*)&val_int; /* Interpret again as float */
}
\end{lstlisting}

\ifdefined\RUSSIAN
В качестве упражнения, вы можете попробовать скомпилировать эту функцию и разобраться, как она работает. \\
\\
Имеется также известный алгоритм быстрого вычисления $\frac{1}{\sqrt{x}}$.
\myindex{Quake III Arena}
Алгоритм стал известным, вероятно потому, что был применен в Quake III Arena.

Описание алгоритма есть в Wikipedia: \url{http://go.yurichev.com/17361}.
\fi % RUSSIAN

\ifdefined\ENGLISH
As an exercise, you can try to compile this function and to understand, how it works. \\
\\
There is also well-known algorithm of fast calculation of $\frac{1}{\sqrt{x}}$.
\myindex{Quake III Arena}
Algorithm became popular, supposedly, because it was used in Quake III Arena.

Algorithm description can be found in Wikipedia: \url{http://go.yurichev.com/17360}.
\fi % ENGLISH

\ifdefined\GERMAN
Versuchen Sie als Übung, diese Funktion zu kompilieren und zu verstehen wie sie funktioniert.\\\\
Es gibt auch einen bekannten Algorithmus zur schnellen Berechnung von $\frac{1}{\sqrt{x}}$.
\myindex{Quake III Arena}
Der Algorithmus wurde vermutlich so populär, weil er in Quake III Arena verwendet wurde.
Eine Beschreibung des Algorithmus' findet man bei Wikipedia: \url{http://go.yurichev.com/17360}.
\fi % GERMAN

\ifdefined\FRENCH
À titre d'exercice, vous pouvez essayez de compiler cette fonction et de comprendre
comme elle fonctionne.\\
\\
C'est un algorithme connu de calcul rapide de $\frac{1}{\sqrt{x}}$.
\myindex{Quake III Arena}
L'algorithme devînt connu, supposément, car il a été utilisé dans Quake III Arena.

La description de l'algorithme peut être trouvée sur Wikipédia: \url{http://go.yurichev.com/17360}.
\fi % FRENCH

\ifdefined\JAPANESE
演習として、この関数をコンパイルして、その機能を理解することを試みることができます。\\
\\
$\frac{1}{\sqrt{x}}$の高速計算のよく知られたアルゴリズムもあります。
\myindex{Quake III Arena}
Quake III Arenaで使用されていたため、アルゴリズムが普及したと考えられます。

アルゴリズムの説明はWikipediaにあります:\url{http://go.yurichev.com/17360}
\fi % JAPANESE
