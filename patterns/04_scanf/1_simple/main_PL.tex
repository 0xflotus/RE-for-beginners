\subsection{Prosty przykład}

\lstinputlisting[style=customc]{patterns/04_scanf/1_simple/ex1.c}

Teraz używanie \scanf do interakcji z użytkownikiem w programach nie jest zbyt popularne, jednak mimo wszystko funkcja ta jest dobrym przykładem użycia wskaźnika na zmienną typu  \Tint.

\subsubsection{O wskaźnikach}
\myindex{\CLanguageElements!\Pointers}

Wskaźniki są jedną z podstawowych koncepcji informatycznych. Często tworząc dużą tablicę, strukturę albo objekt jako argument jakiejś funkcji, przekazywanie wskaźnika do argumentu jest mniej pamięciożerne niż gdyby przekazać całą tablicę/strukturę, np. kiedy chcesz wypisać tekst w konsoli, najprościej będzie wskazać jego adres w pamięci.

W dodatku jeśli wywoływana funkcja potrzebuje zmodyfikować cokolwiek w dużej tablicy lub strukturze danych przekazanej jako parametr i zwrócić potem tą tablicę/strukturę, kopiowanie tylu danych byłoby prawie absurdalne. Dlatego najprościej będzie przekazać adres tej tablicy/struktury do wywoływanej funkcji i wtedy zmodyfikować to co wymagało modyfikacji.

Wskaźnik w \CCpp---jest adresem pewnego miejsca w pamięci.

\myindex{x86-64}
W x86, adresy są reprezentowane przy pomocy 32-bitowych liczb (czyli 4 bajtowych), a w x86-64 jako liczby 64-bitowe (czyli 8 bajtowe). Przy okazji jest to powód dlaczego niektórych ludzi oburza przeskok na x86_64- wszystkie wskaźniki w architekturze x64 wymagają dwa razy więcej miejsca, włączając pamięć cache, co jest bardzo "kosztownym" zużyciem pamięci.

% TODO ... а делать разные версии memcpy для разных типов - абсурд
\myindex{\CStandardLibrary!memcpy()}
Możliwa jest praca z tylko nietypowanymi wskaźnikami, wymagająca nieco wysiłku, np. funkcja z biblioteki standardowej C-\TT{memcpy()}, która kopiuje blok z jednej lokalizacji w pamięci do innej, jako argumenty przyjmuje 2 wskaźniki typu \TT{void*}, co umożliwia kopiowanie dowolnych typów danych. Typy danych nie są istotne, znaczenie mają tylko rozmiary bloków pamięci.

Wskaźniki są także często używane kiedy funkcja potrzebuje zwrócić więcej niż jedną wartość (wrócę do tego później
~(\myref{label_pointers})
).

Funkcja \IT{scanf()} jest takim przypadkiem.

Oprócz tego faktu, funkcja scanf() wymaga podania w argumencie ile wartości ma wczytać, żeby później móc je zwrócić.

W \CCpp typ wskaźnika jest potrzebny tylko do sprawdzania typów podczas kompilacji.

Wewnątrz skompilowanego kodu nie ma żadnej informacji jakiego typu są wskaźniki.
% TODO это сильно затрудняет декомпиляцию

\EN{\subsection{x86}

\subsubsection{MSVC}

Here is what we get after compilation (MSVC 2010 Express):

\lstinputlisting[label=src:passing_arguments_ex_MSVC_cdecl,caption=MSVC 2010 Express,style=customasmx86]{patterns/05_passing_arguments/msvc_EN.asm}

\myindex{x86!\Registers!EBP}

What we see is that the \main function pushes 3 numbers onto the stack and calls \TT{f(int,int,int).} 

Argument access inside \ttf is organized with the help of macros like:\\
\TT{\_a\$ = 8}, 
in the same way as local variables, but with positive offsets (addressed with \IT{plus}).
So, we are addressing the \IT{outer} side of the \gls{stack frame} by adding the \TT{\_a\$} macro to the value in the \EBP register.

\myindex{x86!\Instructions!IMUL}
\myindex{x86!\Instructions!ADD}

Then the value of $a$ is stored into \EAX. After \IMUL instruction execution, the value in \EAX is 
a \gls{product} of the value in \EAX and the content of \TT{\_b}.

After that, \ADD adds the value in \TT{\_c} to \EAX.

The value in \EAX does not need to be moved: it is already where it must be.
On returning to \gls{caller}, it takes the \EAX value and uses it as an argument to \printf.

\clearpage
\myparagraph{\Optimizing MSVC + \olly}
\myindex{\olly}

We can try this (optimized) example in \olly.  Here is the first iteration:

\begin{figure}[H]
\centering
\myincludegraphics{patterns/10_strings/1_strlen/olly1.png}
\caption{\olly: first iteration start}
\label{fig:strlen_olly_1}
\end{figure}

We see that \olly found a loop and, for convenience, \IT{wrapped} its instructions in brackets.
By clicking the right button on \EAX, we can choose 
\q{Follow in Dump} and the memory window scrolls to the right place.
Here we can see the string \q{hello!} in memory.
There is at least
one zero byte after it and then random garbage.

If \olly sees a register with a valid address in it, that points to some string, 
it is shown as a string.

\clearpage
Let's press F8 (\stepover) a few times, to get to the start of the body of the loop:

\begin{figure}[H]
\centering
\myincludegraphics{patterns/10_strings/1_strlen/olly2.png}
\caption{\olly: second iteration start}
\label{fig:strlen_olly_2}
\end{figure}

We see that \EAX contains the address of the second character in the string.

\clearpage

We have to press F8 enough number of times in order to escape from the loop:

\begin{figure}[H]
\centering
\myincludegraphics{patterns/10_strings/1_strlen/olly3.png}
\caption{\olly: pointers difference to be calculated now}
\label{fig:strlen_olly_3}
\end{figure}

We see that \EAX now contains the address of zero byte that's right after the string plus 1 (because INC EAX was executed regardless of whether
we exit from the loop or not).
Meanwhile, \EDX hasn't changed,
so it still pointing to the start of the string.

The difference between these two addresses is being calculated now.

\clearpage
The \SUB instruction just got executed:

\begin{figure}[H]
\centering
\myincludegraphics{patterns/10_strings/1_strlen/olly4.png}
\caption{\olly: \EAX to be decremented now}
\label{fig:strlen_olly_4}
\end{figure}

The difference of pointers is in the \EAX register now---7.
Indeed, the length of the \q{hello!} string is 6, 
but with the zero byte included---7.
But \TT{strlen()} must return the number of non-zero characters in the string.
So the decrement executes and then the function returns.


\subsubsection{GCC}

Let's compile the same in GCC 4.4.1 and see the results in \IDA:

\lstinputlisting[caption=GCC 4.4.1,style=customasmx86]{patterns/05_passing_arguments/gcc_EN.asm}

The result is almost the same with some minor differences discussed earlier.

The \gls{stack pointer} is not set back after the two function calls(f and printf), 
because the penultimate \TT{LEAVE} (\myref{x86_ins:LEAVE}) 
instruction takes care of this at the end.
}
\RU{\subsection{x86}

\subsubsection{MSVC}

Рассмотрим пример, скомпилированный в (MSVC 2010 Express):

\lstinputlisting[label=src:passing_arguments_ex_MSVC_cdecl,caption=MSVC 2010 Express,style=customasmx86]{patterns/05_passing_arguments/msvc_RU.asm}

\myindex{x86!\Registers!EBP}
Итак, здесь видно: в функции \main заталкиваются три числа в стек и вызывается функция \TT{f(int,int,int)}.
 
Внутри \ttf доступ к аргументам, также как и к локальным переменным, происходит через макросы: 
\TT{\_a\$ = 8}, но разница в том, что эти смещения со знаком \IT{плюс}, 
таким образом если прибавить макрос \TT{\_a\$} к указателю на \EBP, то адресуется \IT{внешняя} 
часть \glslink{stack frame}{фрейма} стека относительно \EBP.

\myindex{x86!\Instructions!IMUL}
\myindex{x86!\Instructions!ADD}
Далее всё более-менее просто: значение $a$ помещается в \EAX. 
Далее \EAX умножается при помощи инструкции \IMUL на то, что лежит в \TT{\_b}, 
и в \EAX остается \glslink{product}{произведение} этих двух значений.

Далее к регистру \EAX прибавляется то, что лежит в \TT{\_c}.

Значение из \EAX никуда не нужно перекладывать, оно уже лежит где надо. 
Возвращаем управление вызывающей функции~--- она возьмет значение из \EAX и отправит его в \printf.

\clearpage
\myparagraph{\Optimizing MSVC + \olly}
\myindex{\olly}

Можем попробовать этот (соптимизированный) пример в \olly.  Вот самая первая итерация:

\begin{figure}[H]
\centering
\myincludegraphics{patterns/10_strings/1_strlen/olly1.png}
\caption{\olly: начало первой итерации}
\label{fig:strlen_olly_1}
\end{figure}

Видно, что \olly обнаружил цикл и, для удобства, \IT{свернул} инструкции тела цикла в скобке.

Нажав правой кнопкой на \EAX, можно выбрать \q{Follow in Dump} 
и позиция в окне памяти будет как раз там, где надо.

Здесь мы видим в памяти строку \q{hello!}.
После неё имеется как минимум 1 нулевой байт, затем случайный мусор.
Если \olly видит, что в регистре содержится адрес какой-то строки, он показывает эту строку.

\clearpage
Нажмем F8 (\stepover) столько раз, чтобы текущий адрес снова был в начале тела цикла:

\begin{figure}[H]
\centering
\myincludegraphics{patterns/10_strings/1_strlen/olly2.png}
\caption{\olly: начало второй итерации}
\label{fig:strlen_olly_2}
\end{figure}

Видно, что \EAX уже содержит адрес второго символа в строке.

\clearpage
Будем нажимать F8 достаточное количество раз, чтобы выйти из цикла:

\begin{figure}[H]
\centering
\myincludegraphics{patterns/10_strings/1_strlen/olly3.png}
\caption{\olly: сейчас будет вычисление разницы указателей}
\label{fig:strlen_olly_3}
\end{figure}

Увидим, что \EAX теперь содержит адрес нулевого байта, следующего сразу за строкой плюс 1 (потому что INC EAX исполнился вне зависимости
от того, выходим мы из цикла, или нет).

А \EDX так и не менялся~--- он всё ещё указывает на начало строки.
Здесь сейчас будет вычисляться разница между этими двумя адресами.

\clearpage
Инструкция \SUB исполнилась:

\begin{figure}[H]
\centering
\myincludegraphics{patterns/10_strings/1_strlen/olly4.png}
\caption{\olly: сейчас будет декремент \EAX}
\label{fig:strlen_olly_4}
\end{figure}

Разница указателей сейчас в регистре \EAX~--- 7.

Действительно, длина строки \q{hello!}~--- 6, 
но вместе с нулевым байтом --- 7.
Но \TT{strlen()} должна возвращать количество ненулевых символов в строке.
Так что сейчас будет исполняться декремент и выход из функции.



\subsubsection{GCC}

Скомпилируем то же в GCC 4.4.1 и посмотрим результат в \IDA:

\lstinputlisting[caption=GCC 4.4.1,style=customasmx86]{patterns/05_passing_arguments/gcc_RU.asm}

Практически то же самое, если не считать мелких отличий описанных ранее.

После вызова обоих функций \glslink{stack pointer}{указатель стека} не возвращается назад, 
потому что предпоследняя инструкция \TT{LEAVE} (\myref{x86_ins:LEAVE}) делает это за один раз, в конце исполнения.

}
\PTBR{\input{patterns/04_scanf/1_simple/x86_PTBR}}
\ITA{\input{patterns/04_scanf/1_simple/x86_ITA}}
\DE{\subsubsection{x86}

\myparagraph{MSVC}

Kompilieren wir das Beispiel:

\lstinputlisting[caption=MSVC 2008,style=customasmx86]{patterns/13_arrays/1_simple/simple_msvc.asm}

\myindex{x86!\Instructions!SHL}
Soweit nichts Außergewöhnliches, nur zwei Schleifen: die erste füllt mit Werten auf und die zweite gibt Werte aus.
% TBT
Der Befehl \TT{shl ecx, 1} wird für die Multiplikation mit 2 in \ECX verwendet; mehr dazu unten~\myref{SHR}.

Auf dem Stack werden 80 Bytes für das Array reserviert: 20 Elemente von je 4 Byte.

\clearpage
Untersuchen wir dieses Beispiel in \olly.
\myindex{\olly}

Wir erkennen wie das Array befüllt wird:

jedes Element ist ein 32-Bit-Wort vom Typ \Tint und der Wert ist der Index multipliziert mit 2:

\begin{figure}[H]
\centering
\myincludegraphics{patterns/13_arrays/1_simple/olly.png}
\caption{\olly: nach dem Füllen des Arrays}
\label{fig:array_simple_olly}
\end{figure}
Da sich dieses Array auf dem Stack befindet, finden wir dort alle seine 20 Elemente.

\myparagraph{GCC}

Hier ist was GCC 4.4.1 erzeugt:

\lstinputlisting[caption=GCC 4.4.1,style=customasmx86]{patterns/13_arrays/1_simple/simple_gcc.asm}
Die Variable $a$ ist übrigens vom Typ \IT{int*} (Pointer auf \Tint{})--man kann einen Pointer auf ein Array an eine
andere Funktion übergeben, aber es ist richtiger zu sagen, dass der Pointer auf das erste Element des Arrays übergeben
wird. (Die Adressen der übrigen Elemente werden in bekannter Weise berechnet.)

Wenn man diesen Pointer mittels \IT{a[idx]} indiziert, wird \IT{idx} zum Pointer addiert und das dort abgelegte Element
(auf das der berechnete Pointer zeigt) wird zurückgegeben.

Ein interessantes Beispiel: ein String wie \IT{\q{string}} ist ein Array von Chars und hat den Typ \IT{const
char[]}.

Auch auf diesen Pointer kann ein Index angewendet werden.

Das ist der Grund warum es es möglich ist, Dinge wie \TT{\q{string}[i]} zu schreiben--es handelt sich dabei um einen
korrekten \CCpp Ausdruck!

}
\FR{\subsubsection{x86}

Regardons ce que nous obtenons avec MSVC 2010:

\lstinputlisting[caption=MSVC 2010,style=customasmx86]{patterns/12_FPU/2_passing_floats/MSVC_FR.asm}

\myindex{x86!\Instructions!FLD}
\myindex{x86!\Instructions!FSTP}

\FLD et \FSTP déplacent des variables entre le segment de données et la pile du FPU.
\GTT{pow()}\footnote{une fonction C standard, qui élève un nombre à la puissance
donnée (puissance)} prend deux valeurs depuis la pile et renvoie son résultat dans
le registre \ST{0}.
\printf prend 8 octets de la pile locale et les interprète comme des variables de
type \Tdouble.

À propos, une paire d'instructions \MOV pourrait être utilisée ici pour déplacer
les valeurs depuis la mémoire vers la pile, car les valeurs en mémoire sont stockées
au format IEEE 754, et pow() les prend aussi dans ce format, donc aucune conversion
n'est nécessaire.
C'est fait ainsi dans l'exemple suivant, pour ARM: \myref{FPU_passing_floats_ARM}.

}
\PL{\subsubsection{x86}

\myparagraph{MSVC}

Tutaj znajduje się wynik kompilacji programu w MSVC 2010:

\lstinputlisting[style=customasmx86]{patterns/04_scanf/1_simple/ex1_MSVC_EN.asm}

\TT{x} jest zmienną lokalną.

Według standardu \CCpp zmienna lokalna może być widoczna tylko w konkretnej funkcji. Tradycyjnie zmienne lokalne są przechowywane na stosie. Prawdopodnie są inne moliwości przechowywania tych zmiennych, ale tak akurat jest w x86.

\myindex{x86!\Instructions!PUSH}
Zadaniem instrukcji rozpoczynającej funkcję, \TT{PUSH ECX}, nie jest zapisanie stanu \ECX (można zauważyć brak odpowiadającej instrukcji POP ECX na końcu funkcji).

Tak naprawdę instrukcja ta alokuje 4 bajty na stosie do przechowania zmiennej x.

\label{stack_frame}
\myindex{\Stack!Stack frame}
\myindex{x86!\Registers!EBP}
Dostęp do \TT{x} odbywa się z asystującym makrem \TT{\_x\$} (równym -4) i rejestrem \EBP rwskazującym bieżącą ramkę.

We fragmencie wykonującym funkcje, \EBP wskazuje bieżącą \gls{stack frame}
umożliwiając dostęp do zmiennych lokalnych i argumentów funkcji poprzez \TT{EBP+offset}.

\myindex{x86!\Registers!ESP}
Możliwe jest także użycie \ESP w takim samym celu, le nie jest to zbyt wygodne, ponieważ wartość tego rejestru często się zmienia.
Wartość \EBP może być postrzegana jako \IT{frozen state} wartości w \ESP z początku wykonania funkcji.

% FIXME1 это уже было в 02_stack?
Tutaj znajduje się typowa ramka stosu w układzie środowiska 32-bitowego:

\begin{center}
\begin{tabular}{ | l | l | }
\hline
\dots & \dots \\
\hline
EBP-8 & zmienna lokalna \#2, \MarkedInIDAAs{} \TT{var\_8} \\
\hline
EBP-4 & zmienna lokalna \#1, \MarkedInIDAAs{} \TT{var\_4} \\
\hline
EBP & zapisana wartość \EBP \\
\hline
EBP+4 & adres powrotu \\
\hline
EBP+8 & \argument \#1, \MarkedInIDAAs{} \TT{arg\_0} \\
\hline
EBP+0xC & \argument \#2, \MarkedInIDAAs{} \TT{arg\_4} \\
\hline
EBP+0x10 & \argument \#3, \MarkedInIDAAs{} \TT{arg\_8} \\
\hline
\dots & \dots \\
\hline
\end{tabular}
\end{center}

Funkcja \scanf w naszym przykładzie ma dwa argumenty.

Pierwszy jest wskaźnikiem na string \TT{\%d} a drugi jest adresem zmiennej \TT{x}.

\myindex{x86!\Instructions!LEA}
Na początku adres zmiennej \TT{x} jest ładowany do rejestru \EAX przy pomocy instrukcji \\
\TT{lea eax, DWORD PTR \_x\$[ebp]}.

\LEA oznacza \IT{load effective address} i jest często używana do formowania adresów ~(\myref{sec:LEA}).

Można powiedzieć, że w tym przypadku \LEA po prostu umieszcza sumę rejestru \EBP i makra \TT{\_x\$} w rejestrze \EAX.

To jest to samo co \INS{lea eax, [ebp-4]}.

Więc od rejestru \EBP jest odejmowane 4 i wynik zostaje umieszczony w rejestrze \EAX.
Następnie wartość rejestru \EAX jest odkładana na stosie i funkcja \scanf zostaje wywołana.

\printf wywołuje się z pierwszym argumentem- wskaźnikiem na string:
\TT{You entered \%d...\textbackslash{}n}.

Drugi argument jest przygotowywany za pomocą: \TT{mov ecx, [ebp-4]}.
Instrukcja kopiuje zmienną \TT{x} (nie jej adres) do rejestru \ECX.

Następnie wartość z \ECX jest odkładana na stos, a na koniec zostaje wywołana funkcja  \printf.

\input{patterns/04_scanf/1_simple/olly.tex}

\myparagraph{GCC}

Tak wygląda skompilowany kod w GCC 4.4.1 w systemie Linux:

\lstinputlisting[style=customasmx86]{patterns/04_scanf/1_simple/ex1_GCC.asm}

\myindex{puts() instead of printf()}
GCC zamienia wywołanie funkcji \printf na wywołanie funkcji \puts. Powód tego został wyjaśniony w ~(\myref{puts}).

% TODO: rewrite
%\RU{Почему \scanf переименовали в \TT{\_\_\_isoc99\_scanf}, я честно говоря, пока не знаю.}
%\EN{Why \scanf is renamed to \TT{\_\_\_isoc99\_scanf}, I do not know yet.}
% 
% Apparently it has to do with the ISO c99 standard compliance. By default GCC allows specifying a standard to adhere to.
% For example if you compile with -std=c89 the outputted assmebly file will contain scanf and not __isoc99__scanf. I guess current GCC version adhares to c99 by default.
% According to my understanding the two implementations differ in the set of suported modifyers (See printf man page)

Jak w przykładzie MSVC---argumenty funkcji są umieszczane na stosie przy użyciu instrukcji \MOV.

\myparagraph{By the way}

Ten prosty przykład pokazuje jak faktycznie kompilatory tłumaczą
listy wyrażeń w \CCpp-block na sekwencyjne listy instrukcji.
Nie ma nic pomiędzy wyrażeniami w \CCpp a wynikowym kodem maszynowym.
}
\JPN{\subsubsection{x86}

(MSVC 2010)で見てみましょう

\lstinputlisting[caption=MSVC 2010,style=customasmx86]{patterns/12_FPU/2_passing_floats/MSVC_JPN.asm}

\myindex{x86!\Instructions!FLD}
\myindex{x86!\Instructions!FSTP}

\FLD および \FSTP は、データセグメントとFPUスタックとの間の変数を移動します。
\GTT{pow()} \footnote{標準的なC関数であり、与えられたべき乗(指数関数)}はスタックから両方の値をとり、
その結果を\ST{0}レジスタに返します。  \printf はローカルスタックから8バイトを取り出し、double型の変数として解釈します。

ちなみに、メモリ内の値はIEEE 754形式で格納され、pow()もこの形式で格納されているため、
値をメモリからスタックに移動するための一対の \MOV 命令を使用でき、変換は不要です。
これはARMのための次の例で行われます:\myref{FPU_passing_floats_ARM}
}

\input{patterns/04_scanf/1_simple/x64}
\input{patterns/04_scanf/1_simple/ARM}
\mysection{\oracle}
\label{oracle}

% sections
\EN{\input{examples/oracle/1_version_EN}}\RU{\input{examples/oracle/1_version_RU}}
\EN{\input{examples/oracle/2_ksmlru_EN}}\RU{\input{examples/oracle/2_ksmlru_RU}}
\EN{\input{examples/oracle/3_timer_EN}}\RU{\input{examples/oracle/3_timer_RU}}


