\subsubsection{ARM}

\myparagraph{\OptimizingKeilVI (\ThumbMode)}

\lstinputlisting[style=customasmARM]{patterns/04_scanf/1_simple/ARM_IDA.lst}

\myindex{\CLanguageElements!\Pointers}

\scanf がitemを読み込むためには、 \Tint へのparameter.pointerが必要です。 
\Tint は32ビットなので、メモリのどこかに格納するには4バイトが必要で、32ビットのレジスタに正確に収まります。 
\myindex{IDA!var\_?}
ローカル変数\GTT{x}の場所がスタックに割り当てられ、 
\IDA の名前は\IT{var\_8}です。 ただし、\ac{SP}(\gls{stack pointer})がすでにその領域を指しているため、その領域を直接割り当てることはできません。 

% TBT here
%\INS{PUSH/POP} instructions behaves differently in ARM than in x86 (it's the other way around).
%They are synonyms to \INS{STM/STMDB/LDM/LDMIA} instructions.
%And \INS{PUSH} instruction first writes a value into the stack, \IT{and then} subtracts \ac{SP} by 4.
%\INS{POP} first adds 4 to \ac{SP}, \IT{and then} reads a value from the stack.
%Hence, after \INS{PUSH}, \ac{SP} points to an unused space in stack.
%It is used by \scanf, and by \printf after.

%\INS{LDMIA} means \IT{Load Multiple Registers Increment address After each transfer}.
%\INS{STMDB} means \IT{Store Multiple Registers Decrement address Before each transfer}.

したがって、\ac{SP}の値は\Reg{1}レジスタにコピーされ、フォーマット文字列とともに \scanf に渡されます。 
その後、\INS{LDR}命令の助けを借りて、この値はスタックから\Reg{1}レジスタに移動され、 \printf に渡されます。

\myparagraph{ARM64}

\lstinputlisting[caption=\NonOptimizing GCC 4.9.1 ARM64,numbers=left,style=customasmARM]{patterns/04_scanf/1_simple/ARM64_GCC491_O0_JPN.s}

スタックフレームには32バイトが割り当てられており、必要なサイズよりも大きくなっています。 たぶんメモリのアラインメントの問題でしょうか? 
最も興味深いのはスタックフレーム内の$x$変数のためのスペースを見つけることです(22行目)。 
なぜ28なのでしょう? 何らかの理由で、コンパイラは、この変数をスタックフレームの最後に置きます。 
アドレスは \scanf に渡され、\scanf はユーザ入力値をそのアドレスのメモリに格納するだけです。 
これは \Tint 型の32ビット値です。 
値は27行目から取得され、 \printf に渡されます。
