\subsection{Compter les bits mis à 1}

Voici un exemple simple d'une fonction qui compte le nombre de bits mis à 1 dans
la valeur en entrée.

Cette opération est aussi appelée \q{population count}\footnote{les CPUs x86 modernes
(qui supportent SSE4) ont même une instruction POPCNT pour cela}.

\lstinputlisting[style=customc]{patterns/14_bitfields/4_popcnt/shifts.c}

Dans cette boucle, la variable d'itération $i$ prend les valeurs de 0 à 31, donc
la déclaration $1 \ll i$ prend les valeurs de 1 à \TT{0x80000000}.
Pour décrire cette opération en langage naturel, nous dirions \IT{décaler 1 par n bits à gauche}.
En d'autres mots, la déclaration $1 \ll i$ produit consécutivement toutes les positions
possible pour un bit dans un nombre de 32-bit.
Le bit libéré à droite est toujours à 0.

\label{2n_numbers_table}
Voici une table de tous les $1 \ll i$ possible
for $i=0 \ldots 31$:

\small
\begin{center}
\begin{tabular}{ | l | l | l | l | }
\hline
\HeaderColor \CCpp expression & 
\HeaderColor Puissance de deux & 
\HeaderColor Forme décimale & 
\HeaderColor Forme hexadécimale \\
\hline
$1 \ll 0$ & $2^{0}$ & 1 & 1 \\
\hline
$1 \ll 1$ & $2^{1}$ & 2 & 2 \\
\hline
$1 \ll 2$ & $2^{2}$ & 4 & 4 \\
\hline
$1 \ll 3$ & $2^{3}$ & 8 & 8 \\
\hline
$1 \ll 4$ & $2^{4}$ & 16 & 0x10 \\
\hline
$1 \ll 5$ & $2^{5}$ & 32 & 0x20 \\
\hline
$1 \ll 6$ & $2^{6}$ & 64 & 0x40 \\
\hline
$1 \ll 7$ & $2^{7}$ & 128 & 0x80 \\
\hline
$1 \ll 8$ & $2^{8}$ & 256 & 0x100 \\
\hline
$1 \ll 9$ & $2^{9}$ & 512 & 0x200 \\
\hline
$1 \ll 10$ & $2^{10}$ & 1024 & 0x400 \\
\hline
$1 \ll 11$ & $2^{11}$ & 2048 & 0x800 \\
\hline
$1 \ll 12$ & $2^{12}$ & 4096 & 0x1000 \\
\hline
$1 \ll 13$ & $2^{13}$ & 8192 & 0x2000 \\
\hline
$1 \ll 14$ & $2^{14}$ & 16384 & 0x4000 \\
\hline
$1 \ll 15$ & $2^{15}$ & 32768 & 0x8000 \\
\hline
$1 \ll 16$ & $2^{16}$ & 65536 & 0x10000 \\
\hline
$1 \ll 17$ & $2^{17}$ & 131072 & 0x20000 \\
\hline
$1 \ll 18$ & $2^{18}$ & 262144 & 0x40000 \\
\hline
$1 \ll 19$ & $2^{19}$ & 524288 & 0x80000 \\
\hline
$1 \ll 20$ & $2^{20}$ & 1048576 & 0x100000 \\
\hline
$1 \ll 21$ & $2^{21}$ & 2097152 & 0x200000 \\
\hline
$1 \ll 22$ & $2^{22}$ & 4194304 & 0x400000 \\
\hline
$1 \ll 23$ & $2^{23}$ & 8388608 & 0x800000 \\
\hline
$1 \ll 24$ & $2^{24}$ & 16777216 & 0x1000000 \\
\hline
$1 \ll 25$ & $2^{25}$ & 33554432 & 0x2000000 \\
\hline
$1 \ll 26$ & $2^{26}$ & 67108864 & 0x4000000 \\
\hline
$1 \ll 27$ & $2^{27}$ & 134217728 & 0x8000000 \\
\hline
$1 \ll 28$ & $2^{28}$ & 268435456 & 0x10000000 \\
\hline
$1 \ll 29$ & $2^{29}$ & 536870912 & 0x20000000 \\
\hline
$1 \ll 30$ & $2^{30}$ & 1073741824 & 0x40000000 \\
\hline
$1 \ll 31$ & $2^{31}$ & 2147483648 & 0x80000000 \\
\hline
\end{tabular}
\end{center}
\normalsize

Ces constantes (masques de bit) apparaissent très souvent le code et un rétro-ingénieur
pratiquant doit pouvoir les repérer rapidement.

Les nombres décimaux avant 65536 et les hexadécimaux sont faciles à mémoriser.
Tandis que les nombres décimaux après 65536 ne valent probablement pas la peine de
l'être.

Ces constantes sont utilisées très souvent pour mapper des flags sur des bits spécifiques.
Par exemple, voici un extrait de \TT{ssl\_private.h} du code source d'Apache 2.4.6:

\begin{lstlisting}[style=customc]
/**
 * Define the SSL options
 */
#define SSL_OPT_NONE           (0)
#define SSL_OPT_RELSET         (1<<0)
#define SSL_OPT_STDENVVARS     (1<<1)
#define SSL_OPT_EXPORTCERTDATA (1<<3)
#define SSL_OPT_FAKEBASICAUTH  (1<<4)
#define SSL_OPT_STRICTREQUIRE  (1<<5)
#define SSL_OPT_OPTRENEGOTIATE (1<<6)
#define SSL_OPT_LEGACYDNFORMAT (1<<7)
\end{lstlisting}

Revenons à notre exemple.

La macro \TT{IS\_SET} teste la présence d'un bit dans $a$.
\myindex{x86!\Instructions!AND}

La macro \TT{IS\_SET} est en fait l'opération logique AND (\IT{AND}) et elle renvoie
0 si le bit testé est absent (à 0), ou le masque de bit, si le bit est présent (à 1).
L'opérateur \IT{if()} en \CCpp exécute son code si l'expression n'est pas zéro,
cela peut même être 123456, c'est pourquoi il fonctionne toujours correctement.

% subsections
\subsubsection{x86}

Regardons ce que nous obtenons avec MSVC 2010:

\lstinputlisting[caption=MSVC 2010,style=customasmx86]{patterns/12_FPU/2_passing_floats/MSVC_FR.asm}

\myindex{x86!\Instructions!FLD}
\myindex{x86!\Instructions!FSTP}

\FLD et \FSTP déplacent des variables entre le segment de données et la pile du FPU.
\GTT{pow()}\footnote{une fonction C standard, qui élève un nombre à la puissance
donnée (puissance)} prend deux valeurs depuis la pile et renvoie son résultat dans
le registre \ST{0}.
\printf prend 8 octets de la pile locale et les interprète comme des variables de
type \Tdouble.

À propos, une paire d'instructions \MOV pourrait être utilisée ici pour déplacer
les valeurs depuis la mémoire vers la pile, car les valeurs en mémoire sont stockées
au format IEEE 754, et pow() les prend aussi dans ce format, donc aucune conversion
n'est nécessaire.
C'est fait ainsi dans l'exemple suivant, pour ARM: \myref{FPU_passing_floats_ARM}.


\subsection{x64: MSVC 2013 \Optimizing}

\lstinputlisting[caption=MSVC 2013 x64 \Optimizing,style=customasmx86]{\CURPATH/MSVC2013_x64_Ox_FR.asm}

Tout d'abord, MSVC a inliné le code la fonction \strlen{}, car il en a conclus que
ceci était plus rapide que le \strlen{} habituel + le coût de l'appel et du retour.
Ceci est appelé de l'inlining: \myref{inline_code}.

\myindex{x86!\Instructions!OR}
\myindex{\CStandardLibrary!strlen()}
\label{using_OR_instead_of_MOV}
La première instruction de \strlen{} mis en ligne est\\
\TT{OR RAX, 0xFFFFFFFFFFFFFFFF}. 

MSVC utilise souvent \TT{OR} au lieu de \TT{MOV RAX, 0xFFFFFFFFFFFFFFFF}, car l'opcode
résultant est plus court.

Et bien sûr, c'est équivalent: tous les bits sont mis à 1, et un nombre avec tous
les bits mis vaut $-1$ en complément à 2:\myref{sec:signednumbers}.

On peut se demander pourquoi le nombre $-1$ est utilisé dans \strlen{}.
À des fins d'optimisation, bien sûr.
Voici le code que MSVC a généré:

\lstinputlisting[caption=Inlined \strlen{} by MSVC 2013 x64,style=customasmx86]{\CURPATH/strlen_MSVC_FR.asm}

Essayez d'écrite plus court si vous voulez initialiser le compteur à 0!
OK, essayons:

\lstinputlisting[caption=Our version of \strlen{},style=customasmx86]{\CURPATH/my_strlen_FR.asm}

Nous avons échoué. Nous devons utilisé une instruction \INS{JMP} additionnelle!

Donc, ce que le compilateur de MSVC 2013 a fait, c'est de déplacer l'instruction
\TT{INC} avant le chargement du caractère courant.

Si le premier caractère est 0, c'est OK, \RAX contient 0 à ce moment, donc la longueur
de la chaîne est 0.

Le reste de cette fonction semble facile à comprendre.

\subsection{x64: GCC 4.9.1 \NonOptimizing}

\lstinputlisting[style=customasmx86]{\CURPATH/GCC491_x64_O0_FR.asm}

Les commentaires ont été ajoutés par l'auteur du livre.

Après l'exécution de \strlen{}, le contrôle est passé au label L2, et ici deux clauses
sont vérifiées, l'une après l'autre.

\myindex{\CLanguageElements!Short-circuit}
La seconde ne sera jamais vérifiée, si la première (\IT{str\_len==0}) est fausse
(ceci est un \q{short-circuit} (court-circuit)).

Maintenant regardons la forme courte de cette fonction:

\begin{itemize}
\item Première partie de for() (appel à \strlen{})
\item goto L2
\item L5: corps de for(). sauter à la fin, si besoin
\item troisième partie de for() (décrémenter str\_len)
\item L2: 
deuxième partie de for(): vérifier la première clause, puis la seconde. sauter au
début du corps de la boucle ou sortir.
\item L4: // sortir
\item renvoyer s
\end{itemize}

\subsection{x64: GCC 4.9.1 \Optimizing}
\label{string_trim_GCC_x64_O3}

\lstinputlisting[style=customasmx86]{\CURPATH/GCC491_x64_O3_FR.asm}

Maintenant, c'est plus complexe.

Le code avant le début du corps de la boucle est exécuté une seule fois, mais il contient
le test des caractères \CRLF{} aussi!
À quoi sert cette duplication du code?

La façon courante d'implémenter la boucle principale est sans doute ceci:

\begin{itemize}
\item (début de la boucle) tester la présence des caractères \CRLF{}, décider
\item stocker le caractère zéro
\end{itemize}

Mais GCC a décidé d'inverser ces deux étapes.

Bien sûr,  \IT{stocker le caractère zéro} ne peut pas être la première étape, donc
un autre test est nécessaire:

\begin{itemize}
\item traiter le premier caractère. matcher avec \CRLF{}, sortir si le caractère
n'est pas \CRLF{}

\item (début de la boucle) stocker le caractère zéro

\item tester la présence des caractères \CRLF{}, décider
\end{itemize}

Maintenant la boucle principale est très courte, ce qui est bon pour les derniers
\ac{CPU}s.

Le code n'utilise pas la variable str\_len, mais str\_len-1.
Donc c'est plus comme un index dans un buffer.

Apparemment, GCC a remarqué que l'expression str\_len-1 est utilisée deux fois.

Donc, c'est mieux d'allouer une variable qui contient toujours une valeur qui est
plus petite que la longueur actuelle de la chaîne de un, et la décrémente (ceci a
le même effet que de décrémenter la variable str\_len).

\subsubsection{ARM + \NonOptimizingXcodeIV (\ThumbTwoMode)}
\label{FPU_passing_floats_ARM}

\lstinputlisting[style=customasmARM]{patterns/12_FPU/2_passing_floats/Xcode_thumb_O0.asm}

Comme nous l'avons déjà mentionné, les pointeurs sur des nombres flottants 64-bit
sont passés dans une paire de R-registres.

Ce code est un peu redondant (probablement car l'optimisation est désactivée),
puisqu'il est possible de charger les valeurs directement dans les R-registres sans
toucher les D-registres.

Donc, comme nous le voyons, la fonction \GTT{\_pow} reçoit son premier argument dans
\Reg{0} et \Reg{1}, et le second dans \Reg{2} et \Reg{3}.
La fonction laisse son résultat dans \Reg{0} et \Reg{1}.
Le résultat de \GTT{\_pow} est déplacé dans \GTT{D16}, puis dans la paire \Reg{1}
et \Reg{2}, d'où \printf prend le nombre résultant.

\subsubsection{ARM + \NonOptimizingKeilVI (\ARMMode)}

\lstinputlisting[style=customasmARM]{patterns/12_FPU/2_passing_floats/Keil_ARM_O0.asm}

Les D-registres ne sont pas utilisés ici, juste des paires de R-registres.

\subsubsection{ARM64 + GCC (Linaro) 4.9 \Optimizing}

\lstinputlisting[caption=GCC (Linaro) 4.9 \Optimizing,style=customasmARM]{patterns/12_FPU/2_passing_floats/ARM64_FR.s}

Les constantes sont chargées dans \RegD{0} et \RegD{1}: \TT{pow()} les prend d'ici.
Le résultat sera dans \RegD{0} après l'exécution de \TT{pow()}.
Il est passé à  \printf sans aucune modification ni déplacement, car \printf
prend ces arguments de \glslink{integral type}{type intégral} et pointeurs depuis
des X-registres, et les arguments en virgule flottante depuis des D-registres.


\subsubsection{MIPS}

MIPS peut supporter plusieurs coprocesseurs (jusqu'à 4), le zérotième\footnote{Barbarisme
pour rappeler que les indices commencent à zéro.} est un coprocesseur contrôleur
spécial, et celui d'indice 1 est le FPU.

Comme en ARM, le coprocesseur MIPS n'est pas une machine à pile, il comprend 32 registres
32-bit (\$F0-\$F31):
\myref{MIPS_FPU_registers}.

Lorsque l'on doit travailler avec des valeurs \Tdouble 64-bit, une paire de F-registres
32-bit est utilisée.

\lstinputlisting[caption=GCC 4.4.5 \Optimizing (IDA),style=customasmMIPS]{patterns/12_FPU/1_simple/MIPS_O3_IDA_FR.lst}

Les nouvelles instructions ici sont:

\myindex{MIPS!\Instructions!LWC1}
\myindex{MIPS!\Instructions!DIV.D}
\myindex{MIPS!\Instructions!MUL.D}
\myindex{MIPS!\Instructions!ADD.D}
\begin{itemize}

\item \INS{LWC1} charge un mot de 32-bit dans un registre du premier coprocesseur
(d'où le \q{1} dans le nom de l'instruction).
\myindex{MIPS!\Pseudoinstructions!L.D}

Une paire d'instructions \INS{LWC1} peut être combinée en une pseudo instruction \INS{L.D}.

\item \INS{DIV.D}, \INS{MUL.D}, \INS{ADD.D} effectuent respectivement la division,
la multiplication, et l'addition (\q{.D} est le suffixe standard pour la double précision,
\q{.S} pour la simple précision)

\end{itemize}

\myindex{MIPS!\Instructions!LUI}
\myindex{\CompilerAnomaly}
\label{MIPS_FPU_LUI}

Il y a une anomalie bizarre du compilateur: l'instruction \INS{LUI} que nous avons
marqué avec un point d'interrogation.
Il m'est difficile de comprendre pourquoi charger une partie de la constante de type
64-bit \Tdouble dans le registre \$V0. Cette instruction n'a pas d'effet.
% TODO did you try checking out compiler source code?
Si quelqu'un en sait plus sur ceci, s'il vous plaît, envoyez moi un email\footnote{\EMAIL}.


