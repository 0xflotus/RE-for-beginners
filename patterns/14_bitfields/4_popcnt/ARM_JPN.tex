\subsubsection{ARM + \OptimizingXcodeIV (\ARMMode)}

\lstinputlisting[caption=\OptimizingXcodeIV (\ARMMode),label=ARM_leaf_example4,style=customasmARM]{patterns/14_bitfields/4_popcnt/ARM_Xcode_O3_JPN.lst}

\myindex{ARM!\Instructions!TST}
\TST はx86では \TEST と同じです。

\myindex{ARM!Optional operators!LSL}
\myindex{ARM!Optional operators!LSR}
\myindex{ARM!Optional operators!ASR}
\myindex{ARM!Optional operators!ROR}
\myindex{ARM!Optional operators!RRX}
\myindex{ARM!\Instructions!MOV}
\myindex{ARM!\Instructions!TST}
\myindex{ARM!\Instructions!CMP}
\myindex{ARM!\Instructions!ADD}
\myindex{ARM!\Instructions!SUB}
\myindex{ARM!\Instructions!RSB}
前述のように(\myref{shifts_in_ARM_mode})、
ARMモードでは個別のシフト命令はありません。
ただし、 \MOV 、 \TST 、 \CMP 、 \ADD 、 \SUB 、 \RSB などの命令には、
LSL(\IT{Logical Shift Left})、
LSR(\IT{Logical Shift Right})、
ASR(\IT{Arithmetic Shift Right})、
ROR(\IT{Rotate Right})、
RRX(\IT{Rotate Right with Extend}) 
があります。

これらの変更子は、第2オペランドのシフト方法とビット数を定義します。

\myindex{ARM!\Instructions!TST}
\myindex{ARM!Optional operators!LSL}
したがって、 \TT{\q{TST R1, R2,LSL R3}}命令はここでは $R1 \land (R2 \ll R3)$ として機能します。

\subsubsection{ARM + \OptimizingXcodeIV (\ThumbTwoMode)}

\myindex{ARM!\Instructions!LSL.W}
\myindex{ARM!\Instructions!LSL}
ほぼ同じですが、Thumbモードでは \LSL 修飾子を直接 \TST に定義することはできないため、
1つの \TST の代わりに2つの \INS{LSL.W}/\TST 命令が使用されます。

\begin{lstlisting}[label=ARM_leaf_example5,style=customasmARM]
                MOV             R1, R0
                MOVS            R0, #0
                MOV.W           R9, #1
                MOVS            R3, #0
loc_2F7A
                LSL.W           R2, R9, R3
                TST             R2, R1
                ADD.W           R3, R3, #1
                IT NE
                ADDNE           R0, #1
                CMP             R3, #32
                BNE             loc_2F7A
                BX              LR
\end{lstlisting}

\subsubsection{ARM64 + \Optimizing GCC 4.9}

すでに使用した64ビットの例を見てみましょう:\myref{popcnt_x64_example}

\lstinputlisting[caption=\Optimizing GCC (Linaro) 4.8,style=customasmARM]{patterns/14_bitfields/4_popcnt/ARM64_GCC_O3_JPN.s}

結果は、GCCがx64に対して生成するものと非常によく似ています:\myref{shifts64_GCC_O3}

\myindex{ARM!\Instructions!CSEL}
\CSEL 命令は\q{Conditional SELect}です。 
\TST で設定されたフラグに応じて1つの変数が2つだけ選択され、値が \q{rt}変数を保持する\RegW{2}にコピーされます。

\subsubsection{ARM64 + \NonOptimizing GCC 4.9}

もう一度、すでに使用した64ビットの例について作業します:\myref{popcnt_x64_example}
例によって、コードはより冗長です。

\lstinputlisting[caption=\NonOptimizing GCC (Linaro) 4.8,style=customasmARM]{patterns/14_bitfields/4_popcnt/ARM64_GCC_O0_JPN.s}

