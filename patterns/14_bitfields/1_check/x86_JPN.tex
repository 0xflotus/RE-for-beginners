\subsubsection{x86}
\myindex{Windows!Win32}

Win32 APIの例です。

\begin{lstlisting}[style=customc]
	HANDLE fh;

	fh=CreateFile ("file", GENERIC_WRITE | GENERIC_READ, FILE_SHARE_READ, NULL, OPEN_ALWAYS, FILE_ATTRIBUTE_NORMAL, NULL);
\end{lstlisting}

次の結果を得ます。(MSVC 2010)

\begin{lstlisting}[caption=MSVC 2010,style=customasmx86]
	push	0
	push	128		; 00000080H
	push	4
	push	0
	push	1
	push	-1073741824	; c0000000H
	push	OFFSET $SG78813
	call	DWORD PTR __imp__CreateFileA@28
	mov	DWORD PTR _fh$[ebp], eax
\end{lstlisting}

WinNT.hを見てみましょう。

\begin{lstlisting}[caption=WinNT.h,style=customc]
#define GENERIC_READ                     (0x80000000L)
#define GENERIC_WRITE                    (0x40000000L)
#define GENERIC_EXECUTE                  (0x20000000L)
#define GENERIC_ALL                      (0x10000000L)
\end{lstlisting}

すべてがクリアです。
GENERIC\_READ | GENERIC\_WRITE = 0x80000000 | 0x40000000 = 0xC0000000で、この値が
\TT{CreateFile()}\footnote{\href{http://go.yurichev.com/17065}{msdn.microsoft.com/en-us/library/aa363858(VS.85).aspx}}関数への2番目の引数として使用されています。

\TT{CreateFile()} はこれらのフラグをどのようにチェックしているでしょうか?
\myindex{Windows!KERNEL32.DLL}

Windows XP SP3 x86のKERNEL32.DLLを見てみると、私たちはこのコードの断片を \TT{CreateFileW} で見つけます。

\begin{lstlisting}[caption=KERNEL32.DLL (Windows XP SP3 x86),style=customasmx86]
.text:7C83D429     test    byte ptr [ebp+dwDesiredAccess+3], 40h
.text:7C83D42D     mov     [ebp+var_8], 1
.text:7C83D434     jz      short loc_7C83D417
.text:7C83D436     jmp     loc_7C810817
\end{lstlisting}

\myindex{x86!\Instructions!TEST}

ここでは、 \TEST 命令を参照していますが、第2引数全体を取るのではなく、
最上位バイト(\TT{ebp+dwDesiredAccess+3})のみを取り出し、フラグ\TT{0x40}
(ここでは\TT{GENERIC\_WRITE}フラグを意味します)をチェックします。
\myindex{x86!\Instructions!AND}

\TEST は基本的に \AND と同じ命令ですが、結果を保存することはありません
( \CMP は \SUB と同じですが、結果を保存しないことを思い出してください~(\myref{CMPandSUB}))。

このコードフラグメントのロジックは次のとおりです。

\begin{lstlisting}[style=customc]
if ((dwDesiredAccess&0x40000000) == 0) goto loc_7C83D417
\end{lstlisting}

\myindex{x86!\Instructions!AND}
\myindex{x86!\Registers!ZF}

\AND 命令がこのビットを離れると、 \ZF フラグはクリアされ、
\JZ 条件ジャンプは実行されません。 
条件ジャンプは、\TT{dwDesiredAccess}変数に\TT{0x40000000}ビットが存在しない場合にのみ実行されます。 
\AND の結果は0であり、 \ZF が設定され、条件付きジャンプが実行されます。

GCC 4.4.1とLinuxで試してみましょう。

\begin{lstlisting}[style=customc]
#include <stdio.h>
#include <fcntl.h>

void main()
{
	int handle;

	handle=open ("file", O_RDWR | O_CREAT);
};
\end{lstlisting}

次の結果を得ます。

\lstinputlisting[caption=GCC 4.4.1,style=customasmx86]{patterns/14_bitfields/1_check/check.asm}

\myindex{Linux!libc.so.6}
\myindex{syscall}

\TT{libc.so.6}ライブラリの\TT{open()}関数を見てみると、それは単なるシステムコールです。

\begin{lstlisting}[caption=open() (libc.so.6),style=customasmx86]
.text:000BE69B     mov     edx, [esp+4+mode] ; mode
.text:000BE69F     mov     ecx, [esp+4+flags] ; flags
.text:000BE6A3     mov     ebx, [esp+4+filename] ; filename
.text:000BE6A7     mov     eax, 5
.text:000BE6AC     int     80h             ; LINUX - sys_open
\end{lstlisting}


したがって、\TT{open()}のビットフィールドは、Linuxカーネルのどこかでチェックされるようです。

もちろん、GlibcとLinuxカーネルのソースコードの両方をダウンロードするのは簡単ですが、
それを使わないで問題を理解することに興味があります。

したがって、Linux 2.6では、\TT{sys\_open}システムコールが呼び出されると、制御は最終的に\TT{do\_sys\_open}に渡され、
そこから\TT{do\_filp\_open()}関数(\TT{fs/namei.c}のカーネルソースツリーにあります)に渡されます。

\newcommand{\URLREGPARM}{\href{http://go.yurichev.com/17040}{ohse.de/uwe/articles/gcc-attributes.html\#func-regparm}}

\myindex{fastcall}
\label{regparm}
注意:引数をスタック経由で渡すのとは別に、
レジスタのいくつかをレジスタに渡す方法もあります。
これはfastcall(\myref{fastcall})とも呼ばれます。
これは、引数の値を読み取るためにCPUがメモリ内のスタックにアクセスする必要がないため、より高速に動作します。 
GCCには\IT{regparm}\footnote{\URLREGPARM}というオプションがあります。
これにより、レジスタ経由で渡すことができる引数の数を設定することができます。

\newcommand{\URLKERNELNEWB}{\href{http://go.yurichev.com/17066}{kernelnewbies.org/Linux\_2\_6\_20\#head-042c62f290834eb1fe0a1942bbf5bb9a4accbc8f}}
\newcommand{\CALLINGHFILE}{arch/x86/include/asm/calling.h}

Linux 2.6カーネルは\TT{-mregparm=3}オプション~\footnote{\URLKERNELNEWB}でコンパイルされます。
\footnote{カーネルツリーの\TT{\CALLINGHFILE}ファイルも参照してください}

これが意味するところは、最初の3つの引数はレジスタ \EAX 、 \EDX 、 \ECX を経由し、
残りはスタック経由で渡されるということです。
もちろん、引数の数が3よりも少ない場合、設定されたレジスタの一部のみが使用されます。

ですから、Linux Kernel 2.6.31をダウンロードし、Ubuntuでコンパイルしてみてください。\TT{make vmlinux}し、 \IDA で開き、
\TT{do\_filp\_open()}関数を見つけましょう。以下を見てみます(コメントは私のものです)。

\lstinputlisting[caption=do\_filp\_open() (linux kernel 2.6.31),style=customasmx86]{patterns/14_bitfields/1_check/check2_JPN.asm}

GCCは最初の3つの引数の値をローカルスタックに保存します。 
これが行われなかった場合、コンパイラはこれらのレジスタに触れず、
コンパイラの\gls{register allocator}には厳しい環境になります。

このコードの断片を見てみましょう:

\lstinputlisting[caption=do\_filp\_open() (linux kernel 2.6.31),style=customasmx86]{patterns/14_bitfields/1_check/check3.asm}

\TT{0x40}は\TT{O\_CREAT}マクロと同じことです。
\TT{open\_flag}は\TT{0x40}としてチェックされ、このビットが1の場合、
次の \JNZ 命令が実行されます。
