\subsection{\ShiftsSectionName}

\CCpp でのビットシフトは $\ll$ と $\gg$ 演算子を使って実装されます。
x86 \ac{ISA} はシフトのためにSHL (SHift Left) と SHR (SHift Right)命令を持っています。
シフト命令はしばしば2のべき乗 $2^{n}$ において除算や乗算が使用されます(例えば1,2,4,8など)。
\myref{subsec:mult_using_shifts}、
\myref{division_by_shifting}。

% FIXME: rework this

シフト操作も非常に重要です。シフト演算は特定ビットの分離や複数の
散在ビット値の構築に用いられることが多いからです。
