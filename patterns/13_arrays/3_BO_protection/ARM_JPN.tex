\subsubsection{\OptimizingXcodeIV (\ThumbTwoMode)}

単純な配列の例に戻りましょう(\myref{arrays_simple})。

繰り返しますが、LLVMが\q{カナリア}の正しさをどのようにチェックするのか見ることができます。

% TODO shorten the listing a bit? is full display of unrolled loop necessary?
\lstinputlisting[style=customasmARM]{patterns/13_arrays/3_BO_protection/simple_Xcode_thumb_O3_JPN.asm}

\myindex{Unrolled loop}

まず最初に、見てきたように、LLVMはループを\q{展開し}、LLVMは高速になると結論づけて、
事前に計算されて値はすべて配列に1つ1つ書かれます。
なお、ARMモードでの命令はこれをより高速にする手助けをするかもしれません。
これを見つけるのは宿題にします。

関数の最後で\q{カナリア}の比較を見ます。ローカルスタックのカナリアと \Reg{8}で指し示した正しいものとの。

\myindex{ARM!\Instructions!IT}

それぞれが一致していれば、4命令ブロックが\INS{ITTTT EQ}で実行され、
\Reg{0}に0が書かれ、関数エピローグが終了します。
\q{カナリア}が一致していなければ、ブロックがスキップされ、
\TT{\_\_\_stack\_chk\_fail}関数へのジャンプが実行され、おそらく実行が停止されます。
% TODO1 illustrate this!
