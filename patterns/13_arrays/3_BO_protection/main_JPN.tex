\subsection{バッファオーバーフロー保護手法}
\label{subsec:BO_protection}

There are several methods to protect against this scourge, regardless of the \CCpp programmers' negligence.
MSVC has options like\footnote{compiler-side buffer overflow protection methods:
\href{http://go.yurichev.com/17133}{wikipedia.org/wiki/Buffer\_overflow\_protection}}:

このソースコードに対する保護手法はいくつかあり、 \CCpp プログラマの怠慢にもかかわらず、
MSVCにはオプションがあります。\footnote{コンパイラサイドのバッファオーバーフロー保護手法:
\href{http://go.yurichev.com/17133}{wikipedia.org/wiki/Buffer\_overflow\_protection}}

\begin{lstlisting}
 /RTCs スタックフレームの実行時チェック
 /GZ スタックチェックの有効化 (/RTCs)
\end{lstlisting}

\myindex{x86!\Instructions!RET}
\myindex{Function prologue}
\myindex{Security cookie}

手法の1つに関数プロローグでスタックのローカル変数の間にランダムな値を書き込み、
関数を終了する前に関数エピローグでそれをチェックするというものがあります。
値が同じでなければ、最後の命令 \RET を実行せず、停止(ハング)します。
プロセスは停止しますが、遠隔の攻撃者があなたのホストを攻撃するよりはよいことです。

\newcommand{\CANARYURL}{\href{http://go.yurichev.com/17134}{wikipedia.org/wiki/Domestic\_canary\#Miner.27s\_canary}}

\myindex{Canary}

このランダムな値は しばしば \q{カナリア} と呼ばれ、炭鉱労働でのカナリアに関連しています。\footnote{\CANARYURL}
昔、有毒なガスを一早く検知できるよう、炭鉱労働者に使用されていました。

カナリアは炭鉱のガスにとっても敏感で、危機の際に騒ぎ立て、場合によっては死んでしまいました。

とてもシンプルな配列の例を \ac{MSVC} でRTC1とRTCsオプション付きでコンパイルする場合~(\myref{arrays_simple})、
\q{カナリア}が正しいかどうか、関数の最後に \TT{@\_RTC\_CheckStackVars@8} を呼び出すのを見ることができます。

GCCがこれをどのように扱うかを見てみましょう。
\TT{alloca()}~(\myref{alloca})の例を扱いましょう。

\lstinputlisting[style=customc]{patterns/02_stack/04_alloca/2_1.c}

デフォルトでは、追加のオプションなしに、GCC 4.7.3は\q{カナリア}チェックをコードに挿入します。

\lstinputlisting[caption=GCC 4.7.3,style=customasmx86]{patterns/13_arrays/3_BO_protection/gcc_canary_JPN.asm}

ランダム値が\TT{gs:20}に配置されます。
スタックに書かれて、関数の最後でスタックの値が\TT{gs:20}の\q{カナリア}と一致しているか比較します。
値が一致していなければ、
\TT{\_\_stack\_chk\_fail}
関数が呼び出され、ときどき(Ubuntu 13.04 x86):のようなものをコンソールでみることがあります。

\begin{lstlisting}
*** buffer overflow detected ***: ./2_1 terminated
======= Backtrace: =========
/lib/i386-linux-gnu/libc.so.6(__fortify_fail+0x63)[0xb7699bc3]
/lib/i386-linux-gnu/libc.so.6(+0x10593a)[0xb769893a]
/lib/i386-linux-gnu/libc.so.6(+0x105008)[0xb7698008]
/lib/i386-linux-gnu/libc.so.6(_IO_default_xsputn+0x8c)[0xb7606e5c]
/lib/i386-linux-gnu/libc.so.6(_IO_vfprintf+0x165)[0xb75d7a45]
/lib/i386-linux-gnu/libc.so.6(__vsprintf_chk+0xc9)[0xb76980d9]
/lib/i386-linux-gnu/libc.so.6(__sprintf_chk+0x2f)[0xb7697fef]
./2_1[0x8048404]
/lib/i386-linux-gnu/libc.so.6(__libc_start_main+0xf5)[0xb75ac935]
======= Memory map: ========
08048000-08049000 r-xp 00000000 08:01 2097586    /home/dennis/2_1
08049000-0804a000 r--p 00000000 08:01 2097586    /home/dennis/2_1
0804a000-0804b000 rw-p 00001000 08:01 2097586    /home/dennis/2_1
094d1000-094f2000 rw-p 00000000 00:00 0          [heap]
b7560000-b757b000 r-xp 00000000 08:01 1048602    /lib/i386-linux-gnu/libgcc_s.so.1
b757b000-b757c000 r--p 0001a000 08:01 1048602    /lib/i386-linux-gnu/libgcc_s.so.1
b757c000-b757d000 rw-p 0001b000 08:01 1048602    /lib/i386-linux-gnu/libgcc_s.so.1
b7592000-b7593000 rw-p 00000000 00:00 0
b7593000-b7740000 r-xp 00000000 08:01 1050781    /lib/i386-linux-gnu/libc-2.17.so
b7740000-b7742000 r--p 001ad000 08:01 1050781    /lib/i386-linux-gnu/libc-2.17.so
b7742000-b7743000 rw-p 001af000 08:01 1050781    /lib/i386-linux-gnu/libc-2.17.so
b7743000-b7746000 rw-p 00000000 00:00 0
b775a000-b775d000 rw-p 00000000 00:00 0
b775d000-b775e000 r-xp 00000000 00:00 0          [vdso]
b775e000-b777e000 r-xp 00000000 08:01 1050794    /lib/i386-linux-gnu/ld-2.17.so
b777e000-b777f000 r--p 0001f000 08:01 1050794    /lib/i386-linux-gnu/ld-2.17.so
b777f000-b7780000 rw-p 00020000 08:01 1050794    /lib/i386-linux-gnu/ld-2.17.so
bff35000-bff56000 rw-p 00000000 00:00 0          [stack]
Aborted (core dumped)
\end{lstlisting}

\myindex{MS-DOS}
gsはいわゆるセグメントレジスタです。このレジスタは広くMS-DOSやDOS拡張で使用されました。
今日、この機能は異なっています。
\myindex{TLS}
\myindex{Windows!TIB}

簡単に言うと、Linuxでの\TT{gs}レジスタは常に\ac{TLS}~(\myref{TLS})を指し示します。
スレッド固有の情報がそこに保存されます。
ところで、win32では\TT{fs}レジスタは同じ役割を担い、\ac{TIB}を指し示します。
\footnote{\href{http://go.yurichev.com/17104}{wikipedia.org/wiki/Win32\_Thread\_Information\_Block}}

より詳細はLinuxカーネルソースコード\IT{arch/x86/include/asm/stackprotector.h}の中に
コメントとして記述してあるのを見つけられます(少なくとも3.11バージョンには)。

\subsubsection{\OptimizingXcodeIV (\ThumbTwoMode)}

単純な配列の例に戻りましょう(\myref{arrays_simple})。

繰り返しますが、LLVMが\q{カナリア}の正しさをどのようにチェックするのか見ることができます。

% TODO shorten the listing a bit? is full display of unrolled loop necessary?
\lstinputlisting[style=customasmARM]{patterns/13_arrays/3_BO_protection/simple_Xcode_thumb_O3_JPN.asm}

\myindex{Unrolled loop}

まず最初に、見てきたように、LLVMはループを\q{展開し}、LLVMは高速になると結論づけて、
事前に計算されて値はすべて配列に1つ1つ書かれます。
なお、ARMモードでの命令はこれをより高速にする手助けをするかもしれません。
これを見つけるのは宿題にします。

関数の最後で\q{カナリア}の比較を見ます。ローカルスタックのカナリアと \Reg{8}で指し示した正しいものとの。

\myindex{ARM!\Instructions!IT}

それぞれが一致していれば、4命令ブロックが\INS{ITTTT EQ}で実行され、
\Reg{0}に0が書かれ、関数エピローグが終了します。
\q{カナリア}が一致していなければ、ブロックがスキップされ、
\TT{\_\_\_stack\_chk\_fail}関数へのジャンプが実行され、おそらく実行が停止されます。
% TODO1 illustrate this!

