\subsubsection{3次元配列の例}

多次元配列でも同じです。

\Tint 型の配列で作業していきます。各要素はメモリ上で4バイト必要とします。

見てみましょう。

\lstinputlisting[caption=単純な例,style=customc]{patterns/13_arrays/5_multidimensional/multi.c}

\myparagraph{x86}

MSVC 2010の結果

\lstinputlisting[caption=MSVC 2010,style=customasmx86]{patterns/13_arrays/5_multidimensional/multi_msvc_JPN.asm}

特別なことはありません。インデックスの計算では、式 $address=600 \cdot 4 \cdot x + 30 \cdot 4 \cdot y + 4z$ 
では3つの入力引数が使用され、配列を多次元として表現しています。
\Tint 型は32ビット(4バイト)なので、
係数は4倍する必要があることを忘れないでください。

\lstinputlisting[caption=GCC 4.4.1,style=customasmx86]{patterns/13_arrays/5_multidimensional/multi_gcc_JPN.asm}

GCCコンパイラは異なります。

計算での演算において($30y$)、GCCは乗算命令を使わないコードを生成します。
このようにします。
$(y+y) \ll 4 - (y+y) = (2y) \ll 4 - 2y = 2 \cdot 16 \cdot y - 2y = 32y - 2y = 30y$. 
従って、 $30y$ の計算には、加算命令が1つだけです。
ビットシフト演算と減算が使用されます。
これはより高速です。

\myparagraph{ARM + \NonOptimizingXcodeIV (\ThumbMode)}

\lstinputlisting[caption=\NonOptimizingXcodeIV (\ThumbMode),style=customasmARM]{patterns/13_arrays/5_multidimensional/multi_Xcode_thumb_O0_JPN.asm}

\NonOptimizing LLVMは変数すべてをローカルスタックに保存しますが、冗長です。

配列の要素のアドレスはすでに見た式によって計算されます。

\myparagraph{ARM + \OptimizingXcodeIV (\ThumbMode)}

\lstinputlisting[caption=\OptimizingXcodeIV (\ThumbMode),style=customasmARM]{patterns/13_arrays/5_multidimensional/multi_Xcode_thumb_O3_JPN.asm}

既に見たシフト、加減算による乗算を置き換えるためのトリックもここにあります。

\myindex{ARM!\Instructions!RSB}
\myindex{ARM!\Instructions!SUB}
新しい命令を見てみます:\RSB (\IT{Reverse Subtract})

単純に \SUB として機能しますが、実行前にオペランドをスワップします。
なぜでしょう?
\myindex{ARM!Optional operators!LSL}
\SUB および \RSB は、シフト係数が適用される第2のオペランド(\INS{LSL\#4})への命令です。

ただし、この係数は第2オペランドにのみ適用されます。

これは、加算や乗算のような可換的な(交換可能な)演算の場合は問題ありません。
(結果を変更せずにオペランドを入れ替えてもかまいません)

しかし、減算は非可換的な演算なので、 \RSB が存在します。

\myparagraph{MIPS}

\myindex{MIPS!Global Pointer}
私の例はとても小さいので、GCCコンパイラはグローバルポインタによってアドレス可能な64KiB領域に
配列を配置することに決めました。

\lstinputlisting[caption=\Optimizing GCC 4.4.5 (IDA),style=customasmMIPS]{patterns/13_arrays/5_multidimensional/multi_MIPS_O3_IDA_JPN.lst}

