\subsubsection{x86}

\myparagraph{MSVC}

コンパイルしてみましょう。

\lstinputlisting[caption=MSVC 2008,style=customasmx86]{patterns/13_arrays/1_simple/simple_msvc.asm}

\myindex{x86!\Instructions!SHL}

特別なことは何もなくて、2つのループだけです。1つめは配列に値を詰めるループで2つめは値を表示するループです。
\TT{shl ecx, 1}命令は \ECX の値を2倍するのに使用されます。詳細はこちら~\myref{SHR}

80バイトは4バイトの20要素分の配列用としてスタック上に確保されます。

\clearpage
\olly でこの例を試してみましょう。
\myindex{\olly}

配列がどのように埋まるのか見ていきます。

各要素は32ビットの \Tint 型で値はインデックスを2倍したものです。

\begin{figure}[H]
\centering
\myincludegraphics{patterns/13_arrays/1_simple/olly.png}
\caption{\olly: 要素を埋めた後}
\label{fig:array_simple_olly}
\end{figure}

この配列はスタックに位置しているので、20要素すべてを見ることができます。

\myparagraph{GCC}

GCC 4.4.1ではこのようになります。

\lstinputlisting[caption=GCC 4.4.1,style=customasmx86]{patterns/13_arrays/1_simple/simple_gcc.asm}

なお、変数 $a$ は\IT{int*}型です
(\Tint{} へのポインタ)---別の関数に配列へのポインタを渡すことができます。
しかし、もっと正確には、配列の最初の要素へのポインタが渡されます。
(要素の残りのアドレスは明確なやり方で計算されます)

もしこのポインタを\IT{a[idx]}としてインデックスするなら、\IT{idx}はポインタに加算されるだけで、
配置されている要素(計算されたポインタが示されている)がリターンされます。

面白い例:\IT{\q{string}}のような文字列は(1)文字の配列で\IT{const char[]}の型を持ちます。

インデックスもこのポインタに適用されます。

そしてこれが\TT{\q{string}[i]}のように書き込みが可能な理由です。これは \CCpp の正しい表現です!
