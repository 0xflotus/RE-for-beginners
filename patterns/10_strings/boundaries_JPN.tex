\subsection{文字列境界}

\myindex{win32!GetOpenFileName}
パラメータがwin32の\IT{GetOpenFileName()}関数にどのように渡されるかは興味深いことです。 
それを呼び出すには、許可されたファイル拡張子のリストを設定する必要があります:

\begin{lstlisting}[style=customc]
	OPENFILENAME *LPOPENFILENAME;
	...
	char * filter = "Text files (*.txt)\0*.txt\0MS Word files (*.doc)\0*.doc\0\0";
	...
	LPOPENFILENAME = (OPENFILENAME *)malloc(sizeof(OPENFILENAME));
	...
	LPOPENFILENAME->lpstrFilter = filter;
	...

	if(GetOpenFileName(LPOPENFILENAME))
	{
		...
\end{lstlisting}

ここでは、\IT{GetOpenFileName()}に文字列のリストが渡されます。 
解析するのは問題ではありません。ゼロバイトが1つ出現するたびに、これがアイテムになります。 
ゼロバイトが2バイト出現すれば、リストの最後になります。 
この文字列を \printf に渡すと、最初の項目は単一の文字列として扱われます。

これは文字列か、そうではないのか。
これは、複数のゼロ終端されたC文字列を含むバッファであり、全体として格納して処理することができます。

\myindex{\CStandardLibrary!strtok}
もう一つの例は\IT{strtok()}関数です。文字列をとり、その真ん中にゼロバイトを書き込みます。 
したがって、入力文字列をいくつかの種類のバッファに変換します。バッファには、ゼロで終了するC文字列がいくつかあります。
