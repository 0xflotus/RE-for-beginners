\subsubsection{MIPS}

\lstinputlisting[caption=\Optimizing GCC 4.4.5 (IDA),style=customasmMIPS]{patterns/10_strings/1_strlen/MIPS_O3_IDA_JPN.lst}

\myindex{MIPS!\Instructions!NOR}
\myindex{MIPS!\Pseudoinstructions!NOT}

MIPSには \NOT 命令がありませんが、\NOR は\TT{OR~+~NOT}演算です。

この操作はデジタルエレクトロニクス\footnote{NORは\q{汎用ゲート}と呼ばれます}で広く使用されています。 
\index{Apollo Guidance Computer}

たとえば、Apolloプログラムで使用されているApolloガイダンス・コンピュータは、
5600個のNORゲートを使用して作成されました:
[Jens Eickhoff, \IT{Onboard Computers, Onboard Software and Satellite Operations: An Introduction}, (2011)]を参照。
しかし、NOR要素はコンピュータプログラミングであまり一般的ではありません。

したがって、NOT演算は\TT{NOR~DST,~\$ZERO,~SRC}として実装されています。

基本(\myref{sec:signednumbers})から、符号付き数値のビット反転は、
符号の変更と結果からの1の減算と同じであることがわかります。

ですから、ここでは $str$ の値をとり、それを $-str-1$ に変換することはしません。 
次の加算演算は結果を準備します。
