\mysection{\MinesweeperWinXPExampleChapterName}
\label{minesweeper_winxp}
\myindex{Windows!Windows XP}

Для тех, кто не очень хорошо играет в Сапёра (Minesweeper), можно попробовать найти все скрытые мины в отладчике.

\myindex{\CStandardLibrary!rand()}
\myindex{Windows!PDB}
Как мы знаем, Сапёр располагает мины случайным образом, так что там должен быть генератор случайных чисел
или вызов стандартной функции Си \TT{rand()}.

Вот что хорошо в реверсинге продуктов от Microsoft, так это то что часто есть \gls{PDB}-файл со всеми
символами (имена функций, \etc{}.).

Когда мы загружаем \TT{winmine.exe} в \IDA, она скачивает 
\gls{PDB} файл именно для этого исполняемого файла и добавляет все имена.

И вот оно, только один вызов \TT{rand()} в этой функции:

\begin{lstlisting}[style=customasmx86]
.text:01003940 ; __stdcall Rnd(x)
.text:01003940 _Rnd@4          proc near               ; CODE XREF: StartGame()+53
.text:01003940                                         ; StartGame()+61
.text:01003940
.text:01003940 arg_0           = dword ptr  4
.text:01003940
.text:01003940                 call    ds:__imp__rand
.text:01003946                 cdq
.text:01003947                 idiv    [esp+arg_0]
.text:0100394B                 mov     eax, edx
.text:0100394D                 retn    4
.text:0100394D _Rnd@4          endp
\end{lstlisting}

Так её назвала \IDA и это было имя данное ей разработчиками Сапёра.

Функция очень простая:

\begin{lstlisting}[style=customc]
int Rnd(int limit)
{
    return rand() % limit;
};
\end{lstlisting}

(В \gls{PDB}-файле не было имени \q{limit}; это мы назвали этот аргумент так, вручную.)

Так что она возвращает случайное число в пределах от нуля до заданного предела.

\TT{Rnd()} вызывается только из одного места, это функция с названием \TT{StartGame()}, 
и как видно, это именно тот код, что расставляет мины:

\begin{lstlisting}[style=customasmx86]
.text:010036C7                 push    _xBoxMac
.text:010036CD                 call    _Rnd@4          ; Rnd(x)
.text:010036D2                 push    _yBoxMac
.text:010036D8                 mov     esi, eax
.text:010036DA                 inc     esi
.text:010036DB                 call    _Rnd@4          ; Rnd(x)
.text:010036E0                 inc     eax
.text:010036E1                 mov     ecx, eax
.text:010036E3                 shl     ecx, 5          ; ECX=ECX*32
.text:010036E6                 test    _rgBlk[ecx+esi], 80h
.text:010036EE                 jnz     short loc_10036C7
.text:010036F0                 shl     eax, 5          ; EAX=EAX*32
.text:010036F3                 lea     eax, _rgBlk[eax+esi]
.text:010036FA                 or      byte ptr [eax], 80h
.text:010036FD                 dec     _cBombStart
.text:01003703                 jnz     short loc_10036C7
\end{lstlisting}

Сапёр позволяет задать размеры доски, так что X (xBoxMac) и Y (yBoxMac) это глобальные переменные.

Они передаются в \TT{Rnd()} и генерируются случайные координаты.
Мина устанавливается инструкцией \TT{OR} на \TT{0x010036FA}. 
И если она уже была установлена до этого 
(это возможно, если пара функций \TT{Rnd()} 
сгенерирует пару, которая уже была сгенерирована), 
тогда \TT{TEST} и \TT{JNZ} на \TT{0x010036E6} 
перейдет на повторную генерацию пары.

\TT{cBombStart} это глобальная переменная, содержащая количество мин. Так что это цикл.

Ширина двухмерного массива это 32 (мы можем это вывести, глядя на инструкцию \TT{SHL}, которая умножает
одну из координат на 32).

Размер глобального массива \TT{rgBlk} 
можно легко узнать по разнице между меткой \TT{rgBlk} 
в сегменте данных и следующей известной меткой. 
Это 0x360 (864):

\begin{lstlisting}[style=customasmx86]
.data:01005340 _rgBlk          db 360h dup(?)          ; DATA XREF: MainWndProc(x,x,x,x)+574
.data:01005340                                         ; DisplayBlk(x,x)+23
.data:010056A0 _Preferences    dd ?                    ; DATA XREF: FixMenus()+2
...
\end{lstlisting}

$864/32=27$.

Так что размер массива $27*32$?
Это близко к тому что мы знаем: если попытаемся установить размер доски в установках Сапёра на $100*100$, то он установит размер $24*30$.
Так что это максимальный размер доски здесь.
И размер массива фиксирован для доски любого размера.

Посмотрим на всё это в \olly.
Запустим Сапёр, присоединим (attach) \olly к нему и увидим содержимое памяти по адресу где массив \TT{rgBlk} (\TT{0x01005340})%

\footnote{Все адреса здесь для Сапёра под Windows XP SP3 English. 
Они могут отличаться для других сервис-паков.}.

Так что у нас выходит такой дамп памяти массива:

\lstinputlisting[style=customasmx86]{examples/minesweeper/1.lst}

\olly, как и любой другой шестнадцатеричный редактор, показывает 16 байт на строку.
Так что каждая 32-байтная строка массива занимает ровно 2 строки.

Это уровень для начинающих (доска 9*9).

Тут еще какая-то квадратная структура, заметная визуально (байты 0x10).

Нажмем \q{Run} в \olly чтобы разморозить процесс Сапёра, потом нажмем в случайное место окна Сапёра, попадаемся на мине, но теперь
видны все мины:

\begin{figure}[H]
\centering
\myincludegraphicsSmall{examples/minesweeper/1.png}
\caption{Мины}
\label{fig:minesweeper1}
\end{figure}

Сравнивая места с минами и дамп, мы можем обнаружить что 0x10 это граница, 0x0F --- пустой блок, 
0x8F --- мина.
Вероятно 0x10 это т.н., \IT{sentinel value}.

Теперь добавим комментариев и также заключим все байты 0x8F в квадратные скобки:%

\lstinputlisting[style=customasmx86]{examples/minesweeper/2.lst}

Теперь уберем все байты связанные с границами (0x10) и всё что за ними:%

\lstinputlisting[style=customasmx86]{examples/minesweeper/3.lst}

Да, это всё мины, теперь это очень хорошо видно, в сравнении со скриншотом.

\clearpage
Вот что интересно, это то что мы можем модифицировать массив прямо в \olly.%

Уберем все мины заменив все байты 0x8F на 0x0F, и вот что получится в Сапёре:

\begin{figure}[H]
\centering
\myincludegraphicsSmall{examples/minesweeper/3.png}
\caption{Все мины убраны в отладчике}
\label{fig:minesweeper3}
\end{figure}

Также уберем их все и добавим их в первом ряду: 

\begin{figure}[H]
\centering
\myincludegraphicsSmall{examples/minesweeper/2.png}
\caption{Мины, установленные в отладчике}
\label{fig:minesweeper2}
\end{figure}

Отладчик не очень удобен для подсматривания (а это была наша изначальная цель), так что напишем маленькую
утилиту для показа содержимого доски:

\lstinputlisting[style=customc]{examples/minesweeper/minesweeper_cheater.c}

Просто установите \ac{PID}
\footnote{PID можно увидеть в Task Manager 
(это можно включить в \q{View $\rightarrow$ Select Columns})} 
и адрес массива (\TT{0x01005340} для Windows XP SP3 English) 
и она покажет его
\footnote{Скомпилированная версия здесь: 
\href{http://go.yurichev.com/17165}{beginners.re}}.

Она подключается к win32-процессу по \ac{PID}-у и просто читает из памяти процесса по этому адресу.

\subsection{Автоматический поиск массива}

Задавать адрес каждый раз при запуске нашей утилиты, это неудобно.
К тому же, разные версии ``Сапёра'' могут иметь этот массив по разным адресам.
Зная, что всегда есть рамка (байты 0x10), массив легко найти в памяти:

\lstinputlisting[style=customc]{examples/minesweeper/cheater2_fragment.c}

Полный исходный код: \url{https://github.com/DennisYurichev/RE-for-beginners/blob/master/examples/minesweeper/minesweeper_cheater2.c}.

\subsection{\Exercises}

\begin{itemize}

\item
Почему байты описывающие границы (0x10) (или \IT{sentinel value}) присутствуют вообще?
Зачем они нужны, если они вообще не видимы в интерфейсе Сапёра?
Как можно обойтись без них?

\item
Как выясняется, здесь больше возможных значений (для открытых блоков, для тех на которых игрок установил
флажок, \etc{}.).
	
Попробуйте найти значение каждого.

\item Измените мою утилиту так, чтобы она в запущенном процессе Сапёра убирала все мины, 
или расставляла их в соответствии с каким-то заданным шаблоном.

\end{itemize}
