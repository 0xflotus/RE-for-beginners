\mysection{Overclocking Cointerra Bitcoin miner}
\index{Bitcoin}
\index{BeagleBone}

There was Cointerra Bitcoin miner, looking like that:

\begin{figure}[H]
\centering
\myincludegraphics{examples/bitcoin_miner/board.jpg}
\caption{Board}
\end{figure}

And there was also (possibly leaked) utility\footnote{Can be downloaded here: \url{https://github.com/DennisYurichev/RE-for-beginners/raw/master/examples/bitcoin_miner/files/cointool-overclock}}
which can set clock rate for the board.
It runs on additional BeagleBone Linux ARM board (small board at bottom of the picture).

And the author was once asked, is it possible to hack this utility to see, which frequency can be set and which are not.
And it is possible to tweak it?

The utility must be executed like that: \TT{./cointool-overclock 0 0 900}, where 900 is frequency in MHz.
If the frequency is too high, utility will print \q{Error with arguments} and exit.

This is a fragment of code around reference to \q{Error with arguments} text string:

\begin{lstlisting}[style=customasmARM]

...

.text:0000ABC4         STR      R3, [R11,#var_28]
.text:0000ABC8         MOV      R3, #optind
.text:0000ABD0         LDR      R3, [R3]
.text:0000ABD4         ADD      R3, R3, #1
.text:0000ABD8         MOV      R3, R3,LSL#2
.text:0000ABDC         LDR      R2, [R11,#argv]
.text:0000ABE0         ADD      R3, R2, R3
.text:0000ABE4         LDR      R3, [R3]
.text:0000ABE8         MOV      R0, R3  ; nptr
.text:0000ABEC         MOV      R1, #0  ; endptr
.text:0000ABF0         MOV      R2, #0  ; base
.text:0000ABF4         BL       strtoll
.text:0000ABF8         MOV      R2, R0
.text:0000ABFC         MOV      R3, R1
.text:0000AC00         MOV      R3, R2
.text:0000AC04         STR      R3, [R11,#var_2C]
.text:0000AC08         MOV      R3, #optind
.text:0000AC10         LDR      R3, [R3]
.text:0000AC14         ADD      R3, R3, #2
.text:0000AC18         MOV      R3, R3,LSL#2
.text:0000AC1C         LDR      R2, [R11,#argv]
.text:0000AC20         ADD      R3, R2, R3
.text:0000AC24         LDR      R3, [R3]
.text:0000AC28         MOV      R0, R3  ; nptr
.text:0000AC2C         MOV      R1, #0  ; endptr
.text:0000AC30         MOV      R2, #0  ; base
.text:0000AC34         BL       strtoll
.text:0000AC38         MOV      R2, R0
.text:0000AC3C         MOV      R3, R1
.text:0000AC40         MOV      R3, R2
.text:0000AC44         STR      R3, [R11,#third_argument]
.text:0000AC48         LDR      R3, [R11,#var_28]
.text:0000AC4C         CMP      R3, #0
.text:0000AC50         BLT      errors_with_arguments
.text:0000AC54         LDR      R3, [R11,#var_28]
.text:0000AC58         CMP      R3, #1
.text:0000AC5C         BGT      errors_with_arguments
.text:0000AC60         LDR      R3, [R11,#var_2C]
.text:0000AC64         CMP      R3, #0
.text:0000AC68         BLT      errors_with_arguments
.text:0000AC6C         LDR      R3, [R11,#var_2C]
.text:0000AC70         CMP      R3, #3
.text:0000AC74         BGT      errors_with_arguments
.text:0000AC78         LDR      R3, [R11,#third_argument]
.text:0000AC7C         CMP      R3, #0x31
.text:0000AC80         BLE      errors_with_arguments
.text:0000AC84         LDR      R2, [R11,#third_argument]
.text:0000AC88         MOV      R3, #950
.text:0000AC8C         CMP      R2, R3
.text:0000AC90         BGT      errors_with_arguments
.text:0000AC94         LDR      R2, [R11,#third_argument]
.text:0000AC98         MOV      R3, #0x51EB851F
.text:0000ACA0         SMULL    R1, R3, R3, R2
.text:0000ACA4         MOV      R1, R3,ASR#4
.text:0000ACA8         MOV      R3, R2,ASR#31
.text:0000ACAC         RSB      R3, R3, R1
.text:0000ACB0         MOV      R1, #50
.text:0000ACB4         MUL      R3, R1, R3
.text:0000ACB8         RSB      R3, R3, R2
.text:0000ACBC         CMP      R3, #0
.text:0000ACC0         BEQ      loc_ACEC
.text:0000ACC4
.text:0000ACC4 errors_with_arguments
.text:0000ACC4                                         
.text:0000ACC4         LDR      R3, [R11,#argv]
.text:0000ACC8         LDR      R3, [R3]
.text:0000ACCC         MOV      R0, R3  ; path
.text:0000ACD0         BL       __xpg_basename
.text:0000ACD4         MOV      R3, R0
.text:0000ACD8         MOV      R0, #aSErrorWithArgu ; format
.text:0000ACE0         MOV      R1, R3
.text:0000ACE4         BL       printf
.text:0000ACE8         B        loc_ADD4
.text:0000ACEC ; ------------------------------------------------------------
.text:0000ACEC
.text:0000ACEC loc_ACEC                 ; CODE XREF: main+66C
.text:0000ACEC         LDR      R2, [R11,#third_argument]
.text:0000ACF0         MOV      R3, #499
.text:0000ACF4         CMP      R2, R3
.text:0000ACF8         BGT      loc_AD08
.text:0000ACFC         MOV      R3, #0x64
.text:0000AD00         STR      R3, [R11,#unk_constant]
.text:0000AD04         B        jump_to_write_power
.text:0000AD08 ; ------------------------------------------------------------
.text:0000AD08
.text:0000AD08 loc_AD08                 ; CODE XREF: main+6A4
.text:0000AD08         LDR      R2, [R11,#third_argument]
.text:0000AD0C         MOV      R3, #799
.text:0000AD10         CMP      R2, R3
.text:0000AD14         BGT      loc_AD24
.text:0000AD18         MOV      R3, #0x5F
.text:0000AD1C         STR      R3, [R11,#unk_constant]
.text:0000AD20         B        jump_to_write_power
.text:0000AD24 ; ------------------------------------------------------------
.text:0000AD24
.text:0000AD24 loc_AD24                 ; CODE XREF: main+6C0
.text:0000AD24         LDR      R2, [R11,#third_argument]
.text:0000AD28         MOV      R3, #899
.text:0000AD2C         CMP      R2, R3
.text:0000AD30         BGT      loc_AD40
.text:0000AD34         MOV      R3, #0x5A
.text:0000AD38         STR      R3, [R11,#unk_constant]
.text:0000AD3C         B        jump_to_write_power
.text:0000AD40 ; ------------------------------------------------------------
.text:0000AD40
.text:0000AD40 loc_AD40                 ; CODE XREF: main+6DC
.text:0000AD40         LDR      R2, [R11,#third_argument]
.text:0000AD44         MOV      R3, #999
.text:0000AD48         CMP      R2, R3
.text:0000AD4C         BGT      loc_AD5C
.text:0000AD50         MOV      R3, #0x55
.text:0000AD54         STR      R3, [R11,#unk_constant]
.text:0000AD58         B        jump_to_write_power
.text:0000AD5C ; ------------------------------------------------------------
.text:0000AD5C
.text:0000AD5C loc_AD5C                 ; CODE XREF: main+6F8
.text:0000AD5C         LDR      R2, [R11,#third_argument]
.text:0000AD60         MOV      R3, #1099
.text:0000AD64         CMP      R2, R3
.text:0000AD68         BGT      jump_to_write_power
.text:0000AD6C         MOV      R3, #0x50
.text:0000AD70         STR      R3, [R11,#unk_constant]
.text:0000AD74
.text:0000AD74 jump_to_write_power                     ; CODE XREF: main+6B0
.text:0000AD74                                         ; main+6CC ...
.text:0000AD74         LDR      R3, [R11,#var_28]
.text:0000AD78         UXTB     R1, R3
.text:0000AD7C         LDR      R3, [R11,#var_2C]
.text:0000AD80         UXTB     R2, R3
.text:0000AD84         LDR      R3, [R11,#unk_constant]
.text:0000AD88         UXTB     R3, R3
.text:0000AD8C         LDR      R0, [R11,#third_argument]
.text:0000AD90         UXTH     R0, R0
.text:0000AD94         STR      R0, [SP,#0x44+var_44]
.text:0000AD98         LDR      R0, [R11,#var_24]
.text:0000AD9C         BL       write_power
.text:0000ADA0         LDR      R0, [R11,#var_24]
.text:0000ADA4         MOV      R1, #0x5A
.text:0000ADA8         BL       read_loop
.text:0000ADAC         B        loc_ADD4

...

.rodata:0000B378 aSErrorWithArgu DCB "%s: Error with arguments",0xA,0 ; DATA XREF: main+684

...

\end{lstlisting}

Function names were present in debugging information of the original binary, like \TT{write\_power}, \TT{read\_loop}.
But labels inside functions were named by me.

\myindex{UNIX!getopt}
\myindex{strtoll()}
\TT{optind} name looks familiar. It is from \IT{getopt} *NIX library intended for command-line parsing---well,
this is exactly what happens inside.
Then, the 3rd argument (where frequency value is to be passed) is converted from a string to a number using
a call to \IT{strtoll()} function.

The value is then checked against various constants.
At 0xACEC, it's checked, if it is lesser or equal to 499, and if it is so,
0x64 is to be passed to \TT{write\_power()} function (which sends a command through USB using \TT{send\_msg()}).
If it is greater than 499, jump to 0xAD08 is occurred.

At 0xAD08 it's checked, if it's lesser or equal to 799. 0x5F is then passed to \TT{write\_power()} function in case of success.

There are more checks: for 899 at 0xAD24, for 0x999 at 0xAD40 and finally, for 1099 at 0xAD5C.
If the input frequency is lesser or equal to 1099, 0x50 will be passed (at 0xAD6C) to \TT{write\_power()} function.
And there is some kind of bug.
If the value is still greater than 1099, the value itself is passed into \TT{write\_power()} function.
Oh, it's not a bug, because we can't get here: value is checked first against 950 at 0xAC88, and if it is greater, error message will be displayed and the utility will finish.

Now the table between frequency in MHz and value passed to \TT{write\_power()} function:

\begin{center}
\begin{longtable}{ | l | l | l | }
\hline
\HeaderColor MHz & \HeaderColor hexadecimal & \HeaderColor decimal \\
\hline
499MHz & 0x64 & 100 \\
\hline
799MHz & 0x5f & 95 \\
\hline
899MHz & 0x5a & 90 \\
\hline
999MHz & 0x55 & 85 \\
\hline
1099MHz & 0x50 & 80 \\
\hline
\end{longtable}
\end{center}

As it seems, a value passed to the board is gradually decreasing during frequency increasing.

Now we see that value of 950MHz is a hardcoded limit, at least in this utility. Can we trick it?

Let's back to this piece of code:

\begin{lstlisting}[style=customasmARM]
.text:0000AC84      LDR     R2, [R11,#third_argument]
.text:0000AC88      MOV     R3, #950
.text:0000AC8C      CMP     R2, R3
.text:0000AC90      BGT     errors_with_arguments ; I've patched here to 00 00 00 00
\end{lstlisting}

We must disable \INS{BGT} branch instruction at 0xAC90 somehow. And this is ARM in ARM mode, because, as we see, all addresses are increasing by 4, i.e., each instruction has size of 4 bytes.
\TT{NOP} (no operation) instruction in ARM mode is just four zero bytes: \TT{00 00 00 00}.
So by writing four zeros at 0xAC90 address (or physical offset in file 0x2C90) we can disable the check.

Now it's possible to set frequencies up to 1050MHz. Even more is possible, but due to the bug, if input value is greater than 1099, a value \IT{as is} in MHz will be passed to the board, which is incorrect.

I didn't go further, but if I had to, I would try to decrease a value which is passed to \TT{write\_power()} function.

Now the scary piece of code which I skipped at first:

\begin{lstlisting}[style=customasmARM]
.text:0000AC94       LDR       R2, [R11,#third_argument]
.text:0000AC98       MOV       R3, #0x51EB851F
.text:0000ACA0       SMULL     R1, R3, R3, R2 ; R3=3rg_arg/3.125
.text:0000ACA4       MOV       R1, R3,ASR#4 ; R1=R3/16=3rg_arg/50
.text:0000ACA8       MOV       R3, R2,ASR#31 ; R3=MSB(3rg_arg)
.text:0000ACAC       RSB       R3, R3, R1 ; R3=3rd_arg/50
.text:0000ACB0       MOV       R1, #50
.text:0000ACB4       MUL       R3, R1, R3 ; R3=50*(3rd_arg/50)
.text:0000ACB8       RSB       R3, R3, R2
.text:0000ACBC       CMP       R3, #0
.text:0000ACC0       BEQ       loc_ACEC
.text:0000ACC4
.text:0000ACC4 errors_with_arguments
\end{lstlisting}

Division via multiplication is used here, and constant is 0x51EB851F.
I wrote a simple programmer's calculator\footnote{\url{https://github.com/DennisYurichev/progcalc}} for myself.
And I have there a feature to calculate modulo inverse.

\begin{lstlisting}
modinv32(0x51EB851F)
Warning, result is not integer: 3.125000
(unsigned) dec: 3 hex: 0x3 bin: 11
\end{lstlisting}

That means that \INS{SMULL} instruction at 0xACA0 is basically divides 3rd argument by 3.125.
In fact, all \TT{modinv32()} function in my calculator does, is this:

\[
\frac{1}{\frac{input}{2^{32}}} = \frac{2^{32}}{input}
\]

Then there are additional shifts and now we see than 3rg argument is just divided by 50.
And then it's multiplied by 50 again.
Why?
This is simplest check, if the input value is can be divided by 50 evenly.
If the value of this expression is non-zero, $x$ can't be divided by 50 evenly:

\[
x-((\frac{x}{50}) \cdot 50)
\]

This is in fact simple way to calculate remainder of division.

And then, if the remainder is non-zero, error message is displayed.
So this utility takes frequency values in form like 850, 900, 950, 1000, etc., but not 855 or 911.

That's it! If you do something like that, please be warned that you may damage your board, just as in case of overclocking other devices like \ac{CPU}s, \ac{GPU}s, etc.
If you have a Cointerra board, do this on your own risk!

