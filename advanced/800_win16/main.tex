\mysection{Windows 16-bit}
\myindex{Windows!Windows 3.x}

\RU{16-битные программы под Windows в наше время редки, хотя иногда можно поработать с ними, в смысле ретрокомпьютинга,
либо которые защищенные донглами (\myref{dongles}).}%
\EN{16-bit Windows programs are rare nowadays, but can be used in the cases of retrocomputing
or dongle hacking (\myref{dongles}).}

\RU{16-битные версии Windows были вплоть до}\EN{16-bit Windows versions were up to} 3.11.
95/98/ME \RU{также поддерживает 16-битный код, как и все 32-битные OS линейки}
\EN{also support 16-bit code, as well as the 32-bit versions of the} \gls{Windows NT}\EN{ line}.
\RU{64-битные версии}\EN{The 64-bit versions of} \gls{Windows NT} \RU{не поддерживают 16-битный код вообще}
\EN{line do not support 16-bit executable code at all}.

\RU{Код напоминает тот что под MS-DOS}\EN{The code resembles MS-DOS's one}.

\RU{Исполняемые файлы имеют NE-тип (так называемый \q{new executable}).}
\EN{Executable files are of type NE-type (so-called \q{new executable}).}

\RU{Все рассмотренные здесь примеры скомпилированы компилятором}
\EN{All examples considered here were compiled by the} OpenWatcom 1.9 \RU{используя эти опции}\EN{compiler, using these switches}:\\

\begin{lstlisting}
wcl.exe -i=C:/WATCOM/h/win/ -s -os -bt=windows -bcl=windows example.c
\end{lstlisting}

\subsection{\Example \#1}

\begin{lstlisting}[style=customc]
#include <windows.h>

int PASCAL WinMain( HINSTANCE hInstance,
                    HINSTANCE hPrevInstance,
                    LPSTR lpCmdLine,
                    int nCmdShow )
{
	MessageBeep(MB_ICONEXCLAMATION);
	return 0;
};
\end{lstlisting}

\begin{lstlisting}[style=customasmx86]
WinMain         proc near
                push    bp
                mov     bp, sp
                mov     ax, 30h ; '0'   ; MB_ICONEXCLAMATION constant
                push    ax
                call    MESSAGEBEEP
                xor     ax, ax          ; return 0
                pop     bp
                retn    0Ah
WinMain         endp
\end{lstlisting}

\RU{Пока всё просто}\EN{Seems to be easy, so far}\FR{Ça semble facile, jusqu'ici}.

\input{\CURPATH/ex2.tex}
\subsection{\Example{} \#3}

\lstinputlisting[style=customc]{\CURPATH/ex3.c}

\lstinputlisting[style=customasmx86]{\CURPATH/ex3.lst}

\RU{Немного расширенная версия примера из предыдущей секции}
\EN{Somewhat extended example from the previous section}
\FR{Exemple un peu plus long de la section précédente}.

\input{\CURPATH/ex4.tex}
\input{\CURPATH/ex5.tex}
\EN{\input{\CURPATH/ex6_EN.tex}}\RU{\input{\CURPATH/ex6_RU.tex}}

