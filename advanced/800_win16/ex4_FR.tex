\subsection{\Example{} \#4}

\label{win16_32bit_values}

\lstinputlisting[style=customc]{\CURPATH/ex4.c}

\lstinputlisting[style=customasmx86]{\CURPATH/ex4.lst}

\myindex{MS-DOS}
Les valeurs 32-bit (le type de donnée \TT{long} implique 32 bits, tandis que \Tint
est 16-bit en code 16-bit (à la fois pour MS-DOS et Win16) sont passées par paires.
C'est tout comme lorsqu'une valeur 64-bit est utilisée dans un environnement 32-bit (\myref{sec:64bit_in_32_env}).

\TT{sub\_B2} 
voici une fonction de bibliothèques écrite par les développeurs du compilateurs qui
fait la \q{multiplication des long} (i.e., multiplie deux valeurs 32-bits).
D'autres fonctions de compilateur qui font la même chose sont listées ici: \myref{sec:MSVC_library_func}, \myref{sec:GCC_library_func}.

\myindex{x86!\Instructions!ADD}
\myindex{x86!\Instructions!ADC}
La paire d'instructions \TT{ADD}/\TT{ADC} est utilisée pour l'addition de valeurs
composées: \TT{ADD} peut mettre le flag \TT{CF} à 0/1, et \TT{ADC} l'utilise après.

La paire d'instructions \TT{SUB}/\TT{SBB} est utilisée pour la soustraction: \TT{SUB}
peut mettre la flag \TT{CF} à 0/1, et \TT{SBB} l'utilise après.

Les valeurs 32-bit sont renvoyées de la fonction dans la paire de registres \TT{DX:AX}.

Les constantes sont aussi passées par paires dans \TT{WinMain()} ici.

\myindex{x86!\Instructions!CWD}
La constante 123 typée \Tint{} est d'abord converti suivant le signe de la valeur
32-bit en utilisant l'instruction \TT{CWD}.
