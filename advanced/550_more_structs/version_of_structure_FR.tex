\subsection{Version de structure C}

De nombreux programmeurs Windows ont vu ceci dans MSDN:

\begin{lstlisting}
SizeOfStruct
    The size of the structure, in bytes. This member must be set to sizeof(SYMBOL_INFO).
\end{lstlisting}

( \url{https://msdn.microsoft.com/en-us/library/windows/desktop/ms680686(v=vs.85).aspx} )

En effet, certaines structures comme \IT{SYMBOL\_INFO} commencent en effet avec ce
champ.  Pourquoi ?
C'est une sorte de version de structure.

Imaginez que vous avez une fonction qui dessine un cercle.
Elle prend un unique argument---un pointeur sur une structure avec seulement trois
champs: X, Y et radius.
Et puis, les affichages couleurs ont inondé le marché durant les années 1980. Et
vous voulez ajouter un argument \IT{color} à la fonction.
Mais, disons que vous ne pouvez pas lui ajouter un argument (de nombreux logiciels
utilisent votre \ac{API} et ne peuvent pas être recompilés).
Et si un vieux logiciels utilise votre \ac{API} avec un affichage couleur, faites
que votre fonction dessine un cercle avec par défaut les couleurs noire et blanche.

Un autre jour, vous ajoutez une autre possibilité: le cercle peut maintenant être
rempli, et le type de brosse peut être passé.

Voici une solution à ce problème:

\begin{lstlisting}[style=customc]
#include <stdio.h>

struct ver1
{
	size_t SizeOfStruct;
	int coord_X;
	int coord_Y;
	int radius;
};

struct ver2
{
	size_t SizeOfStruct;
	int coord_X;
	int coord_Y;
	int radius;
	int color;
};

struct ver3
{
	size_t SizeOfStruct;
	int coord_X;
	int coord_Y;
	int radius;
	int color;
	int fill_brush_type; // 0 - ne pas remplir le cercle
};

void draw_circle(struct ver3 *s) // latest struct version is used here
{
	// we presume SizeOfStruct, coord_X and coord_Y fields are always present
	printf ("We are going to draw a circle at %d:%d\n", s->coord_X, s->coord_Y);

	if (s->SizeOfStruct>=sizeof(int)*5)
	{
		// this is at least ver2, color field is present
		printf ("We are going to set color %d\n", s->color);
	}

	if (s->SizeOfStruct>=sizeof(int)*6)
	{
		// this is at least ver3, fill_brush_type field is present
		printf ("We are going to fill it using brush type %d\n", s->fill_brush_type);
	}
};

// early software version
void call_as_ver1()
{
	struct ver1 s;
	s.SizeOfStruct=sizeof(s);
	s.coord_X=123;
	s.coord_Y=456;
	s.radius=10;
	printf ("** %s()\n", __FUNCTION__);
	draw_circle(&s);
};

// next software version
void call_as_ver2()
{
	struct ver2 s;
	s.SizeOfStruct=sizeof(s);
	s.coord_X=123;
	s.coord_Y=456;
	s.radius=10;
	s.color=1;
	printf ("** %s()\n", __FUNCTION__);
	draw_circle(&s);
};

// latest, most advanced version
void call_as_ver3()
{
	struct ver3 s;
	s.SizeOfStruct=sizeof(s);
	s.coord_X=123;
	s.coord_Y=456;
	s.radius=10;
	s.color=1;
	s.fill_brush_type=3;
	printf ("** %s()\n", __FUNCTION__);
	draw_circle(&s);
};

int main()
{
	call_as_ver1();
	call_as_ver2();
	call_as_ver3();
};
\end{lstlisting}

Autrement dit, le champ \IT{SizeOfStruct} prend le rôle d'un champ \IT{version of structure}.
Il pourrait être un type énuméré (1, 2, 3, etc.), mais mettre le champ \IT{SizeOfStruct}
à \IT{sizeof(struct...)} est moins sujet à l'erreur: nous écrivons simplement \IT{s.SizeOfStruct=sizeof(...)}
dans le code de l'appelant.

En C++, ce problème est résolu en utilisant l'\IT{hérutage} (\myref{cpp_inheritance}).
Vous avez seulement à étendre la classe de base (appelons la \IT{Circle}), puis vous
aurez une classe \IT{ColoredCircle}, et ensuite \IT{FilledColoredCircle}, et ainsi
de suite.
La version courante d'un objet (ou, plus précisemment, le \IT{type} courant) sera
déterminé en utilisant la \ac{RTTI} de C++.

Donc lorsque vous voyez \IT{SizeOfStruct} quelque part dans \ac{MSDN}---peut-être
que cette structure a été étendue au moins une fois par le passé.

