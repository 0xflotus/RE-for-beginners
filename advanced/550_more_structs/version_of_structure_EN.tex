\subsection{Version of C structure}

Many Windows programmers have seen this is MSDN:

\begin{lstlisting}
SizeOfStruct
    The size of the structure, in bytes. This member must be set to sizeof(SYMBOL_INFO).
\end{lstlisting}

( \url{https://msdn.microsoft.com/en-us/library/windows/desktop/ms680686(v=vs.85).aspx} )

Some structures like \IT{SYMBOL\_INFO} has started with this field indeed. Why?
This is some kind of structure version.

Imagine you have a function which draws circle.
It takes a single argument---a pointer to a structure with only three fields: X, Y and radius.
And then color displays flooded a market, sometimes in 1980s. And you want to add \IT{color} argument to the function.
But, let's say, you cannot add another argument to it (a lot of software use your \ac{API} and cannot be recompiled).
And if the old piece of software uses your \ac{API} with color display,
let your function draw a circle in (default) black and white colors.

Another day you add another feature: circle now can be filled, and brush type can be set.

Here is one solution to the problem:

\begin{lstlisting}[style=customc]
#include <stdio.h>

struct ver1
{
	size_t SizeOfStruct;
	int coord_X;
	int coord_Y;
	int radius;
};

struct ver2
{
	size_t SizeOfStruct;
	int coord_X;
	int coord_Y;
	int radius;
	int color;
};

struct ver3
{
	size_t SizeOfStruct;
	int coord_X;
	int coord_Y;
	int radius;
	int color;
	int fill_brush_type; // 0 - do not fill circle
};

void draw_circle(struct ver3 *s) // latest struct version is used here
{
	// we presume SizeOfStruct, coord_X and coord_Y fields are always present
	printf ("We are going to draw a circle at %d:%d\n", s->coord_X, s->coord_Y);

	if (s->SizeOfStruct>=sizeof(int)*5)
	{
		// this is at least ver2, color field is present
		printf ("We are going to set color %d\n", s->color);
	}

	if (s->SizeOfStruct>=sizeof(int)*6)
	{
		// this is at least ver3, fill_brush_type field is present
		printf ("We are going to fill it using brush type %d\n", s->fill_brush_type);
	}
};

// early software version
void call_as_ver1()
{
	struct ver1 s;
	s.SizeOfStruct=sizeof(s);
	s.coord_X=123;
	s.coord_Y=456;
	s.radius=10;
	printf ("** %s()\n", __FUNCTION__);
	draw_circle(&s);
};

// next software version
void call_as_ver2()
{
	struct ver2 s;
	s.SizeOfStruct=sizeof(s);
	s.coord_X=123;
	s.coord_Y=456;
	s.radius=10;
	s.color=1;
	printf ("** %s()\n", __FUNCTION__);
	draw_circle(&s);
};

// latest, most advanced version
void call_as_ver3()
{
	struct ver3 s;
	s.SizeOfStruct=sizeof(s);
	s.coord_X=123;
	s.coord_Y=456;
	s.radius=10;
	s.color=1;
	s.fill_brush_type=3;
	printf ("** %s()\n", __FUNCTION__);
	draw_circle(&s);
};

int main()
{
	call_as_ver1();
	call_as_ver2();
	call_as_ver3();
};
\end{lstlisting}

In other words, \IT{SizeOfStruct} field takes a role of \IT{version of structure} field.
It could be enumerate type (1, 2, 3, etc.), but to set \IT{SizeOfStruct} field to \IT{sizeof(struct...)}
is less prone to mistakes/bugs: we just write \IT{s.SizeOfStruct=sizeof(...)} in caller's code.

In C++, this problem is solved using \IT{inheritance} (\myref{cpp_inheritance}).
You just extend your base class (let's call it \IT{Circle}),
and then you will have \IT{ColoredCircle} and then \IT{FilledColoredCircle}, and so on.
A current \IT{version} of an object (or, more precisely, current \IT{type}) will be determined using C++ \ac{RTTI}.

So when you see \IT{SizeOfStruct} somewhere in \ac{MSDN}---perhaps this structure was extended at least once in past.

