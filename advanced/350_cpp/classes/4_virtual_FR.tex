\subsubsection{Méthodes virtuelles}

Encode un exemple simple:

\lstinputlisting[style=customc]{\CURPATH/classes/classes4_virtual.cpp}


La classe \IT{object} a une méthode virtuelle \TT{dump()} qui est remplacée par celle
de la classe héritant \IT{box} et \IT{sphere}.


Si nous sommes dans un environnement où le type de l'objet n'est pas connu, comme
dans la fonction \main de l'exemple, où la méthode virtuelle \TT{dump()} est appelée,
l'information à propos de son type doit être stockée quelque part, afin d'être capable
d'appeler la bonne méthode virtuelle.


Compilons-le dans MSVC 2008 avec les options \Ox et \Obzero, puis regardons le code
de \main:

\lstinputlisting[style=customasmx86]{\CURPATH/classes/classes4_1.asm}


Un pointeur sur la fonction \TT{dump()} est pris quelque part dans l'objet.
Où pourrions-nous stocker l'adresse de la nouvelle méthode?
Seulement quelque part dans le constructeur: il n'y a pas d'autre endroit puisque rien
d'autre n'est appelé dans la fonction \main.
\footnote{Vous pouvez en lire plus sur les pointeurs sur les fonctions dans la section afférente:(\myref{sec:pointerstofunctions}).}


Regardons le code du constructeur de la classe \IT{box}:

\lstinputlisting[style=customasmx86]{\CURPATH/classes/classes4_2.asm}


Ici, nous avons une disposition de la mémoire légèrement différente:
Le premier champ est un pointeur sur une table \TT{box::`vftable'} (le nom a été
mis par le compilateur MSVC).

\label{RTTI}
\myindex{\Cpp!RTTI}

Dans cette table nous voyons un lien sur une table nommée \\
\TT{box::`RTTI Complete Object Locator'} et aussi un lien \\
sur la méthode \TT{box::dump()}.

Elles sont appelées table de méthodes virtuelles et \ac{RTTI}.
La table de méthodes virtuelles a l'adresse des méthodes et la table \ac{RTTI} contient
l'information à propos des types.

À propos, les tables \ac{RTTI} sont utilisées lors de l'appel à \IT{dynamic\_cast}
et \IT{typeid} en \Cpp.
Vous pouvez également voir ici le nom de la classe en chaîne de texte pur.

Ainsi, une méthode de la classe de base \IT{object} peut aussi appelé la méthode
virtuelle \IT{object::dump()}, qui, en fait, va appeler une méthode d'une classe
héritée, puisque cette information est présente juste dans la structure de l'objet.


Du temps CPU additionnel est requis pour faire la recherche dans ces tables et trouver
l'adresse de la bonne  méthode virtuelle, ainsi les méthodes virtuelles sont largement
considérées comme légèrement plus lentes que les méthodes normales.

Dans le code généré par GCC, les tables \ac{RTTI} sont construites légèrement différemment.
% TODO: добавить...
