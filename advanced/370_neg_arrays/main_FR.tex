\mysection{Index de tableau négatifs}
\label{negative_array_indices}

Il est possible d'accéder à l'espace \IT{avant} un tableau en donnant un index négatif,
e.g., $array[-1]$.

\subsection{Accéder à une chaîne depuis la fin}

\myindex{Python}
Python \ac{PL} permet d'accéder aux tableaux depuis la fin.
Par exemple, \IT{string[-1]} renvoie le dernier caractère, \IT{string[-2]} renvoie
le pénultième, etc.
Difficile à croire, mais ceci est aussi possible en \CCpp:

\lstinputlisting[style=customc]{\CURPATH/pythonesque_addressing.c}

Ça fonctionne, mais \textit{s\_end} doit toujours avoir l'adresse du zéro en fin
de la chaîne \textit{s}. Si la taille de la chaîne \textit{s} est modifiée,
\textit{s\_end} doit être modifié aussi.

L'astuce est douteuse, mais, encore une fois, ceci est une démonstration d'indices
négatifs.

\subsection{Accéder à un bloc quelconque depuis la fin}

Rappelons d'abord pourquoi la pile grossit en arrière (\myref{stack_grow_backwards}).
Il y a une sorte de bloc en mémoire et vous voulez y stocker à la fois le heap et
la pile, sans savoir comment ils vont grossir pendant l'exécution.

Vous pouvez mettre un pointeur de \IT{heap} au début du bloc, puis vous mettez un
pointeur de \IT{pile} à la fin du bloc (\IT{heap + size\_of\_block}), et vous pouvez
accéder au \IT{n-ième} élément de la pile avec \IT{stack[-n]}.
Par exemple, \IT{stack[-1]} pour le 1er élément, \IT{stack[-2]} pour le 2nd, etc.

Ceci fonctionnera de la même façon que notre astuce d'accéder la chaîne depuis la
fin.

Vous pouvez facilement vérifier si les structures n'ont pas commencé à se recouvrir:
il suffit d'être sûr que l'adresse du dernier élément dans le \IT{heap} est plus
bas que le dernier élément de la \IT{pile}.

Malheureusement, l'index  $-0$ ne fonctionnera pas, puisque la représentation des
nombres négatifs en complément à deux (\myref{sec:signednumbers}) ne permet pas de
zéro négatif. donc il ne peut pas être distingué d'un zéro positif.

Ceci est aussi mentionné dans ``Transaction processing'', Jim Gray, 1993,
chapitre ``The Tuple-Oriented File System'', p. 755.

\subsection{Tableaux commençants à 1}
\label{arrays_at_one}

\myindex{Fortran}
\myindex{Mathematica}
Fortran et Mathematica définissent le premier élément d'un tableau avec l'indice 1,
sans doute parce que c'est la tradition en mathématiques.
D'autres \ac{PL}s comme \CCpp le définisse avec l'indice 0.
Lequel est le meilleur?
\myindex{Edsger W. Dijkstra}
Edsger W. Dijkstra prétend que le dernier est le meilleur
\footnote{Voir \url{https://www.cs.utexas.edu/users/EWD/transcriptions/EWD08xx/EWD831.html}}.

Mais les programmeurs peuvent toujours avoir l'habitude après le Fortran, donc avec
ce petit truc, il est possible d'accéder au premier élément en \CCpp en utilisant l'indice 1:

\lstinputlisting[style=customc]{\CURPATH/neg_array.c}

\lstinputlisting[caption=MSVC 2010 \NonOptimizing,label=neg_array_c,numbers=left,style=customasmx86]{\CURPATH/neg_array_FR.asm}

Donc nous avons le tableau \TT{array[]} de dix éléments, rempli avec les octets $0 \ldots 9$.

Puis nous avons le pointeur \TT{fakearray[]}, qui pointe un octet avant \TT{array[]}.

\TT{fakearray[1]} pointe exactement sur \TT{array[0]}.
Mais nous sommes toujours curieux, qu'y a-t-il avant \TT{array[]}?
Nous avons ajouté \TT{random\_value} avant \TT{array[]} et l'avons défini à \TT{0x11223344}.
Le compilateur sans optimisation a alloué les variables dans l'ordre dans lequel
elles sont déclarées, donc oui, la valeur 32-bit \TT{random\_value} est juste avant
le tableau.

Nous le lançons, et:

\begin{lstlisting}
first element 0
second element 1
last element 9
array[-1]=11, array[-2]=22, array[-3]=33, array[-4]=44
\end{lstlisting}

Voici le fragment de pile que nous avons copier/coller depuis la fenêtre de pile
d'\olly (avec les commentaires ajoutés par l'auteur):

\lstinputlisting[caption=MSVC 2010 \NonOptimizing]{\CURPATH/stack_FR.txt}

Le pointeur sur \TT{fakearray[]} (\TT{0x001DFBD3}) est en effet l'adresse de \TT{array[]}
dans la pile (\TT{0x001DFBD4}), mais moins 1 octet.

C'est un truc très astucieux et douteux. Personne ne devrait l'utiliser dans du code
de production mais comme démonstration, ça joue parfaitement son rôle.
