\mysection{Boucles: quelques itérateurs}
\label{loop_iterators}

Dans la plupart des cas, les boucles ont un seul itérateur, mais elles peuvent en
avoir plusieurs dans le code résultant.

Voici un exemple très simple:

\lstinputlisting[style=customc]{\CURPATH/iterators_FR.c}

Il y a deux multiplications à chaque itération et ce sont des opérations coûteuses.
Est-ce que ça peut être optimisé d'une certaine façon?

Oui, si l'on remarque que les deux indices de tableau prennent des valeurs qui peuvent
être facilement calculées sans multiplication.

\subsection{Trois itérateurs}

\lstinputlisting[caption=MSVC 2013 x64 \Optimizing,style=customasmx86]{\CURPATH/MSVC_2013_x64_Ox_FR.asm}

Maintenant il y a 3 itérateurs: la variable \IT{cnt} et deux indices, qui sont incrémentés
par 12 et 28 à chaque itération.
Nous pouvons récrire ce code en \CCpp:

\lstinputlisting[style=customc]{\CURPATH/iterators3_FR.c}

Donc, au prix de la mise à jour de 3 itérateurs à chaque itération au lieu d'un,
nous pouvons supprimer deux opérations de multiplication.

\subsection{Deux itérateurs}

GCC 4.9 fait encore mieux, en ne laissant que 2 itérateurs:

\lstinputlisting[caption=GCC 4.9 x64 \Optimizing,style=customasmx86]{\CURPATH/GCC491_x64_O3_FR.asm}

Il n'y a plus de variable \IT{counter}: GCC en a conclu qu'elle n'étais pas nécessaire.

Le dernier élément du tableau \IT{a2} est calculé avant le début de la boucle (ce qui
est facile: $cnt*7$) et c'est ainsi que la boucle est arrêtée: itérer jusqu'à ce que
le second index atteignent cette valeur pré-calculée.

Vous trouverez plus d'informations sur la multiplication en utilisant des décalages/additions/soustractions
ici:
\myref{multiplication_using_shifts_adds_subs}.

Ce code peut être récrit en \CCpp comme ceci:

\lstinputlisting[style=customc]{\CURPATH/iterators2_FR.c}

GCC (Linaro) 4.9 pour ARM64 fait la même chose, mais il pré-calcule le dernier index
de \IT{a1} au lieu de \IT{a2}, ce qui a bien sûr le même effet:

\lstinputlisting[caption=GCC (Linaro) 4.9 ARM64 \Optimizing,label=GCC_no_number_sign,style=customasmARM]{\CURPATH/ARM64_GCC_49_O3_FR.s}

% FIXME1 -> post-increment

GCC 4.4.5 pour MIPS fait la même chose:

\lstinputlisting[caption=GCC 4.4.5 for MIPS \Optimizing (IDA),style=customasmMIPS]{\CURPATH/MIPS_O3_IDA_FR.lst}

\subsection{Cas Intel C++ 2011}
\myindex{\CompilerAnomaly}
\label{loops_iterators_loop_anomaly}

Les optimisations du compilateur peuvent être bizarre, mais néanmoins toujours correctes.
Voici ce qu le compilateur Intel C++ 2011 effectue:

\lstinputlisting[caption=Intel C++ 2011 x64 \Optimizing,style=customasmx86]{\CURPATH/intel_2011_x64_Ox.asm}

Tout d'abord, quelques décisions sont prises, puis une des routines est exécutée.

Il semble qu'il teste si les tableaux se recoupent.

C'est une façons très connue d'optimiser les routines de copie de blocs de mémoire.
Mais les routines de copie sont les même!

ça doit être une erreur de l'optimiseur Intel C++, qui produit néanmoins un code
fonctionnel.

Nous prenons volontairement en compte de tels exemples dans ce livre, afin que lecteur
comprenne que le ce que génère un compilateur est parfois bizarre mais toujours correct,
car lorsque le compilateur a été testé, il a réussi les tests.
