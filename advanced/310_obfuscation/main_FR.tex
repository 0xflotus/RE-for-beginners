\mysection{Obfuscation}

L'obfuscation est une tentative de cacher le code (ou sa signification) aux rétro-ingénieurs.

\subsection{Chaînes de texte}

% TBT
Comme nous l'avons vu dans (\myref{sec:digging_strings}), les chaînes de texte peuvent
être vraiment utiles.

Les programmeurs qui sont conscients de ceci essayent de les cacher, rendant impossible
de trouver la chaîne dans \IDA{} ou tout autre éditeur hexadécimal.

Voici la méthode la plus simple.

La chaîne peut être construite comme ceci:

\begin{lstlisting}[style=customasmx86]
mov     byte ptr [ebx], 'h'
mov     byte ptr [ebx+1], 'e'
mov     byte ptr [ebx+2], 'l'
mov     byte ptr [ebx+3], 'l'
mov     byte ptr [ebx+4], 'o'
mov     byte ptr [ebx+5], ' '
mov     byte ptr [ebx+6], 'w'
mov     byte ptr [ebx+7], 'o'
mov     byte ptr [ebx+8], 'r'
mov     byte ptr [ebx+9], 'l'
mov     byte ptr [ebx+10], 'd'
\end{lstlisting}

La chaîne peut aussi être comparée avec une autre comme ceci:

\begin{lstlisting}[style=customasmx86]
mov	ebx, offset username
cmp	byte ptr [ebx], 'j'
jnz	fail
cmp	byte ptr [ebx+1], 'o'
jnz	fail
cmp	byte ptr [ebx+2], 'h'
jnz	fail
cmp	byte ptr [ebx+3], 'n'
jnz	fail
jz	it_is_john
\end{lstlisting}

Dans les deux cas, il est impossible de trouver ces chaînes directement dans un éditeur
hexadécimal.

\myindex{Shellcode}

À propos, ceci est un moyen de travailler avec des chaînes lorsqu'il est impossible
d'allouer de l'espace pour elles dans le segment de données, par exemple dans un
\ac{PIC} ou un shellcode.

Une autre méthode est d'utiliser \TT{sprintf()} pour la construction:

\begin{lstlisting}[style=customc]
sprintf(buf, "%s%c%s%c%s", "hel",'l',"o w",'o',"rld");
\end{lstlisting}

Le code semble bizarre, mais peut être utile comme simple mesure anti-reversing.

Les chaînes de texte peuvent aussi être présentes dans une forme chiffrée, donc chaque
utilisation d'une chaîne est précédée par une routine de déchiffrement.
Par exemple: \myref{examples_SCO}.

\subsection{Code exécutable}

\subsubsection{Insertion de code inutile}

L'obfuscation de code exécutable implique l'insertion aléatoire de code inutile
dans le code réel, qui s'exécute mais ne fait rien d'utile.

Un simple exemple:

\begin{lstlisting}[caption=code original,style=customasmx86]
add	eax, ebx
mul	ecx
\end{lstlisting}

\lstinputlisting[caption=code obfusqué,style=customasmx86]{\CURPATH/1_FR.asm}

Ici, le code inutile utilise des registres qui ne sont pas utilisés dans le code
réel (\TT{ESI} et \TT{EDX}).
Toutefois, les résultats intermédiaires produit par le code réel peuvent être utilisés
par les instructions inutiles pour brouiller les pistes---pourquoi pas?

\subsubsection{Remplacer des instructions avec des équivalents plus gros}

\begin{itemize}
\item \TT{MOV op1, op2} peut être remplacé par la paire \TT{PUSH op2 / POP op1}.
\item \TT{JMP label} peut être remplacé par la paire \TT{PUSH label / RET}. 
\IDA{} ne montrera pas la référence au label.
\item \TT{CALL label} peut être remplacé par le triplet d'instructions suivant:\\
\TT{PUSH label\_after\_CALL\_instruction / PUSH label / RET}.
\item \TT{PUSH op} peut aussi être remplacé par la paire d'instructions suivante:\\
\TT{SUB ESP, 4 (or 8) / MOV [ESP], op}.%

\end{itemize}

\subsubsection{Code toujours exécuté/jamais exécuté}

Si le développeur est certain que ESI contient toujours 0 à ce point:

\lstinputlisting[style=customasmx86]{\CURPATH/2_FR.asm}

Le rétro-ingénieur a parfois besoin de temps pour le comprendre.

\myindex{opaque predicate}
Ceci est aussi appelé un \IT{prédicat opaque}.

Un autre exemple (et de nouveau, le développeur est certain que ESI vaut toujours
zéro):

\lstinputlisting[style=customasmx86]{\CURPATH/3_FR.asm}

\subsubsection{Mettre beaucoup de bazar}

\begin{lstlisting}
instruction 1
instruction 2
instruction 3
\end{lstlisting}

Peut être remplacé par:

\begin{lstlisting}[style=customasmx86]
begin:		jmp	ins1_label

ins2_label:	instruction 2
		jmp	ins3_label

ins3_label:	instruction 3
		jmp	exit:

ins1_label:	instruction 1
		jmp	ins2_label
exit:
\end{lstlisting}

\subsubsection{Utilisation de pointeurs indirects}

\begin{lstlisting}[style=customasmx86]
dummy_data1	db	100h dup (0)
message1	db	'hello world',0

dummy_data2	db	200h dup (0)
message2	db	'another message',0

func		proc
		...
		mov	eax, offset dummy_data1 ; PE or ELF reloc here
		add	eax, 100h
		push	eax
		call	dump_string
		...
		mov	eax, offset dummy_data2 ; PE or ELF reloc here
		add	eax, 200h
		push	eax
		call	dump_string
		...
func		endp
\end{lstlisting}

\IDA{} montrera seulement les références à \TT{dummy\_data1} et \TT{dummy\_data2},
mais pas aux chaînes de textes.

Les variables globales et même les fonctions peuvent être accédées comme ça.

\subsection{Machine virtuelle / pseudo-code}

Un programmeur peut construire son propre \ac{PL} ou \ac{ISA} et son interpréteur.

(Comme le Visual Basic pre-5.0, .NET ou les machines Java).
Le rétro-ingénieur aura besoin de passer du temps pour comprendre la signification
et les détails de toutes les instructions de l'\ac{ISA}.

Il/elle devra aussi une sorte de désassembleur/décompilateur.
% TODO about obfuscation VMs

\subsection{Autres choses à mentionner}

Ma propre (faible pour le moment) tentative de patch du compilateur Tiny C pour
produire du code obfusqué: \url{http://go.yurichev.com/17220}.

Utiliser l'instruction \MOV pour des choses vraiment compliquées:
[Stephen Dolan, \IT{mov is Turing-complete}, (2013)]
\footnote{\AlsoAvailableAs \url{http://www.cl.cam.ac.uk/~sd601/papers/mov.pdf}}. 

\subsection{\Exercise}

\begin{itemize}
	\item \url{http://challenges.re/29}
\end{itemize}

