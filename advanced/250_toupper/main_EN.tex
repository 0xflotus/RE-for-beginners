\mysection{toupper() function}
\myindex{\CStandardLibrary!toupper()}

Another very popular function transforms a symbol from lower case to upper case, if needed:

\lstinputlisting[style=customc]{\CURPATH/toupper.c}

The \TT{'a'+'A'} expression is left in the source code for better readability, it will be 
optimized by compiler, of course
\footnote{However, to be meticulous, there still could be compilers which can't optimize such expressions
and will leave them right in the code.}.

The \ac{ASCII} code of \q{a} is 97 (or 0x61), and 65 (or 0x41) for \q{A}.

The difference (or distance) between them in the \ac{ASCII} table is 32 (or 0x20).

For better understanding, the reader may take a look at the 7-bit standard \ac{ASCII} table:

\begin{figure}[H]
\centering
\includegraphics[width=0.7\textwidth]{ascii.png}
\caption{7-bit \ac{ASCII} table in Emacs}
\end{figure}

\subsection{x64}

\subsubsection{Two comparison operations}

\NonOptimizing MSVC is straightforward: the code checks if the input symbol is in [97..122] range 
(or in [`a'..`z'] range) and subtracts 32 if it's true.

There are also some minor compiler artifact:

\lstinputlisting[caption=\NonOptimizing MSVC 2013 (x64),numbers=left,style=customasmx86]{\CURPATH/MSVC_2013_x64_EN.asm}

It's important to notice that the input byte is loaded into a 64-bit local stack slot at line 3.

All the remaining bits ([8..63]) are untouched, i.e., contain some random noise (you'll see it in debugger).

% TODO add debugger example
All instructions operate only on byte-level, so it's fine.

The last \TT{MOVZX} instruction at line 15 takes the byte from the local stack slot and zero-extends it to a \Tint 
32-bit data type.

\NonOptimizing GCC does mostly the same:

\lstinputlisting[caption=\NonOptimizing GCC 4.9 (x64),style=customasmx86]{\CURPATH/GCC_49_x64_O0.s}

\subsubsection{One comparison operation}
\label{toupper_one_comparison}

\Optimizing MSVC does a better job, it generates only one comparison operation:

\lstinputlisting[caption=\Optimizing MSVC 2013 (x64),style=customasmx86]{\CURPATH/MSVC_2013_Ox_x64.asm}

It was explained earlier how to replace the two comparison operations with a single one: \myref{one_comparison_instead_of_two}.

We will now rewrite this in \CCpp:

\begin{lstlisting}[style=customc]
int tmp=c-97;

if (tmp>25)
        return c;
else
        return c-32;
\end{lstlisting}

The \IT{tmp} variable must be signed.

This makes two subtraction operations in case of a transformation plus one comparison.

In contrast the original algorithm uses two comparison operations plus one subtracting.

\Optimizing GCC is even better, it gets rid of the jumps (which is good: \myref{branch_predictors}) 
by using the CMOVcc instruction:

\lstinputlisting[caption=\Optimizing GCC 4.9 (x64),numbers=left,style=customasmx86,label=toupper_GCC_O3]{\CURPATH/GCC_49_x64_O3.s}

At line 3 the code prepares the subtracted value in advance, as if the conversion will always happen.

At line 5 the subtracted value in EAX is replaced by the untouched input value if a conversion is not needed.
And then this value (of course incorrect) is dropped.

Advance subtracting is a price the compiler pays for the absence of conditional jumps.

\subsection{ARM}

\Optimizing Keil for ARM mode also generates only one comparison:

\lstinputlisting[caption=\OptimizingKeilVI (\ARMMode),style=customasmARM]{\CURPATH/Keil_ARM_O3.s}

\myindex{ARM!\Instructions!SUBcc}
\myindex{ARM!\Instructions!ANDcc}
The SUBLS and ANDLS instructions are executed only if the value in \Reg{1} is less than 0x19 (or equal).
They also do the actual conversion.

\Optimizing Keil for Thumb mode generates only one comparison operation as well:

\lstinputlisting[caption=\OptimizingKeilVI (\ThumbMode),style=customasmARM]{\CURPATH/Keil_thumb_O3.s}

\myindex{ARM!\Instructions!LSLS}
\myindex{ARM!\Instructions!LSLR}
The last two LSLS and LSRS instructions work like \TT{AND reg, 0xFF}:
they are equivalent to the \CCpp-expression $(i<<24)>>24$.

Seems like that Keil for Thumb mode deduced that two 2-byte instructions are shorter than the code 
that loads the 0xFF constant into a register plus an AND instruction.

\subsubsection{GCC for ARM64}

\lstinputlisting[caption=\NonOptimizing GCC 4.9 (ARM64),style=customasmARM]{\CURPATH/GCC_49_ARM64_O0.s}

\lstinputlisting[caption=\Optimizing GCC 4.9 (ARM64),style=customasmARM]{\CURPATH/GCC_49_ARM64_O3.s}

\subsection{Using bit operations}
\label{toupper_bit}

Given the fact that 5th bit (counting from 0th) is always present after the check, subtracting is merely clearing
this sole bit, but the very same effect can be achieved with ANDing (\myref{AND_OR_as_SUB_ADD}).

Even simpler, with XOR-ing:

\lstinputlisting[style=customc]{\CURPATH/toupper2.c}

The code is close to what the optimized GCC has produced for the previous example (\myref{toupper_GCC_O3}):

\lstinputlisting[caption=\Optimizing GCC 5.4 (x86),style=customasmx86]{\CURPATH/toupper2_GCC540_x86_O3.s}

\dots but \INS{XOR} is used instead of \INS{SUB}.

Flipping 5th bit is just moving a \textit{cursor} in \ac{ASCII} table up and down by two rows.

Some people say that lowercase/uppercase letters has been placed in the \ac{ASCII} table in such a way deliberately,
because:

\begin{framed}
\begin{quotation}
Very old keyboards used to do Shift just by toggling the 32 or 16 bit, depending on the key; this is why the relationship between small and capital letters in ASCII is so regular, and the relationship between numbers and symbols, and some pairs of symbols, is sort of regular if you squint at it.
\end{quotation}
\end{framed}

( Eric S. Raymond, \url{http://www.catb.org/esr/faqs/things-every-hacker-once-knew/} )

Therefore, we can write this piece of code, which just flips the case of letters:

\lstinputlisting[style=customc]{\CURPATH/flip_EN.c}

\subsection{Summary}

All these compiler optimizations are very popular nowadays 
and a practicing reverse engineer usually sees such code patterns often.

