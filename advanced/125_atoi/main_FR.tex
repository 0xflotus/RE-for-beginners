\mysection{Conversion de chaîne en nombre (atoi())}

\myindex{\CStandardLibrary!atoi()}

Essayons de ré-implémenter la fonction C standard atoi().

\subsection{Exemple simple}

Voici la manière la plus simple possible de lire un nombre encodé en \ac{ASCII}.

C'est sujet aux erreurs: un caractère autre qu'un nombre conduit à un résultat incorrect.

\lstinputlisting[style=customc]{\CURPATH/atoi.c}

Donc, tout ce que fait l'algorithme, c'est de lire les chiffres de gauche à droite.

Le caractère \ac{ASCII} zéro est soustrait de chaque chiffre.

Les chiffres de \q{0} à \q{9} sont consécutifs dans la table \ac{ASCII}, donc nous
n'avons même pas besoin de connaître la valeur exacte du caractère \q{0}.

Tout ce que nous avons  besoin de savoir, c'est que \q{0} moins \q{0} vaut 0, \q{9}
moins \q{0} vaut 9, et ainsi de suite.

Soustraire \q{0} de chaque caractère résulte en un nombre de 0 à 9 inclus.

Tout autre caractère conduit à un résultat incorrect, bien sûr!

Chaque chiffre doit être ajouté au résultat final (dans la variable \q{rt}), mais
le résultat final est aussi multiplié par 10 à chaque chiffre.

Autrement dit, le résultat est décalé à gauche d'une position au format décimal à
chaque itération.

Le dernier chiffre est ajouté, mais il n'y a pas de décalage.

\subsubsection{MSVC 2013 x64 \Optimizing}

\lstinputlisting[caption=MSVC 2013 x64 \Optimizing,style=customasmx86]{\CURPATH/atoi.asm.MSVC2013.x64.Ox_FR}

Un caractère peut-être chargé à deux endroits: le premier caractère et tous les caractères subséquents.
Ceci est arrangé de cette manière afin de regrouper les boucles.
\myindex{x86!\Instructions!LEA}

Il n'y a pas d'instructions pour multiplier par 10, à la place, deux instructions
LEA le font.
\myindex{x86!\Instructions!ADD}
\myindex{x86!\Instructions!SUB}

Parfois, MSVC utilise l'instruction ADD avec une constante négative à la place d'un SUB.
C'est le cas.

C'est très difficile de dire pourquoi c'est meilleur que SUB.
Mais MSVC fait souvent ceci.

\subsubsection{GCC 4.9.1 x64 \Optimizing}

GCC 4.9.1 \Optimizing est plus concis, mais il y a une instruction RET redondante
à la fin.
Une suffit.

\lstinputlisting[caption=GCC 4.9.1 x64 \Optimizing,style=customasmx86]{\CURPATH/atoi.s.GCC491.O3.x64_FR}

\subsubsection{\OptimizingKeilVI (\ARMMode)}

\lstinputlisting[caption=\OptimizingKeilVI (\ARMMode),style=customasmARM]{\CURPATH/atoi.s.ARM.O3_FR}

\subsubsection{\OptimizingKeilVI (\ThumbMode)}

\lstinputlisting[caption=\OptimizingKeilVI (\ThumbMode),style=customasmARM]{\CURPATH/atoi.s.thumb.O3_FR}

De façon intéressante, nous pouvons nous rappeler des cours de mathématiques que
l'ordre des opérations d'addition et de soustraction n'a pas d'importance.

C'est notre cas: d'abord, l'expression $rt*10 - '0'$ est calculée, puis la valeur
du caractère en entrée lui est ajoutée.

En effet, le résultat est le même, mais le compilateur a fait quelques regroupements.

\subsubsection{GCC 4.9.1 ARM64 \Optimizing}

Le compilateur ARM64 peut utiliser le suffixe de pré-incrémentation pour les instructions:

\lstinputlisting[caption=GCC 4.9.1 ARM64 \Optimizing,style=customasmARM]{\CURPATH/atoi.s.GCC49.ARM64.O3_FR}

\subsection{Un exemple légèrement avancé}

Mon nouvel extrait de code est plus avancé, maintenant il teste le signe \q{moins}
au premier caractère et renvoie une erreur si un caractère autre qu'un chiffre est
trouvé dans la chaîne en entrée:

\lstinputlisting[style=customc]{\CURPATH/atoi2.c}

\subsubsection{GCC 4.9.1 x64 \Optimizing}

\lstinputlisting[caption=GCC 4.9.1 x64 \Optimizing,style=customasmx86]{\CURPATH/atoi2.s.GCC491.O3.x64_FR}

\myindex{x86!\Instructions!NEG}

Si le signe \q{moins} a été rencontré au début de la chaîne, l'instruction \INS{NEG}
est exécutée à la fin.
Elle rend le nombre négatif.

\label{one_comparison_instead_of_two}
Il y a encore une chose à mentionner.

Comment ferait un programmeur moyen pour tester si le caractère n'est pas un chiffre?
Tout comme nous l'avons dans le code source:

\begin{lstlisting}[style=customc]
if (*s<'0' || *s>'9')
    ...
\end{lstlisting}

Il y a deux opérations de comparaison.

Ce qui est intéressant, c'est que l'on peut remplacer les deux opérations par une
seule:
simplement soustraire \q{0} de la valeur du caractère,

traiter le résultat comme une valeur non-signée (ceci est important) et tester s'il
est plus grand que 9.

Par exemple, disons que l'entrée utilisateur contient le caractère point (\q{.})
qui a pour code \ac{ASCII} 46.
$46-48=-2$ si nous traitons le résultat comme un nombre signé.

En effet, le caractère point est situé deux places avant le caractère \q{0} dans la table \ac{ASCII}.
Mais il correspond à \TT{0xFFFFFFFE} (4294967294)
si nous traitons le résultat comme une valeur non signée, c'est définitivement plus
grand que 9!

Le compilateur fait cela souvent, donc il est important de connaître ces astuces.

Un autre exemple dans ce livre:
\myref{toupper_one_comparison}.

MSVC 2013 x64 \Optimizing utilise les même astuces.

\subsubsection{\OptimizingKeilVI (\ARMMode)}

\lstinputlisting[caption=\OptimizingKeilVI (\ARMMode),numbers=left,style=customasmARM]{\CURPATH/atoi2.s.ARM.O3_FR}

Il n'y a pas d'instruction NEG en ARM 32-bit, donc l'opération \q{Reverse Subtraction}
(ligne 31) est utilisée ici.

Elle est déclenchée si le résultat de l'instruction \INS{CMP} (à la ligne 29) était
\q{Not Equal} (non égal) (d'où le suffixe \TT{-NE}).
\myindex{ARM!\Instructions!RSB}

Donc ce que fait \INS{RSBNE}, c'est soustraire la valeur résultante de 0.

Cela fonctionne comme l'opération de soustraction normale, mais échange les opérandes

Soustraire n'importe quel nombre de 0 donne sa négation: $0-x=-x$.

Le code en mode Thumb est en gros le même.

\myindex{ARM!\Instructions!NEG}
GCC 4.9 pour ARM64 peut utiliser l'instruction NEG, qui est disponible en ARM64.

\subsection{\Exercise{}}

\myindex{Fuzzing}
Oh, à propos, les chercheurs en sécurité sont souvent confrontés à un comportement
imprévisible de programme lorsqu'il traite des données incorrectes.

Par exemple, lors du fuzzing.
À titre d'exercice, vous pouvez essayer d'entrer des caractères qui ne soient pas
des chiffres et de voir ce qui se passe.

Essayez d'expliquer ce qui s'est passé, et pourquoi.

