\mysection{Fonctions inline}
\myindex{Inline code}
\label{inline_code}

Le code inline, c'est lorsque le compilateur, au lieu de mettre une instruction
d'appel à une petite ou à une minuscule fonction, copie son corps à la place.

\lstinputlisting[caption=Un exemple simple,style=customc]{\CURPATH/1.c}

\dots est compilée de façon très prédictive, toutefois, si nous utilisons l'option
d'optimisation de GCC (\Othree), nous voyons:

\lstinputlisting[caption=GCC 4.8.1 \Optimizing,style=customasmx86]{\CURPATH/1.s}

(Ici la division est effectuée avec une multiplication(\myref{sec:divisionbymult}).)

Oui, notre petite fonction \TT{celsius\_to\_fahrenheit()} a été placée juste avant
l'appel à \printf.

Pourquoi? C'est plus rapide que d'exécuter la code de cette fonction plus le surcoût
de l'appel/retour.

Les optimiseurs des compilateurs modernes choisissent de mettre en ligne les petites
fonctions automatiquement.
Mais il est possible de forcer le compilateur à mettre en ligne automatiquement certaines
fonctions, en les marquants avec le mot clef \q{inline} dans sa déclaration.

% sections
\mysection{\oracle}
\label{oracle}

% sections
\EN{\input{examples/oracle/1_version_EN}}\RU{\input{examples/oracle/1_version_RU}}
\EN{\input{examples/oracle/2_ksmlru_EN}}\RU{\input{examples/oracle/2_ksmlru_RU}}
\EN{\input{examples/oracle/3_timer_EN}}\RU{\input{examples/oracle/3_timer_RU}}


