\mysection{Inline functions}
\myindex{Inline code}
\label{inline_code}

Inlined code is when the compiler, instead of placing a call instruction to a small or tiny function,
just places its body right in-place.

\lstinputlisting[caption=A simple example,style=customc]{\CURPATH/1.c}

\dots is compiled in very predictable way, however, if we turn on GCC optimizations (\Othree), we'll see:

\lstinputlisting[caption=\Optimizing GCC 4.8.1,style=customasmx86]{\CURPATH/1.s}

(Here the division is performed by multiplication(\myref{sec:divisionbymult}).)

Yes, our small function \TT{celsius\_to\_fahrenheit()} has just been placed before the \printf call.

Why? It can be faster than executing this function's code plus the overhead of calling/returning.

Modern optimizing compilers are choosing small functions for inlining automatically.
But it's possible to force compiler additionally to inline some function, if to mark
it with the \q{inline} keyword in its declaration.

% sections
\mysection{\oracle}
\label{oracle}

% sections
\EN{\input{examples/oracle/1_version_EN}}\RU{\input{examples/oracle/1_version_RU}}
\EN{\input{examples/oracle/2_ksmlru_EN}}\RU{\input{examples/oracle/2_ksmlru_RU}}
\EN{\input{examples/oracle/3_timer_EN}}\RU{\input{examples/oracle/3_timer_RU}}


