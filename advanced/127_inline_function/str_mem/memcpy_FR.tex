\subsubsection{memcpy()}
\myindex{\CStandardLibrary!memcpy()}

\myparagraph{Petits blocs}
\label{copying_short_blocks}

\myindex{x86!\Instructions!MOV}

La routine pour copier des blocs courts est souvent implémentée comme une séquence
d'instructions \MOV.

\lstinputlisting[caption=exemple memcpy(),style=customc]{\CURPATH/str_mem/memcpy_7.c}

\lstinputlisting[caption=MSVC 2010 \Optimizing,style=customasmx86]{\CURPATH/str_mem/memcpy_7_MSVC_2010_Ox.asm}

\lstinputlisting[caption=GCC 4.8.1 \Optimizing,style=customasmx86]{\CURPATH/str_mem/memcpy_7_GCC_O3.s}

C'est effectué en général ainsi: des blocs de 4-octets sont d'abord copiés, puis
un mot de 16-bit (si nécessaire) et enfin un dernier octet (si nécessaire).

Les structures sont aussi copiées en utilisant
\MOV: \myref{short_struct_copying_using_MOV}.

\myparagraph{Longs blocs}

Les compilateurs se comportent différemment dans ce cas.

\lstinputlisting[caption=memcpy() exemple,style=customc]{\CURPATH/str_mem/memcpy.c}

\myindex{x86!\Instructions!MOVSD}

Pour copier 128 octets, MSVC utilise une seule instruction \TT{MOVSD} (car 128 est
divisible par 4):

\lstinputlisting[caption=MSVC 2010 \Optimizing,style=customasmx86]{\CURPATH/str_mem/memcpy_128_MSVC_2010_Ox.asm}

Lors de la copie de 123 octets, 30 mots de 32-bit sont tout d'abord copiés en utilisant
\TT{MOVSD} (ce qui fait 120 octets), puis 2 octets sont copiés en utilisant \TT{MOVSW},
puis un autre octet en utilisant \TT{MOVSB}.

\lstinputlisting[caption=MSVC 2010 \Optimizing,style=customasmx86]{\CURPATH/str_mem/memcpy_123_MSVC_2010_Ox.asm}


GCC utilise une grosse fonction universelle, qui fonctionne pour n'importe quelle
taille de bloc:

\lstinputlisting[caption=GCC 4.8.1 \Optimizing,style=customasmx86]{\CURPATH/str_mem/memcpy_GCC.s}


Les fonctions de copie de mémoire universelles fonctionnent en général comme suit:
calculer combien de mots de 32-bit peuvent être copiés, puis les copier en utilisant
\TT{MOVSD}, et enfin copier les octets restants.

\myindex{SIMD}

Des fonctions de copie plus avancées et complexes utilisent les instructions \ac{SIMD}
et prennent aussi en compte l'alignement de la mémoire.

Voici un exemple de fonction strlen() SIMD: \myref{SIMD_strlen}.

