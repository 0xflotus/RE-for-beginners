\mysection{Strings trimming}
\newcommand{\CRLF}{\ac{CR}/\ac{LF}}

A very common string processing task is to remove some characters at the start and/or at the end.

In this example, we are going to work with a function which removes all newline characters 
(\CRLF{}) from the end of the input string:

\lstinputlisting[style=customc]{\CURPATH/strtrim_EN.c}

The input argument is always returned on exit, this is convenient when you want to chain 
string processing functions, like it has done here in the \main function.

\myindex{\CLanguageElements!Short-circuit}
The second part of for() (\TT{str\_len>0 \&\& (c=s[str\_len-1])}) is the so called \q{short-circuit} 
in \CCpp and is very convenient \InSqBrackets{\CNotes 1.3.8}.

The \CCpp compilers guarantee an evaluation sequence from left to right.

So if the first clause is false after evaluation, the second one is never to be evaluated.

% subsections
\input{\CURPATH/x64_EN}
\input{\CURPATH/ARM64_EN}
\subsubsection{ARM}

\myparagraph{\OptimizingKeilVI (\ThumbMode)}

\lstinputlisting[style=customasmARM]{patterns/04_scanf/1_simple/ARM_IDA.lst}

\myindex{\CLanguageElements!\Pointers}

In order for \scanf to be able to read item it needs a parameter---pointer to an \Tint.
\Tint is 32-bit, so we need 4 bytes to store it somewhere in memory, and it fits exactly in a 32-bit register.
\myindex{IDA!var\_?}
A place for the local variable \GTT{x} is allocated in the stack and \IDA
has named it \IT{var\_8}. It is not necessary, however, to allocate a such since \ac{SP} (\gls{stack pointer}) is already pointing to that space and it can be used directly.

So, \ac{SP}'s value is copied to the \Reg{1} register and, together with the format-string, passed to \scanf.

\INS{PUSH/POP} instructions behaves differently in ARM than in x86 (it's the other way around).
They are synonyms to \INS{STM/STMDB/LDM/LDMIA} instructions.
And \INS{PUSH} instruction first writes a value into the stack, \IT{and then} subtracts \ac{SP} by 4.
\INS{POP} first adds 4 to \ac{SP}, \IT{and then} reads a value from the stack.
Hence, after \INS{PUSH}, \ac{SP} points to an unused space in stack.
It is used by \scanf, and by \printf after.

\INS{LDMIA} means \IT{Load Multiple Registers Increment address After each transfer}.
\INS{STMDB} means \IT{Store Multiple Registers Decrement address Before each transfer}.

\myindex{ARM!\Instructions!LDR}
Later, with the help of the \INS{LDR} instruction, this value is moved from the stack to the \Reg{1} register in order to be passed to \printf.

\myparagraph{ARM64}

\lstinputlisting[caption=\NonOptimizing GCC 4.9.1 ARM64,numbers=left,style=customasmARM]{patterns/04_scanf/1_simple/ARM64_GCC491_O0_EN.s}

There is 32 bytes are allocated for stack frame, which is bigger than it needed. Perhaps some memory aligning issue?
The most interesting part is finding space for the $x$ variable in the stack frame (line 22).
Why 28? Somehow, compiler decided to place this variable at the end of stack frame instead of beginning.
The address is passed to \scanf, which just stores the user input value in the memory at that address.
This is 32-bit value of type \Tint.
The value is fetched at line 27 and then passed to \printf.


\subsubsection{MIPS}
% FIXME better start at non-optimizing version?

The function uses a lot of S- registers which must be preserved, so that's why its 
values are saved in the function prologue and restored in the epilogue.

\lstinputlisting[caption=\Optimizing GCC 4.4.5 (IDA),style=customasmMIPS]{patterns/13_arrays/1_simple/MIPS_O3_IDA_EN.lst}

Something interesting: there are two loops and the first one doesn't need $i$, it needs only 
$i*2$ (increased by 2 at each iteration) and also the address in memory (increased by 4 at each iteration).

So here we see two variables, one (in \$V0) increasing by 2 each time, and another (in \$V1) --- by 4.

The second loop is where \printf is called and it reports the value of $i$ to the user, 
so there is a variable
which is increased by 1 each time (in \$S0) and also a memory address (in \$S1) increased by 4 each time.

That reminds us of loop optimizations: \myref{loop_iterators}.

Their goal is to get rid of multiplications.



