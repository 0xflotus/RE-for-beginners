\mysection{Ajustement de chaînes}
\newcommand{\CRLF}{\ac{CR}/\ac{LF}}

Un traitement de chaîne très courant est la suppression de certains caractères au
début et/où à la fin.

Dans cet exemple, nous allons travailler avec une fonction qui supprime tous les
caractères newline (\CRLF{}) à la fin de la chaîne entrée:

\lstinputlisting[style=customc]{\CURPATH/strtrim_FR.c}

L'argument en entrée est toujours renvoyé en sortie, ceci est pratique lorsque vous
voulez chaîner les fonctions de traitement de chaîne, comme c'est fait ici dans la
fonction \main.

\myindex{\CLanguageElements!Short-circuit}
La seconde partie de for() (\TT{str\_len>0 \&\& (c=s[str\_len-1])}) est appelé le
\q{short-circuit} en \CCpp et est très pratique \InSqBrackets{\CNotes 1.3.8}.

Les compilateurs \CCpp garantissent une séquence d'évaluation de gauche à droite.

Donc, si la première clause est fausse après l'évaluation, la seconde n'est pas évaluée.

% subsections
\subsection{x64: MSVC 2013 \Optimizing}

\lstinputlisting[caption=MSVC 2013 x64 \Optimizing,style=customasmx86]{\CURPATH/MSVC2013_x64_Ox_FR.asm}

Tout d'abord, MSVC a inliné le code la fonction \strlen{}, car il en a conclus que
ceci était plus rapide que le \strlen{} habituel + le coût de l'appel et du retour.
Ceci est appelé de l'inlining: \myref{inline_code}.

\myindex{x86!\Instructions!OR}
\myindex{\CStandardLibrary!strlen()}
\label{using_OR_instead_of_MOV}
La première instruction de \strlen{} mis en ligne est\\
\TT{OR RAX, 0xFFFFFFFFFFFFFFFF}. 

MSVC utilise souvent \TT{OR} au lieu de \TT{MOV RAX, 0xFFFFFFFFFFFFFFFF}, car l'opcode
résultant est plus court.

Et bien sûr, c'est équivalent: tous les bits sont mis à 1, et un nombre avec tous
les bits mis vaut $-1$ en complément à 2:\myref{sec:signednumbers}.

On peut se demander pourquoi le nombre $-1$ est utilisé dans \strlen{}.
À des fins d'optimisation, bien sûr.
Voici le code que MSVC a généré:

\lstinputlisting[caption=Inlined \strlen{} by MSVC 2013 x64,style=customasmx86]{\CURPATH/strlen_MSVC_FR.asm}

Essayez d'écrite plus court si vous voulez initialiser le compteur à 0!
OK, essayons:

\lstinputlisting[caption=Our version of \strlen{},style=customasmx86]{\CURPATH/my_strlen_FR.asm}

Nous avons échoué. Nous devons utilisé une instruction \INS{JMP} additionnelle!

Donc, ce que le compilateur de MSVC 2013 a fait, c'est de déplacer l'instruction
\TT{INC} avant le chargement du caractère courant.

Si le premier caractère est 0, c'est OK, \RAX contient 0 à ce moment, donc la longueur
de la chaîne est 0.

Le reste de cette fonction semble facile à comprendre.

\subsection{x64: GCC 4.9.1 \NonOptimizing}

\lstinputlisting[style=customasmx86]{\CURPATH/GCC491_x64_O0_FR.asm}

Les commentaires ont été ajoutés par l'auteur du livre.

Après l'exécution de \strlen{}, le contrôle est passé au label L2, et ici deux clauses
sont vérifiées, l'une après l'autre.

\myindex{\CLanguageElements!Short-circuit}
La seconde ne sera jamais vérifiée, si la première (\IT{str\_len==0}) est fausse
(ceci est un \q{short-circuit} (court-circuit)).

Maintenant regardons la forme courte de cette fonction:

\begin{itemize}
\item Première partie de for() (appel à \strlen{})
\item goto L2
\item L5: corps de for(). sauter à la fin, si besoin
\item troisième partie de for() (décrémenter str\_len)
\item L2: 
deuxième partie de for(): vérifier la première clause, puis la seconde. sauter au
début du corps de la boucle ou sortir.
\item L4: // sortir
\item renvoyer s
\end{itemize}

\subsection{x64: GCC 4.9.1 \Optimizing}
\label{string_trim_GCC_x64_O3}

\lstinputlisting[style=customasmx86]{\CURPATH/GCC491_x64_O3_FR.asm}

Maintenant, c'est plus complexe.

Le code avant le début du corps de la boucle est exécuté une seule fois, mais il contient
le test des caractères \CRLF{} aussi!
À quoi sert cette duplication du code?

La façon courante d'implémenter la boucle principale est sans doute ceci:

\begin{itemize}
\item (début de la boucle) tester la présence des caractères \CRLF{}, décider
\item stocker le caractère zéro
\end{itemize}

Mais GCC a décidé d'inverser ces deux étapes.

Bien sûr,  \IT{stocker le caractère zéro} ne peut pas être la première étape, donc
un autre test est nécessaire:

\begin{itemize}
\item traiter le premier caractère. matcher avec \CRLF{}, sortir si le caractère
n'est pas \CRLF{}

\item (début de la boucle) stocker le caractère zéro

\item tester la présence des caractères \CRLF{}, décider
\end{itemize}

Maintenant la boucle principale est très courte, ce qui est bon pour les derniers
\ac{CPU}s.

Le code n'utilise pas la variable str\_len, mais str\_len-1.
Donc c'est plus comme un index dans un buffer.

Apparemment, GCC a remarqué que l'expression str\_len-1 est utilisée deux fois.

Donc, c'est mieux d'allouer une variable qui contient toujours une valeur qui est
plus petite que la longueur actuelle de la chaîne de un, et la décrémente (ceci a
le même effet que de décrémenter la variable str\_len).

\subsubsection{ARM64}

\myparagraph{GCC (Linaro) 4.9 \Optimizing}

\myindex{Multiplication-addition fusionnées} % TODO FIXME verify
\myindex{ARM!\Instructions!MADD}
Tout ce qu'il y a ici est simple.
\TT{MADD} est juste une instruction qui effectue une multiplication/addition fusionnées
(similaire à l'instruction \TT{MLA} que nous avons déjà vue).
Tous les 3 arguments sont passés dans la partie 32-bit de X-registres.
Effectivement, le type des arguments est \IT{int} 32-bit.
Le résultat est renvoyé dans \TT{W0}.

\lstinputlisting[caption=GCC (Linaro) 4.9 \Optimizing,style=customasmARM]{patterns/05_passing_arguments/ARM/ARM64_O3_FR.s}

Étendons le type de toutes les données à 64-bit \TT{uint64\_t} et testons:

\lstinputlisting[style=customc]{patterns/05_passing_arguments/ex64.c}

\begin{lstlisting}[style=customasmARM]
f:
	madd	x0, x0, x1, x2
	ret
main:
	mov	x1, 13396
	adrp	x0, .LC8
	stp	x29, x30, [sp, -16]!
	movk	x1, 0x27d0, lsl 16
	add	x0, x0, :lo12:.LC8
	movk	x1, 0x122, lsl 32
	add	x29, sp, 0
	movk	x1, 0x58be, lsl 48
	bl	printf
	mov	w0, 0
	ldp	x29, x30, [sp], 16
	ret

.LC8:
	.string	"%lld\n"
\end{lstlisting}

La fonction \ttf{} est la même, seulement les X-registres 64-bit sont utilisés entièrement
maintenant.
Les valeurs longues sur 64-bit sont chargées dans les registres par partie, c'est
également décrit ici: \myref{ARM_big_constants_loading}.

\myparagraph{GCC (Linaro) 4.9 \NonOptimizing}

Le code sans optimisation est plus redondant:

\begin{lstlisting}[style=customasmARM]
f:
	sub	sp, sp, #16
	str	w0, [sp,12]
	str	w1, [sp,8]
	str	w2, [sp,4]
	ldr	w1, [sp,12]
	ldr	w0, [sp,8]
	mul	w1, w1, w0
	ldr	w0, [sp,4]
	add	w0, w1, w0
	add	sp, sp, 16
	ret
\end{lstlisting}

Le code sauve ses arguments en entrée dans la pile locale, dans le cas où quelqu'un
(ou quelque chose) dans cette fonction aurait besoin d'utiliser les registres \TT{W0...W2}.
Cela évite d'écraser les arguments originels de la fonction, qui pourraient être
de nouveau utilisés par la suite.

Cela est appelé \IT{Zone de sauvegarde de registre.} (\ARMPCS).
L'appelée, toutefois, n'est pas obligée de les sauvegarder.
C'est un peu similaire au \q{Shadow Space}: \myref{shadow_space}.

Pourquoi est-ce que GCC 4.9 avec l'option d'optimisation supprime ce code de sauvegarde?
Parce qu'il a fait plus d'optimisation et en a conclu que les arguments de la fonction
n'allaient pas être utilisés par la suite et donc que les registres \TT{W0...W2}
ne vont pas être utilisés.

\myindex{ARM!\Instructions!MUL}
\myindex{ARM!\Instructions!ADD}

Nous avons donc une paire d'instructions \TT{MUL}/\TT{ADD} au lieu d'un seul \TT{MADD}.

\subsubsection{ARM + \NonOptimizingXcodeIV (\ThumbTwoMode)}
\label{FPU_passing_floats_ARM}

\lstinputlisting[style=customasmARM]{patterns/12_FPU/2_passing_floats/Xcode_thumb_O0.asm}

Comme nous l'avons déjà mentionné, les pointeurs sur des nombres flottants 64-bit
sont passés dans une paire de R-registres.

Ce code est un peu redondant (probablement car l'optimisation est désactivée),
puisqu'il est possible de charger les valeurs directement dans les R-registres sans
toucher les D-registres.

Donc, comme nous le voyons, la fonction \GTT{\_pow} reçoit son premier argument dans
\Reg{0} et \Reg{1}, et le second dans \Reg{2} et \Reg{3}.
La fonction laisse son résultat dans \Reg{0} et \Reg{1}.
Le résultat de \GTT{\_pow} est déplacé dans \GTT{D16}, puis dans la paire \Reg{1}
et \Reg{2}, d'où \printf prend le nombre résultant.

\subsubsection{ARM + \NonOptimizingKeilVI (\ARMMode)}

\lstinputlisting[style=customasmARM]{patterns/12_FPU/2_passing_floats/Keil_ARM_O0.asm}

Les D-registres ne sont pas utilisés ici, juste des paires de R-registres.

\subsubsection{ARM64 + GCC (Linaro) 4.9 \Optimizing}

\lstinputlisting[caption=GCC (Linaro) 4.9 \Optimizing,style=customasmARM]{patterns/12_FPU/2_passing_floats/ARM64_FR.s}

Les constantes sont chargées dans \RegD{0} et \RegD{1}: \TT{pow()} les prend d'ici.
Le résultat sera dans \RegD{0} après l'exécution de \TT{pow()}.
Il est passé à  \printf sans aucune modification ni déplacement, car \printf
prend ces arguments de \glslink{integral type}{type intégral} et pointeurs depuis
des X-registres, et les arguments en virgule flottante depuis des D-registres.


\subsubsection{MIPS}

MIPS peut supporter plusieurs coprocesseurs (jusqu'à 4), le zérotième\footnote{Barbarisme
pour rappeler que les indices commencent à zéro.} est un coprocesseur contrôleur
spécial, et celui d'indice 1 est le FPU.

Comme en ARM, le coprocesseur MIPS n'est pas une machine à pile, il comprend 32 registres
32-bit (\$F0-\$F31):
\myref{MIPS_FPU_registers}.

Lorsque l'on doit travailler avec des valeurs \Tdouble 64-bit, une paire de F-registres
32-bit est utilisée.

\lstinputlisting[caption=GCC 4.4.5 \Optimizing (IDA),style=customasmMIPS]{patterns/12_FPU/1_simple/MIPS_O3_IDA_FR.lst}

Les nouvelles instructions ici sont:

\myindex{MIPS!\Instructions!LWC1}
\myindex{MIPS!\Instructions!DIV.D}
\myindex{MIPS!\Instructions!MUL.D}
\myindex{MIPS!\Instructions!ADD.D}
\begin{itemize}

\item \INS{LWC1} charge un mot de 32-bit dans un registre du premier coprocesseur
(d'où le \q{1} dans le nom de l'instruction).
\myindex{MIPS!\Pseudoinstructions!L.D}

Une paire d'instructions \INS{LWC1} peut être combinée en une pseudo instruction \INS{L.D}.

\item \INS{DIV.D}, \INS{MUL.D}, \INS{ADD.D} effectuent respectivement la division,
la multiplication, et l'addition (\q{.D} est le suffixe standard pour la double précision,
\q{.S} pour la simple précision)

\end{itemize}

\myindex{MIPS!\Instructions!LUI}
\myindex{\CompilerAnomaly}
\label{MIPS_FPU_LUI}

Il y a une anomalie bizarre du compilateur: l'instruction \INS{LUI} que nous avons
marqué avec un point d'interrogation.
Il m'est difficile de comprendre pourquoi charger une partie de la constante de type
64-bit \Tdouble dans le registre \$V0. Cette instruction n'a pas d'effet.
% TODO did you try checking out compiler source code?
Si quelqu'un en sait plus sur ceci, s'il vous plaît, envoyez moi un email\footnote{\EMAIL}.



