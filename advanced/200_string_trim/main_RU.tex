\mysection{Обрезка строк}
\newcommand{\CRLF}{\ac{CR}/\ac{LF}}

Весьма востребованная операция со строками --- это удаление некоторых символов в начале и/или конце
строки.

В этом примере, мы будем работать с функцией, удаляющей все символы перевода строки 
(\CRLF{}) в конце входной строки:

\lstinputlisting[style=customc]{\CURPATH/strtrim_RU.c}

Входной аргумент всегда возвращается на выходе, это удобно, когда вам нужно объединять
функции обработки строк в цепочки, как это сделано здесь в функции \main.

\myindex{\CLanguageElements!Short-circuit}
Вторая часть for() (\TT{str\_len>0 \&\& (c=s[str\_len-1])}) называется в \CCpp \q{short-circuit} 
(короткое замыкание) и это очень удобно: \InSqBrackets{\CNotes 1.3.8}.

Компиляторы \CCpp гарантируют последовательное вычисление слева направо.

Так что если первое условие не истинно после вычисления, второе никогда не будет
вычисляться.

% subsections
\input{\CURPATH/x64_RU}
\input{\CURPATH/ARM64_RU}
\subsubsection{ARM}

\myparagraph{\OptimizingKeilVI (\ThumbMode)}

\lstinputlisting[style=customasmARM]{patterns/04_scanf/1_simple/ARM_IDA.lst}

\myindex{\CLanguageElements!\Pointers}
Чтобы \scanf мог вернуть значение, ему нужно передать указатель на переменную типа \Tint.
\Tint~--- 32-битное значение, для его хранения нужно только 4 байта, и оно помещается в 32-битный регистр.

\myindex{IDA!var\_?}
Место для локальной переменной \GTT{x} выделяется в стеке, \IDA наименовала её \IT{var\_8}. 
Впрочем, место для неё выделять не обязательно, т.к. \glslink{stack pointer}{указатель стека} \ac{SP} уже указывает на место, 
свободное для использования.
Так что значение указателя \ac{SP} копируется в регистр \Reg{1}, и вместе с format-строкой, 
передается в \scanf.

Инструкции \INS{PUSH/POP} в ARM работают иначе, чем в x86 (тут всё наоборот).
Это синонимы инструкций \INS{STM/STMDB/LDM/LDMIA}.
И инструкция \INS{PUSH} в начале записывает в стек значение, \IT{затем} вычитает 4 из \ac{SP}.
\INS{POP} в начале прибавляет 4 к \ac{SP}, \IT{затем} читает значение из стека.
Так что после \INS{PUSH}, \ac{SP} указывает на неиспользуемое место в стеке.
Его и использует \scanf, а затем и \printf.

\INS{LDMIA} означает \IT{Load Multiple Registers Increment address After each transfer}.
\INS{STMDB} означает \IT{Store Multiple Registers Decrement address Before each transfer}.

\myindex{ARM!\Instructions!LDR}
Позже, при помощи инструкции \INS{LDR}, это значение перемещается из стека в регистр \Reg{1}, чтобы быть переданным в \printf.

\myparagraph{ARM64}

\lstinputlisting[caption=\NonOptimizing GCC 4.9.1 ARM64,numbers=left,style=customasmARM]{patterns/04_scanf/1_simple/ARM64_GCC491_O0_RU.s}

Под стековый фрейм выделяется 32 байта, что больше чем нужно. Может быть, это связано с выравниваем по границе памяти?
Самая интересная часть~--- это поиск места под переменную $x$ в стековом фрейме (строка 22).
Почему 28? Почему-то, компилятор решил расположить эту переменную в конце стекового фрейма, а не в начале.
Адрес потом передается в \scanf, которая просто сохраняет значение, введенное пользователем, в памяти по этому адресу.
Это 32-битное значение типа \Tint.
Значение загружается в строке 27 и затем передается в \printf.


\subsubsection{MIPS}
% FIXME better start at non-optimizing version?
Функция использует много S-регистров, которые должны быть сохранены. Вот почему их значения сохраняются
в прологе функции и восстанавливаются в эпилоге.

\lstinputlisting[caption=\Optimizing GCC 4.4.5 (IDA),style=customasmMIPS]{patterns/13_arrays/1_simple/MIPS_O3_IDA_RU.lst}

Интересная вещь: здесь два цикла и в первом не нужна переменная $i$, а нужна только переменная
$i*2$ (скачущая через 2 на каждой итерации) и ещё адрес в памяти (скачущий через 4 на каждой итерации).

Так что мы видим здесь две переменных: одна (в \$V0) увеличивается на 2 каждый раз, и вторая (в \$V1) --- на 4.

Второй цикл содержит вызов \printf. Он должен показывать значение $i$ пользователю,
поэтому здесь есть переменная, увеличивающаяся на 1 каждый раз (в \$S0), а также адрес в памяти (в \$S1) 
увеличивающийся на 4 каждый раз.

Это напоминает нам оптимизацию циклов: \myref{loop_iterators}.
Цель оптимизации в том, чтобы избавиться от операций умножения.



