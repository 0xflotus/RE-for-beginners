\mysection{CRC32 calculation example}
\myindex{CRC32}
\label{sec:CRC32}

\newcommand{\URLCRCSRC}{\url{http://go.yurichev.com/17327}}

This is a very popular table-based CRC32 hash calculation 
technique\footnote{The source code has been taken from here: \URLCRCSRC}.

\lstinputlisting[style=customc]{\CURPATH/CRC.c}

\myindex{\CLanguageElements!for}

We are interested in the \TT{crc()} function only.
By the way, pay attention to the two loop initializers in the \TT{for()} statement: \TT{hash=len, i=0}.
The \CCpp standard allows this, of course.
The emitted code will contain two operations in the loop initialization part
instead of one.

Let's compile it in MSVC with optimization (\Ox).
For the sake of brevity, only the \TT{crc()} function is listed here, with my comments.

\lstinputlisting[style=customasmx86]{\CURPATH/CRC_2_EN.asm}

Let's try the same in GCC 4.4.1 with \Othree option:

\lstinputlisting[style=customasmx86]{\CURPATH/CRC_gcc_O3_EN.asm}

\myindex{x86!\Instructions!NOP}
\myindex{x86!\Instructions!LEA}

GCC has aligned the loop start on a 8-byte boundary by adding \NOP and \TT{lea esi, [esi+0]}
(that is an \IT{idle operation} too).
Read more about it in npad section~(\myref{sec:npad}).

