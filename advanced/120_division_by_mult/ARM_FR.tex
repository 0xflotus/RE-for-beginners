\subsection{ARM}

Le processeur ARM, tout comme un autre processeur \q{pur} RISC n'a pas d'instruction
pour la division.
Il manque aussi une simple instruction pour la multiplication avec une constante 32-bit
(rappelez-vous qu'une constante 32-bit ne tient pas dans un opcode 32-bit).

En utilisant de cette astuce intelligente (ou \IT{hack}), il est possible d'effectuer
la division en utilisant seulement trois instructions: addition, soustraction et
décalages de bit~(\myref{sec:bitfields}).

Voici un exemple qui divise un nombre 32-bit par 10, tiré de
\InSqBrackets{\ARMCookBook 3.3 Division by a Constant}.
La sortie est constituée du quotient et du reste.

\begin{lstlisting}[style=customasmARM]
; prend l'argument dans a1
; renvoie le quotient dans a1, le reste dans a2
; on peut utiliser moins de cycles si seul le quotient ou le reste est requis
    SUB    a2, a1, #10             ; garde (x-10) pour plus tard
    SUB    a1, a1, a1, lsr #2
    ADD    a1, a1, a1, lsr #4
    ADD    a1, a1, a1, lsr #8
    ADD    a1, a1, a1, lsr #16
    MOV    a1, a1, lsr #3
    ADD    a3, a1, a1, asl #2
    SUBS   a2, a2, a3, asl #1      ; calcule (x-10) - (x/10)*10
    ADDPL  a1, a1, #1              ; fix-up quotient
    ADDMI  a2, a2, #10             ; fix-up reste
    MOV    pc, lr
\end{lstlisting}

\subsubsection{\OptimizingXcodeIV (\ARMMode)}

\begin{lstlisting}[style=customasmARM]
__text:00002C58 39 1E 08 E3 E3 18 43 E3  MOV    R1, 0x38E38E39
__text:00002C60 10 F1 50 E7              SMMUL  R0, R0, R1
__text:00002C64 C0 10 A0 E1              MOV    R1, R0,ASR#1
__text:00002C68 A0 0F 81 E0              ADD    R0, R1, R0,LSR#31
__text:00002C6C 1E FF 2F E1              BX     LR
\end{lstlisting}

Ce code est presque le même que celui généré par MSVC avec optimisation et GCC.

Il semble que LLVM utilise le même algorithme pour générer des constantes.

\myindex{ARM!\Instructions!MOV}
\myindex{ARM!\Instructions!MOVT}

Le lecteur attentif pourrait se demander comment \MOV écrit une valeur 32-bit dans
un registre, alors que ceci n'est pas possible en mode ARM.

C'est impossible, en effet, mais on voit qu'il y a 8 octets par instruction, au lieu
des 4 standards, en fait, ce sont deux instructions.

La première instruction charge \TT{0x8E39} dans les 16 bits bas du registre et la
seconde instruction est \TT{MOVT}, qui charge \TT{0x383E} dans les 16 bits hauts
du registre.
\IDA reconnaît de telles séquences, et par concision, il les réduit a une seule
\q{pseudo-instruction}.

\myindex{ARM!\Instructions!SMMUL}
L'instruction \TT{SMMUL} (\IT{Signed Most Significant Word Multiply} mot le plus
significatif d'une multiplication signée), multiplie deux nombres, les traitant comme
des nombres signés et laisse la partie 32-bit haute dans le registre \Reg{0},
en ignorant la partie 32-bit basse du résultat.

\myindex{ARM!Optional operators!ASR}
L'instruction \TT{\q{MOV R1, R0,ASR\#1}} est le décalage arithmétique à droite d'un bit.

\myindex{ARM!\Instructions!ADD}
\myindex{ARM!Data processing instructions}
\myindex{ARM!Optional operators!LSR}
\TT{\q{ADD R0, R1, R0,LSR\#31}} est $R0=R1 + R0>>31$

% FIXME какие именно инструкции? \myref{} ->
\label{shifts_in_ARM_mode}

Il n'y a pas d'instruction de décalage séparée en mode ARM.
A la place, des instructions comme
(\MOV, \ADD, \SUB, \TT{RSB})\footnote{\DataProcessingInstructionsFootNote}
peuvent avoir un suffixe, indiquant si le second argument doit être décalé, et si
oui, de quelle valeur et comment.
\TT{ASR} signifie \IT{Arithmetic Shift Right}, \TT{LSR}---\IT{Logical Shift Right}.


\subsubsection{\OptimizingXcodeIV (\ThumbTwoMode)}

\begin{lstlisting}[style=customasmARM]
MOV             R1, 0x38E38E39
SMMUL.W         R0, R0, R1
ASRS            R1, R0, #1
ADD.W           R0, R1, R0,LSR#31
BX              LR
\end{lstlisting}

\myindex{ARM!\Instructions!ASRS}

Il y a dex instructions de décalage séparées en mode Thumb, et l'une d'elle est
utilisée ici---\TT{ASRS} (arithmetic shift right).

\subsubsection{\NonOptimizing Xcode 4.6.3 (LLVM) and Keil 6/2013}

LLVM \NonOptimizing
ne génère pas le code que nous avons vu avant dans cette section, mais insère à la
place un appel à la fonction de bibliothèque \IT{\_\_\_divsi3}.

À propos de Keil: il insère un appel à la fonction de bibliothèque \IT{\_\_aeabi\_idivmod}
dans tous les cas.
