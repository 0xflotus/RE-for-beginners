\mysection{More about pointers}
\myindex{\CLanguageElements!\Pointers}
\label{label_pointers}

\epigraph{The way C handles pointers, for example, was a brilliant innovation;
it solved a lot of problems that we had before in data structuring and
made the programs look good afterwards.}{Donald Knuth, interview (1993)}

For those, who still have hard time understanding \CCpp pointers, here are more examples.
Some of them are weird and serves only demonstration purpose:
use them in production code only if you really know what you're doing.

\subsection{Accessing arguments/local variables of caller}

From C/C++ basics we know that this is impossible for a function to access arguments of caller function or its
local variables.

Nevertheless, it's possible using dirty hacks.
For example:

\begin{lstlisting}[style=customc]
#include <stdio.h>

void f(char *text)
{
	// print stack
	int *tmp=&text;
	for (int i=0; i<20; i++)
	{
		printf ("0x%x\n", *tmp);
		tmp++;
	};
};

void draw_text(int X, int Y, char* text)
{
	f(text);

	printf ("We are going to draw [%s] at %d:%d\n", text, X, Y);
};

int main()
{
	printf ("address of main()=0x%x\n", &main);
	printf ("address of draw_text()=0x%x\n", &draw_text);
	draw_text(100, 200, "Hello!");
};
\end{lstlisting}

On 32-bit Ubuntu 16.04 and GCC 5.4.0, I got this:

\begin{lstlisting}
address of main()=0x80484f8
address of draw_text()=0x80484cb
0x8048645	first argument to f()
0x8048628
0xbfd8ab98
0xb7634590
0xb779eddc
0xb77e4918
0xbfd8aba8
0x8048547	return address into the middle of main()
0x64		first argument to draw_text()
0xc8		second argument to draw_text()
0x8048645	third argument to draw_text()
0x8048581
0xb779d3dc
0xbfd8abc0
0x0
0xb7603637
0xb779d000
0xb779d000
0x0
0xb7603637
\end{lstlisting}

(Comments are mine.)

Since \IT{f()} starting to enumerate stack elements at its first argument, the first stack element is indeed a pointer
to \q{Hello!} string. We see its address is also used as third argument to \IT{draw\_text()} function.

In \IT{f()} we could read all functions arguments and local variables if we know exact stack layout, but it's always
changed, from compiler to compiler.
Various optimization levels affect stack layout greatly.

But if we can somehow detect information we need, we can use it and even modify it.
As an example, I'll rework \IT{f()} function:

\begin{lstlisting}[style=customc]
void f(char *text)
{
	...

	// find 100, 200 values pair and modify the second on
	tmp=&text;
	for (int i=0; i<20; i++)
	{
		if (*tmp==100 && *(tmp+1)==200)
		{
			printf ("found\n");
			*(tmp+1)=210; // change 200 to 210
			break;
		};
		tmp++;
	};
};
\end{lstlisting}

Holy moly, it works:

\begin{lstlisting}
found
We are going to draw [Hello!] at 100:210
\end{lstlisting}

\myparagraph{Summary}

It's extremely dirty hack, intended to demonstrate stack internals.
I never ever seen or heard that anyone used this in a real code.
But still, this is a good example.

\myparagraph{\Exercise}

The example has been compiled without optimization on 32-bit Ubuntu using GCC 5.4.0 and it works.
But when I turn on \TT{-O3} maximum optimization, it's failed.
Try to find why.

Use your favorite compiler and OS, try various optimization levels, find if it works and if it doesn't, find why.


\subsection{Returning string}

This is classic bug from \RobPikePractice{}:

\begin{lstlisting}[style=customc]
#include <stdio.h>

char* amsg(int n, char* s)
{
        char buf[100];

        sprintf (buf, "error %d: %s\n", n, s) ;

        return buf;
};

int main()
{
        printf ("%s\n", amsg (1234, "something wrong!"));
};
\end{lstlisting}

It would crash.
First, let's understand, why.

This is a stack state before amsg() return:

\begin{lstlisting}
(lower addresses)

[amsg(): 100 bytes]
[RA]                   <- current SP
[two amsg arguments]
[something else]
[main() local variables]

(upper addresses)
\end{lstlisting}

When amsg() returns control flow to \main, so far so good.
But \printf is called from \main, which is, in turn, use stack for its own needs, zapping 100-byte buffer.
A random garbage will be printed at the best.

Hard to believe, but I know how to fix this problem:

\begin{lstlisting}[style=customc]
#include <stdio.h>

char* amsg(int n, char* s)
{
        char buf[100];

        sprintf (buf, "error %d: %s\n", n, s) ;

        return buf;
};

char* interim (int n, char* s)
{
        char large_buf[8000];
        // make use of local array.
        // it will be optimized away otherwise, as useless.
        large_buf[0]=0;
        return amsg (n, s);
};

int main()
{
        printf ("%s\n", interim (1234, "something wrong!"));
};
\end{lstlisting}

It will work if compiled by MSVC 2013 with no optimizations and with \TT{/GS-} option\footnote{Turn off buffer security check}.
MSVC will warn: ``warning C4172: returning address of local variable or temporary'', but the code will run and message
will be printed.
Let's see stack state at the moment when amsg() returns control to interim():

\begin{lstlisting}
(lower addresses)

[amsg(): 100 bytes]
[RA]                                     <- current SP
[two amsg() arguments]
[interim() stuff, incl. 8000 bytes]
[something else]
[main() local variables]

(upper addresses)
\end{lstlisting}

Now the stack state at the moment when interim() returns control to \main{}:

\begin{lstlisting}
(lower addresses)

[amsg(): 100 bytes]
[RA]
[two amsg() arguments]
[interim() stuff, incl. 8000 bytes]
[something else]                         <- current SP
[main() local variables]

(upper addresses)
\end{lstlisting}

So when \main calls \printf, it uses stack at the place where interim()'s buffer was allocated,
and doesn't zap 100 bytes with error message inside, because 8000 bytes (or maybe much less) is just enough for everything
\printf and other descending functions do!

It may also work if there are many functions between, like:
\main $\rightarrow$ f1() $\rightarrow$ f2() $\rightarrow$ f3() ... $\rightarrow$ amsg(),
and then the result of amsg() is used in \main.
The distance between \ac{SP} in \main and address of \TT{buf[]} must be long enough,

This is why bugs like these are dangerous: sometimes your code works (and bug can be hiding unnoticed), sometimes not.
\label{heisenbug}
\myindex{Heisenbug}
Bugs like these are jokingly called heisenbugs or schrödinbugs\footnote{\url{https://en.wikipedia.org/wiki/Heisenbug}}.


\input{advanced/450_more_ptrs/3_EN}
\input{advanced/450_more_ptrs/4_EN}
\input{advanced/450_more_ptrs/5_EN}
\input{advanced/450_more_ptrs/6_EN}

\subsection{Pointer as object identificator}

Both assembly language and C has no \ac{OOP} features, but it's possible to write a code in \ac{OOP} style
(just treat structure as an object).

It's interesting, that sometimes, pointer to an object (or its address) is called as ID
(in sense of data hiding/encapsulation).

\myindex{win32!LoadLibrary()}
\myindex{win32!GetProcAddress()}
For example, LoadLibrary(), according to \ac{MSDN}, returns ``handle to the module''
\footnote{\url{https://msdn.microsoft.com/ru-ru/library/windows/desktop/ms684175(v=vs.85).aspx}}.
Then you pass this ``handle'' to other functions like GetProcAddress().
But in fact, LoadLibrary() returns pointer to DLL file mapped into memory
\footnote{\url{https://blogs.msdn.microsoft.com/oldnewthing/20041025-00/?p=37483}}.
You can read two bytes from the address LoadLibrary() returns, and that would be ``MZ'' (first two bytes of any
.EXE/.DLL file in Windows).

\myindex{win32!HMODULE}
\myindex{win32!HINSTANCE}
Apparently, Microsoft ``hides'' that fact to provide better forward compatibility.
Also, HMODULE and HINSTANCE data types had another meaning in 16-bit Windows.

Probably, this is reason why \printf has ``\%p'' modifier, which is used for printing pointers (32-bit integers
on 32-bit architectures, 64-bit on 64-bit, etc) in hexadecimal form.
Address of a structure dumped into debug log may help in finding it in another place of log.

\myindex{SQLite}
Here is also from SQLite source code:

\begin{lstlisting}

...

struct Pager {
  sqlite3_vfs *pVfs;          /* OS functions to use for IO */
  u8 exclusiveMode;           /* Boolean. True if locking_mode==EXCLUSIVE */
  u8 journalMode;             /* One of the PAGER_JOURNALMODE_* values */
  u8 useJournal;              /* Use a rollback journal on this file */
  u8 noSync;                  /* Do not sync the journal if true */

....

static int pagerLockDb(Pager *pPager, int eLock){
  int rc = SQLITE_OK;

  assert( eLock==SHARED_LOCK || eLock==RESERVED_LOCK || eLock==EXCLUSIVE_LOCK );
  if( pPager->eLock<eLock || pPager->eLock==UNKNOWN_LOCK ){
    rc = sqlite3OsLock(pPager->fd, eLock);
    if( rc==SQLITE_OK && (pPager->eLock!=UNKNOWN_LOCK||eLock==EXCLUSIVE_LOCK) ){
      pPager->eLock = (u8)eLock;
      IOTRACE(("LOCK %p %d\n", pPager, eLock))
    }
  }
  return rc;
}

...

  PAGER_INCR(sqlite3_pager_readdb_count);
  PAGER_INCR(pPager->nRead);
  IOTRACE(("PGIN %p %d\n", pPager, pgno));
  PAGERTRACE(("FETCH %d page %d hash(%08x)\n",
               PAGERID(pPager), pgno, pager_pagehash(pPg)));

...

\end{lstlisting}

