% TODO translate
% TODO TikZ
\mysection[Packer des valeurs 12-bit dans un tableau]{Packer des valeurs 12-bit dans un tableau en utilisant des opérations sur les bits (x64, ARM/ARM64, MIPS)}

(Cette partie est apparue tout d'abord dans mon blog le 4 septembre 2015.)

\subsection{Introduction}

Le système de fichier File Allocation Table (FAT) a été très répandu.
C'est difficile à croire, mais il est encore utilisé avec les disques flash, peut-être
par simplicité et compatibilité.
La table FAT en elle-même est un tableau d'éléments, chacun d'eux pointant sur le
numéro de cluster suivant d'un fichier (FAT supporte les fichiers dispersés sur l'ensemble
du disque).
Ceci implique que le maximum de chaque élément est le nombre maximum de clusters
sur le disque.
Sur MS-DOS, la plupart des disques durs ont un système de fichiers FAT16, car le
nombre de cluster peut être mis dans une valeur 16-bit.
Les disques durs sont devenus moins chers, et FAT32 est apparu, où des valeurs 32-bit
sont allouées pour le numéro de cluster.
Mais il fût aussi un temps, lorsque les disquettes n'étaient pas aussi bon marché
et n'avaient pas beaucoup d'espace, donc FAT12 était utilisé dessus, afin de faire
tenir toutes les structures du système de fichiers dans aussi peu d'espace que possible.

Donc la table FAT dans un système de fichiers FAT12 est un tableau où deux éléments
de 12-bit subséquents sont stockés dans 3 octets (triplet).
Voici comment 6 valeurs 12-bit (AAA, BBB, CCC, DDD, EEE et FFF) sont arrangées dans
9 octets:

\begin{lstlisting}
 +0 +1 +2 +3 +4 +5 +6 +7 +8
|AA|AB|BB|CC|CD|DD|EE|EF|FF|...
\end{lstlisting}

Stocker des valeurs dans un tableau et les récupérer peut être un bon exemple d'opération
de manipulation de bit (tant en C/C++ qu'en code machine bas niveau), c'est donc pourquoi
je vais utiliser FAT12 comme exemple ici.

\subsection{Structure de données}

Nous pouvons rapidement voir que chaque triplet d'octets stockera 2 valeurs 12-bit:
la première est située du côté gauche et la seconde, à droite:

\begin{lstlisting}
 +0 +1 +2
|11|12|22|...
\end{lstlisting}

Nous allons stocker les morceaux (chunks de 4 bit) de la façon suivante (1 - morceau
le plus haut, 3 - le plus bas):

(Pair)

\begin{lstlisting}
 +0 +1 +2
|12|3.|..|...
\end{lstlisting}

(Impair)

\begin{lstlisting}
 +0 +1 +2
|..|.1|23|...
\end{lstlisting}

\subsection{L' algorithme}

Donc, l'algorithme fonctionne comme suit: si l'index de l'élément est pair, le mettre
du côté gauche, si l'index est impair, le mettre du côté droit.
L'octet du milieu: si l'index de l'élément est pair, mettre cette partie dans les
4 bits hauts, si il est impair, dans les 4 bits bas.
Mais tout d'abord, trouver le bon triplet, c'est facile: $\frac{index}{2}$.
Trouver l'adresse de l'octet de droite dans un tableau d'octets est aussi facile: $\frac{index}{2} \cdot 3$
ou $index \cdot \frac{3}{2}$ ou juste $index \cdot 1.5$.

Prendre des valeurs du tableau: si l'index est pair, prendre le plus à gauche, celui
du milieu et combiner ces parties.
Si l'index est impair, prendre les octets du milieu et celui le plus à droite et
les combiner.
Ne pas oublier d'isoler les bits inutiles de l'octet du milieu.

Mettre des valeurs se fait presque de la même façon, mais faire attention de ne pas
écraser les \IT{autres} bits de l'octet du milieu, ne modifier que les \IT{votre}.

\subsection{Le code C/C++}

\lstinputlisting[style=customc]{advanced/380_FAT12/FAT12.c}

Pendant le test, tous les éléments de 12-bit sont remplis avec des valeurs dans l'intervalle
0..0xFFF.
Et voici une liste de tous les triplets, chaque ligne a 3 octets:

\begin{lstlisting}
0x000001
0x002003
0x004005
0x006007
0x008009
0x00A00B
0x00C00D
0x00E00F
0x010011
0x012013
0x014015

...

0xFECFED
0xFEEFEF
0xFF0FF1
0xFF2FF3
0xFF4FF5
0xFF6FF7
0xFF8FF9
0xFFAFFB
0xFFCFFD
0xFFEFFF
\end{lstlisting}

Voici une sortie niveau octet de GDB de 300 octets (ou 100 triplets) commençant en
512/2*3, i.e., l'adresse où le 512ème élément (0x200) commence.
J'ai ajouté des crochets dans mon éditeur de texte pour montrer des triplets explicitement.
Remarquez bien l'octet du milieu, où le dernier élément se termine et le suivant commence,
Autrement dit, chaque octet du milieu a les 4 bits bas d'un élément pair et les 4
bits hauts d'un élément impair.

\lstinputlisting{advanced/380_FAT12/gdb.txt}

\subsection{Comment ça fonctionne}

Disons que le tableau est un buffer global pour faciliter l'accès.

\subsubsection{Getter}

Commençons avec la fonction prenant les valeurs du tableau, car elle est plus simple.

La méthode pour trouver le numéro d'un triplet est simplement de diviser l'index
entré par 2, mais nous pouvons faire cela en décalant à droite de 1 bit.
C'est un moyen très répandu de diviser/multiplier par des nombres de la forme $2^n$.

Je peux monter comment ça marche. Disons que vous voulez diviser 123 par 10.
Supprimons le dernier chiffre (3, qui est le reste de la division) et il reste 12
La division par 2 consiste à supprimer le bit le moins significatif. La suppression
peut être faite en décalant à droite.

Maintenant la fonction doit décider si l'index d'entrée est pair (donc la valeur
12-bit est placé à gauche) ou impair (à droite).
La façon la plus simple est d'isoler le bit le plus bas (x\&1). S'il vaut zéro, notre
valeur est paire, autrement elle est impaire.

Ce fait peut être illustré facilement, regardez le bit de poids le plus faible:

\begin{lstlisting}
decimal binary even/odd

0       0000   even
1       0001   odd
2       0010   even
3       0011   odd
4       0100   even
5       0101   odd
6       0110   even
7       0111   odd
8       1000   even
9       1001   odd
10      1010   even
11      1011   odd
12      1100   even
13      1101   odd
14      1110   even
15      1111   odd
...
\end{lstlisting}

Zéro est aussi un nombre pair, c'est ainsi dans le système de complément à deux,
où il se trouve entre deux nombres impairs (-1 et 1).

Pour les matheux, je peux aussi dire que le signe pair ou impair est aussi le
reste de la division par 2.
La division d'un nombre par 2 est simplement supprimer le dernier bit, qui est le
reste de la division.
Bon, nous n'avons pas besoin de décaler ici, juste isoler le bit de poids faible.

Si l'élément est impair, nous prenons les octets du milieu et de droite:\\ (\IT{array[array\_idx+1]}
et \IT{array[array\_idx+2]}).
Les 4 bits de poids faible de l'octet du milieu sont isolés.
L'octet de droite est pris en entier.
Maintenant nous devons combiner ces parties en une valeur 12-bit.
Pour cela, décaler les 4 bits de l'octet du milieu de 8 bits à gauche, afin que ces
4 bits soient situés juste après le 8ème bit le plus haut d'un octet.
Puis, en utilisant l'opération OR, nous pouvons ajouter ces parties.

Si l'élément est pair, les 8 bits hauts de la valeur 12-bit sont situés dans l'octet
de gauche, tandis que les 4 bits bas sont situés dans les 4 bits hauts de l'octet
du milieu.
Nous isolons les 4 bits les plus hauts de l'octet du milieu en le décalant de 4 bits
à droite et ensuite appliquons l'opération AND, afin d'être sûr qu'il ne reste rien.
Nous prenons ensuite l'octet de gauche et décalons sa valeur de 4 bits à gauche,
car il a les bits d'indices 11 à 4 (inclus, en commençant à 0).
En utilisant l'opération OR, nous combinons les deux parties.

\subsubsection{Setter}

Le setter calcule l'adresse du triplet de la même façon.
Il opère sur les octets gauche/droit de la même façon.
Mais il n'est pas correct de simplement écrire l'octet du milieu, car l'opération
d'écriture détruirait l'information relative à l'autre élément.
Donc la façon répandue de faire est de charger l'octet, de supprimer les octets que
vous voulez écrire, les y écrire en laissant l'autre partie intacte.
En utilisant l'opération AND (\TT{\&} in C/C++), nous supprimons tout excepté \IT{notre} partie.
En utilisant l'opération OR (\TT{|} in C/C++), nous modifions l'octet du milieu.

\subsection{GCC 4.8.2 avec optimisation pour x86-64}

Voyons ce que fait GCC 4.8.2 avec optimisation pour Linux x64.
J'ai ajouté des commentaires.
Parfois, les lecteurs sont troublés car l'ordre des instructions n'est pas logique.
C'est okay, car les compilateurs qui optimisent prennent en compte le mécanisme d'exécution
en désordre des CPUs , et parfois,  échanger des instructions permet d'aller plus vite.

\subsubsection{Getter}

\lstinputlisting[style=customasmx86]{advanced/380_FAT12/GCC_getter.asm}

\subsubsection{Setter}

\lstinputlisting[style=customasmx86]{advanced/380_FAT12/GCC_setter.asm}

\subsubsection{Autres commentaires}

Toutes les adresses dans Linux x64 sont sur 64-bit, donc dans l'arithmétique des
pointeurs, toutes les valeurs doivent être 64-bit.
Le code calculant l'offset à l'intérieur d'un  tableau opère sur des valeurs 32-bit
(l'argument d'entrée \IT{idx} a un type \IT{int}, qui a une largeur de 32 bits),
et donc ces valeurs doivent être converties en des adresses 64-bit avant d'effectuer
la lecture/l'écriture en mémoire.
Donc, il y a beaucoup d'instructions d'extensions de signe (comme \INS{CDQE}, \INS{MOVSX})
utilisées pour la conversion.
Pourquoi étendre le signe? D'après les standards C/C++, l'arithmétique des pointeurs
peut opérer sur des valeurs négatives (il est possible d'accéder un tableau en utilisant
des index négatifs comme \IT{array[-123]}, voir: \myref{negative_array_indices}).
Comme le compilateur GCC ne peut pas être certain que les indices sont toujours positifs,
il ajoute des instructions d'extension de signe.

\subsection{Keil 5.05 avec optimisation (mode Thumb)}

\subsubsection{Getter}

Le code suivant a une opération OR finale dans l'épilogue de la fonction.
En effet, il s'exécute à la fin des deux branches, ce qui permet d'économiser un
peu d'espace.

\lstinputlisting[style=customasmARM]{advanced/380_FAT12/Keil_thumb_O3_getter.asm}

Il y a au moins une redondance: \IT{idx*1.5} est calculé deux fois.
À titre d'exercice, le lecteur peut essayer de retravailler la fonction assembleur
afin de la rendre plus courte. N'oubliez pas de tester!

Une autre chose à mentionner est qu'il est difficile de générer de grandes constantes
avec des instructions 16-bit Thumb, donc donc le compilateur Keil génère souvent
du code complexe en utilisant des instructions de décalage afin d'obtenir le même
effet.
Par exemple, il est compliqué de générer le code \INS{AND Rdest, Rsrc, 1} ou \INS{TST Rsrc, 1}
en mode Thumb, donc Keil génère le code \IT{idx} qui décale l'entrée de 31 bits à
gauche et ensuite vérifie si le résultat est zéro ou non.

\subsubsection{Setter}

La première moitié du code du setter est très similaire au getter, l'adresse du triplet
est calculée en premier, puis un saut se produit pour diriger vers le code de gestion
adéquat.

\lstinputlisting[style=customasmARM]{advanced/380_FAT12/Keil_thumb_O3_setter.asm}

\subsection{Keil 5.05 avec optimisation (mode ARM)}

\subsubsection{Getter}

La fonction getter pour ARM n'a pas de branche conditionnelle du tout!
Grâce au suffixe (comme \IT{-EQ}, \IT{-NE}), qui peut être ajouté à de nombreuses
instructions en mode ARM, elles ne sont exécutées que si le flag correspondant est
mis.

De nombreuses instructions en mode ARM peuvent avoir un suffixe de décalage comme
\INS{LSL \#1} (qui signifie que le dernier opérande est décalé de 1 bit).

\lstinputlisting[style=customasmARM]{advanced/380_FAT12/Keil_ARM_O3_getter.asm}

\subsubsection{Setter}

\lstinputlisting[style=customasmARM]{advanced/380_FAT12/Keil_ARM_O3_setter.asm}

La valeur de \IT{idx*1.5} est calculée deux fois, cette redondance produite par le
compilateur Keil peut être éliminée.
Vous pouvez aussi retravailler cette fonction assembleur afin de la raccourcir. N'oubliez
pas de tester!

\subsection{(32-bit ARM) Compariison de la densité du code en modes Thumb et ARM}

Le mode Thumb des CPUs ARM a été introduit afin de rendre les instructions plus courte
(16-bit) au lieu des instructions 32-bit en mode ARM.
Mais comme nous le voyons, il est difficile de dire si ça vaut la peine: le code en
mode ARM est toujours plus court (toutefois, les instructions sont plus longues).

\subsection{GCC 4.9.3 avec optimisation pour ARM64}

\subsubsection{Getter}

\lstinputlisting[style=customasmARM]{advanced/380_FAT12/ARM64_getter.lst}

ARM64 possède la nouvelle instruction cool \INS{UBFIZ} (\IT{Unsigned bitfield insert
in zero, with zeros to left and right}), qui peut être utilisée pour mettre un nombre
spécifié de bits d'un registre vers un autre.
C'est un alias d'une autre instruction, \INS{UBFM} (\IT{Unsigned bitfield move, with
 zeros to left and right}).
L'instruction \INS{UBFM} est l'instruction utilisée en interne en ARM64 au lieu de
\INS{LSL/LSR} (bit shifts).

\subsubsection{Setter}

\lstinputlisting[style=customasmARM]{advanced/380_FAT12/ARM64_setter.lst}

\subsection{GCC 4.4.5 avec optimisation pour MIPS}

Il faut garder à l'esprit que chaque instruction après une instruction
saut/branchement est d'abord exécutée.
C'est appelé \IT{branch delay slot} (slot de délai de branchement) dans le jargon
des CPUs RISC.
Pour rendre les choses plus simple, il suffit d'échange (mentalement) chaque paire
d'instructions qui commence avec une instruction de saut ou de branchement.

Les MIPS n'ont pas de flag (apparemment, pour simplifier les dépendances des données),
donc une instruction de branchement (comme \INS{BNE}) effectue à la fois la comparaison
et le branchement.

Il y a aussi le code de mise en place de GP (Global Pointer), qui peut-être ignoré
jusqu'ici.

\subsubsection{Getter}

\lstinputlisting[style=customasmMIPS]{advanced/380_FAT12/MIPS_getter.asm}

\subsubsection{Setter}

\lstinputlisting[style=customasmMIPS]{advanced/380_FAT12/MIPS_setter.asm}

\subsection{Différence avec le FAT12 réel}

La table FAT12 réelle est légèrement différente: \url{https://en.wikipedia.org/wiki/Design_of_the_FAT_file_system\#Cluster_map}.

Pour les éléments pairs:

\begin{lstlisting}
 +0 +1 +2
|23|.1|..|..
\end{lstlisting}

Pour les éléments impairs:

\begin{lstlisting}
 +0 +1 +2
|..|3.|12|..
\end{lstlisting}

\href{https://github.com/torvalds/linux/blob/de182468d1bb726198abaab315820542425270b7/fs/fat/fatent.c#L117}{fat12\_ent\_get()},
\href{https://github.com/torvalds/linux/blob/de182468d1bb726198abaab315820542425270b7/fs/fat/fatent.c#L153}{fat12\_ent\_put()}.


\subsection{Exercice}

Peut-être qu'il y a un moyen de stocker les données d'une certaine façon, afin que
les fonctions setter/getter soient plus rapide.
Si nous placions les valeurs de cette façon:

(Éléments pairs)

\begin{lstlisting}
 +0 +1 +2
|23|1.|..|..
\end{lstlisting}

(Éléments impairs)

\begin{lstlisting}
 +0 +1 +2
|..|.1|23|..
\end{lstlisting}

Cette façon de stocker les données permet d'éliminer au moins une instruction de
décalage.
Comme exercice, vous pouvez retravailler mon code \CCpp dans ce sens et voir ce que
le compilateur va produire.

\subsection{Résumé}

Les opérations de décalage de bits (\TT{<<} et \TT{>>} en \CCpp, \INS{SHL}/{SHR}/{SAL}/{SAR}
en x86, \INS{LSL}/\INS{LSR} en ARM, \INS{SLL}/\INS{SRL} en MIPS) sont utilisés pour
placer un ou des bit(s) à des endroits spécifiques.

L'opération AND (\TT{\&} en \CCpp, \INS{AND} en x86/ARM) est utilisée pour supprimer
des bits non voulus, aussi pendant l'isolement.

L'opération OR (\TT{|} en \CCpp, \INS{OR} en x86/ARM) est utilisé pour fusionner
ou combiner plusieurs valeurs en une seule.
Une valeur en entrée ne doit pas utiliser l'espace à une place où une autre valeur
a ses bits.

ARM64 a les nouvelles instructions \INS{UBFM}, \INS{UFBIZ} pour déplacer des bits
spécifiques d'un registre vers un autre.

\subsection{Conclusion}

FAT12 n'est plus guère utilisé de nos jours, mais si vous avez des contraintes d'espaces
et devez stocker des valeurs limitées à 12 bits, vous pouvez envisager un tableau
dense à la façon dont cela est fait dans la table FAT12.

