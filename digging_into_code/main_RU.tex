\chapter{Поиск в коде того что нужно}

Современное ПО, в общем-то, минимализмом не отличается.

\myindex{\Cpp!STL}
Но не потому, что программисты слишком много пишут, 
а потому что к исполняемым файлам обыкновенно прикомпилируют все подряд библиотеки. 
Если бы все вспомогательные библиотеки всегда выносили во внешние DLL, мир был бы иным.
(Еще одна причина для Си++ --- \ac{STL} и прочие библиотеки шаблонов.)

\newcommand{\FOOTNOTEBOOST}{\footnote{\url{http://go.yurichev.com/17036}}}
\newcommand{\FOOTNOTELIBPNG}{\footnote{\url{http://go.yurichev.com/17037}}}

Таким образом, очень полезно сразу понимать, какая функция из стандартной библиотеки или 
более-менее известной (как Boost\FOOTNOTEBOOST, libpng\FOOTNOTELIBPNG), 
а какая --- имеет отношение к тому что мы пытаемся найти в коде.

Переписывать весь код на \CCpp, чтобы разобраться в нем, безусловно, не имеет никакого смысла.

Одна из важных задач reverse engineer-а это быстрый поиск в коде того что собственно его интересует.

\myindex{\GrepUsage}
Дизассемблер \IDA позволяет делать поиск как минимум строк, последовательностей байт, констант.
Можно даже сделать экспорт кода в текстовый файл .lst или .asm и затем натравить на него \TT{grep}, \TT{awk}, итд.

Когда вы пытаетесь понять, что делает тот или иной код, это запросто может быть какая-то 
опенсорсная библиотека вроде libpng. Поэтому, когда находите константы, или текстовые строки, которые 
выглядят явно знакомыми, всегда полезно их \IT{погуглить}.
А если вы найдете искомый опенсорсный проект где это используется, 
то тогда будет достаточно будет просто сравнить вашу функцию с ней. 
Это решит часть проблем.

К примеру, если программа использует какие-то XML-файлы, первым шагом может быть
установление, какая именно XML-библиотека для этого используется, ведь часто используется какая-то
стандартная (или очень известная) вместо самодельной.

\myindex{SAP}
\myindex{Windows!PDB}
К примеру, автор этих строк однажды пытался разобраться как происходит компрессия/декомпрессия сетевых пакетов в SAP 6.0. 
Это очень большая программа, но к ней идет подробный .\gls{PDB}-файл с отладочной информацией, и это очень удобно. 
Он в конце концов пришел к тому что одна из функций декомпрессирующая пакеты называется CsDecomprLZC(). 
Не сильно раздумывая, он решил погуглить и оказалось, что функция с таким же названием имеется в MaxDB
(это опен-сорсный проект SAP) \footnote{Больше об этом в соответствующей секции~(\myref{sec:SAPGUI})}.

\url{http://www.google.com/search?q=CsDecomprLZC}

Каково же было мое удивление, когда оказалось, что в MaxDB используется точно такой же алгоритм, 
скорее всего, с таким же исходником.

\mysection{Идентификация исполняемых файлов}

\subsection{Microsoft Visual C++}
\label{MSVC_versions}

Версии MSVC и DLL которые могут быть импортированы:

\small
\begin{center}
\begin{tabular}{ | l | l | l | l | l | }
\hline
\HeaderColor Маркетинговая вер. & 
\HeaderColor Внутренняя вер. & 
\HeaderColor Вер. CL.EXE &
\HeaderColor Импорт.DLL &
\HeaderColor Дата выхода \\
\hline
% 4.0, April 1995
% 97 & 5.0 & February 1997
6		&  6.0	& 12.00	& msvcrt.dll	& June 1998		\\
		&	&	& msvcp60.dll	&			\\
\hline
.NET (2002)	&  7.0	& 13.00	& msvcr70.dll	& February 13, 2002	\\
		&	&	& msvcp70.dll	&			\\
\hline
.NET 2003	&  7.1	& 13.10 & msvcr71.dll	& April 24, 2003	\\
		&	&	& msvcp71.dll	&			\\
\hline
2005		&  8.0	& 14.00 & msvcr80.dll	& November 7, 2005	\\
		&	&	& msvcp80.dll	&			\\
\hline
2008		&  9.0	& 15.00 & msvcr90.dll	& November 19, 2007	\\
		&	&	& msvcp90.dll	&			\\
\hline
2010		& 10.0	& 16.00 & msvcr100.dll	& April 12, 2010 	\\
		&	&	& msvcp100.dll	&			\\
\hline
2012		& 11.0	& 17.00 & msvcr110.dll	& September 12, 2012 	\\
		&	&	& msvcp110.dll	&			\\
\hline
2013		& 12.0	& 18.00 & msvcr120.dll	& October 17, 2013 	\\
		&	&	& msvcp120.dll	&			\\
\hline
\end{tabular}
\end{center}
\normalsize

msvcp*.dll содержит функции связанные с \Cpp{}, так что если она импортируется, скорее всего, 
вы имеете дело с программой на \Cpp.

\subsubsection{Name mangling}

Имена обычно начинаются с символа \TT{?}.

О \gls{name mangling} в MSVC читайте также здесь: \myref{namemangling}.

\subsection{GCC}
\myindex{GCC}

Кроме компиляторов под *NIX, GCC имеется так же и для win32-окружения: в виде Cygwin и MinGW.

\subsubsection{Name mangling}

Имена обычно начинаются с символов \TT{\_Z}.

О \gls{name mangling} в GCC читайте также здесь: \myref{namemangling}.

\subsubsection{Cygwin}
\myindex{Cygwin}

cygwin1.dll часто импортируется.

\subsubsection{MinGW}
\myindex{MinGW}

msvcrt.dll может импортироваться.

\subsection{Intel Fortran}
\myindex{Фортран}

libifcoremd.dll, libifportmd.dll и libiomp5md.dll (поддержка OpenMP) могут импортироваться.

В libifcoremd.dll много функций с префиксом \TT{for\_}, что значит \IT{Fortran}.

\subsection{Watcom, OpenWatcom}
\myindex{Watcom}
\myindex{OpenWatcom}

\subsubsection{Name mangling}

Имена обычно начинаются с символа \TT{W}.

Например, так кодируется метод \q{method} класса \q{class} не имеющий аргументов и возвращающий \Tvoid{}:

\begin{lstlisting}
W?method$_class$n__v
\end{lstlisting}

\subsection{Borland}
\myindex{Borland Delphi}
\myindex{Borland C++Builder}

Вот пример \gls{name mangling} в Borland Delphi и C++Builder:

\lstinputlisting{digging_into_code/identification/borland_mangling.txt}

Имена всегда начинаются с символа \TT{@} 
затем следует имя класса, имя метода
и закодированные типы аргументов.

Эти имена могут присутствовать с импортах .exe, экспортах .dll, отладочной информации, итд.

Borland Visual Component Libraries (VCL) находятся в файлах .bpl вместо .dll, например, vcl50.dll, rtl60.dll.

Другие DLL которые могут импортироваться: BORLNDMM.DLL.

\subsubsection{Delphi}

Почти все исполняемые файлы имеют текстовую строку \q{Boolean} 
в самом начале сегмента кода, среди остальных имен типов.

Вот очень характерное для Delphi начало сегмента \TT{CODE}, 
этот блок следует сразу за заголовком win32 PE-файла:

\lstinputlisting{digging_into_code/identification/delphi.txt}

Первые 4 байта сегмента данных (\TT{DATA}) в исполняемых файлах могут быть \TT{00 00 00 00}, \TT{32 13 8B C0} или \TT{FF FF FF FF}.
Эта информация может помочь при работе с запакованными/зашифрованными программами на Delphi.

\subsection{Другие известные DLL}

\myindex{OpenMP}
\begin{itemize}
\item vcomp*.dll --- Реализация OpenMP от Microsoft.
\end{itemize}


% binary files might be also here

\mysection{Связь с внешним миром (на уровне функции)}
Очень желательно следить за аргументами ф-ции и возвращаемыми значениями, в отладчике или \ac{DBI}.
Например, автор этих строк однажды пытался понять значение некоторой очень запутанной ф-ции, которая, как потом оказалось,
была неверно реализованной пузырьковой сортировкой\footnote{\url{https://yurichev.com/blog/weird_sort/}}.
(Она работала правильно, но медленнее.)
В то же время, наблюдение за входами и выходами этой ф-ции помогает мгновенно понять, что она делает.

Часто, когда вы видите деление через умножение (\myref{sec:divisionbymult}),
но забыли все детали о том, как оно работает, вы можете просто наблюдать за входом и выходом, и так быстро найти делитель.

% sections:
\mysection{Связь с внешним миром (win32)}

Иногда, чтобы понять, что делает та или иная функция, можно её не разбирать, а просто посмотреть на её входы и выходы.
Так можно сэкономить время.

Обращения к файлам и реестру: 
для самого простого анализа может помочь утилита Process Monitor\footnote{\url{http://go.yurichev.com/17301}}
от SysInternals.

Для анализа обращения программы к сети, может помочь  Wireshark\footnote{\url{http://go.yurichev.com/17303}}.

Затем всё-таки придётся смотреть внутрь. \\
\\
Первое на что нужно обратить внимание, это какие функции из \ac{API} \ac{OS}
и какие функции стандартных библиотек используются.
Если программа поделена на главный исполняемый файл и группу DLL-файлов, то имена функций в этих DLL, бывает так, могут помочь.

Если нас интересует, что именно приводит к вызову \TT{MessageBox()} с определенным текстом, 
то первое, что можно попробовать сделать: найти в сегменте данных этот текст, найти ссылки на него, и найти, 
откуда может передаться управление к интересующему нас вызову \TT{MessageBox()}.

\myindex{\CStandardLibrary!rand()}
Если речь идет о компьютерной игре, и нам интересно какие события в ней более-менее случайны, 
мы можем найти функцию \rand или её заменитель (как алгоритм Mersenne twister), и посмотреть, 
из каких мест эта функция вызывается и что самое главное: как используется результат этой функции.%

% BUG in varioref: http://tex.stackexchange.com/questions/104261/varioref-vref-or-vpageref-at-page-boundary-may-loop
Один пример: \ref{chap:color_lines}. 

Но если это не игра, а \rand используется, то также весьма любопытно, зачем. 
Бывают неожиданные случаи вроде использования \rand в алгоритме для сжатия данных (для имитации шифрования):
\href{http://go.yurichev.com/17221}{blog.yurichev.com}.

\subsection{Часто используемые функции Windows API}

Это функции которые можно увидеть в числе импортируемых.
Но также нельзя забывать, что далеко не все они были использованы в коде написанном автором.
Немалая часть может вызываться из библиотечных функций и \ac{CRT}-кода.
	
Многие ф-ции могут иметь суффикс \GTT{-A} для ASCII-версии и \GTT{-W} для Unicode-версии.

\begin{itemize}

\item
Работа с реестром (advapi32.dll): 
RegEnumKeyEx, RegEnumValue, RegGetValue, RegOpenKeyEx, RegQueryValueEx.

\item
Работа с текстовыми .ini-файлами (kernel32.dll):\\
GetPrivateProfileString.

\item
Диалоговые окна (user32.dll):\\
MessageBox, MessageBoxEx, CreateDialog, SetDlgItemText, GetDlgItemText.

\item
Работа с ресурсами (\myref{PEresources}): (user32.dll):
LoadMenu.

\item
Работа с TCP/IP-сетью (ws2\_32.dll):
WSARecv, WSASend.

\item
Работа с файлами (kernel32.dll):
CreateFile, ReadFile, ReadFileEx, WriteFile, WriteFileEx.

\item
Высокоуровневая работа с Internet
(wininet.dll):
WinHttpOpen.

\item
Проверка цифровой подписи исполняемого файла (wintrust.dll):
WinVerifyTrust.

\item
Стандартная библиотека MSVC (в случае динамического связывания)%
 (msvcr*.dll):
assert, itoa, ltoa, open, printf, read, strcmp, atol, atoi, fopen, fread, fwrite, memcmp, rand,
strlen, strstr, strchr.

\end{itemize}

\subsection{Расширение триального периода}

Ф-ции доступа к реестру это частая цель тех, кто пытается расширить триальный период ПО, которое
может сохранять дату/время инсталляции в реестре.

Другая популярная цель это ф-ции GetLocalTime() и GetSystemTime():
триальное ПО, при каждом запуске, должно как-то проверять текущую дату/время.

% FIXME language!
\subsection{Удаление nag-окна}

Популярный метод поиска того, что заставляет выводить nag-окно это перехват ф-ций MessageBox(),
CreateDialog() и CreateWindow().

\subsection{tracer: Перехват всех функций в отдельном модуле}
\myindex{tracer}

\myindex{x86!\Instructions!INT3}
В \tracer есть поддержка точек останова INT3, хотя и срабатывающие только один раз, но зато их можно установить на все
сразу функции в некоей DLL.

\begin{lstlisting}
--one-time-INT3-bp:somedll.dll!.*
\end{lstlisting}

Либо, поставим INT3-прерывание на все функции, имена которых начинаются с префикса \TT{xml}:

\begin{lstlisting}
--one-time-INT3-bp:somedll.dll!xml.*
\end{lstlisting}

В качестве обратной стороны медали, такие прерывания срабатывают только один раз.
Tracer покажет вызов какой-либо функции, если он случится, но только один раз.
Еще один недостаток --- увидеть аргументы функции также нельзя.

Тем не менее, эта возможность очень удобна для тех ситуаций, 
когда вы знаете что некая программа использует некую DLL,
но не знаете какие именно функции в этой DLL.

И функций много. 

\par
\myindex{Cygwin}
Например, попробуем узнать, что использует cygwin-утилита uptime:

\begin{lstlisting}
tracer -l:uptime.exe --one-time-INT3-bp:cygwin1.dll!.*
\end{lstlisting}

Так мы можем увидеть все функции из библиотеки cygwin1.dll, которые были вызваны хотя бы один раз, и откуда:

\lstinputlisting{digging_into_code/uptime_cygwin.txt}


\mysection{Строки}
\label{sec:digging_strings}

\mysection{Оптимизации циклов}

% subsections:
\subsection{Странная оптимизация циклов}

Это самая простая (из всех возможных) реализация memcpy():

\begin{lstlisting}[style=customc]
void memcpy (unsigned char* dst, unsigned char* src, size_t cnt)
{
	size_t i;
	for (i=0; i<cnt; i++)
		dst[i]=src[i];
};
\end{lstlisting}

Как минимум MSVC 6.0 из конца 90-х вплоть до MSVC 2013 может выдавать вот такой странный код (этот листинг создан MSVC 2013
x86):

\lstinputlisting[style=customasmx86]{advanced/500_loop_optimizations/1_1_RU.lst}

Это всё странно, потому что как люди работают с двумя указателями? Они сохраняют два адреса в двух регистрах или двух
ячейках памяти.
Компилятор MSVC в данном случае сохраняет два указателя как один указатель (\IT{скользящий dst} в \EAX)
и разницу между указателями \IT{src} и \IT{dst} (она остается неизменной во время исполнения цикла, в \ESI).
\myindex{\CLanguageElements!ptrdiff\_t}
(Кстати, это тот редкий случай, когда можно использовать тип ptrdiff\_t.)
Когда нужно загрузить байт из \IT{src}, он загружается на \IT{diff + скользящий dst} и сохраняет байт просто на
\IT{скользящем dst}.

Должно быть это какой-то трюк для оптимизации. Но я переписал эту ф-цию так:

\lstinputlisting[style=customasmx86]{advanced/500_loop_optimizations/1_2.lst}

\dots и она работает также быстро как и \IT{соптимизированная} версия на моем Intel Xeon E31220 @ 3.10GHz.
Может быть, эта оптимизация предназначалась для более старых x86-процессоров 90-х, т.к., этот трюк использует
как минимум древний MS VC 6.0?

Есть идеи?

\myindex{Hex-Rays}
Hex-Rays 2.2 не распознает такие шаблонные фрагменты кода (будем надеятся, это временно?):

\begin{lstlisting}[style=customc]
void __cdecl f1(char *dst, char *src, size_t size)
{
  size_t counter; // edx@1
  char *sliding_dst; // eax@2
  char tmp; // cl@3

  counter = size;
  if ( size )
  {
    sliding_dst = dst;
    do
    {
      tmp = (sliding_dst++)[src - dst];         // разница (src-dst) вычисляется один раз, перед телом цикла
      *(sliding_dst - 1) = tmp;
      --counter;
    }
    while ( counter );
  }
}
\end{lstlisting}

Тем не менее, этот трюк часто используется в MSVC (и не только в самодельных ф-циях \IT{memcpy()}, но также и во многих
циклах, работающих с двумя или более массивами), так что для реверс-инжиниров стоит помнить об этом.

% <!-- As of why writting occurred after <b>dst</b> incrementing? -->


\subsection{Возврат строки}

Классическая ошибка из \RobPikePractice{}:

\begin{lstlisting}[style=customc]
#include <stdio.h>

char* amsg(int n, char* s)
{
        char buf[100];

        sprintf (buf, "error %d: %s\n", n, s) ;

        return buf;
};

int main()
{
        printf ("%s\n", amsg (1234, "something wrong!"));
};
\end{lstlisting}

Она упадет.
В начале, попытаемся понять, почему.

Это состояние стека перед возвратом из amsg():

% FIXME! TikZ or whatever
\begin{lstlisting}
§(низкие адреса)§

§[amsg(): 100 байт]§
§[RA]                               <- текущий SP§
§[два аргумента amsg]§
§[что-то еще]§
§[локальные переменные main()]§

§(высокие адреса)§
\end{lstlisting}

Когда управление возвращается из amsg() в \main, пока всё хорошо.
Но когда \printf вызывается из \main, который, в свою очередь, использует стек для своих нужд, затирая 100-байтный буфер.
В лучшем случае, будет выведен случайный мусор.

Трудно поверить, но я знаю, как это исправить:

\begin{lstlisting}[style=customc]
#include <stdio.h>

char* amsg(int n, char* s)
{
        char buf[100];

        sprintf (buf, "error %d: %s\n", n, s) ;

        return buf;
};

char* interim (int n, char* s)
{
        char large_buf[8000];
        // используем локальный массив.
        // а иначе компилятор выбросит его при оптимизации, как неиспользуемый.
        large_buf[0]=0;
        return amsg (n, s);
};

int main()
{
        printf ("%s\n", interim (1234, "something wrong!"));
};
\end{lstlisting}

Это заработает если скомпилировано в MSVC 2013 без оптимизаций и с опцией \TT{/GS-}\footnote{Выключить защиту от переполнения буфера}.
MSVC предупредит: ``warning C4172: returning address of local variable or temporary'', но код запустится и сообщение выведется.
Посмотрим состояние стека в момент, когда amsg() возвращает управление в interim():

\begin{lstlisting}
§(низкие адреса)§

§[amsg(): 100 байт]§
§[RA]                                      <- текущий SP§
§[два аргумента amsg()]§
§[вледения interim(), включая 8000 байт]§
§[еще что-то]§
§[локальные переменные main()]§

§(высокие адреса)§
\end{lstlisting}

Теперь состояние стека на момент, когда interim() возвращает управление в \main{}:

\begin{lstlisting}
§(низкие адреса)§

§[amsg(): 100 байт]§
§[RA]§
§[два аргумента amsg()]§
§[вледения interim(), включая 8000 байт]§
§[еще что-то]                              <- текущий SP§
§[локальные переменные main()]§

§(высокие адреса)§
\end{lstlisting}

Так что когда \main вызывает \printf, он использует стек в месте, где выделен буфер в interim(),
и не затирает 100 байт с сообщение об ошибке внутри, потому что 8000 байт (или может быть меньше) это достаточно для всего,
что делает \printf и другие нисходящие ф-ции!

Это также может сработать, если между ними много ф-ций, например:
\main $\rightarrow$ f1() $\rightarrow$ f2() $\rightarrow$ f3() ... $\rightarrow$ amsg(),
и тогда результат amsg() используется в \main.
Дистанция между \ac{SP} в \main и адресом буфера \TT{buf[]} должна быть достаточно длинной.

Вот почему такие ошибки опасны: иногда ваш код работает (и бага прячется незамеченной). иногда нет.
\label{heisenbug}
\myindex{Хейзенбаги}
Такие баги в шутку называют хейзенбаги или шрёдинбаги\footnote{\url{https://en.wikipedia.org/wiki/Heisenbug}}.





\subsection{Поиск строк в бинарном файле}

\epigraph{Actually, the best form of Unix documentation is frequently running the
\textbf{strings} command over a program’s object code. Using \textbf{strings}, you can get
a complete list of the program’s hard-coded file name, environment variables,
undocumented options, obscure error messages, and so forth.}{The Unix-Haters Handbook}

\myindex{UNIX!strings}
Стандартная утилита в UNIX \IT{strings} это самый простой способ увидеть строки в файле.
Например, это строки найденные в исполняемом файле sshd из OpenSSH 7.2:

\lstinputlisting{digging_into_code/sshd_strings.txt}

Тут опции, сообщения об ошибках, пути к файлам, импортируемые модули, функции, и еще какие-то странные строки (ключи?)
Присутствует также нечитаемый шум---иногда в x86-коде бывают целые куски состоящие из печатаемых ASCII-символов,
вплоть до ~8 символов.

Конечно, OpenSSH это опенсорсная программа.
Но изучение читаемых строк внутри некоторого неизвестного бинарного файла это зачастую самый первый шаг в анализе.
\myindex{UNIX!grep}

Также можно использовать \IT{grep}.

\myindex{Hiew}
\myindex{Sysinternals}
В Hiew есть такая же возможность (Alt-F6), также как и в Sysinternals ProcessMonitor.

\subsection{Сообщения об ошибках и отладочные сообщения}

Очень сильно помогают отладочные сообщения, если они имеются. В некотором смысле, отладочные сообщения, 
это отчет о том, что сейчас происходит в программе.
Зачастую, это \printf-подобные функции, 
которые пишут куда-нибудь в лог, а бывает так что и не пишут ничего, но вызовы остались, так как эта сборка --- не
отладочная, а \IT{release}.

\myindex{\oracle}
Если в отладочных сообщениях дампятся значения некоторых локальных или глобальных переменных, 
это тоже может помочь, как минимум, узнать их имена. 
Например, в \oracle одна из таких функций: \TT{ksdwrt()}.

Осмысленные текстовые строки вообще очень сильно могут помочь. 
Дизассемблер \IDA может сразу указать, из какой функции и из какого её места используется эта строка. 
Встречаются и смешные случаи
\footnote{\href{http://go.yurichev.com/17223}{blog.yurichev.com}}.

Сообщения об ошибках также могут помочь найти то что нужно. 
В \oracle сигнализация об ошибках проходит при помощи вызова некоторой группы функций. \\
Тут еще немного об этом: \href{http://go.yurichev.com/17224}{blog.yurichev.com}.

\myindex{Error messages}
Можно довольно быстро найти, какие функции сообщают о каких ошибках, и при каких условиях.

Это, кстати, одна из причин, почему в защите софта от копирования, 
бывает так, что сообщение об ошибке заменяется 
невнятным кодом или номером ошибки. Мало кому приятно, если взломщик быстро поймет, 
из-за чего именно срабатывает защита от копирования, просто по сообщению об ошибке.

Один из примеров шифрования сообщений об ошибке, здесь: \myref{examples_SCO}.

\subsection{Подозрительные магические строки}

Некоторые магические строки, используемые в бэкдорах выглядят очень подозрительно.
Например, в домашних роутерах TP-Link WR740 был бэкдор
\footnote{\url{http://sekurak.pl/tp-link-httptftp-backdoor/}, на русском: \url{http://m.habrahabr.ru/post/172799/}}.
Бэкдор активировался при посещении следующего URL:\\
\url{http://192.168.0.1/userRpmNatDebugRpm26525557/start_art.html}.\\
Действительно, строка \q{userRpmNatDebugRpm26525557} присутствует в прошивке.

Эту строку нельзя было нагуглить до распространения информации о бэкдоре.

Вы не найдете ничего такого ни в одном \ac{RFC}.

Вы не найдете ни одного алгоритма, который бы использовал такие странные последовательности байт.

И это не выглядит как сообщение об ошибке, или отладочное сообщение.

Так что проверить использование подобных странных строк --- это всегда хорошая идея.
\\
\myindex{base64}
Иногда такие строки кодируются при помощи 
base64\footnote{Например, бэкдор в кабельном модеме Arris: 
\url{http://www.securitylab.ru/analytics/461497.php}}.
Так что неплохая идея их всех декодировать и затем просмотреть глазами, пусть даже бегло.
\\
\myindex{Security through obscurity}

Более точно, такой метод сокрытия бэкдоров называется \q{security through obscurity} (безопасность через
запутанность).

\mysection{Вызовы assert()}
\myindex{\CStandardLibrary!assert()}
Может также помочь наличие \TT{assert()} в коде: обычно этот макрос оставляет название файла-исходника, 
номер строки, и условие.

Наиболее полезная информация содержится в assert-условии, по нему можно судить по именам переменных
или именам полей структур. Другая полезная информация --- это имена файлов, по их именам можно попытаться
предположить, что там за код. Также, по именам файлов можно опознать какую-либо очень известную опен-сорсную
библиотеку.

\lstinputlisting[caption=Пример информативных вызовов assert(),style=customasmx86]{digging_into_code/assert_examples.lst}

Полезно \q{гуглить} и условия и имена файлов, это может вывести вас к опен-сорсной библиотеке.
Например, если \q{погуглить} \q{sp->lzw\_nbits <= BITS\_MAX}, 
это вполне предсказуемо выводит на опенсорсный код, что-то связанное с LZW-компрессией.


\mysection{Константы}

Люди, включая программистов, часто используют круглые числа вроде 10, 100, 1000, в т.ч. и в коде.

Практикующие реверсеры, обычно, хорошо знают их в шестнадцатеричном представлении:
10=0xA, 100=0x64, 1000=0x3E8, 10000=0x2710.

Иногда попадаются константы \TT{0xAAAAAAAA} \\
(0b10101010101010101010101010101010) и
\TT{0x55555555} (0b01010101010101010101010101010101) --- это чередующиеся биты.
Это помогает отличить некоторый сигнал от сигнала где все биты включены (0b1111 \dots) или выключены (0b0000 \dots).

Например, константа \TT{0x55AA} используется как минимум в бут-секторе, \ac{MBR}, 
и в \ac{ROM} плат-расширений IBM-компьютеров.

Некоторые алгоритмы, особенно криптографические, используют хорошо различимые константы, 
которые при помощи \IDA легко находить в коде.

\myindex{MD5}
\newcommand{\URLMD}{http://go.yurichev.com/17110}

Например, алгоритм MD5\footnote{\href{\URLMD}{wikipedia}} инициализирует свои внутренние переменные так:

\begin{verbatim}
var int h0 := 0x67452301
var int h1 := 0xEFCDAB89
var int h2 := 0x98BADCFE
var int h3 := 0x10325476
\end{verbatim}

Если в коде найти использование этих четырех констант подряд --- очень высокая вероятность что эта функция имеет отношение к MD5.

\par
Еще такой пример это алгоритмы CRC16/CRC32, часто, алгоритмы вычисления контрольной суммы по CRC 
используют заранее заполненные таблицы, вроде:

\begin{lstlisting}[caption=linux/lib/crc16.c,style=customc]
/** CRC table for the CRC-16. The poly is 0x8005 (x^16 + x^15 + x^2 + 1) */
u16 const crc16_table[256] = {
	0x0000, 0xC0C1, 0xC181, 0x0140, 0xC301, 0x03C0, 0x0280, 0xC241,
	0xC601, 0x06C0, 0x0780, 0xC741, 0x0500, 0xC5C1, 0xC481, 0x0440,
	0xCC01, 0x0CC0, 0x0D80, 0xCD41, 0x0F00, 0xCFC1, 0xCE81, 0x0E40,
	...
\end{lstlisting}

См. также таблицу CRC32: \myref{sec:CRC32}.

В бестабличных алгоритмах CRC используются хорошо известные полиномы, например 0xEDB88320 для CRC32.

\subsection{Магические числа}
\label{magic_numbers}

\newcommand{\FNURLMAGIC}{\footnote{\href{http://go.yurichev.com/17112}{wikipedia}}}

Немало форматов файлов определяет стандартный заголовок файла где используются \IT{магическое число} (magic number)\FNURLMAGIC{}, один или даже несколько.

\myindex{MS-DOS}
Скажем, все исполняемые файлы для Win32 и MS-DOS начинаются с двух символов \q{MZ}\footnote{\href{http://go.yurichev.com/17113}{wikipedia}}.

\myindex{MIDI}
В начале MIDI-файла должно быть \q{MThd}. Если у нас есть использующая для чего-нибудь MIDI-файлы программа,
наверняка она будет проверять MIDI-файлы на правильность хотя бы проверяя первые 4 байта.

Это можно сделать при помощи:
(\IT{buf} указывает на начало загруженного в память файла)

\begin{lstlisting}[style=customasmx86]
cmp [buf], 0x6468544D ; "MThd"
jnz _error_not_a_MIDI_file
\end{lstlisting}

\myindex{\CStandardLibrary!memcmp()}
\myindex{x86!\Instructions!CMPSB}
\dots либо вызвав функцию сравнения блоков памяти \TT{memcmp()} или любой аналогичный код, 
вплоть до инструкции \TT{CMPSB} (\myref{REPE_CMPSx}).

Найдя такое место мы получаем как минимум информацию о том, где начинается загрузка MIDI-файла, во-вторых, 
мы можем увидеть где располагается буфер с содержимым файла, и что еще оттуда берется, и как используется.

\subsubsection{Даты}

\myindex{UFS2}
\myindex{FreeBSD}
\myindex{HASP}

Часто, можно встретить число вроде \TT{0x19870116}, которое явно выглядит как дата (1987-й год, 1-й месяц (январь), 16-й день).
Это может быть чей-то день рождения (программиста, его/её родственника, ребенка), либо какая-то другая важная дата.
Дата может быть записана и в другом порядке, например \TT{0x16011987}.
Даты в американском стиле также популярны, например \TT{0x01161987}.

Известный пример это \TT{0x19540119} (магическое число используемое в структуре суперблока UFS2), это день рождения Маршала Кирка МакКузика, видного разработчика FreeBSD.

\myindex{Stuxnet}
В Stuxnet используется число ``19790509'' (хотя и не как 32-битное число, а как строка), и это привело к догадкам,
что этот зловред связан с Израелем\footnote{Это дата казни персидского еврея Habib Elghanian-а}.

Также, числа вроде таких очень популярны в любительской криптографии, например, это отрывок из внутренностей \IT{секретной функции} донглы HASP3
\footnote{\url{https://web.archive.org/web/20160311231616/http://www.woodmann.com/fravia/bayu3.htm}}:

\begin{lstlisting}[style=customc]
void xor_pwd(void) 
{ 
	int i; 
	
	pwd^=0x09071966;
	for(i=0;i<8;i++) 
	{ 
		al_buf[i]= pwd & 7; pwd = pwd >> 3; 
	} 
};

void emulate_func2(unsigned short seed)
{ 
	int i, j; 
	for(i=0;i<8;i++) 
	{ 
		ch[i] = 0; 
		
		for(j=0;j<8;j++)
		{ 
			seed *= 0x1989; 
			seed += 5; 
			ch[i] |= (tab[(seed>>9)&0x3f]) << (7-j); 
		}
	} 
}
\end{lstlisting}

\subsubsection{DHCP}

Это касается также и сетевых протоколов. 
Например, сетевые пакеты протокола DHCP содержат так называемую \IT{magic cookie}: \TT{0x63538263}. 
Какой-либо код, генерирующий пакеты по протоколу DHCP где-то и как-то должен внедрять в пакет также и эту константу. 
Найдя её в коде мы сможем найти место где происходит это и не только это. 
Любая программа, получающая DHCP-пакеты, должна где-то как-то проверять \IT{magic cookie}, 
сравнивая это поле пакета с константой.

Например, берем файл dhcpcore.dll из Windows 7 x64 и ищем эту константу. 
И находим, два раза: оказывается, эта константа используется в функциях с красноречивыми названиями \\
\TT{DhcpExtractOptionsForValidation()} и \TT{DhcpExtractFullOptions()}:

\begin{lstlisting}[caption=dhcpcore.dll (Windows 7 x64),style=customasmx86]
.rdata:000007FF6483CBE8 dword_7FF6483CBE8 dd 63538263h          ; DATA XREF: DhcpExtractOptionsForValidation+79
.rdata:000007FF6483CBEC dword_7FF6483CBEC dd 63538263h          ; DATA XREF: DhcpExtractFullOptions+97
\end{lstlisting}

А вот те места в функциях где происходит обращение к константам:

\begin{lstlisting}[caption=dhcpcore.dll (Windows 7 x64),style=customasmx86]
.text:000007FF6480875F  mov     eax, [rsi]
.text:000007FF64808761  cmp     eax, cs:dword_7FF6483CBE8
.text:000007FF64808767  jnz     loc_7FF64817179
\end{lstlisting}

И:

\begin{lstlisting}[caption=dhcpcore.dll (Windows 7 x64),style=customasmx86]
.text:000007FF648082C7  mov     eax, [r12]
.text:000007FF648082CB  cmp     eax, cs:dword_7FF6483CBEC
.text:000007FF648082D1  jnz     loc_7FF648173AF
\end{lstlisting}

\subsection{Специфические константы}

Иногда, бывают какие-то специфические константы для некоторого типа кода.
Например, однажды автор сих строк пытался разобраться с кодом, где подозрительно часто встречалось число 12.
Размеры многих массивов также были 12, или кратные 12 (24, итд).
Оказалось, этот код брал на вход 12-канальный аудиофайл и обрабатывал его.

И наоборот: например, если программа работает с текстовым полем длиной 120 байт, значит где-то в коде должна
быть константа 120, или 119.
Если используется UTF-16, то тогда $2 \cdot 120$.
Если код работает с сетевыми пакетами фиксированной длины, то хорошо бы и такую константу поискать в коде.

Это также справедливо для любительской криптографии (ключи с лицензией, итд).
Если зашифрованный блок имеет размер в $n$ байт, вы можете попробовать поискать это число в коде.
Также, если вы видите фрагмент кода, который при исполнении, повторяется $n$ раз в цикле,
это может быть ф-ция шифрования/дешифрования.

\subsection{Поиск констант}

В \IDA это очень просто, Alt-B или Alt-I.

\myindex{binary grep}
А для поиска константы в большом количестве файлов, либо для поиска их в неисполняемых файлах, имеется небольшая утилита
\IT{binary grep}\footnote{\BGREPURL}.


\mysection{Поиск нужных инструкций}

Если программа использует инструкции сопроцессора, и их не очень много, 
то можно попробовать вручную проверить отладчиком какую-то из них.

\par К примеру, нас может заинтересовать, при помощи чего Microsoft Excel считает 
результаты формул, введенных пользователем. Например, операция деления.

\myindex{\GrepUsage}
\myindex{x86!\Instructions!FDIV}
Если загрузить excel.exe (из Office 2010) версии 14.0.4756.1000 в \IDA, затем сделать полный листинг 
и найти все инструкции \FDIV (но кроме тех, которые в качестве второго операнда используют константы --- они, 
очевидно, не подходят нам):

\begin{lstlisting}
cat EXCEL.lst | grep fdiv | grep -v dbl_ > EXCEL.fdiv
\end{lstlisting}

\dots то окажется, что их всего 144.

\par Мы можем вводить в Excel строку вроде \TT{=(1/3)} и проверить все эти инструкции.

\myindex{tracer}
\par Проверяя каждую инструкцию в отладчике или \tracer 
(проверять эти инструкции можно по 4 за раз), 
окажется, что нам везет и срабатывает всего лишь 14-я по счету:

\begin{lstlisting}[style=customasmx86]
.text:3011E919 DC 33          fdiv    qword ptr [ebx]
\end{lstlisting}

\begin{lstlisting}
PID=13944|TID=28744|(0) 0x2f64e919 (Excel.exe!BASE+0x11e919)
EAX=0x02088006 EBX=0x02088018 ECX=0x00000001 EDX=0x00000001
ESI=0x02088000 EDI=0x00544804 EBP=0x0274FA3C ESP=0x0274F9F8
EIP=0x2F64E919
FLAGS=PF IF
FPU ControlWord=IC RC=NEAR PC=64bits PM UM OM ZM DM IM 
FPU StatusWord=
FPU ST(0): 1.000000
\end{lstlisting}

В \ST{0} содержится первый аргумент (1), второй содержится в \TT{[EBX]}.\\
\\
\myindex{x86!\Instructions!FDIV}
Следующая за \FDIV инструкция (\TT{FSTP}) записывает результат в память: \\

\begin{lstlisting}[style=customasmx86]
.text:3011E91B DD 1E          fstp    qword ptr [esi]
\end{lstlisting}

Если поставить breakpoint на ней, то мы можем видеть результат:

\begin{lstlisting}
PID=32852|TID=36488|(0) 0x2f40e91b (Excel.exe!BASE+0x11e91b)
EAX=0x00598006 EBX=0x00598018 ECX=0x00000001 EDX=0x00000001
ESI=0x00598000 EDI=0x00294804 EBP=0x026CF93C ESP=0x026CF8F8
EIP=0x2F40E91B
FLAGS=PF IF
FPU ControlWord=IC RC=NEAR PC=64bits PM UM OM ZM DM IM 
FPU StatusWord=C1 P 
FPU ST(0): 0.333333
\end{lstlisting}

А также, в рамках пранка\footnote{practical joke}, модифицировать его на лету:

\begin{lstlisting}
tracer -l:excel.exe bpx=excel.exe!BASE+0x11E91B,set(st0,666)
\end{lstlisting}

\begin{lstlisting}
PID=36540|TID=24056|(0) 0x2f40e91b (Excel.exe!BASE+0x11e91b)
EAX=0x00680006 EBX=0x00680018 ECX=0x00000001 EDX=0x00000001
ESI=0x00680000 EDI=0x00395404 EBP=0x0290FD9C ESP=0x0290FD58
EIP=0x2F40E91B
FLAGS=PF IF
FPU ControlWord=IC RC=NEAR PC=64bits PM UM OM ZM DM IM 
FPU StatusWord=C1 P 
FPU ST(0): 0.333333
Set ST0 register to 666.000000
\end{lstlisting}

Excel показывает в этой ячейке 666, что окончательно убеждает нас в том, что мы нашли нужное место.

\begin{figure}[H]
\centering
\includegraphics[width=0.6\textwidth]{digging_into_code/Excel_prank.png}
\caption{Пранк сработал}
\end{figure}

Если попробовать ту же версию Excel, только x64, то окажется что там инструкций \FDIV всего 12, 
причем нужная нам --- третья по счету.

\begin{lstlisting}
tracer.exe -l:excel.exe bpx=excel.exe!BASE+0x1B7FCC,set(st0,666)
\end{lstlisting}

\myindex{x86!\Instructions!DIVSD}
Видимо, все дело в том, что много операций деления переменных типов \Tfloat и \Tdouble 
компилятор заменил на SSE-инструкции вроде \TT{DIVSD}, 
коих здесь теперь действительно много (\TT{DIVSD} присутствует в количестве 268 инструкций).


\mysection{Подозрительные паттерны кода}

\subsection{Инструкции XOR}
\myindex{x86!\Instructions!XOR}

Инструкции вроде \TT{XOR op, op} (например, \TT{XOR EAX, EAX}) 
обычно используются для обнуления регистра,
однако, если операнды разные, то применяется операция именно \q{исключающего или}.
Эта операция очень редко применяется в обычном программировании, но применяется очень часто в криптографии,
включая любительскую.

Особенно подозрительно, если второй операнд --- это большое число.
Это может указывать на шифрование, вычисление контрольной суммы, итд.  \\
\\
Одно из исключений из этого наблюдения о котором стоит сказать, то, что генерация и проверка значения \q{канарейки}
(\myref{subsec:BO_protection}) часто происходит, используя инструкцию \XOR.  \\
\\
\myindex{AWK}
Этот AWK-скрипт можно использовать для обработки листингов (.lst) созданных \IDA{}:

\lstinputlisting{digging_into_code/awk.sh}

Нельзя также забывать,
что если использовать подобный скрипт, то, возможно, он захватит и неверно дизассемблированный
код 
(\myref{sec:incorrectly_disasmed_code}).

\subsection{Вручную написанный код на ассемблере}

\myindex{Function prologue}
\myindex{Function epilogue}
\myindex{x86!\Instructions!LOOP}
\myindex{x86!\Instructions!RCL}
Современные компиляторы не генерируют инструкции \TT{LOOP} и \TT{RCL}. 
С другой стороны, эти инструкции хорошо знакомы кодерам, предпочитающим писать прямо на ассемблере. 
Подобные инструкции отмечены как (M) в списке инструкций в приложении: 
\myref{sec:x86_instructions}.
Если такие инструкции встретились, можно сказать с какой-то вероятностью, что этот фрагмент кода написан вручную.

\par
Также, пролог/эпилог функции обычно не встречается в ассемблерном коде, написанном вручную.

\par
Как правило, во вручную написанном коде, нет никакого четкого метода передачи аргументов в функцию.

\par
Пример из ядра Windows 2003 
(файл ntoskrnl.exe):

\lstinputlisting[style=customasmx86]{digging_into_code/ntoskrnl.lst}

Действительно, если заглянуть в исходные коды 
\ac{WRK} v1.2, данный код можно найти в файле \\
\IT{WRK-v1.2\textbackslash{}base\textbackslash{}ntos\textbackslash{}ke\textbackslash{}i386\textbackslash{}cpu.asm}.

\mysection{Использование magic numbers для трассировки}

Нередко бывает нужно узнать, как используется то или иное значение, прочитанное из файла либо взятое из пакета,
принятого по сети. Часто, ручное слежение за нужной переменной это трудный процесс. Один из простых методов (хотя и не
полностью надежный на 100\%) это использование вашей собственной \IT{magic number}.

Это чем-то напоминает компьютерную томографию: пациенту перед сканированием вводят в кровь 
рентгеноконтрастный препарат, хорошо отсвечивающий в рентгеновских лучах.
Известно, как кровь нормального человека
расходится, например, по почкам, и если в этой крови будет препарат, то при томографии будет хорошо видно,
достаточно ли хорошо кровь расходится по почкам и нет ли там камней, например, и прочих образований.

Мы можем взять 32-битное число вроде \TT{0x0badf00d}, либо чью-то дату рождения вроде \TT{0x11101979} 
и записать это, занимающее 4 байта число, в какое-либо место файла используемого исследуемой нами программой.

\myindex{\GrepUsage}
\myindex{tracer}
Затем, при трассировке этой программы, в том числе, при помощи \tracer в режиме 
\IT{code coverage}, а затем при помощи
\IT{grep} или простого поиска по текстовому файлу с результатами трассировки, мы можем легко увидеть, в каких местах кода использовалось 
это значение, и как.

Пример результата работы \tracer в режиме \IT{cc}, к которому легко применить утилиту \IT{grep}:

\begin{lstlisting}[style=customasmx86]
0x150bf66 (_kziaia+0x14), e=       1 [MOV EBX, [EBP+8]] [EBP+8]=0xf59c934 
0x150bf69 (_kziaia+0x17), e=       1 [MOV EDX, [69AEB08h]] [69AEB08h]=0 
0x150bf6f (_kziaia+0x1d), e=       1 [FS: MOV EAX, [2Ch]] 
0x150bf75 (_kziaia+0x23), e=       1 [MOV ECX, [EAX+EDX*4]] [EAX+EDX*4]=0xf1ac360 
0x150bf78 (_kziaia+0x26), e=       1 [MOV [EBP-4], ECX] ECX=0xf1ac360 
\end{lstlisting}
% TODO: good example!
Это справедливо также и для сетевых пакетов.
Важно только, чтобы наш \IT{magic number} был как можно более уникален и не присутствовал в самом коде.

\newcommand{\DOSBOXURL}{\href{http://go.yurichev.com/17222}{blog.yurichev.com}}

\myindex{DosBox}
\myindex{MS-DOS}
Помимо \tracer, такой эмулятор MS-DOS как DosBox, в режиме heavydebug, может писать в отчет информацию обо всех
состояниях регистра на каждом шаге исполнения программы\footnote{См. также мой пост в блоге об этой возможности в 
DosBox: \DOSBOXURL{}}, так что этот метод может пригодиться и для исследования программ под DOS.


\mysection{Циклы}

Когда ваша программа работает с некоторым файлом, или буфером некоторой длины,
внутри кода где-то должен быть цикл с дешифровкой/обработкой.

Вот реальный пример выхода инструмента \tracer.
Был код, загружающий некоторый зашифрованный файл размером 258 байт.
Я могу запустить его с целью подсчета исполнения каждой инструкции (в наше время \ac{DBI} послужила бы куда лучше).
И я могу быстро найти место в коде, которое было исполнено 259/258 раз:

\lstinputlisting{digging_into_code/crypto_loop.txt}

Как потом оказалось, это цикл дешифрования.


\mysection{Оптимизации циклов}

% subsections:
\subsection{Странная оптимизация циклов}

Это самая простая (из всех возможных) реализация memcpy():

\begin{lstlisting}[style=customc]
void memcpy (unsigned char* dst, unsigned char* src, size_t cnt)
{
	size_t i;
	for (i=0; i<cnt; i++)
		dst[i]=src[i];
};
\end{lstlisting}

Как минимум MSVC 6.0 из конца 90-х вплоть до MSVC 2013 может выдавать вот такой странный код (этот листинг создан MSVC 2013
x86):

\lstinputlisting[style=customasmx86]{advanced/500_loop_optimizations/1_1_RU.lst}

Это всё странно, потому что как люди работают с двумя указателями? Они сохраняют два адреса в двух регистрах или двух
ячейках памяти.
Компилятор MSVC в данном случае сохраняет два указателя как один указатель (\IT{скользящий dst} в \EAX)
и разницу между указателями \IT{src} и \IT{dst} (она остается неизменной во время исполнения цикла, в \ESI).
\myindex{\CLanguageElements!ptrdiff\_t}
(Кстати, это тот редкий случай, когда можно использовать тип ptrdiff\_t.)
Когда нужно загрузить байт из \IT{src}, он загружается на \IT{diff + скользящий dst} и сохраняет байт просто на
\IT{скользящем dst}.

Должно быть это какой-то трюк для оптимизации. Но я переписал эту ф-цию так:

\lstinputlisting[style=customasmx86]{advanced/500_loop_optimizations/1_2.lst}

\dots и она работает также быстро как и \IT{соптимизированная} версия на моем Intel Xeon E31220 @ 3.10GHz.
Может быть, эта оптимизация предназначалась для более старых x86-процессоров 90-х, т.к., этот трюк использует
как минимум древний MS VC 6.0?

Есть идеи?

\myindex{Hex-Rays}
Hex-Rays 2.2 не распознает такие шаблонные фрагменты кода (будем надеятся, это временно?):

\begin{lstlisting}[style=customc]
void __cdecl f1(char *dst, char *src, size_t size)
{
  size_t counter; // edx@1
  char *sliding_dst; // eax@2
  char tmp; // cl@3

  counter = size;
  if ( size )
  {
    sliding_dst = dst;
    do
    {
      tmp = (sliding_dst++)[src - dst];         // разница (src-dst) вычисляется один раз, перед телом цикла
      *(sliding_dst - 1) = tmp;
      --counter;
    }
    while ( counter );
  }
}
\end{lstlisting}

Тем не менее, этот трюк часто используется в MSVC (и не только в самодельных ф-циях \IT{memcpy()}, но также и во многих
циклах, работающих с двумя или более массивами), так что для реверс-инжиниров стоит помнить об этом.

% <!-- As of why writting occurred after <b>dst</b> incrementing? -->


\subsection{Возврат строки}

Классическая ошибка из \RobPikePractice{}:

\begin{lstlisting}[style=customc]
#include <stdio.h>

char* amsg(int n, char* s)
{
        char buf[100];

        sprintf (buf, "error %d: %s\n", n, s) ;

        return buf;
};

int main()
{
        printf ("%s\n", amsg (1234, "something wrong!"));
};
\end{lstlisting}

Она упадет.
В начале, попытаемся понять, почему.

Это состояние стека перед возвратом из amsg():

% FIXME! TikZ or whatever
\begin{lstlisting}
§(низкие адреса)§

§[amsg(): 100 байт]§
§[RA]                               <- текущий SP§
§[два аргумента amsg]§
§[что-то еще]§
§[локальные переменные main()]§

§(высокие адреса)§
\end{lstlisting}

Когда управление возвращается из amsg() в \main, пока всё хорошо.
Но когда \printf вызывается из \main, который, в свою очередь, использует стек для своих нужд, затирая 100-байтный буфер.
В лучшем случае, будет выведен случайный мусор.

Трудно поверить, но я знаю, как это исправить:

\begin{lstlisting}[style=customc]
#include <stdio.h>

char* amsg(int n, char* s)
{
        char buf[100];

        sprintf (buf, "error %d: %s\n", n, s) ;

        return buf;
};

char* interim (int n, char* s)
{
        char large_buf[8000];
        // используем локальный массив.
        // а иначе компилятор выбросит его при оптимизации, как неиспользуемый.
        large_buf[0]=0;
        return amsg (n, s);
};

int main()
{
        printf ("%s\n", interim (1234, "something wrong!"));
};
\end{lstlisting}

Это заработает если скомпилировано в MSVC 2013 без оптимизаций и с опцией \TT{/GS-}\footnote{Выключить защиту от переполнения буфера}.
MSVC предупредит: ``warning C4172: returning address of local variable or temporary'', но код запустится и сообщение выведется.
Посмотрим состояние стека в момент, когда amsg() возвращает управление в interim():

\begin{lstlisting}
§(низкие адреса)§

§[amsg(): 100 байт]§
§[RA]                                      <- текущий SP§
§[два аргумента amsg()]§
§[вледения interim(), включая 8000 байт]§
§[еще что-то]§
§[локальные переменные main()]§

§(высокие адреса)§
\end{lstlisting}

Теперь состояние стека на момент, когда interim() возвращает управление в \main{}:

\begin{lstlisting}
§(низкие адреса)§

§[amsg(): 100 байт]§
§[RA]§
§[два аргумента amsg()]§
§[вледения interim(), включая 8000 байт]§
§[еще что-то]                              <- текущий SP§
§[локальные переменные main()]§

§(высокие адреса)§
\end{lstlisting}

Так что когда \main вызывает \printf, он использует стек в месте, где выделен буфер в interim(),
и не затирает 100 байт с сообщение об ошибке внутри, потому что 8000 байт (или может быть меньше) это достаточно для всего,
что делает \printf и другие нисходящие ф-ции!

Это также может сработать, если между ними много ф-ций, например:
\main $\rightarrow$ f1() $\rightarrow$ f2() $\rightarrow$ f3() ... $\rightarrow$ amsg(),
и тогда результат amsg() используется в \main.
Дистанция между \ac{SP} в \main и адресом буфера \TT{buf[]} должна быть достаточно длинной.

Вот почему такие ошибки опасны: иногда ваш код работает (и бага прячется незамеченной). иногда нет.
\label{heisenbug}
\myindex{Хейзенбаги}
Такие баги в шутку называют хейзенбаги или шрёдинбаги\footnote{\url{https://en.wikipedia.org/wiki/Heisenbug}}.




\input{digging_into_code/snapshots_comparing_RU}
\mysection{Определение \ac{ISA}}
\label{ISA_detect}

Часто, вы можете иметь дело с бинарным файлом для неизвестной \ac{ISA}.
Вероятно, простейший способ определить \ac{ISA} это пробовать разные в IDA, objdump или другом дизассемблере.

Чтобы этого достичь, нужно понимать разницу между некорректно дизассемблированным кодом, и корректно дизассемблированным.

% subsection:
\renewcommand{\CURPATH}{digging_into_code/incorrect_disassembly}
\mysection{Оптимизации циклов}

% subsections:
\subsection{Странная оптимизация циклов}

Это самая простая (из всех возможных) реализация memcpy():

\begin{lstlisting}[style=customc]
void memcpy (unsigned char* dst, unsigned char* src, size_t cnt)
{
	size_t i;
	for (i=0; i<cnt; i++)
		dst[i]=src[i];
};
\end{lstlisting}

Как минимум MSVC 6.0 из конца 90-х вплоть до MSVC 2013 может выдавать вот такой странный код (этот листинг создан MSVC 2013
x86):

\lstinputlisting[style=customasmx86]{advanced/500_loop_optimizations/1_1_RU.lst}

Это всё странно, потому что как люди работают с двумя указателями? Они сохраняют два адреса в двух регистрах или двух
ячейках памяти.
Компилятор MSVC в данном случае сохраняет два указателя как один указатель (\IT{скользящий dst} в \EAX)
и разницу между указателями \IT{src} и \IT{dst} (она остается неизменной во время исполнения цикла, в \ESI).
\myindex{\CLanguageElements!ptrdiff\_t}
(Кстати, это тот редкий случай, когда можно использовать тип ptrdiff\_t.)
Когда нужно загрузить байт из \IT{src}, он загружается на \IT{diff + скользящий dst} и сохраняет байт просто на
\IT{скользящем dst}.

Должно быть это какой-то трюк для оптимизации. Но я переписал эту ф-цию так:

\lstinputlisting[style=customasmx86]{advanced/500_loop_optimizations/1_2.lst}

\dots и она работает также быстро как и \IT{соптимизированная} версия на моем Intel Xeon E31220 @ 3.10GHz.
Может быть, эта оптимизация предназначалась для более старых x86-процессоров 90-х, т.к., этот трюк использует
как минимум древний MS VC 6.0?

Есть идеи?

\myindex{Hex-Rays}
Hex-Rays 2.2 не распознает такие шаблонные фрагменты кода (будем надеятся, это временно?):

\begin{lstlisting}[style=customc]
void __cdecl f1(char *dst, char *src, size_t size)
{
  size_t counter; // edx@1
  char *sliding_dst; // eax@2
  char tmp; // cl@3

  counter = size;
  if ( size )
  {
    sliding_dst = dst;
    do
    {
      tmp = (sliding_dst++)[src - dst];         // разница (src-dst) вычисляется один раз, перед телом цикла
      *(sliding_dst - 1) = tmp;
      --counter;
    }
    while ( counter );
  }
}
\end{lstlisting}

Тем не менее, этот трюк часто используется в MSVC (и не только в самодельных ф-циях \IT{memcpy()}, но также и во многих
циклах, работающих с двумя или более массивами), так что для реверс-инжиниров стоит помнить об этом.

% <!-- As of why writting occurred after <b>dst</b> incrementing? -->


\subsection{Возврат строки}

Классическая ошибка из \RobPikePractice{}:

\begin{lstlisting}[style=customc]
#include <stdio.h>

char* amsg(int n, char* s)
{
        char buf[100];

        sprintf (buf, "error %d: %s\n", n, s) ;

        return buf;
};

int main()
{
        printf ("%s\n", amsg (1234, "something wrong!"));
};
\end{lstlisting}

Она упадет.
В начале, попытаемся понять, почему.

Это состояние стека перед возвратом из amsg():

% FIXME! TikZ or whatever
\begin{lstlisting}
§(низкие адреса)§

§[amsg(): 100 байт]§
§[RA]                               <- текущий SP§
§[два аргумента amsg]§
§[что-то еще]§
§[локальные переменные main()]§

§(высокие адреса)§
\end{lstlisting}

Когда управление возвращается из amsg() в \main, пока всё хорошо.
Но когда \printf вызывается из \main, который, в свою очередь, использует стек для своих нужд, затирая 100-байтный буфер.
В лучшем случае, будет выведен случайный мусор.

Трудно поверить, но я знаю, как это исправить:

\begin{lstlisting}[style=customc]
#include <stdio.h>

char* amsg(int n, char* s)
{
        char buf[100];

        sprintf (buf, "error %d: %s\n", n, s) ;

        return buf;
};

char* interim (int n, char* s)
{
        char large_buf[8000];
        // используем локальный массив.
        // а иначе компилятор выбросит его при оптимизации, как неиспользуемый.
        large_buf[0]=0;
        return amsg (n, s);
};

int main()
{
        printf ("%s\n", interim (1234, "something wrong!"));
};
\end{lstlisting}

Это заработает если скомпилировано в MSVC 2013 без оптимизаций и с опцией \TT{/GS-}\footnote{Выключить защиту от переполнения буфера}.
MSVC предупредит: ``warning C4172: returning address of local variable or temporary'', но код запустится и сообщение выведется.
Посмотрим состояние стека в момент, когда amsg() возвращает управление в interim():

\begin{lstlisting}
§(низкие адреса)§

§[amsg(): 100 байт]§
§[RA]                                      <- текущий SP§
§[два аргумента amsg()]§
§[вледения interim(), включая 8000 байт]§
§[еще что-то]§
§[локальные переменные main()]§

§(высокие адреса)§
\end{lstlisting}

Теперь состояние стека на момент, когда interim() возвращает управление в \main{}:

\begin{lstlisting}
§(низкие адреса)§

§[amsg(): 100 байт]§
§[RA]§
§[два аргумента amsg()]§
§[вледения interim(), включая 8000 байт]§
§[еще что-то]                              <- текущий SP§
§[локальные переменные main()]§

§(высокие адреса)§
\end{lstlisting}

Так что когда \main вызывает \printf, он использует стек в месте, где выделен буфер в interim(),
и не затирает 100 байт с сообщение об ошибке внутри, потому что 8000 байт (или может быть меньше) это достаточно для всего,
что делает \printf и другие нисходящие ф-ции!

Это также может сработать, если между ними много ф-ций, например:
\main $\rightarrow$ f1() $\rightarrow$ f2() $\rightarrow$ f3() ... $\rightarrow$ amsg(),
и тогда результат amsg() используется в \main.
Дистанция между \ac{SP} в \main и адресом буфера \TT{buf[]} должна быть достаточно длинной.

Вот почему такие ошибки опасны: иногда ваш код работает (и бага прячется незамеченной). иногда нет.
\label{heisenbug}
\myindex{Хейзенбаги}
Такие баги в шутку называют хейзенбаги или шрёдинбаги\footnote{\url{https://en.wikipedia.org/wiki/Heisenbug}}.





\subsection{Корректино дизассемблированный код}
\label{correctly_disasmed_code}

Каждая \ac{ISA} имеет десяток самых используемых инструкций, остальные используются куда реже.

Интересно знать тот факт, что в x86, инструкции вызовов ф-ций (\PUSH/\CALL/\ADD) и \MOV
это наиболее часто исполняющиеся инструкции в коде почти во всем ПО что мы используем.
Другими словами, \ac{CPU} очень занят передачей информации между уровнями абстракции, или, можно сказать, очень занят
переключением между этими уровнями.
Вне зависимости от \ac{ISA}.
Это цена расслоения программ на разные уровни абстракций (чтобы человеку было легче с ними управляться).



\mysection{Прочее}

\subsection{Общая идея}

Нужно стараться как можно чаще ставить себя на место программиста и задавать себе вопрос, 
как бы вы сделали ту или иную вещь в этом случае и в этой программе.

\subsection{Порядок функций в бинарном коде}

Все функции расположеные в одном .c или .cpp файле компилируются в соответствующий объектный (.o) файл.
Линкер впоследствии складывает все нужные объектные файлы вместе, не меняя порядок ф-ций в них.
Как следствие, если вы видите в коде две или более идущих подряд ф-ций, то это означает, что и в исходном коде они 
были расположены в одном и том же файле (если только вы не на границе двух объектных файлов, конечно).
Это может означать, что эти ф-ции имеют что-то общее между собой, что они из одного слоя \ac{API}, из одной библиотеки, итд.%

\subsection{Крохотные функции}

Крохотные ф-ции, такие как пустые ф-ции (\myref{empty_func})
или ф-ции возвращающие только ``true'' (1) или ``false'' (0) (\myref{ret_val_func}) очень часто встречаются,
и почти все современные компиляторы, как правило, помещают только одну такую ф-цию в исполняемый код,
даже если в исходном их было много одинаковых.
Так что если вы видите ф-цию состояющую только из \TT{mov eax, 1 / ret}, которая может вызываться из разных мест,
которые, судя по всему, друг с другом никак не связаны, это может быть результат подобной оптимизации.

\subsection{\Cpp}

\ac{RTTI}~(\myref{RTTI})-информация также может быть полезна для идентификации 
классов в \Cpp.

