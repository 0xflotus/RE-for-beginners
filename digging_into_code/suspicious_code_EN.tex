\mysection{Suspicious code patterns}

\subsection{XOR instructions}
\myindex{x86!\Instructions!XOR}

Instructions like \TT{XOR op, op} (for example, \TT{XOR EAX, EAX}) 
are usually used for setting the register value
to zero, but if the operands are different, the \q{exclusive or} operation
is executed.

This operation is rare in common programming, but widespread in cryptography,
including amateur one.
It's especially suspicious if the
second operand is a big number.

This may point to encrypting/decrypting, checksum computing, etc.\\
\\

One exception to this observation worth noting is the \q{canary} (\myref{subsec:BO_protection}). 
Its generation and checking are often done using the \XOR instruction. \\
\\
\myindex{AWK}

This AWK script can be used for processing \IDA{} listing (.lst) files:

\lstinputlisting{digging_into_code/awk.sh}

It is also worth noting that this kind of script can also match incorrectly disassembled code 
(\myref{sec:incorrectly_disasmed_code}).

\subsection{Hand-written assembly code}

\myindex{Function prologue}
\myindex{Function epilogue}
\myindex{x86!\Instructions!LOOP}
\myindex{x86!\Instructions!RCL}

Modern compilers do not emit the \TT{LOOP} and \TT{RCL} instructions.
On the other hand, these instructions are well-known to coders who like to code directly in assembly language.
If you spot these, it can be said that there is a high probability that this fragment of code was hand-written.
Such instructions are marked as (M) in the instructions list in this appendix: \myref{sec:x86_instructions}.

\par

Also the function prologue/epilogue are not commonly present in hand-written assembly.
\par

Commonly there is no fixed system for passing arguments to functions in the hand-written code.

\par
Example from the Windows 2003 kernel 
(ntoskrnl.exe file):

\lstinputlisting[style=customasmx86]{digging_into_code/ntoskrnl.lst}

Indeed, if we look in the 
\ac{WRK} v1.2 source code, this code
can be found easily in file \\
\IT{WRK-v1.2\textbackslash{}base\textbackslash{}ntos\textbackslash{}ke\textbackslash{}i386\textbackslash{}cpu.asm}.
