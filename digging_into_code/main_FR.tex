\chapter{Trouver des choses importantes/intéressantes dans le code}

Le minimalisme n'est pas une caractéristique prépondérante des logiciels modernes.

\myindex{\Cpp!STL}

Pas parce que les programmeurs écrivent beaucoup, mais parce que de nombreuses bibliothèques
sont couramment liées statiquement aux fichiers exécutable.
Si toutes les bibliothèques externes étaient déplacées dans des fichiers DLL externes,
le monde serait différent. (Une autre raison pour C++ sont la \ac{STL} et autres
bibliothèques templates.)

\newcommand{\FOOTNOTEBOOST}{\footnote{\url{http://go.yurichev.com/17036}}}
\newcommand{\FOOTNOTELIBPNG}{\footnote{\url{http://go.yurichev.com/17037}}}

Ainsi, il est très important de déterminer l'origine de la fonction, si elle provient
d'une bibliothèque standard ou d'une bibliothèque bien connue (comme Boost\FOOTNOTEBOOST,
libpng\FOOTNOTELIBPNG), ou si elle est liée à ce que l'on essaye de trouver dans
le code.

Il est simplement absurde de tout récrire le code en \CCpp pour trouver ce que l'on
cherche.

Une des premières tâches d'un rétro-ingénieur est de trouver rapidement le code dont
il a besoin.

\myindex{\GrepUsage}

Le dés-assembleur \IDA nous permet de chercher parmi les chaînes de texte, les séquences
d'octets et les constantes.
Il est même possible d'exporter le code dans un fichier texte .lst ou .asm et d'utiliser
\TT{grep}, \TT{awk}, etc.

Lorsque vous essayez de comprendre ce que fait un certain code, ceci peut être facile
avec une bibliothèque open-source comme libpng.
Donc, lorsque vous voyez certaines constantes ou chaînes de texte qui vous semblent
familières, il vaut toujours la peine de les \IT{googler}.
Et si vous trouvez le projet open-source où elles sont utilisées, alors il suffit
de comparer les fonctions.
Ceci peut permettre de résoudre certaines parties du problème.

Par exemple, si un programme utilise des fichiers XML, la premières étape peut-être
de déterminer quelle bibliothèque XML est utilisée pour le traitement, puisque les
bibliothèques standards (ou bien connues) sont en général utilisées au lieu de code
fait maison.

\myindex{SAP}
\myindex{Windows!PDB}

Par exemple, l'auteur de ces lignes a essayé une fois de comprendre comment la compression/décompression
des paquets réseau fonctionne dans SAP 6.0.
C'est un logiciel gigantesque, mais un .\gls{PDB} détaillé avec des informations
de débogage est présent, et c'est pratique.
Il en est finalement arrivé à l'idée que l'une des fonctions, qui était appelée par
\IT{CsDecomprLZC}, effectuait la décompression des paquets réseau.
Immédiatement, il a essayé de googler le nom et a rapidement trouvé que la fonction
était utilisée dans MaxDB (c'est un projet open-source de SAP)
\footnote{Plus sur ce sujet dans la section concernée~(\myref{sec:SAPGUI})}.

\url{http://www.google.com/search?q=CsDecomprLZC}

Étonnament, les logiciels MaxDB et SAP 6.0 partagent du code comme ceci pour la compression/
décompression des paquets réseau.

% binary files might be also here

\mysection{Communication avec le mondes extérieur (niveau fonction)}
Il est souvent recommandé de suivre les arguments de la fonction et sa valeur de
retour dans un débogueur ou \ac{DBI}.
Par exemple, l'auteur a essayé une fois de comprendre la signification d'une fonction
obscure, qui s'est avérée être un tri à bulles mal implémenté\footnote{\url{https://yurichev.com/blog/weird_sort/}}.
(Il fonctionnait correctement, mais plus lentement.)
En même temps, regarder les entrées et sorties de cette fonction aide instantanément
à comprendre ce quelle fait.

Souvent, lorsque vous voyez une division par la multiplication (\myref{sec:divisionbymult}),
mais avez oublié tous les détails du mécanisme, vous pouvez seulement observer l'entrée
et la sortie, et trouver le diviseur rapidement.

% sections:
% TODO move section...

\subsection{Quelques schémas de fichier binaire}

Tous les exemples ici ont été préparé sur Windows, avec la page de code 437 activée
\footnote{\url{https://en.wikipedia.org/wiki/Code_page_437}} dans la console.
L'intérieur des fichiers binaires peut avoir l'air différent avec une autre page
de code.

\clearpage
\subsubsection{Tableaux}

Parfois, nous pouvons clairement localiser visuellement un tableau de valeurs 16/32/64-bit,
dans un éditeur hexadécimal.

Voici un exemple de tableau de valeurs 16-bit.
Nous voyons que le premier octet d'une paire est 7 ou 8, et que le second semble
aléatoire:

\begin{figure}[H]
\centering
\myincludegraphics{digging_into_code/binary/16bit_array.png}
\caption{FAR: tableau de valeurs 16-bit}
\end{figure}

J'ai utilisé un fichier contenant un signal 12-canaux numérisé en utilisant 16-bit \ac{ADC}.

\clearpage
\myindex{MIPS}
\par Et voici un exemple ce code MIPS très typique.

Comme nous pouvons nous en souvenir, chaque instruction MIPS (et aussi ARM en mode
ARM ou ARM64) a une taille de 32 bits (ou 4 octets), donc un tel code est un tableau
de valeurs 32-bit.

En regardant cette copie d'écran, nous voyons des sortes de schémas.

Les lignes rouge verticales ont été ajoutées pour la clarté:

\begin{figure}[H]
\centering
\myincludegraphics{digging_into_code/binary/typical_MIPS_code.png}
\caption{Hiew: code MIPS très typique}
\end{figure}

Il y a un autre exemple de tel schéma ici dans le livre:
\myref{Oracle_SYM_files_example}.

\clearpage
\subsubsection{Fichiers clairsemés}

Ceci est un fichier clairsemé avec des données éparpillées dans un fichier presque vide.
Chaque caractère espace est en fait l'octet zéro (qui rend comme un espace).
Ceci est un fichier pour programmer des FPGA (Altera Stratix GX device).
Bien sûr, de tels fichiers peuvent être compressés facilement, mais des formats comme
celui-ci sont très populaire dans les logiciels scientifiques et d'ingénierie, où
l'efficience des accès est importante, tandis que la compacité ne l'est pas.

\begin{figure}[H]
\centering
\myincludegraphics{digging_into_code/binary/sparse_FPGA.png}
\caption{FAR: Fichier sparse}
\end{figure}

\clearpage
\subsubsection{Fichiers compressés}

% FIXME \ref{} ->
Ce fichier est juste une archive compressée.
Il a une entropie relativement haute et visuellement, il à l'air chaotique.
Ceci est ce à quoi ressemble les fichiers compressés et/ou chiffrés.

\begin{figure}[H]
\centering
\myincludegraphics{digging_into_code/binary/compressed.png}
\caption{FAR: Fichier compressé}
\end{figure}

\clearpage
\subsubsection{\ac{CDFS}}

Les fichiers d'installation d'un \ac{OS} sont en général distribués sous forme de
fichiers ISO, qui sont des copies de disques CD/DVD.
Le système de fichiers utilisé est appelé \ac{CDFS}, ce que vous voyez ici sont des
noms de fichiers mixés avec des données additionnelles.
Ceci peut-être la taille des fichiers, des pointeurs sur d'autres répertoires, des
attributs de fichier, etc.
C'est l'aspect typique de ce à quoi ressemble un système de fichiers en interne.

\begin{figure}[H]
\centering
\myincludegraphics{digging_into_code/binary/cdfs.png}
\caption{FAR: Fichier ISO: \ac{CD} d'installation d'Ubuntu 15}
\end{figure}

\clearpage
\subsubsection{Code exécutable x86 32-bit}

Voici l'allure de code exécutable x86 32-bit.
Il n'a pas une grande entropie, car certains octets reviennent plus souvent que d'autres.

\begin{figure}[H]
\centering
\myincludegraphics{digging_into_code/binary/x86_32.png}
\caption{FAR: Code exécutable x86 32-bit}
\end{figure}

% TODO: Read more about x86 statistics: \ref{}. % FIXME blog post about decryption...

\clearpage
\subsubsection{Fichiers graphique BMP}

% TODO: bitmap, bit, group of bits...

Les fichiers BMP ne sont pas compressés, donc chaque octet (ou groupe d'octet) représente
chaque pixel.
J'ai trouvé cette image quelque part dans mon installation de Windows 8.1:

\begin{figure}[H]
\centering
\myincludegraphicsSmall{digging_into_code/binary/bmp.png}
\caption{Image exemple}
\end{figure}

Vous voyez que cette image a des pixels qui ne doivent pas pouvoir être compressés
beaucoup (autour du centre), mais il y a de longues lignes monochromes au haut et
en bas.
En effet, de telles lignes ressemblent à des lignes lorsque l'on regarde le fichier:

\begin{figure}[H]
\centering
\myincludegraphics{digging_into_code/binary/bmp_FAR.png}
\caption{Fragment de fichier BMP}
\end{figure}



\mysection{Autres choses}

\subsection{Idée générale}

Un rétro-ingénieur doit essayer se se mettre dans la peau d'un programmeur aussi
souvent que possible.
Pour prendre son point de vue et se demander comment il aurait résolu des taches
d'un cas spécifique.

\subsection{Ordre des fonctions dans le code binaire}

Toutes les fonctions situées dans un unique fichier .c ou .cpp sont compilées dans
le fichier objet (.o) correspondant.
Plus tard, l'éditeur de liens mets tous les fichiers dont il a besoin ensemble, sans
changer l'ordre ni les fonctions.
Par conséquent, si vous voyez deux ou plus fonctions consécutives, cela signifie
qu'elles étaient situées dans le même fichier source (à moins que vous ne soyez en
limite de deux fichiers objet, bien sûr).
Ceci signifie que ces fonctions ont quelque chose en commun, qu'elles sont des fonctions
du même niveau d'\ac{API}, de la même bibliothèque, etc.

\subsection{Fonctions minuscules}

Les fonctions minuscules comme les fonctions vides (\myref{empty_func})
ou les fonctions qui renvoient juste ``true'' (1) ou ``false'' (0) (\myref{ret_val_func})
sont très communes, et presque tous les compilateurs corrects tendent à ne mettre
qu'une seule fonction de ce genre dans le code de l'exécutable résultant, même si
il y avait plusieurs fonctions similaires dans le code source.
Donc, à chaque fois que vous voyez une fonction minuscule consistant seulement en
\TT{mov eax, 1 / ret} qui est référencée (et peut être appelée) dans plusieurs endroits
qui ne semblent pas reliés les uns au autres, ceci peut résulter d'une telle optimisation.%

\subsection{\Cpp}

Les données \ac{RTTI}~(\myref{RTTI})- peuvent être utiles pour l'identification des
classes \Cpp.

