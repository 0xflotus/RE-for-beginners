\subsubsection{Base64}
\myindex{Base64}

The base64 encoding is highly popular for the cases when you have to transfer binary data as a text string.

In essence, this algorithm encodes 3 binary bytes into 4 printable characters:
all 26 Latin letters (both lower and upper case), digits, plus sign (\q{+}) and slash sign (\q{/}),
64 characters in total.

One distinctive feature of base64 strings is that they often (but not always) end with 1 or 2 \gls{padding}
equality symbol(s) (\q{=}), for example:

\begin{lstlisting}
AVjbbVSVfcUMu1xvjaMgjNtueRwBbxnyJw8dpGnLW8ZW8aKG3v4Y0icuQT+qEJAp9lAOuWs=
\end{lstlisting}

\begin{lstlisting}
WVjbbVSVfcUMu1xvjaMgjNtueRwBbxnyJw8dpGnLW8ZW8aKG3v4Y0icuQT+qEJAp9lAOuQ==
\end{lstlisting}

The equality sign (\q{=}) is never encounter in the middle of base64-encoded strings.

Now example of manual encoding.
Let's encode 0x00, 0x11, 0x22, 0x33 hexadecimal bytes into base64 string:

\lstinputlisting{digging_into_code/strings/base64_ex.sh}

Let's put all 4 bytes in binary form, then regroup them into 6-bit groups:

\begin{lstlisting}
|  00  ||  11  ||  22  ||  33  ||      ||      |
00000000000100010010001000110011????????????????
| A  || B  || E  || i  || M  || w  || =  || =  |
\end{lstlisting}

Three first bytes (0x00, 0x11, 0x22) can be encoded into 4 base64 characters (``ABEi''),
but the last one (0x33) --- cannot be,
so it's encoded using two characters (``Mw'') and \gls{padding} symbol (``='')
is added twice to pad the last group to 4 characters.
Hence, length of all correct base64 strings are always divisible by 4.

\myindex{XML}
\myindex{PGP}
Base64 is often used when binary data needs to be stored in XML.
``Armored'' (i.e., in text form) PGP keys and signatures are encoded using base64.

Some people tries to use base64 to obfuscate strings:
\url{http://blog.sec-consult.com/2016/01/deliberately-hidden-backdoor-account-in.html}
\footnote{\url{http://archive.is/nDCas}}.

\myindex{base64scanner}
There are utilities for scanning an arbitrary binary files for base64 strings.
One such utility is base64scanner\footnote{\url{https://github.com/DennisYurichev/base64scanner}}.

\myindex{UseNet}
\myindex{FidoNet}
\myindex{Uuencoding}
\myindex{Phrack}
Another encoding system which was much more popular in UseNet and FidoNet is Uuencoding.
Binary files are still encoded in Uuencode format in Phrack magazine.
It offers mostly the same features, but is different from base64 in the sense that file name
is also stored in header.

\myindex{Tor}
\myindex{base32}
By the way: there is also close sibling to base64: base32, alphabet of which has ~10 digits and ~26 Latin characters.
One well-known usage of it is onion addresses
\footnote{\url{https://trac.torproject.org/projects/tor/wiki/doc/HiddenServiceNames}},
like: \url{http://3g2upl4pq6kufc4m.onion/}.
\ac{URL} can't have mixed-case Latin characters, so apparently, this is why Tor developers used base32.

