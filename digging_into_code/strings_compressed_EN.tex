\mysection{Text strings right in the middle of compressed data}

\myindex{Linux kernel}
You can download Linux kernels and find English words right in the middle of compressed data:

\begin{lstlisting}
% wget https://www.kernel.org/pub/linux/kernel/v4.x/linux-4.10.2.tar.gz

% xxd -g 1 -seek 0x515c550 -l 0x30 linux-4.10.2.tar.gz

0515c550: c5 59 43 cf 41 27 85 54 35 4a 57 90 73 89 b7 6a  .YC.A'.T5JW.s..j
0515c560: 15 af 03 db 20 df 6a 51 f9 56 49 52 55 53 3d da  .... .jQ.VIRUS=.
0515c570: 0e b9 29 24 cc 6a 38 e2 78 66 09 33 72 aa 88 df  ..)$.j8.xf.3r...
\end{lstlisting}

\begin{lstlisting}
% wget https://cdn.kernel.org/pub/linux/kernel/v2.3/linux-2.3.3.tar.bz2

% xxd -g 1 -seek 0xa93086 -l 0x30 linux-2.3.3.tar.bz2

00a93086: 4d 45 54 41 4c cd 44 45 2d 2c 41 41 54 94 8b a1  METAL.DE-,AAT...
00a93096: 5d 2b d8 d0 bd d8 06 91 74 ab 41 a0 0a 8a 94 68  ]+......t.A....h
00a930a6: 66 56 86 81 68 0d 0e 25 6b b6 80 a4 28 1a 00 a4  fV..h..%k...(...
\end{lstlisting}

One of Linux kernel patches in compressed form has the ``Linux'' word itself:

\begin{lstlisting}
% wget https://cdn.kernel.org/pub/linux/kernel/v4.x/testing/patch-4.6-rc4.gz

% xxd -g 1 -seek 0x4d03f -l 0x30 patch-4.6-rc4.gz

0004d03f: c7 40 24 bd ae ef ee 03 2c 95 dc 65 eb 31 d3 f1  .@$.....,..e.1..
0004d04f: 4c 69 6e 75 78 f2 f3 70 3c 3a bd 3e bd f8 59 7e  Linux..p<:.>..Y~
0004d05f: cd 76 55 74 2b cb d5 af 7a 35 56 d7 5e 07 5a 67  .vUt+...z5V.^.Zg
\end{lstlisting}

Other English words I've found in other compressed Linux kernel trees:

\begin{lstlisting}
linux-4.6.2.tar.gz: [maybe] at 0x68e78ec
linux-4.10.14.tar.xz: [OCEAN] at 0x6bf0a8
linux-4.7.8.tar.gz: [FUNNY] at 0x29e6e20
linux-4.6.4.tar.gz: [DRINK] at 0x68dc314
linux-2.6.11.8.tar.bz2: [LUCKY] at 0x1ab5be7
linux-3.0.68.tar.gz: [BOOST] at 0x11238c7
linux-3.0.16.tar.bz2: [APPLE] at 0x34c091
linux-3.0.26.tar.xz: [magic] at 0x296f7d9
linux-3.11.8.tar.bz2: [TRUTH] at 0xf635ba
linux-3.10.11.tar.bz2: [logic] at 0x4a7f794
\end{lstlisting}

\myindex{Apophenia}
\myindex{Pareidolia}
\myindex{Lurkmore}
There is a nice illustration of apophenia and pareidolia
There is a nice illustration of apophenia and pareidolia
(human's mind ability to see faces in clouds, etc) in Lurkmore, Russian counterpart of Encyclopedia Dramatica.
As they wrote in the article about electronic voice phenomenon\footnote{\url{http://archive.is/gYnFL}},
you can open any long enough compressed file in hex editor and find well-known 3-letter Russian obscene word, and you'll find it a lot: but that means nothing, just a mere coincidence.

And I was interested in calculation, how big compressed file must be to contain all possible 3-letter, 4-letter, etc, words?
In my naive calculations, I've got this: probability of the first specific byte in the middle of compressed data stream with maximal entropy is $\frac{1}{256}$, probability of the 2nd is also $\frac{1}{256}$,
and probability of specific byte pair is $\frac{1}{256 \cdot 256} = \frac{1}{256^2}$.
Probabilty of specific triple is $\frac{1}{256^3}$.
If the file has maximal entropy (which is almost unachievable, but \dots) and we live in an ideal world, you've got to have a file of size just $256^3=16777216$, which is 16-17MB.
\myindex{rafind2}
You can check: get any compressed file, and use \IT{rafind2} to search for any 3-letter word (not just that Russian obscene one).

It took $\approx$ 8-9 GB of my downloaded movies/TV series files to find the word ``beer'' in them (case sensitive).
Perhaps, these movies wasn't compressed good enough?
This is also true for a well-known 4-letter English obscene word.

My approach is naive, so I googled for mathematically grounded one, and have find this question:
``Time until a consecutive sequence of ones in a random bit sequence''
\footnote{\url{http://math.stackexchange.com/questions/27989/time-until-a-consecutive-sequence-of-ones-in-a-random-bit-sequence/27991#27991}}.
The answer is: $(p^{−n}−1)/(1−p)$, where $p$ is probability of each event and $n$ is number of consecutive events.
Plug $\frac{1}{256}$ and $3$ and you'll get almost the same as my naive calculations.

So any 3-letter word can be found in the compressed file (with ideal entropy) of length $256^3 = \approx 17MB$, any 4-letter word --- $256^4 = 4.7GB$ (size of DVD).
Any 5-letter word --- $256^5 = \approx 1TB$.

For the piece of text you are reading now, I mirrored the whole \href{https://www.kernel.org/}{kernel.org} website (hopefully, sysadmins can forgive me),
and it has $\approx$ 430GB of compressed Linux Kernel source trees.
It has enough compressed data to contain these words, however, I cheated a bit: I searched for both lowercase and uppercase strings, thus compressed data set I need is almost halved.

This is quite interesting thing to think about: 1TB of compressed data with maximal entropy has all possible 5-byte chains,
but the data is encoded not in chains itself, but in the order of chains (no matter of compression algorithm, etc).

Now the information for gamblers: one should throw a dice $\approx 42$ times to get a pair of six, but no one will tell you, when exactly this will happen.
\myindex{Rosencrantz \& Guildenstern Are Dead}
I don't remember, how many times coin was tossed in the ``Rosencrantz \& Guildenstern Are Dead'' movie, but one should toss it $\approx 2048$ times and at some point, you'll get 10 heads in a row,
and at some other point, 10 tails in a row. Again, no one will tell you, when exactly this will happen.

Compressed data can also be treated as a stream of random data, so we can use the same mathematics to determine probabilities, etc.

If you can live with strings of mixed case, like ``bEeR'', probabilities and compressed data sets are much lower:
$128^3=2MB$ for all 3-letter words of mixed case,
$128^4=268MB$ for all 4-letter words,
$128^5=34GB$ for all 5-letter words, etc.

\myindex{Jorge Luis Borges}
Moral of the story: whenever you search for some patterns, you can find it in the middle of compressed blob, but that means nothing else then coincidence.
In philosophical sense, this is a case of selection/confirmation bias: you find what you search for in ``The Library of Babel''\footnote{Short story by Jorge Luis Borges}.

