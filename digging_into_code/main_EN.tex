\chapter{Finding important/interesting stuff in the code}

Minimalism it is not a prominent feature of modern software.

\myindex{\Cpp!STL}

But not because the programmers are writing a lot, but because a lot of libraries are commonly linked statically
to executable files.
If all external libraries were shifted into an external DLL files, the world would be different.
(Another reason for C++ are the \ac{STL} and other template libraries.)

\newcommand{\FOOTNOTEBOOST}{\footnote{\url{http://go.yurichev.com/17036}}}
\newcommand{\FOOTNOTELIBPNG}{\footnote{\url{http://go.yurichev.com/17037}}}

Thus, it is very important to determine the origin of a function, if it is from standard library or 
well-known library (like Boost\FOOTNOTEBOOST, libpng\FOOTNOTELIBPNG),
or if it is related to what we are trying to find in the code.

It is just absurd to rewrite all code in \CCpp to find what we're looking for.

One of the primary tasks of a reverse engineer is to find quickly the code he/she needs.

\myindex{\GrepUsage}

The \IDA disassembler allow us to search among text strings, byte sequences and constants.
It is even possible to export the code to .lst or .asm text files and then use \TT{grep}, \TT{awk}, etc.

When you try to understand what some code is doing, this easily could be some open-source library like libpng.
So when you see some constants or text strings which look familiar, it is always worth to \IT{google} them.
And if you find the opensource project where they are used, 
then it's enough just to compare the functions.
It may solve some part of the problem.

For example, if a program uses XML files, the first step may be determining which
XML library is used for processing, since the standard (or well-known) libraries are usually used
instead of self-made one.

\myindex{SAP}
\myindex{Windows!PDB}

For example, the author of these lines once tried to understand how the compression/decompression of network packets works in SAP 6.0. 
It is a huge software, but a detailed .\gls{PDB} with debugging information is present, 
and that is convenient.
He finally came to the idea that one of the functions, that was called \IT{CsDecomprLZC}, was doing the decompression of network packets.
Immediately he tried to google its name and he quickly found the function was used in MaxDB
(it is an open-source SAP project) \footnote{More about it in relevant section~(\myref{sec:SAPGUI})}.

\url{http://www.google.com/search?q=CsDecomprLZC}

Astoundingly, MaxDB and SAP 6.0 software shared likewise code for the compression/decompression of network packets.

\mysection{Identification of executable files}

\subsection{Microsoft Visual C++}
\label{MSVC_versions}

MSVC versions and DLLs that can be imported:

\small
\begin{center}
\begin{tabular}{ | l | l | l | l | l | }
\hline
\HeaderColor Marketing ver. & 
\HeaderColor Internal ver. & 
\HeaderColor CL.EXE ver. &
\HeaderColor DLLs imported &
\HeaderColor Release date \\
\hline
% 4.0, April 1995
% 97 & 5.0 & February 1997
6		&  6.0	& 12.00	& msvcrt.dll	& June 1998		\\
		&	&	& msvcp60.dll	&			\\
\hline
.NET (2002)	&  7.0	& 13.00	& msvcr70.dll	& February 13, 2002	\\
		&	&	& msvcp70.dll	&			\\
\hline
.NET 2003	&  7.1	& 13.10 & msvcr71.dll	& April 24, 2003	\\
		&	&	& msvcp71.dll	&			\\
\hline
2005		&  8.0	& 14.00 & msvcr80.dll	& November 7, 2005	\\
		&	&	& msvcp80.dll	&			\\
\hline
2008		&  9.0	& 15.00 & msvcr90.dll	& November 19, 2007	\\
		&	&	& msvcp90.dll	&			\\
\hline
2010		& 10.0	& 16.00 & msvcr100.dll	& April 12, 2010 	\\
		&	&	& msvcp100.dll	&			\\
\hline
2012		& 11.0	& 17.00 & msvcr110.dll	& September 12, 2012 	\\
		&	&	& msvcp110.dll	&			\\
\hline
2013		& 12.0	& 18.00 & msvcr120.dll	& October 17, 2013 	\\
		&	&	& msvcp120.dll	&			\\
\hline
\end{tabular}
\end{center}
\normalsize

msvcp*.dll has \Cpp{}-related functions, so if it is imported, 
this is probably a \Cpp program.

\subsubsection{Name mangling}

The names usually start with the \TT{?} symbol.

You can read more about MSVC's \gls{name mangling} here: \myref{namemangling}.

\subsection{GCC}
\myindex{GCC}

Aside from *NIX targets, GCC is also present in the win32 environment, in the form of Cygwin and MinGW.

\subsubsection{Name mangling}

Names usually start with the \TT{\_Z} symbols.

You can read more about GCC's \gls{name mangling} here: \myref{namemangling}.

\subsubsection{Cygwin}
\myindex{Cygwin}

cygwin1.dll is often imported.

\subsubsection{MinGW}
\myindex{MinGW}

msvcrt.dll may be imported.

\subsection{Intel Fortran}
\myindex{Fortran}

libifcoremd.dll, libifportmd.dll and libiomp5md.dll (OpenMP support) may be imported.

libifcoremd.dll has a lot of functions prefixed with \TT{for\_}, which means \IT{Fortran}.

\subsection{Watcom, OpenWatcom}
\myindex{Watcom}
\myindex{OpenWatcom}

\subsubsection{Name mangling}

Names usually start with the \TT{W} symbol.

For example, that is how the method named \q{method} of the class \q{class} that does not have any arguments and returns
\Tvoid is encoded:

\begin{lstlisting}
W?method$_class$n__v
\end{lstlisting}

\subsection{Borland}
\myindex{Borland Delphi}
\myindex{Borland C++Builder}

Here is an example of Borland Delphi's and C++Builder's \gls{name mangling}:

\lstinputlisting{digging_into_code/identification/borland_mangling.txt}

The names always start with the \TT{@} 
symbol, then we have the class name came, method name, and encoded the types of the arguments of the method.

These names can be in the .exe imports, .dll exports, debug data, etc.

Borland Visual Component Libraries (VCL) 
are stored in .bpl files instead of .dll ones, for example, vcl50.dll, rtl60.dll.

Another DLL that might be imported: BORLNDMM.DLL.

\subsubsection{Delphi}

Almost all Delphi executables has the \q{Boolean} text string at the beginning of the code segment, along with other type names.

This is a very typical beginning of the \TT{CODE} 
segment of a Delphi program, this block came right after the win32 PE file header:

\lstinputlisting{digging_into_code/identification/delphi.txt}

The first 4 bytes of the data segment (\TT{DATA}) can be \TT{00 00 00 00}, \TT{32 13 8B C0} or \TT{FF FF FF FF}.%

This information can be useful when dealing with packed/encrypted Delphi executables.

\subsection{Other known DLLs}

\myindex{OpenMP}
\begin{itemize}
\item vcomp*.dll---Microsoft's implementation of OpenMP.
\end{itemize}



% binary files might be also here

\mysection{Communication with outer world (function level)}
It's often advisable to track function arguments and return values in debugger or \ac{DBI}.
For example, the author once tried to understand meaning of some obscure function, which happens to be incorrectly
implemented bubble sort\footnote{\url{https://yurichev.com/blog/weird_sort/}}.
(It worked correctly, but slower.)
Meanwhile, watching inputs and outputs of this function helps instantly to understand what it does.

Often, when you see division by multiplication (\myref{sec:divisionbymult}),
but forgot all details about its mechanics, you can just observe input
and output and quickly find divisor.

% sections:
\mysection{Communication with the outer world (win32)}

Sometimes it's enough to observe some function's inputs and outputs in order to understand what it does.
That way you can save time.

Files and registry access: 
for the very basic analysis, Process Monitor\footnote{\url{http://go.yurichev.com/17301}}
utility from SysInternals can help.

For the basic analysis of network accesses, Wireshark\footnote{\url{http://go.yurichev.com/17303}} can be useful.

But then you will have to look inside anyway. \\
\\
The first thing to look for is which functions from the \ac{OS}'s \ac{API}s and standard libraries are used.

If the program is divided into a main executable file and a group of DLL files, sometimes the names of the functions in these DLLs can help.

If we are interested in exactly what can lead to a call to \TT{MessageBox()} with specific text, 
we can try to find this text in the data segment, find the references to it and find the points
from which the control may be passed to the \TT{MessageBox()} call we're interested in.

\myindex{\CStandardLibrary!rand()}
If we are talking about a video game and we're interested in which events are more or less random in it,
we may try to find the \rand function or its replacements (like the Mersenne twister algorithm) and find the places
from which those functions are called, and more importantly, how are the results used.
% BUG in varioref: http://tex.stackexchange.com/questions/104261/varioref-vref-or-vpageref-at-page-boundary-may-loop
One example: \ref{chap:color_lines}. 

But if it is not a game, and \rand is still used, it is also interesting to know why.
There are cases of unexpected \rand usage in data compression algorithms (for encryption imitation):
\href{http://go.yurichev.com/17221}{blog.yurichev.com}.

\subsection{Often used functions in the Windows API}

These functions may be among the imported.
It is worth to note that not every function might be used in the code that was written by the programmer.
A lot of functions might be called from library functions and \ac{CRT} code.

Some functions may have the \GTT{-A} suffix for the ASCII version and \GTT{-W} for the Unicode version.

\begin{itemize}

\item
Registry access (advapi32.dll): 
RegEnumKeyEx, RegEnumValue, RegGetValue, RegOpenKeyEx, RegQueryValueEx.

\item
Access to text .ini-files (kernel32.dll): 
GetPrivateProfileString.

\item
Dialog boxes (user32.dll): 
MessageBox, MessageBoxEx, CreateDialog, SetDlgItemText, GetDlgItemText.

\item
Resources access (\myref{PEresources}): (user32.dll): LoadMenu.

\item
TCP/IP networking (ws2\_32.dll):
WSARecv, WSASend.

\item
File access (kernel32.dll):
CreateFile, ReadFile, ReadFileEx, WriteFile, WriteFileEx.

\item
High-level access to the Internet (wininet.dll): WinHttpOpen.

\item
Checking the digital signature of an executable file (wintrust.dll):
WinVerifyTrust.

\item
The standard MSVC library (if it's linked dynamically) (msvcr*.dll):
assert, itoa, ltoa, open, printf, read, strcmp, atol, atoi, fopen, fread, fwrite, memcmp, rand,
strlen, strstr, strchr.

\end{itemize}

\subsection{Extending trial period}

Registry access functions are frequent targets for those who try to crack trial period of some software, which may save
installation date/time into registry.

Another popular target are GetLocalTime() and GetSystemTime() functions:
a trial software, at each startup, must check current date/time somehow anyway.

\subsection{Removing nag dialog box}

A popular way to find out what causing popping nag dialog box is intercepting MessageBox(), 
CreateDialog() and CreateWindow() functions.

\subsection{tracer: Intercepting all functions in specific module}
\myindex{tracer}

\myindex{x86!\Instructions!INT3}
There are INT3 breakpoints in the \tracer, that are triggered only once, however, they can be set for all functions
in a specific DLL.

\begin{lstlisting}
--one-time-INT3-bp:somedll.dll!.*
\end{lstlisting}

Or, let's set INT3 breakpoints on all functions with the \TT{xml} prefix in their name:

\begin{lstlisting}
--one-time-INT3-bp:somedll.dll!xml.*
\end{lstlisting}

On the other side of the coin, such breakpoints are triggered only once.
Tracer will show the call of a function, if it happens, but only once.
Another drawback---it is impossible to see the function's arguments.

Nevertheless, this feature is very useful when you know that the program uses a DLL,
but you do not know which functions are actually used.
And there are a lot of functions. 

\par
\myindex{Cygwin}
For example, let's see, what does the uptime utility from cygwin use:

\begin{lstlisting}
tracer -l:uptime.exe --one-time-INT3-bp:cygwin1.dll!.*
\end{lstlisting}

Thus we may see all that cygwin1.dll library functions that were called at least once, and where from:

\lstinputlisting{digging_into_code/uptime_cygwin.txt}


\mysection{Strings}
\label{sec:digging_strings}

\subsection{Text strings}

\subsubsection{\CCpp}

\label{C_strings}
The normal C strings are zero-terminated (\ac{ASCIIZ}-strings).

The reason why the C string format is as it is (zero-terminated) is apparently historical.
In [Dennis M. Ritchie, \IT{The Evolution of the Unix Time-sharing System}, (1979)]
we read:

\begin{framed}
\begin{quotation}
A minor difference was that the unit of I/O was the word, not the byte, because the PDP-7 was a word-addressed
machine. In practice this meant merely that all programs dealing with character streams ignored null
characters, because null was used to pad a file to an even number of characters.
\end{quotation}
\end{framed}

\myindex{Hiew}

In Hiew or FAR Manager these strings look like this:

\begin{lstlisting}[style=customc]
int main()
{
	printf ("Hello, world!\n");
};
\end{lstlisting}

\begin{figure}[H]
\centering
\includegraphics[width=0.6\textwidth]{digging_into_code/strings/C-string.png}
\caption{Hiew}
\end{figure}

% FIXME видно \n в конце, потом пробел

\subsubsection{Borland Delphi}
\myindex{Pascal}
\myindex{Borland Delphi}

The string in Pascal and Borland Delphi is preceded by an 8-bit or 32-bit string length.

For example:

\begin{lstlisting}[caption=Delphi,style=customasmx86]
CODE:00518AC8                 dd 19h
CODE:00518ACC aLoading___Plea db 'Loading... , please wait.',0

...

CODE:00518AFC                 dd 10h
CODE:00518B00 aPreparingRun__ db 'Preparing run...',0
\end{lstlisting}

\subsubsection{Unicode}

\myindex{Unicode}

Often, what is called Unicode is a methods for encoding strings where each character occupies 2 bytes or 16 bits.
This is a common terminological mistake.
Unicode is a standard for assigning a number to each character in the many writing systems of the 
world, but does not describe the encoding method.

\myindex{UTF-8}
\myindex{UTF-16LE}
The most popular encoding methods are: UTF-8 (is widespread in Internet and *NIX systems) and UTF-16LE (is used in Windows).

\myparagraph{UTF-8}

\myindex{UTF-8}
UTF-8 is one of the most successful methods for
encoding characters.
All Latin symbols are encoded just like in ASCII,
and the symbols beyond the ASCII table are encoded using several bytes.
0 is encoded as
before, so all standard C string functions work with UTF-8 strings just like any other string.

Let's see how the symbols in various languages are encoded in UTF-8 and how it looks like in FAR, using the 437 codepage
\footnote{The example and translations was taken from here: 
\url{http://go.yurichev.com/17304}}:

\begin{figure}[H]
\centering
\includegraphics[width=0.6\textwidth]{digging_into_code/strings/multilang_sampler.png}
\end{figure}

% FIXME: cut it
\begin{figure}[H]
\centering
\myincludegraphics{digging_into_code/strings/multilang_sampler_UTF8.png}
\caption{FAR: UTF-8}
\end{figure}

As you can see, the English language string looks the same as it is in ASCII.

The Hungarian language uses some Latin symbols plus symbols with diacritic marks.

These symbols are encoded using several bytes, these are underscored with red.
It's the same story with the Icelandic and Polish languages.

There is also the \q{Euro} currency symbol at the start, which is encoded with 3 bytes.

The rest of the writing systems here have no connection with Latin.

At least in Russian, Arabic, Hebrew and Hindi we can see some recurring bytes, and that is not surprise:
all symbols from a writing system are usually located in the same Unicode table, so their code begins with
the same numbers.

At the beginning, before the \q{How much?} string we see 3 bytes, which are in fact the \ac{BOM}.
The \ac{BOM} defines the encoding system to be
used.

\myparagraph{UTF-16LE}

\myindex{UTF-16LE}
\myindex{Windows!Win32}
Many win32 functions in Windows have the suffixes \TT{-A} and \TT{-W}.
The first type of functions works
with normal strings, the other with UTF-16LE strings (\IT{wide}).

In the second case, each symbol is usually stored in a 16-bit value of type \IT{short}.

The Latin symbols in UTF-16 strings look in Hiew or FAR like they are interleaved with zero byte:

\begin{lstlisting}[style=customc]
int wmain()
{
	wprintf (L"Hello, world!\n");
};
\end{lstlisting}

\begin{figure}[H]
\centering
\includegraphics[width=0.6\textwidth]{digging_into_code/strings/UTF16-string.png}
\caption{Hiew}
\end{figure}

We can see this often in \gls{Windows NT} system files:

\begin{figure}[H]
\centering
\includegraphics[width=0.6\textwidth]{digging_into_code/strings/ntoskrnl_UTF16.png}
\caption{Hiew}
\end{figure}

\myindex{IDA}
Strings with characters that occupy exactly 2 bytes are called \q{Unicode} in \IDA:

\begin{lstlisting}[style=customasmx86]
.data:0040E000 aHelloWorld:
.data:0040E000                 unicode 0, <Hello, world!>
.data:0040E000                 dw 0Ah, 0
\end{lstlisting}

Here is how the Russian language string is encoded in UTF-16LE:

\begin{figure}[H]
\centering
\includegraphics[width=0.6\textwidth]{digging_into_code/strings/russian_UTF16.png}
\caption{Hiew: UTF-16LE}
\end{figure}

What we can easily spot is that the symbols are interleaved by the diamond character (which has the ASCII code of 4).
Indeed, the Cyrillic symbols are located in the fourth Unicode plane
\footnote{\href{http://go.yurichev.com/17003}{wikipedia}}.
Hence, all Cyrillic symbols in UTF-16LE are located in the \TT{0x400-0x4FF} range.

Let's go back to the example with the string written in multiple languages.
Here is how it looks like in UTF-16LE.

% FIXME: cut it
\begin{figure}[H]
\centering
\myincludegraphics{digging_into_code/strings/multilang_sampler_UTF16.png}
\caption{FAR: UTF-16LE}
\end{figure}

Here we can also see the \ac{BOM} at the beginning.
All Latin characters are interleaved with a zero byte.

Some characters with diacritic marks (Hungarian and Icelandic languages) are also underscored in red.

% subsection:
\subsubsection{Base64}
\myindex{Base64}

The base64 encoding is highly popular for the cases when you have to transfer binary data as a text string.

In essence, this algorithm encodes 3 binary bytes into 4 printable characters:
all 26 Latin letters (both lower and upper case), digits, plus sign (\q{+}) and slash sign (\q{/}),
64 characters in total.

One distinctive feature of base64 strings is that they often (but not always) end with 1 or 2 \gls{padding}
equality symbol(s) (\q{=}), for example:

\begin{lstlisting}
AVjbbVSVfcUMu1xvjaMgjNtueRwBbxnyJw8dpGnLW8ZW8aKG3v4Y0icuQT+qEJAp9lAOuWs=
\end{lstlisting}

\begin{lstlisting}
WVjbbVSVfcUMu1xvjaMgjNtueRwBbxnyJw8dpGnLW8ZW8aKG3v4Y0icuQT+qEJAp9lAOuQ==
\end{lstlisting}

The equality sign (\q{=}) is never encounter in the middle of base64-encoded strings.

Now example of manual encoding.
Let's encode 0x00, 0x11, 0x22, 0x33 hexadecimal bytes into base64 string:

\lstinputlisting{digging_into_code/strings/base64_ex.sh}

Let's put all 4 bytes in binary form, then regroup them into 6-bit groups:

\begin{lstlisting}
|  00  ||  11  ||  22  ||  33  ||      ||      |
00000000000100010010001000110011????????????????
| A  || B  || E  || i  || M  || w  || =  || =  |
\end{lstlisting}

Three first bytes (0x00, 0x11, 0x22) can be encoded into 4 base64 characters (``ABEi''),
but the last one (0x33) --- cannot be,
so it's encoded using two characters (``Mw'') and \gls{padding} symbol (``='')
is added twice to pad the last group to 4 characters.
Hence, length of all correct base64 strings are always divisible by 4.

\myindex{XML}
\myindex{PGP}
Base64 is often used when binary data needs to be stored in XML.
``Armored'' (i.e., in text form) PGP keys and signatures are encoded using base64.

Some people tries to use base64 to obfuscate strings:
\url{http://blog.sec-consult.com/2016/01/deliberately-hidden-backdoor-account-in.html}
\footnote{\url{http://archive.is/nDCas}}.

\myindex{base64scanner}
There are utilities for scanning an arbitrary binary files for base64 strings.
One such utility is base64scanner\footnote{\url{https://github.com/DennisYurichev/base64scanner}}.

\myindex{UseNet}
\myindex{FidoNet}
\myindex{Uuencoding}
\myindex{Phrack}
Another encoding system which was much more popular in UseNet and FidoNet is Uuencoding.
Binary files are still encoded in Uuencode format in Phrack magazine.
It offers mostly the same features, but is different from base64 in the sense that file name
is also stored in header.

\myindex{Tor}
\myindex{base32}
By the way: there is also close sibling to base64: base32, alphabet of which has ~10 digits and ~26 Latin characters.
One well-known usage of it is onion addresses
\footnote{\url{https://trac.torproject.org/projects/tor/wiki/doc/HiddenServiceNames}},
like: \url{http://3g2upl4pq6kufc4m.onion/}.
\ac{URL} can't have mixed-case Latin characters, so apparently, this is why Tor developers used base32.





\subsection{Finding strings in binary}

\epigraph{Actually, the best form of Unix documentation is frequently running the
\textbf{strings} command over a program’s object code. Using \textbf{strings}, you can get
a complete list of the program’s hard-coded file name, environment variables,
undocumented options, obscure error messages, and so forth.}{The Unix-Haters Handbook}

\myindex{UNIX!strings}
The standard UNIX \IT{strings} utility is quick-n-dirty way to see strings in file.
For example, these are some strings from OpenSSH 7.2 sshd executable file:

\lstinputlisting{digging_into_code/sshd_strings.txt}

There are options, error messages, file paths, imported dynamic modules and functions, some other strange strings (keys?)
There is also unreadable noise---x86 code sometimes has chunks consisting of printable ASCII characters, up to ~8 characters.

Of course, OpenSSH is open-source program.
But looking at readable strings inside of some unknown binary is often a first step of analysis.
\myindex{UNIX!grep}

\IT{grep} can be applied as well.

\myindex{Hiew}
\myindex{Sysinternals}
Hiew has the same capability (Alt-F6), as well as Sysinternals ProcessMonitor.

\subsection{Error/debug messages}

Debugging messages are very helpful if present.
In some sense, the debugging messages are reporting
what's going on in the program right now. Often these are \printf-like functions,
which write to log-files, or sometimes do not writing anything but the calls are still present 
since the build is not a debug one but \IT{release} one.
\myindex{\oracle}

If local or global variables are dumped in debug messages, it might be helpful as well 
since it is possible to get at least the variable names.
For example, one of such function in \oracle is \TT{ksdwrt()}.

Meaningful text strings are often helpful.
The \IDA disassembler may show from which function and from which point this specific string is used.
Funny cases sometimes happen\footnote{\href{http://go.yurichev.com/17223}{blog.yurichev.com}}.

The error messages may help us as well.
In \oracle, errors are reported using a group of functions.\\
You can read more about them here: \href{http://go.yurichev.com/17224}{blog.yurichev.com}.

\myindex{Error messages}

It is possible to find quickly which functions report errors and in which conditions.

By the way, this is often the reason for copy-protection systems to inarticulate cryptic error messages 
or just error numbers. No one is happy when the software cracker quickly understand why the copy-protection
is triggered just by the error message.

One example of encrypted error messages is here: \myref{examples_SCO}.

\subsection{Suspicious magic strings}

Some magic strings which are usually used in backdoors looks pretty suspicious.

For example, there was a backdoor in the TP-Link WR740 home router\footnote{\url{http://sekurak.pl/tp-link-httptftp-backdoor/}}.
The backdoor can activated using the following URL:\\
\url{http://192.168.0.1/userRpmNatDebugRpm26525557/start_art.html}.\\

Indeed, the \q{userRpmNatDebugRpm26525557} string is present in the firmware.

This string was not googleable until the wide disclosure of information about the backdoor.

You would not find this in any \ac{RFC}.

You would not find any computer science algorithm which uses such strange byte sequences.

And it doesn't look like an error or debugging message.

So it's a good idea to inspect the usage of such weird strings.\\
\\
\myindex{base64}

Sometimes, such strings are encoded using base64.

So it's a good idea to decode them all and to scan them visually, even a glance should be enough.\\
\\
\myindex{Security through obscurity}
More precise, this method of hiding backdoors is called \q{security through obscurity}.


\mysection{Calls to assert()}
\myindex{\CStandardLibrary!assert()}

Sometimes the presence of the \TT{assert()} macro is useful too: 
commonly this macro leaves source file name, line number and condition in the code.

The most useful information is contained in the assert's condition, we can deduce variable names or structure field
names from it. Another useful piece of information are the file names---we can try to deduce what type of
code is there.
Also it is possible to recognize well-known open-source libraries by the file names.

\lstinputlisting[caption=Example of informative assert() calls,style=customasmx86]{digging_into_code/assert_examples.lst}

It is advisable to \q{google} both the conditions and file names, which can lead us to an open-source library.
For example, if we \q{google} \q{sp->lzw\_nbits <= BITS\_MAX}, this predictably 
gives us some open-source code that's related to the LZW compression.

\mysection{Constants}

Humans, including programmers, often use round numbers like 10, 100, 1000, 
in real life as well as in the code.

The practicing reverse engineer usually know them well in hexadecimal representation:
10=0xA, 100=0x64, 1000=0x3E8, 10000=0x2710.

The constants \TT{0xAAAAAAAA} (0b10101010101010101010101010101010) and \\
\TT{0x55555555} (0b01010101010101010101010101010101)  are also popular---those
are composed of alternating bits.

That may help to distinguish some signal from a signal where all bits are turned on (0b1111 \dots) or off (0b0000 \dots).
For example, the \TT{0x55AA} constant
is used at least in the boot sector, \ac{MBR}, 
and in the \ac{ROM} of IBM-compatible extension cards.

Some algorithms, especially cryptographical ones use distinct constants, which are easy to find
in code using \IDA.

\myindex{MD5}
\newcommand{\URLMD}{http://go.yurichev.com/17111}

For example, the MD5\footnote{\href{\URLMD}{wikipedia}} algorithm initializes its own internal variables like this:

\begin{verbatim}
var int h0 := 0x67452301
var int h1 := 0xEFCDAB89
var int h2 := 0x98BADCFE
var int h3 := 0x10325476
\end{verbatim}

If you find these four constants used in the code in a row, it is highly probable that this function is related to MD5.

\par Another example are the CRC16/CRC32 algorithms, 
whose calculation algorithms often use precomputed tables like this one:

\begin{lstlisting}[caption=linux/lib/crc16.c,style=customc]
/** CRC table for the CRC-16. The poly is 0x8005 (x^16 + x^15 + x^2 + 1) */
u16 const crc16_table[256] = {
	0x0000, 0xC0C1, 0xC181, 0x0140, 0xC301, 0x03C0, 0x0280, 0xC241,
	0xC601, 0x06C0, 0x0780, 0xC741, 0x0500, 0xC5C1, 0xC481, 0x0440,
	0xCC01, 0x0CC0, 0x0D80, 0xCD41, 0x0F00, 0xCFC1, 0xCE81, 0x0E40,
	...
\end{lstlisting}

See also the precomputed table for CRC32: \myref{sec:CRC32}.

In tableless CRC algorithms well-known polynomials are used, for example, 0xEDB88320 for CRC32.

\subsection{Magic numbers}
\label{magic_numbers}

\newcommand{\FNURLMAGIC}{\footnote{\href{http://go.yurichev.com/17112}{wikipedia}}}

A lot of file formats define a standard file header where a \IT{magic number(s)}\FNURLMAGIC{} is used, single one or even several.

\myindex{MS-DOS}

For example, all Win32 and MS-DOS executables start with the two characters \q{MZ}\footnote{\href{http://go.yurichev.com/17113}{wikipedia}}.

\myindex{MIDI}

At the beginning of a MIDI file the \q{MThd} signature must be present. 
If we have a program which uses MIDI files for something,
it's very likely that it must check the file for validity by checking at least the first 4 bytes.

This could be done like this:
(\IT{buf} points to the beginning of the loaded file in memory)

\begin{lstlisting}[style=customasmx86]
cmp [buf], 0x6468544D ; "MThd"
jnz _error_not_a_MIDI_file
\end{lstlisting}

\myindex{\CStandardLibrary!memcmp()}
\myindex{x86!\Instructions!CMPSB}

\dots or by calling a function for comparing memory blocks like \TT{memcmp()} or any other equivalent code
up to a \TT{CMPSB} (\myref{REPE_CMPSx}) instruction.

When you find such point you already can say where the loading of the MIDI file starts,
also, we could see the location
of the buffer with the contents of the MIDI file, what is used from the buffer, and how.

\subsubsection{Dates}

\myindex{UFS2}
\myindex{FreeBSD}
\myindex{HASP}

Often, one may encounter number like \TT{0x19870116}, which is clearly looks like a date (year 1987, 1th month (January), 16th day).
This may be someone's birthday (a programmer, his/her relative, child), or some other important date.
The date may also be written in a reverse order, like \TT{0x16011987}.
American-style dates are also popular, like \TT{0x01161987}.

Well-known example is \TT{0x19540119} (magic number used in UFS2 superblock structure), which is a birthday of Marshall Kirk McKusick, prominent FreeBSD contributor.

\myindex{Stuxnet}
Stuxnet uses the number ``19790509'' (not as 32-bit number, but as string, though), and this led to speculation
that the malware is connected to Israel
\footnote{This is a date of execution of Habib Elghanian, persian jew.}

Also, numbers like those are very popular in amateur-grade cryptography, for example, excerpt from the \IT{secret function} internals from HASP3 dongle
\footnote{\url{https://web.archive.org/web/20160311231616/http://www.woodmann.com/fravia/bayu3.htm}}:

\begin{lstlisting}[style=customc]
void xor_pwd(void) 
{ 
	int i; 
	
	pwd^=0x09071966;
	for(i=0;i<8;i++) 
	{ 
		al_buf[i]= pwd & 7; pwd = pwd >> 3; 
	} 
};

void emulate_func2(unsigned short seed)
{ 
	int i, j; 
	for(i=0;i<8;i++) 
	{ 
		ch[i] = 0; 
		
		for(j=0;j<8;j++)
		{ 
			seed *= 0x1989; 
			seed += 5; 
			ch[i] |= (tab[(seed>>9)&0x3f]) << (7-j); 
		}
	} 
}
\end{lstlisting}

\subsubsection{DHCP}

This applies to network protocols as well.
For example, the DHCP protocol's network packets contains the so-called \IT{magic cookie}: \TT{0x63538263}.
Any code that generates DHCP packets somewhere must embed this constant into the packet.
If we find it in the code we may find where this happens and, not only that.
Any program which can receive DHCP packet must verify the \IT{magic cookie}, comparing it with the constant.

For example, let's take the dhcpcore.dll file from Windows 7 x64 and search for the constant.
And we can find it, twice:
it seems that the constant is used in two functions with descriptive names\\
\TT{DhcpExtractOptionsForValidation()} and \TT{DhcpExtractFullOptions()}:

\begin{lstlisting}[caption=dhcpcore.dll (Windows 7 x64),style=customasmx86]
.rdata:000007FF6483CBE8 dword_7FF6483CBE8 dd 63538263h          ; DATA XREF: DhcpExtractOptionsForValidation+79
.rdata:000007FF6483CBEC dword_7FF6483CBEC dd 63538263h          ; DATA XREF: DhcpExtractFullOptions+97
\end{lstlisting}

And here are the places where these constants are accessed:

\begin{lstlisting}[caption=dhcpcore.dll (Windows 7 x64),style=customasmx86]
.text:000007FF6480875F  mov     eax, [rsi]
.text:000007FF64808761  cmp     eax, cs:dword_7FF6483CBE8
.text:000007FF64808767  jnz     loc_7FF64817179
\end{lstlisting}

And:

\begin{lstlisting}[caption=dhcpcore.dll (Windows 7 x64),style=customasmx86]
.text:000007FF648082C7  mov     eax, [r12]
.text:000007FF648082CB  cmp     eax, cs:dword_7FF6483CBEC
.text:000007FF648082D1  jnz     loc_7FF648173AF
\end{lstlisting}

\subsection{Specific constants}

Sometimes, there is a specific constant for some type of code.
For example, the author once dug into a code, where number 12 was encountered suspiciously often.
Size of many arrays is 12, or multiple of 12 (24, etc).
As it turned out, that code takes 12-channel audio file at input and process it.

And vice versa: for example, if a program works with text field which has length of 120 bytes,
there has to be a constant 120 or 119 somewhere in the code.
If UTF-16 is used, then $2 \cdot 120$.
If a code works with network packets of fixed size, it's good idea to search for this constant in the code as well.

This is also true for amateur cryptography (license keys, etc).
If encrypted block has size of $n$ bytes, you may want to try to find occurences of this number throughout the code.
Also, if you see a piece of code which is been repeated $n$ times in loop during execution,
this may be encryption/decryption routine.

\subsection{Searching for constants}

It is easy in \IDA: Alt-B or Alt-I.
\myindex{binary grep}
And for searching for a constant in a big pile of files, or for searching in non-executable files,
there is a small utility called \IT{binary grep}\footnote{\BGREPURL}.


\mysection{Finding the right instructions}

If the program is utilizing FPU instructions and there are very few of them in the code,
one can try to check each one manually with a debugger.

\par For example, we may be interested how Microsoft Excel calculates the formulae entered by user.
For example, the division operation.

\myindex{\GrepUsage}
\myindex{x86!\Instructions!FDIV}

If we load excel.exe (from Office 2010) version 14.0.4756.1000 into \IDA, make a full listing
and to find every \FDIV instruction (except the ones which use constants as a second 
operand---obviously, they do not suit us):

\begin{lstlisting}
cat EXCEL.lst | grep fdiv | grep -v dbl_ > EXCEL.fdiv
\end{lstlisting}

\dots then we see that there are 144 of them.

\par We can enter a string like \TT{=(1/3)} in Excel and check each instruction.

\myindex{tracer}

\par By checking each instruction in a debugger or \tracer
(one may check 4 instruction at a time),
we get lucky and the sought-for instruction is just the 14th:

\begin{lstlisting}[style=customasmx86]
.text:3011E919 DC 33          fdiv    qword ptr [ebx]
\end{lstlisting}

\begin{lstlisting}
PID=13944|TID=28744|(0) 0x2f64e919 (Excel.exe!BASE+0x11e919)
EAX=0x02088006 EBX=0x02088018 ECX=0x00000001 EDX=0x00000001
ESI=0x02088000 EDI=0x00544804 EBP=0x0274FA3C ESP=0x0274F9F8
EIP=0x2F64E919
FLAGS=PF IF
FPU ControlWord=IC RC=NEAR PC=64bits PM UM OM ZM DM IM 
FPU StatusWord=
FPU ST(0): 1.000000
\end{lstlisting}

\ST{0} holds the first argument (1) and second one is in \TT{[EBX]}.\\
\\
\myindex{x86!\Instructions!FDIV}

The instruction after \FDIV (\TT{FSTP}) writes the result in memory:\\

\begin{lstlisting}[style=customasmx86]
.text:3011E91B DD 1E          fstp    qword ptr [esi]
\end{lstlisting}

If we set a breakpoint on it, we can see the result:

\begin{lstlisting}
PID=32852|TID=36488|(0) 0x2f40e91b (Excel.exe!BASE+0x11e91b)
EAX=0x00598006 EBX=0x00598018 ECX=0x00000001 EDX=0x00000001
ESI=0x00598000 EDI=0x00294804 EBP=0x026CF93C ESP=0x026CF8F8
EIP=0x2F40E91B
FLAGS=PF IF
FPU ControlWord=IC RC=NEAR PC=64bits PM UM OM ZM DM IM 
FPU StatusWord=C1 P 
FPU ST(0): 0.333333
\end{lstlisting}

Also as a practical joke, we can modify it on the fly:

\begin{lstlisting}
tracer -l:excel.exe bpx=excel.exe!BASE+0x11E91B,set(st0,666)
\end{lstlisting}

\begin{lstlisting}
PID=36540|TID=24056|(0) 0x2f40e91b (Excel.exe!BASE+0x11e91b)
EAX=0x00680006 EBX=0x00680018 ECX=0x00000001 EDX=0x00000001
ESI=0x00680000 EDI=0x00395404 EBP=0x0290FD9C ESP=0x0290FD58
EIP=0x2F40E91B
FLAGS=PF IF
FPU ControlWord=IC RC=NEAR PC=64bits PM UM OM ZM DM IM 
FPU StatusWord=C1 P 
FPU ST(0): 0.333333
Set ST0 register to 666.000000
\end{lstlisting}

Excel shows 666 in the cell, finally convincing us that we have found the right point.

\begin{figure}[H]
\centering
\includegraphics[width=0.6\textwidth]{digging_into_code/Excel_prank.png}
\caption{The practical joke worked}
\end{figure}

If we try the same Excel version, but in x64,
we will find only 12 \FDIV instructions there,
and the one we looking for is the third one.

\begin{lstlisting}
tracer.exe -l:excel.exe bpx=excel.exe!BASE+0x1B7FCC,set(st0,666)
\end{lstlisting}

\myindex{x86!\Instructions!DIVSD}

It seems that a lot of division operations of \Tfloat and \Tdouble types, were replaced by the compiler with SSE instructions
like \TT{DIVSD} (\TT{DIVSD} is present 268 times in total).

\mysection{Suspicious code patterns}

\subsection{XOR instructions}
\myindex{x86!\Instructions!XOR}

Instructions like \TT{XOR op, op} (for example, \TT{XOR EAX, EAX}) 
are usually used for setting the register value
to zero, but if the operands are different, the \q{exclusive or} operation
is executed.

This operation is rare in common programming, but widespread in cryptography,
including amateur one.
It's especially suspicious if the
second operand is a big number.

This may point to encrypting/decrypting, checksum computing, etc.\\
\\

One exception to this observation worth noting is the \q{canary} (\myref{subsec:BO_protection}). 
Its generation and checking are often done using the \XOR instruction. \\
\\
\myindex{AWK}

This AWK script can be used for processing \IDA{} listing (.lst) files:

\lstinputlisting{digging_into_code/awk.sh}

It is also worth noting that this kind of script can also match incorrectly disassembled code 
(\myref{sec:incorrectly_disasmed_code}).

\subsection{Hand-written assembly code}

\myindex{Function prologue}
\myindex{Function epilogue}
\myindex{x86!\Instructions!LOOP}
\myindex{x86!\Instructions!RCL}

Modern compilers do not emit the \TT{LOOP} and \TT{RCL} instructions.
On the other hand, these instructions are well-known to coders who like to code directly in assembly language.
If you spot these, it can be said that there is a high probability that this fragment of code was hand-written.
Such instructions are marked as (M) in the instructions list in this appendix: \myref{sec:x86_instructions}.

\par

Also the function prologue/epilogue are not commonly present in hand-written assembly.
\par

Commonly there is no fixed system for passing arguments to functions in the hand-written code.

\par
Example from the Windows 2003 kernel 
(ntoskrnl.exe file):

\lstinputlisting[style=customasmx86]{digging_into_code/ntoskrnl.lst}

Indeed, if we look in the 
\ac{WRK} v1.2 source code, this code
can be found easily in file \\
\IT{WRK-v1.2\textbackslash{}base\textbackslash{}ntos\textbackslash{}ke\textbackslash{}i386\textbackslash{}cpu.asm}.

\mysection{Using magic numbers while tracing}

Often, our main goal is to understand how the program uses a value that has been either read from file or received via network. 
The manual tracing of a value is often a very labor-intensive task. One of the simplest techniques for this (although not 100\% reliable) 
is to use your own \IT{magic number}.

This resembles X-ray computed tomography is some sense: a radiocontrast agent is injected into the patient's blood,
which is then used to improve the visibility of the patient's internal structure in to the X-rays.
It is well known how the blood of healthy humans
percolates in the kidneys and if the agent is in the blood, it can be easily seen on tomography, how blood is percolating,
and are there any stones or tumors.

We can take a 32-bit number like \TT{0x0badf00d}, or someone's birth date like \TT{0x11101979}
and write this 4-byte number to some point in a file used by the program we investigate.

\myindex{\GrepUsage}
\myindex{tracer}

Then, while tracing this program with \tracer in \IT{code coverage} mode, with the help of \IT{grep}
or just by searching in the text file (of tracing results), we can easily see where the value has been used and how.

Example 
of \IT{grepable} \tracer results in \IT{cc} mode:

\begin{lstlisting}[style=customasmx86]
0x150bf66 (_kziaia+0x14), e=       1 [MOV EBX, [EBP+8]] [EBP+8]=0xf59c934 
0x150bf69 (_kziaia+0x17), e=       1 [MOV EDX, [69AEB08h]] [69AEB08h]=0 
0x150bf6f (_kziaia+0x1d), e=       1 [FS: MOV EAX, [2Ch]] 
0x150bf75 (_kziaia+0x23), e=       1 [MOV ECX, [EAX+EDX*4]] [EAX+EDX*4]=0xf1ac360 
0x150bf78 (_kziaia+0x26), e=       1 [MOV [EBP-4], ECX] ECX=0xf1ac360 
\end{lstlisting}
% TODO: good example!

This can be used for network packets as well.
It is important for the \IT{magic number} to be unique and not to be present in the program's code.

\newcommand{\DOSBOXURL}{\href{http://go.yurichev.com/17222}{blog.yurichev.com}}

\myindex{DosBox}
\myindex{MS-DOS}
Aside of 
the \tracer, DosBox (MS-DOS emulator) in heavydebug mode
is able to write information about all registers' states for each executed instruction of the program to a plain text file\footnote{See also my 
blog post about this DosBox feature: \DOSBOXURL{}}, so this technique may be useful for DOS programs as well.


\mysection{Loops}

Whenever your program works with some kind of file, or buffer of some size,
it has to be some kind of decrypting/processing loop inside of the code.

This is a real example of \tracer tool output.
There was a code which loads some kind of encryted file of 258 bytes.
I run it with the intention to get each instruction counts (a \ac{DBI} tool will serve much better these days).
And I quickly found a piece of code, which executed 259/258 times:

\lstinputlisting{digging_into_code/crypto_loop.txt}

As it turns out, this is the decrypting loop.


\mysection{Returning Values}
\label{ret_val_func}

Another simple function is the one that simply returns a constant value:

\lstinputlisting[caption=\EN{\CCpp Code},style=customc]{patterns/011_ret/1.c}

Let's compile it.

\subsection{x86}

Here's what both the GCC and MSVC compilers produce (with optimization) on the x86 platform:

\lstinputlisting[caption=\Optimizing GCC/MSVC (\assemblyOutput),style=customasmx86]{patterns/011_ret/1.s}

\myindex{x86!\Instructions!RET}
There are just two instructions: the first places the value 123 into the \EAX register,
which is used by convention for storing the return
value, and the second one is \RET, which returns execution to the \gls{caller}.

The caller will take the result from the \EAX register.

\subsection{ARM}

There are a few differences on the ARM platform:

\lstinputlisting[caption=\OptimizingKeilVI (\ARMMode) ASM Output,style=customasmARM]{patterns/011_ret/1_Keil_ARM_O3.s}

ARM uses the register \Reg{0} for returning the results of functions, so 123 is copied into \Reg{0}.

\myindex{ARM!\Instructions!MOV}
\myindex{x86!\Instructions!MOV}
It is worth noting that \MOV is a misleading name for the instruction in both the x86 and ARM \ac{ISA}s.

The data is not in fact \IT{moved}, but \IT{copied}.

\subsection{MIPS}

\label{MIPS_leaf_function_ex1}

The GCC assembly output below lists registers by number:

\lstinputlisting[caption=\Optimizing GCC 4.4.5 (\assemblyOutput),style=customasmMIPS]{patterns/011_ret/MIPS.s}

\dots while \IDA does it by their pseudo names:

\lstinputlisting[caption=\Optimizing GCC 4.4.5 (IDA),style=customasmMIPS]{patterns/011_ret/MIPS_IDA.lst}

The \$2 (or \$V0) register is used to store the function's return value.
\myindex{MIPS!\Pseudoinstructions!LI}
\INS{LI} stands for ``Load Immediate'' and is the MIPS equivalent to \MOV.

\myindex{MIPS!\Instructions!J}
The other instruction is the jump instruction (J or JR) which returns the execution flow to the \gls{caller}.

\myindex{MIPS!Branch delay slot}
You might be wondering why the positions of the load instruction (LI) and the jump instruction (J or JR) are swapped. This is due to a \ac{RISC} feature called ``branch delay slot''.

The reason this happens is a quirk in the architecture of some RISC \ac{ISA}s and isn't important for our
purposes---we must simply keep in mind that in MIPS, the instruction following a jump or branch instruction
is executed \IT{before} the jump/branch instruction itself.

As a consequence, branch instructions always swap places with the instruction executed immediately beforehand.

In practice, functions which merely return 1 (\IT{true}) or 0 (\IT{false}) are very frequent.

The smallest ever of the standard UNIX utilities, \IT{/bin/true} and \IT{/bin/false} return 0 and 1 respectively, as an exit code.
(Zero as an exit code usually means success, non-zero means error.)

\input{digging_into_code/snapshots_comparing_EN}
\mysection{\ac{ISA} detection}
\label{ISA_detect}

Often, you can deal with a binary file for an unknown \ac{ISA}.
Perhaps, easiest way to detect \ac{ISA} is to try various ones in IDA, objdump or another disassembler.

To achieve this, one should understand a difference between incorrectly disassembled code and correctly one.

% subsection:
\renewcommand{\CURPATH}{digging_into_code/incorrect_disassembly}
\mysection{Returning Values}
\label{ret_val_func}

Another simple function is the one that simply returns a constant value:

\lstinputlisting[caption=\EN{\CCpp Code},style=customc]{patterns/011_ret/1.c}

Let's compile it.

\subsection{x86}

Here's what both the GCC and MSVC compilers produce (with optimization) on the x86 platform:

\lstinputlisting[caption=\Optimizing GCC/MSVC (\assemblyOutput),style=customasmx86]{patterns/011_ret/1.s}

\myindex{x86!\Instructions!RET}
There are just two instructions: the first places the value 123 into the \EAX register,
which is used by convention for storing the return
value, and the second one is \RET, which returns execution to the \gls{caller}.

The caller will take the result from the \EAX register.

\subsection{ARM}

There are a few differences on the ARM platform:

\lstinputlisting[caption=\OptimizingKeilVI (\ARMMode) ASM Output,style=customasmARM]{patterns/011_ret/1_Keil_ARM_O3.s}

ARM uses the register \Reg{0} for returning the results of functions, so 123 is copied into \Reg{0}.

\myindex{ARM!\Instructions!MOV}
\myindex{x86!\Instructions!MOV}
It is worth noting that \MOV is a misleading name for the instruction in both the x86 and ARM \ac{ISA}s.

The data is not in fact \IT{moved}, but \IT{copied}.

\subsection{MIPS}

\label{MIPS_leaf_function_ex1}

The GCC assembly output below lists registers by number:

\lstinputlisting[caption=\Optimizing GCC 4.4.5 (\assemblyOutput),style=customasmMIPS]{patterns/011_ret/MIPS.s}

\dots while \IDA does it by their pseudo names:

\lstinputlisting[caption=\Optimizing GCC 4.4.5 (IDA),style=customasmMIPS]{patterns/011_ret/MIPS_IDA.lst}

The \$2 (or \$V0) register is used to store the function's return value.
\myindex{MIPS!\Pseudoinstructions!LI}
\INS{LI} stands for ``Load Immediate'' and is the MIPS equivalent to \MOV.

\myindex{MIPS!\Instructions!J}
The other instruction is the jump instruction (J or JR) which returns the execution flow to the \gls{caller}.

\myindex{MIPS!Branch delay slot}
You might be wondering why the positions of the load instruction (LI) and the jump instruction (J or JR) are swapped. This is due to a \ac{RISC} feature called ``branch delay slot''.

The reason this happens is a quirk in the architecture of some RISC \ac{ISA}s and isn't important for our
purposes---we must simply keep in mind that in MIPS, the instruction following a jump or branch instruction
is executed \IT{before} the jump/branch instruction itself.

As a consequence, branch instructions always swap places with the instruction executed immediately beforehand.

In practice, functions which merely return 1 (\IT{true}) or 0 (\IT{false}) are very frequent.

The smallest ever of the standard UNIX utilities, \IT{/bin/true} and \IT{/bin/false} return 0 and 1 respectively, as an exit code.
(Zero as an exit code usually means success, non-zero means error.)


\subsection{Correctly disassembled code}
\label{correctly_disasmed_code}

Each \ac{ISA} has a dozen of a most used instructions, all the rest are used much less often.

As of x86, it is interesting to know that the fact that function calls (\PUSH/\CALL/\ADD) and \MOV
instructions are the most frequently executed pieces of code in almost all
programs we use.
In other words, \ac{CPU} is very busy passing information between levels of abstractions, or,
it can be said, it's very busy switching between these levels.
Regardless type of \ac{ISA}.
This is a cost of splitting problems into several levels of abstractions (so humans could work with them easier).


\section{Text strings right in the middle of compressed data}

\myindex{Linux kernel}
You can download Linux kernels and find English words right in the middle of compressed data:

\begin{lstlisting}
% wget https://www.kernel.org/pub/linux/kernel/v4.x/linux-4.10.2.tar.gz

% xxd -g 1 -seek 0x515c550 -l 0x30 linux-4.10.2.tar.gz

0515c550: c5 59 43 cf 41 27 85 54 35 4a 57 90 73 89 b7 6a  .YC.A'.T5JW.s..j
0515c560: 15 af 03 db 20 df 6a 51 f9 56 49 52 55 53 3d da  .... .jQ.VIRUS=.
0515c570: 0e b9 29 24 cc 6a 38 e2 78 66 09 33 72 aa 88 df  ..)$.j8.xf.3r...
\end{lstlisting}

\begin{lstlisting}
% wget https://cdn.kernel.org/pub/linux/kernel/v2.3/linux-2.3.3.tar.bz2

% xxd -g 1 -seek 0xa93086 -l 0x30 linux-2.3.3.tar.bz2

00a93086: 4d 45 54 41 4c cd 44 45 2d 2c 41 41 54 94 8b a1  METAL.DE-,AAT...
00a93096: 5d 2b d8 d0 bd d8 06 91 74 ab 41 a0 0a 8a 94 68  ]+......t.A....h
00a930a6: 66 56 86 81 68 0d 0e 25 6b b6 80 a4 28 1a 00 a4  fV..h..%k...(...
\end{lstlisting}

One of Linux kernel patches in compressed form has the ``Linux'' word itself:

\begin{lstlisting}
% wget https://cdn.kernel.org/pub/linux/kernel/v4.x/testing/patch-4.6-rc4.gz

% xxd -g 1 -seek 0x4d03f -l 0x30 patch-4.6-rc4.gz

0004d03f: c7 40 24 bd ae ef ee 03 2c 95 dc 65 eb 31 d3 f1  .@$.....,..e.1..
0004d04f: 4c 69 6e 75 78 f2 f3 70 3c 3a bd 3e bd f8 59 7e  Linux..p<:.>..Y~
0004d05f: cd 76 55 74 2b cb d5 af 7a 35 56 d7 5e 07 5a 67  .vUt+...z5V.^.Zg
\end{lstlisting}

Other English words I've found in other compressed Linux kernel trees:

\begin{lstlisting}
linux-4.6.2.tar.gz: [maybe] at 0x68e78ec
linux-4.10.14.tar.xz: [OCEAN] at 0x6bf0a8
linux-4.7.8.tar.gz: [FUNNY] at 0x29e6e20
linux-4.6.4.tar.gz: [DRINK] at 0x68dc314
linux-2.6.11.8.tar.bz2: [LUCKY] at 0x1ab5be7
linux-3.0.68.tar.gz: [BOOST] at 0x11238c7
linux-3.0.16.tar.bz2: [APPLE] at 0x34c091
linux-3.0.26.tar.xz: [magic] at 0x296f7d9
linux-3.11.8.tar.bz2: [TRUTH] at 0xf635ba
linux-3.10.11.tar.bz2: [logic] at 0x4a7f794
\end{lstlisting}

\myindex{Apophenia}
\myindex{Pareidolia}
\myindex{Lurkmore}
There is a nice illustration of apophenia and pareidolia
There is a nice illustration of apophenia and pareidolia
(human's mind ability to see faces in clouds, etc) in Lurkmore, Russian counterpart of Encyclopedia Dramatica.
As they wrote in the article about electronic voice phenomenon\footnote{\url{http://archive.is/gYnFL}},
you can open any long enough compressed file in hex editor and find well-known 3-letter Russian obscene word, and you'll find it a lot: but that means nothing, just a mere coincidence.

And I was interested in calculation, how big compressed file must be to contain all possible 3-letter, 4-letter, etc, words?
In my naive calculations, I've got this: probability of the first specific byte in the middle of compressed data stream with maximal entropy is $\frac{1}{256}$, probability of the 2nd is also $\frac{1}{256}$,
and probability of specific byte pair is $\frac{1}{256 \cdot 256} = \frac{1}{256^2}$.
Probabilty of specific triple is $\frac{1}{256^3}$.
If the file has maximal entropy (which is almost unachievable, but \dots) and we live in an ideal world, you've got to have a file of size just $256^3=16777216$, which is 16-17MB.
\myindex{rafind2}
You can check: get any compressed file, and use \IT{rafind2} to search for any 3-letter word (not just that Russian obscene one).

It took $\approx$ 8-9 GB of my downloaded movies/TV series files to find the word ``beer'' in them (case sensitive).
Perhaps, these movies wasn't compressed good enough?
This is also true for a well-known 4-letter English obscene word.

My approach is naive, so I googled for mathematically grounded one, and have find this question:
``Time until a consecutive sequence of ones in a random bit sequence''
\footnote{\url{http://math.stackexchange.com/questions/27989/time-until-a-consecutive-sequence-of-ones-in-a-random-bit-sequence/27991#27991}}.
The answer is: $(p^{−n}−1)/(1−p)$, where $p$ is probability of each event and $n$ is number of consecutive events.
Plug $\frac{1}{256}$ and $3$ and you'll get almost the same as my naive calculations.

So any 3-letter word can be found in the compressed file (with ideal entropy) of length $256^3 = \approx 17MB$, any 4-letter word --- $256^4 = 4.7GB$ (size of DVD).
Any 5-letter word --- $256^5 = \approx 1TB$.

For the piece of text you are reading now, I mirrored the whole \href{https://www.kernel.org/}{kernel.org} website (hopefully, sysadmins can forgive me),
and it has $\approx$ 430GB of compressed Linux Kernel source trees.
It has enough compressed data to contain these words, however, I cheated a bit: I searched for both lowercase and uppercase strings, thus compressed data set I need is almost halved.

This is quite interesting thing to think about: 1TB of compressed data with maximal entropy has all possible 5-byte chains,
but the data is encoded not in chains itself, but in the order of chains (no matter of compression algorithm, etc).

Now the information for gamblers: one should throw a dice $\approx 42$ times to get a pair of six, but no one will tell you, when exactly this will happen.
\myindex{Rosencrantz \& Guildenstern Are Dead}
I don't remember, how many times coin was tossed in the ``Rosencrantz \& Guildenstern Are Dead'' movie, but one should toss it $\approx 2048$ times and at some point, you'll get 10 heads in a row,
and at some other point, 10 tails in a row. Again, no one will tell you, when exactly this will happen.

Compressed data can also be treated as a stream of random data, so we can use the same mathematics to determine probabilities, etc.

If you can live with strings of mixed case, like ``bEeR'', probabilities and compressed data sets are much lower:
$128^3=2MB$ for all 3-letter words of mixed case,
$128^4=268MB$ for all 4-letter words,
$128^5=34GB$ for all 5-letter words, etc.

\myindex{Jorge Luis Borges}
Moral of the story: whenever you search for some patterns, you can find it in the middle of compressed blob, but that means nothing else then coincidence.
In philosophical sense, this is a case of selection/confirmation bias: you find what you search for in ``The Library of Babel''\footnote{Short story by Jorge Luis Borges}.



\mysection{Other things}

\subsection{General idea}

A reverse engineer should try to be in programmer's shoes as often as possible. 
To take his/her viewpoint and ask himself, how would one solve some task the specific case.

\subsection{Order of functions in binary code}

All functions located in a single .c or .cpp-file are compiled into corresponding object (.o) file.
Later, linker puts all object files it needs together, not changing order or functions in them.
As a consequence, if you see two or more consecutive functions, it means, that they were placed together
in a single source code file (unless you're on border of two object files, of course.)
This means these functions have something in common, that they are from the same \ac{API} level, from same library, etc.

\subsection{Tiny functions}

Tiny functions like empty functions (\myref{empty_func})
or function which returns just ``true'' (1) or ``false'' (0) (\myref{ret_val_func}) are very common,
and almost all decent compilers tend to put only one such function into resulting executable code even if there were several
similar functions in source code.
So, whenever you see a tiny function consisting just of \TT{mov eax, 1 / ret}
which is referenced (and can be called) from many places,
which are seems unconnected to each other, this may be a result of such optimization.%

\subsection{\Cpp}

\ac{RTTI}~(\myref{RTTI})-data may be also useful for \Cpp class identification.

