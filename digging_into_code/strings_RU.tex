\mysection{Строки}
\label{sec:digging_strings}

\mysection{Оптимизации циклов}

% subsections:
\subsection{Странная оптимизация циклов}

Это самая простая (из всех возможных) реализация memcpy():

\begin{lstlisting}[style=customc]
void memcpy (unsigned char* dst, unsigned char* src, size_t cnt)
{
	size_t i;
	for (i=0; i<cnt; i++)
		dst[i]=src[i];
};
\end{lstlisting}

Как минимум MSVC 6.0 из конца 90-х вплоть до MSVC 2013 может выдавать вот такой странный код (этот листинг создан MSVC 2013
x86):

\lstinputlisting[style=customasmx86]{advanced/500_loop_optimizations/1_1_RU.lst}

Это всё странно, потому что как люди работают с двумя указателями? Они сохраняют два адреса в двух регистрах или двух
ячейках памяти.
Компилятор MSVC в данном случае сохраняет два указателя как один указатель (\IT{скользящий dst} в \EAX)
и разницу между указателями \IT{src} и \IT{dst} (она остается неизменной во время исполнения цикла, в \ESI).
\myindex{\CLanguageElements!ptrdiff\_t}
(Кстати, это тот редкий случай, когда можно использовать тип ptrdiff\_t.)
Когда нужно загрузить байт из \IT{src}, он загружается на \IT{diff + скользящий dst} и сохраняет байт просто на
\IT{скользящем dst}.

Должно быть это какой-то трюк для оптимизации. Но я переписал эту ф-цию так:

\lstinputlisting[style=customasmx86]{advanced/500_loop_optimizations/1_2.lst}

\dots и она работает также быстро как и \IT{соптимизированная} версия на моем Intel Xeon E31220 @ 3.10GHz.
Может быть, эта оптимизация предназначалась для более старых x86-процессоров 90-х, т.к., этот трюк использует
как минимум древний MS VC 6.0?

Есть идеи?

\myindex{Hex-Rays}
Hex-Rays 2.2 не распознает такие шаблонные фрагменты кода (будем надеятся, это временно?):

\begin{lstlisting}[style=customc]
void __cdecl f1(char *dst, char *src, size_t size)
{
  size_t counter; // edx@1
  char *sliding_dst; // eax@2
  char tmp; // cl@3

  counter = size;
  if ( size )
  {
    sliding_dst = dst;
    do
    {
      tmp = (sliding_dst++)[src - dst];         // разница (src-dst) вычисляется один раз, перед телом цикла
      *(sliding_dst - 1) = tmp;
      --counter;
    }
    while ( counter );
  }
}
\end{lstlisting}

Тем не менее, этот трюк часто используется в MSVC (и не только в самодельных ф-циях \IT{memcpy()}, но также и во многих
циклах, работающих с двумя или более массивами), так что для реверс-инжиниров стоит помнить об этом.

% <!-- As of why writting occurred after <b>dst</b> incrementing? -->


\subsection{Возврат строки}

Классическая ошибка из \RobPikePractice{}:

\begin{lstlisting}[style=customc]
#include <stdio.h>

char* amsg(int n, char* s)
{
        char buf[100];

        sprintf (buf, "error %d: %s\n", n, s) ;

        return buf;
};

int main()
{
        printf ("%s\n", amsg (1234, "something wrong!"));
};
\end{lstlisting}

Она упадет.
В начале, попытаемся понять, почему.

Это состояние стека перед возвратом из amsg():

% FIXME! TikZ or whatever
\begin{lstlisting}
§(низкие адреса)§

§[amsg(): 100 байт]§
§[RA]                               <- текущий SP§
§[два аргумента amsg]§
§[что-то еще]§
§[локальные переменные main()]§

§(высокие адреса)§
\end{lstlisting}

Когда управление возвращается из amsg() в \main, пока всё хорошо.
Но когда \printf вызывается из \main, который, в свою очередь, использует стек для своих нужд, затирая 100-байтный буфер.
В лучшем случае, будет выведен случайный мусор.

Трудно поверить, но я знаю, как это исправить:

\begin{lstlisting}[style=customc]
#include <stdio.h>

char* amsg(int n, char* s)
{
        char buf[100];

        sprintf (buf, "error %d: %s\n", n, s) ;

        return buf;
};

char* interim (int n, char* s)
{
        char large_buf[8000];
        // используем локальный массив.
        // а иначе компилятор выбросит его при оптимизации, как неиспользуемый.
        large_buf[0]=0;
        return amsg (n, s);
};

int main()
{
        printf ("%s\n", interim (1234, "something wrong!"));
};
\end{lstlisting}

Это заработает если скомпилировано в MSVC 2013 без оптимизаций и с опцией \TT{/GS-}\footnote{Выключить защиту от переполнения буфера}.
MSVC предупредит: ``warning C4172: returning address of local variable or temporary'', но код запустится и сообщение выведется.
Посмотрим состояние стека в момент, когда amsg() возвращает управление в interim():

\begin{lstlisting}
§(низкие адреса)§

§[amsg(): 100 байт]§
§[RA]                                      <- текущий SP§
§[два аргумента amsg()]§
§[вледения interim(), включая 8000 байт]§
§[еще что-то]§
§[локальные переменные main()]§

§(высокие адреса)§
\end{lstlisting}

Теперь состояние стека на момент, когда interim() возвращает управление в \main{}:

\begin{lstlisting}
§(низкие адреса)§

§[amsg(): 100 байт]§
§[RA]§
§[два аргумента amsg()]§
§[вледения interim(), включая 8000 байт]§
§[еще что-то]                              <- текущий SP§
§[локальные переменные main()]§

§(высокие адреса)§
\end{lstlisting}

Так что когда \main вызывает \printf, он использует стек в месте, где выделен буфер в interim(),
и не затирает 100 байт с сообщение об ошибке внутри, потому что 8000 байт (или может быть меньше) это достаточно для всего,
что делает \printf и другие нисходящие ф-ции!

Это также может сработать, если между ними много ф-ций, например:
\main $\rightarrow$ f1() $\rightarrow$ f2() $\rightarrow$ f3() ... $\rightarrow$ amsg(),
и тогда результат amsg() используется в \main.
Дистанция между \ac{SP} в \main и адресом буфера \TT{buf[]} должна быть достаточно длинной.

Вот почему такие ошибки опасны: иногда ваш код работает (и бага прячется незамеченной). иногда нет.
\label{heisenbug}
\myindex{Хейзенбаги}
Такие баги в шутку называют хейзенбаги или шрёдинбаги\footnote{\url{https://en.wikipedia.org/wiki/Heisenbug}}.





\subsection{Поиск строк в бинарном файле}

\epigraph{Actually, the best form of Unix documentation is frequently running the
\textbf{strings} command over a program’s object code. Using \textbf{strings}, you can get
a complete list of the program’s hard-coded file name, environment variables,
undocumented options, obscure error messages, and so forth.}{The Unix-Haters Handbook}

\myindex{UNIX!strings}
Стандартная утилита в UNIX \IT{strings} это самый простой способ увидеть строки в файле.
Например, это строки найденные в исполняемом файле sshd из OpenSSH 7.2:

\lstinputlisting{digging_into_code/sshd_strings.txt}

Тут опции, сообщения об ошибках, пути к файлам, импортируемые модули, функции, и еще какие-то странные строки (ключи?)
Присутствует также нечитаемый шум---иногда в x86-коде бывают целые куски состоящие из печатаемых ASCII-символов,
вплоть до ~8 символов.

Конечно, OpenSSH это опенсорсная программа.
Но изучение читаемых строк внутри некоторого неизвестного бинарного файла это зачастую самый первый шаг в анализе.
\myindex{UNIX!grep}

Также можно использовать \IT{grep}.

\myindex{Hiew}
\myindex{Sysinternals}
В Hiew есть такая же возможность (Alt-F6), также как и в Sysinternals ProcessMonitor.

\subsection{Сообщения об ошибках и отладочные сообщения}

Очень сильно помогают отладочные сообщения, если они имеются. В некотором смысле, отладочные сообщения, 
это отчет о том, что сейчас происходит в программе.
Зачастую, это \printf-подобные функции, 
которые пишут куда-нибудь в лог, а бывает так что и не пишут ничего, но вызовы остались, так как эта сборка --- не
отладочная, а \IT{release}.

\myindex{\oracle}
Если в отладочных сообщениях дампятся значения некоторых локальных или глобальных переменных, 
это тоже может помочь, как минимум, узнать их имена. 
Например, в \oracle одна из таких функций: \TT{ksdwrt()}.

Осмысленные текстовые строки вообще очень сильно могут помочь. 
Дизассемблер \IDA может сразу указать, из какой функции и из какого её места используется эта строка. 
Встречаются и смешные случаи
\footnote{\href{http://go.yurichev.com/17223}{blog.yurichev.com}}.

Сообщения об ошибках также могут помочь найти то что нужно. 
В \oracle сигнализация об ошибках проходит при помощи вызова некоторой группы функций. \\
Тут еще немного об этом: \href{http://go.yurichev.com/17224}{blog.yurichev.com}.

\myindex{Error messages}
Можно довольно быстро найти, какие функции сообщают о каких ошибках, и при каких условиях.

Это, кстати, одна из причин, почему в защите софта от копирования, 
бывает так, что сообщение об ошибке заменяется 
невнятным кодом или номером ошибки. Мало кому приятно, если взломщик быстро поймет, 
из-за чего именно срабатывает защита от копирования, просто по сообщению об ошибке.

Один из примеров шифрования сообщений об ошибке, здесь: \myref{examples_SCO}.

\subsection{Подозрительные магические строки}

Некоторые магические строки, используемые в бэкдорах выглядят очень подозрительно.
Например, в домашних роутерах TP-Link WR740 был бэкдор
\footnote{\url{http://sekurak.pl/tp-link-httptftp-backdoor/}, на русском: \url{http://m.habrahabr.ru/post/172799/}}.
Бэкдор активировался при посещении следующего URL:\\
\url{http://192.168.0.1/userRpmNatDebugRpm26525557/start_art.html}.\\
Действительно, строка \q{userRpmNatDebugRpm26525557} присутствует в прошивке.

Эту строку нельзя было нагуглить до распространения информации о бэкдоре.

Вы не найдете ничего такого ни в одном \ac{RFC}.

Вы не найдете ни одного алгоритма, который бы использовал такие странные последовательности байт.

И это не выглядит как сообщение об ошибке, или отладочное сообщение.

Так что проверить использование подобных странных строк --- это всегда хорошая идея.
\\
\myindex{base64}
Иногда такие строки кодируются при помощи 
base64\footnote{Например, бэкдор в кабельном модеме Arris: 
\url{http://www.securitylab.ru/analytics/461497.php}}.
Так что неплохая идея их всех декодировать и затем просмотреть глазами, пусть даже бегло.
\\
\myindex{Security through obscurity}

Более точно, такой метод сокрытия бэкдоров называется \q{security through obscurity} (безопасность через
запутанность).
