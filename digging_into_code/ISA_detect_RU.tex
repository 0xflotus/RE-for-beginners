\mysection{Определение \ac{ISA}}
\label{ISA_detect}

Часто, вы можете иметь дело с бинарным файлом для неизвестной \ac{ISA}.
Вероятно, простейший способ определить \ac{ISA} это пробовать разные в IDA, objdump или другом дизассемблере.

Чтобы этого достичь, нужно понимать разницу между некорректно дизассемблированным кодом, и корректно дизассемблированным.

% subsection:
\renewcommand{\CURPATH}{digging_into_code/incorrect_disassembly}
\mysection{Оптимизации циклов}

% subsections:
\subsection{Странная оптимизация циклов}

Это самая простая (из всех возможных) реализация memcpy():

\begin{lstlisting}[style=customc]
void memcpy (unsigned char* dst, unsigned char* src, size_t cnt)
{
	size_t i;
	for (i=0; i<cnt; i++)
		dst[i]=src[i];
};
\end{lstlisting}

Как минимум MSVC 6.0 из конца 90-х вплоть до MSVC 2013 может выдавать вот такой странный код (этот листинг создан MSVC 2013
x86):

\lstinputlisting[style=customasmx86]{advanced/500_loop_optimizations/1_1_RU.lst}

Это всё странно, потому что как люди работают с двумя указателями? Они сохраняют два адреса в двух регистрах или двух
ячейках памяти.
Компилятор MSVC в данном случае сохраняет два указателя как один указатель (\IT{скользящий dst} в \EAX)
и разницу между указателями \IT{src} и \IT{dst} (она остается неизменной во время исполнения цикла, в \ESI).
\myindex{\CLanguageElements!ptrdiff\_t}
(Кстати, это тот редкий случай, когда можно использовать тип ptrdiff\_t.)
Когда нужно загрузить байт из \IT{src}, он загружается на \IT{diff + скользящий dst} и сохраняет байт просто на
\IT{скользящем dst}.

Должно быть это какой-то трюк для оптимизации. Но я переписал эту ф-цию так:

\lstinputlisting[style=customasmx86]{advanced/500_loop_optimizations/1_2.lst}

\dots и она работает также быстро как и \IT{соптимизированная} версия на моем Intel Xeon E31220 @ 3.10GHz.
Может быть, эта оптимизация предназначалась для более старых x86-процессоров 90-х, т.к., этот трюк использует
как минимум древний MS VC 6.0?

Есть идеи?

\myindex{Hex-Rays}
Hex-Rays 2.2 не распознает такие шаблонные фрагменты кода (будем надеятся, это временно?):

\begin{lstlisting}[style=customc]
void __cdecl f1(char *dst, char *src, size_t size)
{
  size_t counter; // edx@1
  char *sliding_dst; // eax@2
  char tmp; // cl@3

  counter = size;
  if ( size )
  {
    sliding_dst = dst;
    do
    {
      tmp = (sliding_dst++)[src - dst];         // разница (src-dst) вычисляется один раз, перед телом цикла
      *(sliding_dst - 1) = tmp;
      --counter;
    }
    while ( counter );
  }
}
\end{lstlisting}

Тем не менее, этот трюк часто используется в MSVC (и не только в самодельных ф-циях \IT{memcpy()}, но также и во многих
циклах, работающих с двумя или более массивами), так что для реверс-инжиниров стоит помнить об этом.

% <!-- As of why writting occurred after <b>dst</b> incrementing? -->


\subsection{Возврат строки}

Классическая ошибка из \RobPikePractice{}:

\begin{lstlisting}[style=customc]
#include <stdio.h>

char* amsg(int n, char* s)
{
        char buf[100];

        sprintf (buf, "error %d: %s\n", n, s) ;

        return buf;
};

int main()
{
        printf ("%s\n", amsg (1234, "something wrong!"));
};
\end{lstlisting}

Она упадет.
В начале, попытаемся понять, почему.

Это состояние стека перед возвратом из amsg():

% FIXME! TikZ or whatever
\begin{lstlisting}
§(низкие адреса)§

§[amsg(): 100 байт]§
§[RA]                               <- текущий SP§
§[два аргумента amsg]§
§[что-то еще]§
§[локальные переменные main()]§

§(высокие адреса)§
\end{lstlisting}

Когда управление возвращается из amsg() в \main, пока всё хорошо.
Но когда \printf вызывается из \main, который, в свою очередь, использует стек для своих нужд, затирая 100-байтный буфер.
В лучшем случае, будет выведен случайный мусор.

Трудно поверить, но я знаю, как это исправить:

\begin{lstlisting}[style=customc]
#include <stdio.h>

char* amsg(int n, char* s)
{
        char buf[100];

        sprintf (buf, "error %d: %s\n", n, s) ;

        return buf;
};

char* interim (int n, char* s)
{
        char large_buf[8000];
        // используем локальный массив.
        // а иначе компилятор выбросит его при оптимизации, как неиспользуемый.
        large_buf[0]=0;
        return amsg (n, s);
};

int main()
{
        printf ("%s\n", interim (1234, "something wrong!"));
};
\end{lstlisting}

Это заработает если скомпилировано в MSVC 2013 без оптимизаций и с опцией \TT{/GS-}\footnote{Выключить защиту от переполнения буфера}.
MSVC предупредит: ``warning C4172: returning address of local variable or temporary'', но код запустится и сообщение выведется.
Посмотрим состояние стека в момент, когда amsg() возвращает управление в interim():

\begin{lstlisting}
§(низкие адреса)§

§[amsg(): 100 байт]§
§[RA]                                      <- текущий SP§
§[два аргумента amsg()]§
§[вледения interim(), включая 8000 байт]§
§[еще что-то]§
§[локальные переменные main()]§

§(высокие адреса)§
\end{lstlisting}

Теперь состояние стека на момент, когда interim() возвращает управление в \main{}:

\begin{lstlisting}
§(низкие адреса)§

§[amsg(): 100 байт]§
§[RA]§
§[два аргумента amsg()]§
§[вледения interim(), включая 8000 байт]§
§[еще что-то]                              <- текущий SP§
§[локальные переменные main()]§

§(высокие адреса)§
\end{lstlisting}

Так что когда \main вызывает \printf, он использует стек в месте, где выделен буфер в interim(),
и не затирает 100 байт с сообщение об ошибке внутри, потому что 8000 байт (или может быть меньше) это достаточно для всего,
что делает \printf и другие нисходящие ф-ции!

Это также может сработать, если между ними много ф-ций, например:
\main $\rightarrow$ f1() $\rightarrow$ f2() $\rightarrow$ f3() ... $\rightarrow$ amsg(),
и тогда результат amsg() используется в \main.
Дистанция между \ac{SP} в \main и адресом буфера \TT{buf[]} должна быть достаточно длинной.

Вот почему такие ошибки опасны: иногда ваш код работает (и бага прячется незамеченной). иногда нет.
\label{heisenbug}
\myindex{Хейзенбаги}
Такие баги в шутку называют хейзенбаги или шрёдинбаги\footnote{\url{https://en.wikipedia.org/wiki/Heisenbug}}.





\subsection{Корректино дизассемблированный код}
\label{correctly_disasmed_code}

Каждая \ac{ISA} имеет десяток самых используемых инструкций, остальные используются куда реже.

Интересно знать тот факт, что в x86, инструкции вызовов ф-ций (\PUSH/\CALL/\ADD) и \MOV
это наиболее часто исполняющиеся инструкции в коде почти во всем ПО что мы используем.
Другими словами, \ac{CPU} очень занят передачей информации между уровнями абстракции, или, можно сказать, очень занят
переключением между этими уровнями.
Вне зависимости от \ac{ISA}.
Это цена расслоения программ на разные уровни абстракций (чтобы человеку было легче с ними управляться).

