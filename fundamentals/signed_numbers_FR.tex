\mysection{\SignedNumbersSectionName}
\label{sec:signednumbers}
\myindex{Signed numbers}

\newcommand{\URLS}{\href{http://go.yurichev.com/17117}{wikipedia}}

Il existe plusieurs méthodes pour représenter les nombres signées\footnote{\URLS},
mais le \q{complément à deux} est la plus populaire sur les ordinateurs.

Voici une table pour quelques valeurs d'octet:

\begin{center}
\begin{tabular}{ | l | l | l | l | }
\hline
\HeaderColor binaire & \HeaderColor hexadécimal & \HeaderColor non-signé & \HeaderColor signé \\
\hline
01111111 & 0x7f & 127 & 127 \\
\hline
01111110 & 0x7e & 126 & 126 \\
\hline
\multicolumn{4}{ |c| }{...} \\
\hline
00000110 & 0x6 & 6 & 6 \\
\hline
00000101 & 0x5 & 5 & 5 \\
\hline
00000100 & 0x4 & 4 & 4 \\
\hline
00000011 & 0x3 & 3 & 3 \\
\hline
00000010 & 0x2 & 2 & 2 \\
\hline
00000001 & 0x1 & 1 & 1 \\
\hline
00000000 & 0x0 & 0 & 0 \\
\hline
11111111 & 0xff & 255 & -1 \\
\hline
11111110 & 0xfe & 254 & -2 \\
\hline
11111101 & 0xfd & 253 & -3 \\
\hline
11111100 & 0xfc & 252 & -4 \\
\hline
11111011 & 0xfb & 251 & -5 \\
\hline
11111010 & 0xfa & 250 & -6 \\
\hline
\multicolumn{4}{ |c| }{...} \\
\hline
10000010 & 0x82 & 130 & -126 \\
\hline
10000001 & 0x81 & 129 & -127 \\
\hline
10000000 & 0x80 & 128 & -128 \\
\hline
\end{tabular}
\end{center}

\myindex{x86!\Instructions!JA}
\myindex{x86!\Instructions!JB}
\myindex{x86!\Instructions!JL}
\myindex{x86!\Instructions!JG}
La différence entre nombres signé et non-signé est que si l'on représente \TT{0xFFFFFFFE}
et \TT{0x00000002} comme non signées, alors le premier nombre (4294967294) est plus
grand que le second (2).
Si nous les représentons comme signés, le premier devient $-2$, et il est plus petit
que le second.
C'est la raison pour laquelle les sauts conditionnels~(\myref{sec:Jcc}) existent
à la fois pour des opérations signées (p. ex. \JG, \JL) et non-signées (\JA, \JB).

Par souci de simplicité, voici ce qu'il faut retenir:

\begin{itemize}
\item Les nombres peuvent être signés ou non-signés.

\item Types \CCpp signés:

  \begin{itemize}
    \item \TT{int64\_t} (-9,223,372,036,854,775,808 .. 9,223,372,036,854,775,807)
	  (-~9.2..~9.2 quintillions) ou \\
                \TT{0x8000000000000000..0x7FFFFFFFFFFFFFFF}),
    \item \Tint (-2,147,483,648..2,147,483,647 (-~2.15..~2.15Gb) ou \TT{0x80000000..0x7FFFFFFF}),
    \item \Tchar (-128..127 ou \TT{0x80..0x7F}),
    \item \TT{ssize\_t}.
   \end{itemize}

	Non-signés:
  \begin{itemize}
	  \item \TT{uint64\_t} (0..18,446,744,073,709,551,615 
		  (~18 quintillions) ou \TT{0..0xFFFFFFFFFFFFFFFF}),
   \item \TT{unsigned int} (0..4,294,967,295 (~4.3Gb) ou \TT{0..0xFFFFFFFF}),
   \item \TT{unsigned char} (0..255 ou \TT{0..0xFF}),
   \item \TT{size\_t}.
  \end{itemize}

\item Les types signés ont le signe dans le \ac{MSB}: 1 signifie \q{moins}, 0 signifie \q{plus}.

\item Étendre à un type de données plus large est facile:
\myref{subsec:sign_extending_32_to_64}.

\label{sec:signednumbers:negation}
\item La négation est simple: il suffit d'inverser tous les bits et d'ajouter 1.

Nous pouvons garder à l'esprit qu'un nombre de signe opposé se trouve de l'autre côté,
à la même distance de zéro.
L'addition d'un est nécessaire car zéro se trouve au milieu.

\myindex{x86!\Instructions!IDIV}
\myindex{x86!\Instructions!DIV}
\myindex{x86!\Instructions!IMUL}
\myindex{x86!\Instructions!MUL}
\myindex{x86!\Instructions!CBW}
\myindex{x86!\Instructions!CWD}
\myindex{x86!\Instructions!CWDE}
\myindex{x86!\Instructions!CDQ}
\myindex{x86!\Instructions!CDQE}
\myindex{x86!\Instructions!MOVSX}
\myindex{x86!\Instructions!SAR}
\item
	Les opérations d'addition et de soustraction fonctionnent bien pour les valeurs signées et non-signées.
	Mais pour la multiplication et la division, le x86 possède des instructions différentes:
	\TT{IDIV}/\TT{IMUL} pour les signés et \TT{DIV}/\TT{MUL} pour les non-signés.
\item
	Voici d'autres instructions qui fonctionnent avec des nombres signés:\\
	\TT{CBW/CWD/CWDE/CDQ/CDQE} (\myref{ins:CBW_CWD_etc}), \TT{MOVSX} (\myref{MOVSX}), \TT{SAR} (\myref{ins:SAR}).
\end{itemize}

Une table avec quelques valeurs négatives et positives (\ref{signed_tbl}) ressemble
à un thermomètre avec une échelle Celsius.
C'est pourquoi l'addition et la soustraction fonctionnent bien pour les nombres signés
et non-signés:
si le premier opérande est représenté par une marque sur un thermomètre, et que l'on
doit ajouter un second opérande, et qu'il est positif, nous devons juste augmenter
la marque sur le thermomètre de la valeur du second opérande.
Si le second opérande est négatif, alors nous baissons la marque de la valeur absolue
du second opérande.

L'addition de deux nombres négatifs fonctionne comme suit.
Par exemple, nous devons ajouter -2 et -3 en utilisant des registres 16-bit.
-2 et -3 sont respectivement 0xfffe et 0xfffd.
si nous les ajoutons comme nombres non-signés, nous obtenons 0xfffe+0xfffd=0x1fffb.
Mais nous travaillons avec des registres 16-bit, le résultat est \IT{tronqué},
le premier 1 est perdu, et il reste 0xfffb et c'est -5.
Ceci fonctionne car -2 (ou 0xfffe) peut être représenté en utilisant des mots simples
comme suit:
``il manque 2 à la valeur maximale d'un registre 16-bit + 1''.
-3 peut être représenté comme ``\dots il manque 3 à la valeur maximale jusqu'à \dots''.
La valeur maximale d'un registre 16-bit + 1 est 0x10000.
Pendant l'addition de deux nombres et en \IT{tronquant} modulo $2^{16}$, il manquera
$2+3=5$.

% subsections:
\subsection{Utiliser IMUL au lieu de MUL}
\label{IMUL_over_MUL}

\myindex{x86!\Instructions!MUL}
\myindex{x86!\Instructions!IMUL}
Un exemple comme \lstref{unsigned_multiply_C} où deux valeurs non signées sont multipliées
compile en \lstref{unsigned_multiply_lst} où \IMUL est utilisé à la place de \MUL.

Ceci est une propriété importante des instructions \MUL et \IMUL.
Tout d'abord, les deux produisent une valeur 64-bit si deux valeurs 32-bit sont multipliées,
ou une valeur 128-bit si deux valeurs 64-bit sont multipliées (le plus grand \glslink{product}{produit}
dans un environnement 32-bit est \\
\GTT{0xffffffff*0xffffffff=0xfffffffe00000001}).
Mais les standards \CCpp n'ont pas de moyen d'accèder à la moitié supérieure du résultat,
et un \glslink{product}{produit} a toujours la même taille que ses multiplicandes.
Et les deux instructions \MUL et \IMUL fonctionnent de la même manière si la moitié
supérieure est ignorée, i.e, elles produisent le même résultat dans la partie inférieure.
Ceci est une propriété importante de la façon de représenter les nombre en \q{complément
à deux}.

Donc, le compilateur \CCpp peut utiliser indifféremment ces deux instructions.

Mais \IMUL est plus flexible que \MUL, car elle prend n'importe quel(s) registre(s) comme
source, alors que \MUL nécessite que l'un des multiplicandes soit stocké dans le
registre \AX/\EAX/\RAX
Et même plus que ça: \MUL stocke son résultat dans la paire \GTT{EDX:EAX} en environnement
32-bit, ou \GTT{RDX:RAX} dans un 64-bit, donc elle calcule toujours le résultat complet.
Au contraire, il est possible de ne mettre qu'un seul registre de destination lorsque
l'on utilise \IMUL, au lieu d'une paire, et alors le \ac{CPU} calculera seulement
la partie basse, ce qui fonctionne plus rapidement [voir Torborn Granlund,
\IT{Instruction latencies and throughput for AMD and Intel x86 processors}\footnote{\url{http://yurichev.com/mirrors/x86-timing.pdf}}].

Cela étant considéré, les compilateurs \CCpp peuvent générer l'instruction \IMUL
plus souvent que \MUL.

\myindex{Compiler intrinsic}
Néanmoins, en utilisant les fonctions intrinsèques du compilateur, il est toujours
possible d'effectuer une multiplication non signée et d'obtenir le résultat \IT{complet}.
Ceci est parfois appelé \IT{multiplication étendue}.
MSVC a une fonction intrinsèque pour ceci, appelée \IT{\_\_emul}\footnote{\url{https://msdn.microsoft.com/en-us/library/d2s81xt0(v=vs.80).aspx}}
et une autre: \IT{\_umul128}\footnote{\url{https://msdn.microsoft.com/library/3dayytw9%28v=vs.100%29.aspx}}.
GCC offre le type de données \IT{\_\_int128}, et dans le cas de multiplicandes 64-bit,
ils sont déjà promus en 128-bit, puis le \glslink{product}{produit} est stocké dans
une autre valeur \IT{\_\_int128}, puis le résultat est décalé de 64 bits à droite,
et vous obtenez la moitié haute du résultat\footnote{Exemple: \url{http://stackoverflow.com/a/13187798}}.

\subsubsection{Fonction MulDiv() dans Windows}
\myindex{Windows!Win32!MulDiv()}

Windows possède la fonction MulDiv()\footnote{\url{https://msdn.microsoft.com/en-us/library/windows/desktop/aa383718(v=vs.85).aspx}},
fonction qui fusionne une multiplication et une division, elle multiplie deux entiers
32-bit dans une valeur 64-bit intermédiaire et la divise par un troisième entier 32-bit.
C'est plus facile que d'utiliser deux fonctions intrinsèques, donc les développeurs
de Microsoft ont écrit une fonction spéciale pour cela.
Et il semble que ça soit une fonction très utilisée, à en juger par son utilisation.


\subsection{Quelques ajouts à propos du complément à deux}

\epigraph{Exercice 2-1. Écrire un programme pour déterminer les intervalles des variables
\TT{char}, \TT{short}, \TT{int}, et \TT{long}, signées et non signées, en affichant
les valeurs appropriées depuis les headers standards et par calcul direct.}{\KRBook}

\subsubsection{Obtenir le nombre maximum de quelques \glslink{word}{mots}}

Le maximum d'un nombre non signé est simplement un nombre où tous les bits sont mis:
\IT{0xFF....FF} (ceci est -1 si le \glslink{word}{mot} est traité comme un entier
signé).
Donc, vous prenez un \glslink{word}{mot}, vous mettez tous les bits et vous obtenez
la valeur:

\begin{lstlisting}[style=customc]
#include <stdio.h>

int main()
{
	unsigned int val=~0; // changer à "unsigned char" pour obtenir la valeur maximale pour un octet 8-bit non-signé
	// 0-1 fonctionnera aussi, ou juste -1
	printf ("%u\n", val); // %u pour unsigned
};
\end{lstlisting}

C'est 4294967295 pour un entier 32-bit.

\subsubsection{Obtenir le nombre maximum de quelques \glslink{word}{mots} signés}

Le nombre signé minimum est encodé en \IT{0x80....00}, i.e., le bit le plus significatif
est mis, tandis que tous les autres sont à zéro.
Le nombre maximum signé est encodé de la même manière, mais tous les bits sont
inversés: \IT{0x7F....FF}.

Déplaçons un seul bit jusqu'à ce qu'il disparaisse:

\begin{lstlisting}[style=customc]
#include <stdio.h>

int main()
{
	signed int val=1; // changer à "signed char" pour trouver les valeurs pour un octet signé
	while (val!=0)
	{
		printf ("%d %d\n", val, ~val);
		val=val<<1;
	};
};
\end{lstlisting}

La sortie est:

\begin{lstlisting}
...

536870912 -536870913
1073741824 -1073741825
-2147483648 2147483647
\end{lstlisting}

Les deux dernier nombres sont respectivement le minimum et le maximum d'un entier
signé 32-bit \IT{int}.



