\subsection{hachage Zobrist / hachage de tabulation}
\myindex{Chess}
\myindex{Zobrist hashing}
\myindex{Tabulation hashing}

Si vous travaillez sur un moteur de jeu d'échec, vous traversez l'arbre de jeu de
très nombreuses fois par seconde, et souvent, vous pouvez rencontrer une même position,
qui a déjà été étudiée.

Donc, vous devez utiliser un méthode pour stocker quelque part les positions déjà calculées.
Mais les positions d'un jeu d'échec demandent beaucoup de mémoire, et une fonction
de hachage peut être utilisée à la place.

Voici un moyen de compresser une position d'échecs dans une valeur 64-bit, appelée
le hachage de Zobrist:

\begin{lstlisting}[style=customc]
// nous avons un échiquier de 8*8 et 12 pièces (6 pour le côté blanc et 6 pour le noir)

uint64_t table[12][8][8]; // remplir avec des valeurs aléatoires

int position[8][8]; // pour chaque case de l'échiquier. 0 - no piece. 1..12 - piece

uint64_t hash;

for (int row=0; row<8; row++)
	for (int col=0; col<8; col++)
	{
		int piece=position[row][col];

		if (piece!=0)
			hash=hash^table[piece][row][col];
	};

return hash;
\end{lstlisting}

Maintenant la partie la plus intéressante: si la position suivante (modifiée) diffère
seulement d'une pièce (déplacée), vous ne devez pas recalculer le hachage pour la
position complète, tout ce que vous devez faire est:

\begin{lstlisting}[style=customc]
hash=...; // (déjà calculé)

// soustraire l'information à propos de l'ancienne pièce:
hash=hash^table[old_piece][old_row][old_col];

// ajouter l'information à propos de la nouvelle pièce:
hash=hash^table[new_piece][new_row][new_col];
\end{lstlisting}
