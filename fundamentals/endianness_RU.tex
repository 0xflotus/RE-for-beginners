\mysection{Endianness (порядок байт)}
\label{sec:endianness}

Endianness (порядок байт) это способ представления чисел в памяти.

\subsection{Big-endian (от старшего к младшему)}

Число \TT{0x12345678} представляется в памяти так:

\begin{center}
\begin{tabular}{ | l | l | }
\hline
\HeaderColor адрес в памяти & \HeaderColor значение байта \\
\hline
+0 & 0x12 \\
\hline
+1 & 0x34 \\
\hline
+2 & 0x56 \\
\hline
+3 & 0x78 \\
\hline
\end{tabular}
\end{center}

CPU с таким порядком включают в себя Motorola 68k, IBM POWER.

\subsection{Little-endian (от младшего к старшему)}

Число \TT{0x12345678} представляется в памяти так:

\begin{center}
\begin{tabular}{ | l | l | }
\hline
\HeaderColor адрес в памяти & \HeaderColor значение байта \\
\hline
+0 & 0x78 \\
\hline
+1 & 0x56 \\
\hline
+2 & 0x34 \\
\hline
+3 & 0x12 \\
\hline
\end{tabular}
\end{center}

CPU с таким порядком байт включают в себя Intel x86.

\subsection{\Example}

Возьмем big-endian Linux для MIPS заинсталированный в QEMU
\footnote{Доступен для скачивания здесь: \url{http://go.yurichev.com/17008}}.

И скомпилируем этот простой пример:

\begin{lstlisting}[style=customc]
#include <stdio.h>

int main()
{
	int v;

	v=123;

	printf ("%02X %02X %02X %02X\n", 
		*(char*)&v,
		*(((char*)&v)+1),
		*(((char*)&v)+2),
		*(((char*)&v)+3));
};
\end{lstlisting}

И запустим его:

\begin{lstlisting}
root@debian-mips:~# ./a.out 
00 00 00 7B
\end{lstlisting}

Это оно и есть.
0x7B это 123 в десятичном виде.
В little-endian-архитектуре, 7B это первый байт (вы можете это проверить в x86 или x86-64),
но здесь он последний, потому что старший байт идет первым.

Вот почему имеются разные дистрибутивы Linux для MIPS
(\q{mips} (big-endian) и \q{mipsel} (little-endian)).
Программа скомпилированная для одного соглашения об endiannes, не сможет работать в OS использующей
другое соглашение.

Еще один пример связанный с big-endian в MIPS в этой книге: \myref{MIPS_structure_big_endian}.

\subsection{Bi-endian (переключаемый порядок)}

CPU поддерживающие оба порядка, и его можно переключать, включают в себя ARM, PowerPC, SPARC, MIPS, \ac{IA64}, итд.

\subsection{Конвертирование}

\myindex{x86!\Instructions!BSWAP}
Инструкция \TT{BSWAP} может использоваться для конвертирования.

\myindex{TCP/IP}
Сетевые пакеты TCP/IP используют соглашение big-endian, вот почему программа, работающая на little-endian архитектуре
должна конвертировать значения.

Обычно, используются функции \TT{htonl()} и \TT{htons()}.

Порядок байт big-endian в среде TCP/IP также называется, \q{network byte order},
а порядок байт на компьютере \q{host byte order}.
На архитектуре Intel x86, и других little-endian архитектурах, \q{host byte order} это little-endian, 
а вот на IBM POWER это может быть big-endian, так что на последней, 
\TT{htonl()} и \TT{htons()} не меняют порядок байт.

