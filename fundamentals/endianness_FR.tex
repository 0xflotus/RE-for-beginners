\mysection{Endianness}
\label{sec:endianness}

L'endianness (boutisme) est la façon de représenter les valeurs en mémoire.

\subsection{Big-endian}

La valeur \TT{0x12345678} est représentée en mémoire comme:

\begin{center}
\begin{tabular}{ | l | l | }
\hline
\HeaderColor adresse en mémoire & \HeaderColor valeur de l'octet \\
\hline
+0 & 0x12 \\
\hline
+1 & 0x34 \\
\hline
+2 & 0x56 \\
\hline
+3 & 0x78 \\
\hline
\end{tabular}
\end{center}

Les CPUs big-endian comprennent les Motorola 68k, IBM POWER.

\subsection{Little-endian}

La valeur \TT{0x12345678} est représentée en mémoire comme:

\begin{center}
\begin{tabular}{ | l | l | }
\hline
\HeaderColor adresse en mémoire & \HeaderColor valeur de l'octet \\
\hline
+0 & 0x78 \\
\hline
+1 & 0x56 \\
\hline
+2 & 0x34 \\
\hline
+3 & 0x12 \\
\hline
\end{tabular}
\end{center}

Les CPUS little-endian comprennent les Intel x86.

\subsection{\Example}

Prenons une système Linux MIPS big-endian déjà installé et prêt dans QEMU
\footnote{Disponible au téléchargement ici: \url{http://go.yurichev.com/17008}}.


Et compilons cet exemple simple:

\begin{lstlisting}[style=customc]
#include <stdio.h>

int main()
{
	int v;

	v=123;

	printf ("%02X %02X %02X %02X\n", 
		*(char*)&v,
		*(((char*)&v)+1),
		*(((char*)&v)+2),
		*(((char*)&v)+3));
};
\end{lstlisting}

Après l'avoir lancé nous obtenons:

\begin{lstlisting}
root@debian-mips:~# ./a.out 
00 00 00 7B
\end{lstlisting}

C'est ça.
0x7B est 123 en décimal.
En architecture little-endian, 7B est le premier octet (vous pouvez vérifier en x86
ou en x86-64, mais ici c'est le dernier, car l'octet le plus significatif vient en
premier.

C'est pourquoi il y a des distributions Liux séparées pour MIPS (\q{mips} (big-endian)
et \q{mipsel} (little-endian)).
Il est impossible pour un binaire compilé pour une architecture de fonctionner sur
un \ac{OS} avec une architecture différente.

Il y a un exemple de MIPS big-endian dans ce livre: \myref{MIPS_structure_big_endian}.

\subsection{Bi-endian}

Les CPUs qui peuvent changer d'endianness sont les ARM, PowerPC, SPARC, MIPS, \ac{IA64}, etc.

\subsection{Convertir des données}

\myindex{x86!\Instructions!BSWAP}
L'instruction \TT{BSWAP} peut être utilisée pour la conversion.

\myindex{TCP/IP}
Les paquets de données des réseaux TCP/IP utilisent la convention bit-endian, c'est
donc pourquoi un programme travaillant en architecture little-endian doit convertir
les valeurs.
Les fonctions \TT{htonl()} et \TT{htons()} sont utilisées en général.

En TCP/IP, big-endian est aussi appelé \q{network byte order}, tandis que l'ordre
des octets sur l'ordinateur \q{host byte order}.
Le \q{host byte order} est little-endian sur les x86 Intel et les autres architectures
little-endian, mais est big-endian sur les IBM POWER, donc \TT{htonl()} et \TT{htons()}
ne modifient aucun octet sur cette dernière.

