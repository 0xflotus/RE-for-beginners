\newcommand{\HashFuncChapterName}{Fonctions de hachage}
\mysection{\HashFuncChapterName}
\label{hash_func}

\myindex{\HashFuncChapterName}
\myindex{CRC32}
Un exemple très simple est CRC32, un algorithme qui fournit des checksum plus \q{fort}
à des fins de vérifications d'intégrité.
Il est impossible de restaurer le texte d'origine depuis la valeur du hash, il a
beaucoup moins d'informations:
Mais CRC32 n'est pas cryptographiquement sûr: on sait comment modifier un texte afin
que son hash CRC32 résultant soit celui que l'on veut.
Les fonctions cryptographiques sont protégées contre cela.\\
\\
\myindex{MD5}
\myindex{SHA1}
MD5, SHA1, etc. sont de telles fonctions et elles sont largement utilisées pour hacher
les mots de passe des utilisateurs afin de les stocker dans une base de données.
En effet: la base de données d'un forum Internet ne doit pas contenir les mots de
passe des utilisateurs (une base de données volée compromettrait tous les mots de
passe des utilisateurs) mais seulement les hachages (donc un cracker ne pourrait
pas révéler les mots de passe).
En outre, un forum Internet n'a pas besoin de connaître votre mot de passe exactement,
il a seulement besoin de vérifier si son hachage est le même que celui dans la base
de données, et vous donne accès s'ils correspondent.
Une des méthodes de cracking la plus simple est simplement d'essayer de hacher tous
les mots de passe possible pour voir celui qui correspond à la valeur recherchée.
D'autres méthodes sont beaucoup plus complexes.
% TODO1 add about Rainbow tables

\subsection{Comment fonctionnent les fonctions à sens unique?}

Une fonction à sens unique est une fonction qui est capable de transformer une valeur
en une autre, tandis qu'il est impossible (ou très difficile) de l'inverser.
Certaines personnes éprouvent des difficultés à comprendre comment ceci est possible.
Voici une démonstration simple.

Nous avons un vecteur de 10 nombres dans l'intervalle 0..9, chacun est présent une
seule fois, par exemple:

\begin{lstlisting}
4 6 0 1 3 5 7 8 9 2
\end{lstlisting}

L'algorithme pour une fonction à sens unique la plus simple possible est:

\begin{itemize}
\item prendre le nombre à l'indice zéro (4 dans notre cas);
\item prendre le nombre à l'indice 1 (6 dans notre cas);
\item échanger les nombres aux positions 4 et 6.
\end{itemize}

Marquons les nombres aux positions 4 et 6:

\begin{lstlisting}
4 6 0 1 3 5 7 8 9 2
        ^   ^
\end{lstlisting}

Échangeons-les et nous obtenons ce résultat:

\begin{lstlisting}
4 6 0 1 7 5 3 8 9 2
\end{lstlisting}

En regardant le résultat, et même si nous connaissons l'algorithme, nous ne pouvons
pas connaître l'état initial de façon certaine, car les deux premiers nombres pourraient
être 0 et/ou 1, et pourraient donc participer à la procédure d'échange.

Ceci est un exemple extrêmement simplifié pour la démonstration. Les fonctions à
sens unique réelles sont bien plus complexes.
