\subsection{Utiliser IMUL au lieu de MUL}
\label{IMUL_over_MUL}

\myindex{x86!\Instructions!MUL}
\myindex{x86!\Instructions!IMUL}
Un exemple comme \lstref{unsigned_multiply_C} où deux valeurs non signées sont multipliées
compile en \lstref{unsigned_multiply_lst} où \IMUL est utilisé à la place de \MUL.

Ceci est une propriété importante des instructions \MUL et \IMUL.
Tout d'abord, les deux produisent une valeur 64-bit si deux valeurs 32-bit sont multipliées,
ou une valeur 128-bit si deux valeurs 64-bit sont multipliées (le plus grand \glslink{product}{produit}
dans un environnement 32-bit est \\
\GTT{0xffffffff*0xffffffff=0xfffffffe00000001}).
Mais les standards \CCpp n'ont pas de moyen d'accèder à la moitié supérieure du résultat,
et un \glslink{product}{produit} a toujours la même taille que ses multiplicandes.
Et les deux instructions \MUL et \IMUL fonctionnent de la même manière si la moitié
supérieure est ignorée, i.e, elles produisent le même résultat dans la partie inférieure.
Ceci est une propriété importante de la façon de représenter les nombre en \q{complément
à deux}.

Donc, le compilateur \CCpp peut utiliser indifféremment ces deux instructions.

Mais \IMUL est plus flexible que \MUL, car elle prend n'importe quel(s) registre(s) comme
source, alors que \MUL nécessite que l'un des multiplicandes soit stocké dans le
registre \AX/\EAX/\RAX
Et même plus que ça: \MUL stocke son résultat dans la paire \GTT{EDX:EAX} en environnement
32-bit, ou \GTT{RDX:RAX} dans un 64-bit, donc elle calcule toujours le résultat complet.
Au contraire, il est possible de ne mettre qu'un seul registre de destination lorsque
l'on utilise \IMUL, au lieu d'une paire, et alors le \ac{CPU} calculera seulement
la partie basse, ce qui fonctionne plus rapidement [voir Torborn Granlund,
\IT{Instruction latencies and throughput for AMD and Intel x86 processors}\footnote{\url{http://yurichev.com/mirrors/x86-timing.pdf}}].

Cela étant considéré, les compilateurs \CCpp peuvent générer l'instruction \IMUL
plus souvent que \MUL.

\myindex{Compiler intrinsic}
Néanmoins, en utilisant les fonctions intrinsèques du compilateur, il est toujours
possible d'effectuer une multiplication non signée et d'obtenir le résultat \IT{complet}.
Ceci est parfois appelé \IT{multiplication étendue}.
MSVC a une fonction intrinsèque pour ceci, appelée \IT{\_\_emul}\footnote{\url{https://msdn.microsoft.com/en-us/library/d2s81xt0(v=vs.80).aspx}}
et une autre: \IT{\_umul128}\footnote{\url{https://msdn.microsoft.com/library/3dayytw9%28v=vs.100%29.aspx}}.
GCC offre le type de données \IT{\_\_int128}, et dans le cas de multiplicandes 64-bit,
ils sont déjà promus en 128-bit, puis le \glslink{product}{produit} est stocké dans
une autre valeur \IT{\_\_int128}, puis le résultat est décalé de 64 bits à droite,
et vous obtenez la moitié haute du résultat\footnote{Exemple: \url{http://stackoverflow.com/a/13187798}}.

\subsubsection{Fonction MulDiv() dans Windows}
\myindex{Windows!Win32!MulDiv()}

Windows possède la fonction MulDiv()\footnote{\url{https://msdn.microsoft.com/en-us/library/windows/desktop/aa383718(v=vs.85).aspx}},
fonction qui fusionne une multiplication et une division, elle multiplie deux entiers
32-bit dans une valeur 64-bit intermédiaire et la divise par un troisième entier 32-bit.
C'est plus facile que d'utiliser deux fonctions intrinsèques, donc les développeurs
de Microsoft ont écrit une fonction spéciale pour cela.
Et il semble que ça soit une fonction très utilisée, à en juger par son utilisation.

