\mysection{Целочисленное переполнение (integer overflow)}

Я сознательно расположил эту секцию после секции о представлении знаковых чисел.

В начале, взгляние на эту реализацию ф-ции \IT{itoa()} из \InSqBrackets{\KRBook}:

\begin{lstlisting}[style=customc]
void itoa(int n, char s[])
{
        int i, sign;
        if ((sign = n) < 0) /* record sign */
                n = -n; /* make n positive */
        i = 0;
        do { /* generate digits in reverse order */
                s[i++] = n % 10 + '0'; /* get next digit */
        } while ((n /= 10) > 0); /* delete it */
        if (sign < 0)
                s[i++] = '-';
        s[i] = '\0';
        strrev(s);
}
\end{lstlisting}

( Полный текст: \url{https://github.com/DennisYurichev/RE-for-beginners/blob/master/fundamentals/itoa_KR.c} )

Здесь есть малозаметная ошибка. Попробуйте её найти. Можете скачать исходный код, скомпилировать его, итд.
Ответ на следующей странице.

\clearpage

Из \InSqBrackets{\KRBook}:

\begin{framed}
\begin{quotation}
Упражнение 3-4. В представлении чисел с помощью дополнения до двойки наша версия функции \IT{itoa}
не умеет обрабатывать самое большое по модулю отрицательное число, т.е., значение 
\IT{n}, равное $-(2^{wordsize-1})$. Объясните, почему это так. Доработайте функцию так, чтобы она
выводила это число правильно независимо от системы, в которой она работает.
\end{quotation}
\end{framed}

Ответ: ф-ция не может корректно обработать самое большое отрицательное число (INT\_MIN или 0x80000000 или -2147483648).

Как изменить знак? Инвертируйте все биты и прибавьте 1.
Если инвертировать все биты в значении INT\_MIN (0x80000000), это 0x7fffffff. Прибавьте 1 и это снова 0x80000000.
Так что смена знака не дает никакого эффекта.
Это важный артефакт two's complement-системы.

Еще об этом:

\begin{itemize}
\item blexim -- Basic Integer Overflows\footnote{\url{http://phrack.org/issues/60/10.html}}

\item Yannick Moy, Nikolaj Bjørner, and David Sielaff -- Modular Bug-finding for Integer Overflows in the Large: Sound, Efficient, Bit-precise Static Analysis\footnote{\url{https://yurichev.com/mirrors/SMT/z3prefix.pdf}}
\end{itemize}

