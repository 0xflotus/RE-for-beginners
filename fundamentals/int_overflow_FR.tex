\section{Dépassement d'entier}

%%I intentionally put this section after the section about signed number representation.
J'ai intentionnellement mis cette section après celle sur la représentation des nombres
signés.

%%First, take a look at this implementation of \IT{itoa()} function from \InSqBrackets{\KRBook}:
Tout d'abord, regardons l'implémentation de la fonction  \IT{itoa()} dans \InSqBrackets{\KRBook}:

\begin{lstlisting}[style=customc]
void itoa(int n, char s[])
{
        int i, sign;
        if ((sign = n) < 0) /* record sign */
                n = -n; /* make n positive */
        i = 0;
        do { /* generate digits in reverse order */
                s[i++] = n % 10 + '0'; /* get next digit */
        } while ((n /= 10) > 0); /* delete it */
        if (sign < 0)
                s[i++] = '-';
        s[i] = '\0';
        strrev(s);
}
\end{lstlisting}

%%( The full source code: \url{https://github.com/DennisYurichev/RE-for-beginners/blob/master/fundamentals/itoa_KR.c} )
( Le code soutce complet: \url{https://github.com/DennisYurichev/RE-for-beginners/blob/master/fundamentals/itoa_KR.c} )

%%It has a subtle bug. Try to find it. You can download source code, compile it, etc.
%%The answer on the next page.
Elle a un bogue subtil. Essayez de le trouver. Vous pouvez téléchargez le code source,
le compiler, etc.
La réponse se trouve à la page suivante.

\clearpage

%%From \InSqBrackets{\KRBook}:
De \InSqBrackets{\KRBook}:

\begin{framed}
\begin{quotation}
Exercice 3-4. Dans un système de représentation des nombres par complément à deux
notre version de \IT{itoa} ne peut pas traiter le plus grand nombre négatif. c'est-à-
dire la valeur de \IT{n} égale à $-(2^{wordsize-1})$. Pourquoi ? Modifiez \IT{itoa}
de façon à ce qu'elle traite ce cas correctement, quelle que soit la machine utilisée.
\end{quotation}
\end{framed}

La réponse est: la fonction ne peut pas traiter le plus grand nombre négatif (INT\_MIN
ou 0x80000000 ou -2147483648) correctement.

Comment changer le signe? Inverser tous les bits et ajouter 1.
Si vous inversez tous les bits de la valeur INT\_MIN (0x80000000), ça donne 0x7fffffff.
Ajouter 1 et vous obtenez à nouveau 0x80000000.
C'est un artefact important du système de complément à deux.

Lectures  complémentaires:

\begin{itemize}
\item blexim -- Basic Integer Overflows\footnote{\url{http://phrack.org/issues/60/10.html}}

\item Yannick Moy, Nikolaj Bjørner, et David Sielaff -- Modular Bug-finding for Integer Overflows in the Large: Sound, Efficient, Bit-precise Static Analysis\footnote{\url{https://yurichev.com/mirrors/SMT/z3prefix.pdf}}
\end{itemize}

